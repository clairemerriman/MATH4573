\documentclass{ximera}
\usepackage{amssymb, latexsym, amsmath, amsthm, graphicx, amsthm,alltt,color, listings,multicol,hyperref,xr-hyper,aliascnt,enumitem}
\usepackage{xfrac}


\usepackage{parskip}
\usepackage{graphicx}
\usepackage[,margin=0.7in]{geometry}
\setlength{\textheight}{8.5in}
 
\usepackage{tkz-euclide}
%\usetkzobj{all}
\tikzstyle geometryDiagrams=[rounded corners=.5pt,ultra thick,color=black]
\colorlet{penColor}{black} % Color of a curve in a plot


\usepackage{subcaption}
\usepackage{float}
\usepackage{fancyhdr}
\usepackage{pdfpages}
\newcounter{includepdfpage}


\newcommand{\semester}{%
  \ifcase\month
  \or Spring %1
  \or Spring %2
  \or Spring %3
  \or Spring %4
  \or Spring  %5
  \or Fall %8
  \or Fall %9
  \or Fall %10
  \or Fall %11
  \or Fall %12
  \fi
}
\usepackage{currfile}
\usepackage{xstring}


\lhead{\large{Number Theory: MAT-255}}
%Put your Document Title (Camp: Topic) Here
\chead{}
\rhead{\semester 24}
\lfoot{}
\cfoot{}
\rfoot{Page \thepage}
\renewcommand\headrulewidth{0pt}
\renewcommand\footrulewidth{0pt}

\headheight 50pt
\headsep 30pt

\author{Claire Merriman}
\date{Spring 2024}


%%%%%%%%%%%%%%%%%%%%%
% Counters and autoref for unnumbered environments
%%%%%%%%%%%%%%%%%%%%%
\theoremstyle{plain}


\newtheorem*{namedthm}{Theorem}
\newcounter{thm}%makes pointer correct
\providecommand{\thmname}{Proposition}

\makeatletter
\NewDocumentEnvironment{thm*}{o}
 {%
  \IfValueTF{#1}
    {\namedthm[#1]\refstepcounter{thm}\def\@currentlabel{(#1)}}%
    {\namedthm}%
 }
 {%
  \endnamedthm
 }
\makeatother


\newtheorem*{namedprop}{Proposition}
\newcounter{prop}%makes pointer correct
\providecommand{\propname}{Proposition}

\makeatletter
\NewDocumentEnvironment{prop*}{o}
 {%
  \IfValueTF{#1}
    {\namedprop[#1]\refstepcounter{prop}\def\@currentlabel{(#1)}}%
    {\namedprop}%
 }
 {%
  \endnamedprop
 }
\makeatother

\newtheorem*{namedlem}{Lemma}
\newcounter{lem}%makes pointer correct
\providecommand{\lemname}{Lemma}

\makeatletter
\NewDocumentEnvironment{lem*}{o}
 {%
  \IfValueTF{#1}
    {\namedlem[#1]\refstepcounter{lem}\def\@currentlabel{(#1)}}%
    {\namedlem}%
 }
 {%
  \endnamedlem
 }
\makeatother

\newtheorem*{namedcor}{Corollary}
\newcounter{cor}%makes pointer correct
\providecommand{\corname}{Corollary}

\makeatletter
\NewDocumentEnvironment{cor*}{o}
 {%
  \IfValueTF{#1}
    {\namedcor[#1]\refstepcounter{cor}\def\@currentlabel{(#1)}}%
    {\namedcor}%
 }
 {%
  \endnamedcor
 }
\makeatother

\theoremstyle{definition}

\newtheorem*{innerrem}{Remark}
\newcounter{rem}%makes pointer correct
\providecommand{\remname}{Remark}

\makeatletter
\NewDocumentEnvironment{rem}{o}
 {%
  \IfValueTF{#1}
    {\innerrem[#1]\refstepcounter{rem}\def\@currentlabel{(#1)}}%
    {\innerrem}%
 }
 {%
  \endinnerrem
 }
\makeatother

\newtheorem*{innerdefn}{Definition}%%placeholder
\newcounter{defn}%makes pointer correct
\providecommand{\defnname}{Definition}

\makeatletter
\NewDocumentEnvironment{defn}{o}
 {%
  \IfValueTF{#1}
    {\innerdefn[#1]\refstepcounter{defn}\def\@currentlabel{(#1)}}%
    {\innerdefn}%
 }
 {%
  \endinnerdefn
 }
\makeatother

\newtheorem*{scratch}{Scratch Work}


\newtheorem*{namedconj}{Conjecture}
\newcounter{conj}%makes pointer correct
\providecommand{\conjname}{Conjecture}
\makeatletter
\NewDocumentEnvironment{conj}{o}
 {%
  \IfValueTF{#1}
    {\innerconj[#1]\refstepcounter{conj}\def\@currentlabel{(#1)}}%
    {\innerconj}%
 }
 {%
  \endinnerconj
 }
\makeatother

\newtheorem*{poll}{Poll question}
\newtheorem{tps}{Think-Pair-Share}[section]
%\newtheorem{br}{In-class Problem}[section]
\newtheorem*{cs}{Crowd Sourced Proof}

\newlist{checklist}{itemize}{2}
\setlist[checklist]{label=$\square$}

\newenvironment{obj}{
	\textbf{Learning Objectives.} By the end of class, students will be able to:
		\begin{itemize}}
		{\!.\end{itemize}
		}

\newenvironment{pre}{
	\begin{description}
	}{
	\end{description}
}


\newcounter{br}%makes pointer correct
\providecommand{\brname}{In-class Problem}

\newenvironment{br}[1][2in]%
{%Env start code
\problemEnvironmentStart{#1}{In-class Problem}
\refstepcounter{br}
}
{%Env end code
\problemEnvironmentEnd
}

\newcounter{ex}%makes pointer correct
\providecommand{\exname}{Homework Problem}
\newenvironment{ex}[1][2in]%
{%Env start code
\problemEnvironmentStart{#1}{Homework Problem}
\refstepcounter{ex}
}
{%Env end code
\problemEnvironmentEnd
}



\newenvironment{sketch}
 {\begin{proof}[Sketch of Proof]}
 {\end{proof}}
%\newenvironment{hint}
%  {\begin{proof}[Hint]}
%  {\end{proof}}

\newcommand{\gt}{>}
\newcommand{\lt}{<}
\newcommand{\N}{\mathbb N}
\newcommand{\Q}{\mathbb Q}
\newcommand{\Z}{\mathbb Z}
\newcommand{\C}{\mathbb C}
\newcommand{\R}{\mathbb R}
\renewcommand{\H}{\mathbb{H}}
\newcommand{\lcm}{\operatorname{lcm}}
\newcommand{\nequiv}{\not\equiv}
\newcommand{\ord}{\operatorname{ord}}
\newcommand{\ds}{\displaystyle}
\newcommand{\floor}[1]{\left\lfloor #1\right\rfloor}
\newcommand{\legendre}[2]{\left(\frac{#1}{#2}\right)}



%%%%%%%%%%%%



\newtheorem*{prob}{Homework Problem}
%\newtheorem*{algorithm}{Algorithm}

\StrBetween*[1,1]{\currfilename}{Week}{Sp}[\week]

\title{Week \week--MAT-255 Number Theory}

\begin{document}

%\maketitle
%\tableofcontents
%%%%%%%%%%%%%%%%%%%%%%%%%%
%%%%%%%%%%%%%%%%%%%%%%%%%%
\section{Monday, January 29: More algebraic proofs and the Euclidean Algorithm}
%%%%%%%%%%%%%%%%%%%%%%%%%%
%%%%%%%%%%%%%%%%%%%%%%%%%%

\begin{obj}
\item  Understand turning scratch work into proof for algebraic proofs
\item  Approach the floor function problems from Homework 1
\item Prove the Euclidean Algorithm halts and generates the greatest common divisor of two positive integers
\end{obj}

\begin{pre}
 \item[Read] None 
 \item[Turn in] The Learn TeX assignment using a copy of the Homework Template.
\end{pre}
%%%%%%%%%%%%%%%%%%%%%%%%%%
\subsection{Example of proof like the floor function (30 minutes)}
%%%%%%%%%%%%%%%%%%%%%%%%%%
I did not go over the entire example in class. This also took the place of in class problems.

Let's explicitly think about the floor function as a function. That is, $f(x):\mathbb{R}\to\mathbb{Z}$ (a function from the real numbers to the integers). Here is a restatement of the homework problem:

\begin{prob}
For each of the following equations, find a domain for $f(x)=\lfloor x \rfloor$ make the statement true. Prove your statement. 
	\begin{enumerate}
 		\item $f( x ) + f( x ) =f( 2x)$
		\item $f( x + 3 )  = 3 +f( x)$
		\item $f( x +3 ) = 	3 + x$
	\end{enumerate} 
\end{prob}

Here is a similar problem with proofs where $g:\mathbb{R}\to\mathbb{R}$. Note that the scratch work is one way to think about solving the problem but would not be included in the homework writeup.

\begin{example}
 For each of the following equations, find the full domain for $g(x)=3\sin(\pi x)$ that
 makes the statement true. Prove that the equation is always true on this domain. 
	\begin{enumerate}
 		\item $g( x ) + g( x ) =g( 2x)$
		
		\begin{scratch}
		We want to find a restriction such that $3\sin(\pi x)+3\sin(\pi x)=3\sin(2\pi x)$. Using the double angle formula, we get \[3\sin(2\pi x)=6\sin(\pi x)\cos(2\pi x).\] This means we are actually looking for 
			\begin{align*}
 				6\sin(\pi x)\cos(2\pi x)&=6\sin(\pi x)\\
				\cos(2\pi x)&=1.
			\end{align*}
		\end{scratch}

		\begin{solution}
 		If $x\in\mathbb{Z}$, then $g(x)+g(x)=g(2 x)$.
 
		\begin{proof}
 		Let $x\in\mathbb{Z}$. Then $g(x)+g(x)=3\sin(\pi x)+3\sin(\pi x)=0$ and $g(2x)=3\sin(2\pi x)=0$. Thus, $g(x)+g(x)=g(2 x)$.
		\end{proof}
		\end{solution}

		\item $g( x + 3 )  = 3 +g( x)$
		
		\begin{scratch}
 			We want to find a restriction such that $3\sin(\pi (x+3))=3+3\sin(\pi x)$.
			Using the angle addition formula, we get 
			\[3\sin(\pi x+3\pi))=3\sin(\pi x)\cos(3\pi) +3\cos(\pi x)\sin(3\pi)=-3\sin(\pi x).\] This means we are actually looking for 
			\begin{align*}
 				-3\sin(\pi x)&=3\sin(\pi x)+3\\
				-1&=2\sin(\pi x)\\.
			\end{align*}
		\end{scratch}
		
		
		\begin{solution}
 		If $x=2k+\frac{7}{6}$ or $x=2k-\frac{1}{6}$ for some $k\in\mathbb{Z}$, then $g (x+3)=g(x)+3.$
		
		\begin{proof}
		
		\begin{description} Note that $g(x+3)=3\sin(\pi x+3\pi)=-3\sin(\pi x)$ by the angle addition formula. We will consider two cases:
 			\item{Case 1:} Let $x=2k+\frac{7}{6}$ for some $k\in\mathbb{Z}$. Then 			
			\begin{align*}
 			g(x+3)&=-3\sin(\pi x)\\
			&=-3\sin(2k\pi+\frac{7\pi}{6})\\
			&=\frac{3}{2},
			\end{align*}
			and $g(x)+3=3\sin(2k\pi+\frac{7\pi}{6})+3=\frac{3}{2}$.
			\item{Case 2:} Let $x=2k-\frac{1}{6}$ for some $k\in\mathbb{Z}$. Then 			
			\begin{align*}
 			g(x+3)&=-3\sin(\pi x)\\
			&=-3\sin(2k\pi-\frac{\pi}{6})\\
			&=\frac{3}{2},
			\end{align*}
			and $g(x)+3=3\sin(2k\pi-\frac{\pi}{6})+3=\frac{3}{2}$.
		\end{description}
		Thus, $g(x+3)=g(x)+3$ when $x=2k+\frac{7}{6}$ or $x=2k-\frac{1}{6}$ for some $k\in\Z$.
		\end{proof}
		\end{solution}
		\item Let $h(x)=x^2-1$. For each of the following equations, find the full domain for $h(x)$ that makes the statement true. Prove your statement.
		
		\begin{enumerate}
			\item $h(x+3)=h(x)+3$
			\item $h(x+3)=x+3$
		\end{enumerate}


		\begin{scratch}
			First, $h(x+3)=(x+3)^2-1=x^2+6x+8$.
			
			For (a) 
			\begin{align*}
				x^2+6x+8&= x^2 -1+3\\
				6x& = -6
			\end{align*}
			For (b)
			\begin{align*}
				x^2+6x+8&= x+3\\
				x^2 +5x+5& = 0\\
				x&=\frac{-5\pm\sqrt{5}}{2}
			\end{align*}
		\end{scratch}
		
		\begin{solution}
			
			\begin{enumerate}
				\item For $h(x)=x^2-1,$ $h(x+3)=h(x)+3$ if and only if $x=-1$.
				\begin{proof}
					Let $h(x)=x^2-1$ and $x=-1$. Then $h(x+3)=3=0+3=h(x)+3$.

					To prove that this is the only such $x$, let $h(x+3)=h(x)+3$. Then $x^2+6x+8=x^2+2$, which simplifies to $x=-1$.
				\end{proof}
				\item For $h(x)=x^2-1,$ $h(x+3)=x+3$ if and only if $x=\frac{-5\pm\sqrt{5}}{2}$.
				\begin{proof}
					Let $h(x)=x^2-1$. First we consider the case $x=\frac{-5+\sqrt{5}}{2}$. Then $h(x+3)=\frac{1+\sqrt{5}}{2}=
					\frac{-5+\sqrt{5}}{2}+3=x+3$. 
					
					Next, we consider the case $x=\frac{-5-\sqrt{5}}{2}$. Then $h(x+3)=\frac{1-\sqrt{5}}{2}=
					\frac{-5-\sqrt{5}}{2}+3=x+3$.

					To prove that these are the only such $x$, let $h(x+3)=x+3$. Then $x^2+6x+8=x+3$, which  hasthe solutions  $x=\frac{-5\pm\sqrt{5}}{2}$.
				\end{proof}
			\end{enumerate}
		\end{solution}
	\end{enumerate}
\end{example}

%%%%%%%%%%%%%%%%%%%%%%%%%%
\subsection{The Euclidean Algorithm (20 minutes)}

Typically by \emph{Euclidean Algorithm}, we mean  both the algorithm and the theorem that the algorithm always generates the greatest common divisor of two (positive) integers.


\begin{thm*}[Euclidean algorithm] 
	Let $a,b\in\Z$ with $a\geq b>0$. By the division algorithm, there exist $q_1,r_1\in\Z$ such that 
	\[a=b q_1+r_1,\quad 0\leq r_1<b.\]
	If $r_1>0$, there exist $q_2,r_2\in\Z$ such that 
	\[b=r_1 q_2+r_2,\quad 0\leq r_2<r_1.\]
	If $r_2>0$, there exist $q_3,r_3\in\Z$ such that 
	\[r_1=r_2 q_3+r_3,\quad 0\leq r_3<r_2.\]
	
	Continuing this process, $r_n=0$ for some $n$. If $n>1$, then $\gcd(a,b)=r_{n-1}$. If $n=1$, then $\gcd(a,b)=b$.
	\end{thm*}
	\begin{proof}
	 Note that $r_1>r_2>r_3>\dots\geq0$ by construction. If the sequence did not stop, then we would have an infinite, decreasing sequence of positive integers, which is not possible. Thus, $r_n=0$ for some $n$. 
	 
	 When $n=1$, $a=bq+0$ and $\gcd(a,b)=b$.
	 
	 Lemma 1.12 states that for $a=bq_1+r_1$, $\gcd(a,b)=\gcd(b,r_1)$. This is because any common divisor of $a$ and $b$  is also a divisor of $r_1=a-bq_1$. 
	 
	 If $n>1$, then by repeated application of the Lemma 1.12, we have 
	 \[\gcd(a,b)=\gcd(b,r_1)=\gcd(r_1,r_2)=\cdots=\gcd(r_{n-2},r_{n-1})\]
	 Then $r_{n-2}=r_{n-1} q_n+0$. Thus $\gcd(r_{n-2},r_{n-1})=r_{n-1}$.
	\end{proof}
	
	When using the Euclidean algorithm, it can be tricky to keep track of what is happening. Doing a lot of examples can help.
	
%%%%%%%%%%%%%%%%%%%%%%%%%%
%%%%%%%%%%%%%%%%%%%%%%%%%%
\section{Wednesday, January 31: Practice with the Euclidean Algorithm and the Fundamental Theorem of Arithmetic}
%%%%%%%%%%%%%%%%%%%%%%%%%%
%%%%%%%%%%%%%%%%%%%%%%%%%%

\begin{obj}
\item  Use the (extended) Euclidean Algorithm to write $(a,b)$ as a linear combination of $a$ and $b$
\item Prove the Fundamental Theorem of Arithmetic
\item  Prove $\sqrt{2}$ is irrational
\end{obj}

\begin{pre}
 \item[Read] Strayer, Section 1.5 through Proposition 1.17
 \item[Turn in] 
 \begin{itemize}
	\item Answer these questions about the proof of the Fundamental Theorem of Arithmetic (taken from \href{https://maa.org/node/121566}{Helping Undergraduates Learn to Read Mathematics}):
	
	\begin{itemize}
		\item Can you write a brief outline (maybe 1/10 as long as the theorem) giving the logic of the argument -- proof by contradiction, induction on n, etc.? (This is KEY.)
		\item What mathematical raw materials are used in the proof? (Do we need a lemma? Do we need a new definition? A powerful theorem? and do you recall how to prove it? Is the full generality of that theorem needed, or just a weak version?)
		\item What does the proof tell you about why the theorem holds?
		\item Where is each of the hypotheses used in the proof?
		\item Can you think of other questions to ask yourself?
	\end{itemize}

\item Strayer states that the proof of Proposition 1.17 is ``obvious from the Fundamental Theorem of Arithmetic and the definitions of $(a,b)$ and $[a,b]$." Is this true? If so, why? If not, fill in the gaps.
 \end{itemize}

 
\begin{solution}
Answers to both questions will vary between students.
\end{solution}
\end{pre}
%%%%%%%%%%%%%%%%%%%%%%%%%

%%%%%%%%%%%%%%%%%%%%%%%%%
\subsection{Announcements}
%%%%%%%%%%%%%%%%%%%%%%%%%
We have gotten a bit ahead of the reading. I adjusted the reading assignments--make sure to read Section 6.1 for Friday! This section does not use any facts beyond divisibility.

%%%%%%%%%%%%%%%%%%%%%%%%%
\subsection{Practice with the Euclidean Algorithm (15 minutes)}
%%%%%%%%%%%%%%%%%%%%%%%%%%

Work in pairs to answer the following. Each pair will be assigned parts the following question.

\begin{br}[Strayer Exercises 32 and 54]
Find the greatest common divisors of the pairs of integers below and write the greatest common divisor as a linear combination of the integers.
\begin{enumerate}[label=32(\alph*)]
	% \item $(21,28)$
	
	% \begin{solution}
	% 	By inspection: $28-21=7$.

	% 	Using the Euclidean Algorithm:
	% 	$a=28,b=21$
	% 	\begin{align*}
	% 		28 & = 21(1)+7 &q_1=1,r_1=7 &&7=21(1)+28(-1)\\
	% 		21 & = 7(3) +0 & q_2=3, r_2=0
	% 	\end{align*}
	% 	so $28+(-1)21=7=(28,21)$
	% \end{solution}
	\setcounter{enumi}{1}
	\item $(32,56)$
	 \begin{solution}
	 	Using the Euclidean Algorithm:
	 	$a=56,b=32$
	 	\begin{align*}
	 		56 & = 32(1)+24 &q_1=1,r_1=24 &&24=56(1)+32(-1)\\
	 		32 & = 24(1) +8 & q_2=1, r_2=8 &&8=32(1)+24(-1)=32(1)+(56(1)+32(-1))(-1)=32(2)+56(-1)\\
	 		32&=8(4)+0 & q_3=4, r_3=0.
	 	\end{align*}
	 	so $56(-1)+32(2)=8=(56,32)$
	 \end{solution}

	\setcounter{enumi}{3}
	\item $(0,113)$
	 \begin{solution}
	 	Since $0=113(0)$, $(0,113)=113=0(0)=113(1)$.
	 \end{solution}
\end{enumerate}


\begin{enumerate}[label=54(\alph*)]
	\setcounter{enumi}{1}
	\item $(78,708)$
	 \begin{solution}
	 	Using the Euclidean Algorithm:
	 	$a=708,b=78$
	 	\begin{align*}
	 		708 & = 78(9)+6 &q_1=9,r_1=6 &&6=708(1)+78(-9)\\
	 		78 & = 6(13) +0 & q_2=13, r_2=0.
	 	\end{align*}
	 	so $708(1)+78(-6)=6=(78,708)$
	 \end{solution}
\end{enumerate}
\end{br}

%%%%%%%%%%%%%%%%%%%%%%%%%%
\subsection{Fundamental Theorem of Arithmetic (35 minutes)}

Observations from reading assignment
\begin{itemize}
 \item Lemma 1.14 and Corollary 1.15 (if $p\mid a_1a_2\dots a_n$ then $p\mid a_i$ for some $i$)
\begin{itemize}
\item Lemma 1.14 is used as the base case in the proof of Corollary 1.15
 \item Prove factorization is unique by contradiction
\end{itemize}
\end{itemize}

\begin{thm*}[Fundamental Theorem of Arithmetic]
	Every integer greater than one can be written in the form $p_1^{a_1}p_2^{a_2}\cdots p_r^{a_r}$ where the $p_i$ are distinct prime numbers and the $a_i$ are positive integers. This factorization into primes is unique up to the ordering of the terms.
\end{thm*}
\begin{proof}[Alternate proof for existence]
 We will show that every integer $n$ greater than $1$ has a prime factorization. First, note that all primes are already in the desired form. We will use induction to show that every composite integer can be factored into the product of primes. When $n=4$, we can write $n=2^2$, so $4$ has the desired form.
 
Assume that for all integers $k$ with $1<k<n$, $k$ can be written in the form  $p_1^{a_1}p_2^{a_2}\cdots p_r^{a_r}$ where the $p_i$ are distinct prime numbers and the $a_i$ are positive integers. If $n$ is prime, we are done, otherwise there exists $a,b\in\Z$ with $1<a,b<n$ such that $n=ab$. By the induction hypothesis, there exist primes $p_1,p_2,\dots,p_r,q_1,q_2,\dots,q_s$ and positive integers $a_1,a_2,\dots,a_r,b_1,b_2,\dots b_s$ such that $a=p_1^{a_1}p_2^{a_2}\cdots p_r^{a_r}$ and $b=q_1^{b_1}q_2^{a_2}\cdots q_s^{b_a}$. Then \[n=p_1^{a_1}p_2^{a_2}\cdots p_r^{a_r}q_1^{b_1}q_2^{a_2}\cdots q_s^{b_a}.\]
\end{proof}

We will use an idea similar to the proof of the Fundamental Theorem of Arithmetic to proof the following:

\begin{br}
	\begin{prop*}
		$\sqrt{2}$ is irrational
	\end{prop*}

	As class, put the steps of the proof in order, then fill in the missing information.
\end{br}

% Finally, work two groups. Each group will be assigned one of the following question.


%\begin{br}
%	Let $p$ be prime.
%	\begin{enumerate}
%		\item If $(a,b)=p$, what are the possible values of $(a^2,b)$? Of $(a^3,b)$? Of $(a^2,b^3)$?
%		
%		% \begin{solution}
%		% 	If $(a,b)=p$, then there exist $j,k\in\Z$ such that $a=pj, b=pk$, and $p\nmid j$ or $p\nmid k$ (otherwise $(a,b)=p^2$). 
%		% 	\[a^2=p^2j^2,\quad
%		% 	a^3=p^3j^3,\quad
%		% 	b^3=p^3k^3\]
%		% 	Then $(a^2,b)$ is $p$ if $p\nmid k$ or $p^2$ if $p\mid k$; and
%		% 	$(a^3,b)$ is $p$ if $p\nmid k,$ $p^2$ if $p\mid k$ and $p^2\nmid k,$ or $p^3$ if $p^2\mid k$. 
%			
%		% 	If $p\mid j,$ then $p\nmid k$ and   
%		% 	$(a^2,b^3)=p^3$.
%		% 	If $p\nmid j,$ then   
%		% 	$(a^2,b^3)=p^2.$
%		% \end{solution}
%		\item If $(a,b)=p$ and $(b,p^3)=p^2$, find $(ab,p^4)$ and $(a+b,p^4)$.
%		
%		% \begin{solution}
%		% 	There exists $j,k\in\Z$ such that $a=pj, b=p^2k,$ and $p\nmid k, p\nmid k$. 
%		% 	Then $ab=p^3jk$ and $a+b=pj+p^2k=p(j+pk)$. Thus, $(ab,p^4)=p^3$ and $(a+b,p^4)=p.$
%		% \end{solution}
%	\end{enumerate}
%\end{br}

%%%%%%%%%%%%%%%%%%%%%%%%%%
\section{Friday, February 2: Quiz 2, Greatest Common Divisors and Diophantine Equations}
%%%%%%%%%%%%%%%%%%%%%%%%%%
%%%%%%%%%%%%%%%%%%%%%%%%%%

\begin{obj}
	\item Prove the formula for integer solutions to $ax+by=c$.
	\item State when integer solution exist for $a_1x_1+\cdots+a_kx_k=c$.
\end{obj}

\begin{pre}
 \item[Read] Strayer, Section 6.1
 \item[Turn in] Exercise 2a.
 Find all integer solutions to $18x+28y=10$
% \begin{solution}
% Notice that $18(-1)+28(1)=10$. This can be done by inspection or the Euclidean Algorithm. Then our specific solution is $x_0=-1, y_0=1$ and all solutions have the form \[x=-1+\frac{28n}{2}=-1+14n,\quad y=1-\frac{18n}{2}=1-9n \qquad \textnormal{for all $n\in\Z$.}\]
% \end{solution}
\end{pre}


%%%%%%%%%%%%%%%%%%%%%%%%%
\subsection{Quiz (10 minutes)}
%%%%%%%%%%%%%%%%%%%%%%%%%%
%%%%%%%%%%%%%%%%%%%%%%%%%%%
\subsection{More greatest common divisor (40 min)} 
%%%%%%%%%%%%%%%%%%%%%%%%%%%

\begin{lem}\label{gcd_3case}
	Let $a,b,c\in\Z,$ with $a\neq 0$. Then $(a,b,c)=((a,b),c).$
\end{lem}

 \begin{proof}
 	Let $a,b,c\in\Z,$ with $a\neq 0$. Define $d=(a,b,c)$ and $e=((a,b),c).$ We will show that $d\mid e$ and $e\mid d$. Since the greatest common divisor is positive, we can conclude that $d=e$\footnote{This is not true in general and a common mistake on the homework. In general $d=\pm e$}.

 	Since $d=(a,b,c),$ we know $d\mid a, d\mid b,$ and $d\mid c$. By the lemma we are about to prove, $d\mid (a,b)$. Thus, $d$ is a common divisor of $(a,b)$ and $c$, so $d\mid e$.

 	Since $e=((a,b),c)$, $e\mid  (a,b)$ and $e\mid c$. Since $e\mid (a,b),$ we know $e\mid a$ and $e\mid b$ by the same lemma. Thus, $e$ is a common divides of $a,b$ and $c$
 \end{proof}

\begin{lem}\label{gcd_mult}
	Let $a,b\in\Z,$ not both zero. Then any  common divisor of $a$ and $b$ divides the greatest common divisor.
\end{lem}

 \begin{proof}
 	Let $a,b\in\Z,$ not both zero. By Proposition 1.11, $(a,b)=am+bn$ for some $n,m\in\Z$. Thus, $d\mid (a,b)$ by linear combination.
 \end{proof}

\begin{lem}\label{gcd_trans}
 	Let $a,b\in\Z,$ not both zero. Then any divisor of $(a,b)$ is a common divisor of $a$ and $b$.
\end{lem}

\begin{proof}
 Let $c$ be a divisor of $(a,b)$. Since $(a,b)\mid a$ and $(a,b )\mid b,$ then $c\mid a$ and $c\mid b$ by transitivity.
\end{proof}
%\begin{prop*}
%	Let $a,b,c\in\Z,$ at least one not equal to zero, and $d=(a,b,c)$. Then there exists integers $x,y,z$ such that \[ax+by+cz=d.\]
%\end{prop*}


\begin{prop*}
 Let $a_1,\dots,a_n\in\Z$ with $a_1\neq 0$.  Then 
	\[(a_1,\dots,a_n)=((a_1,a_2,a_3,\dots,a_{n-1}),a_n).\]
\end{prop*}
\begin{proof}
 Let $k=2$. The since $((a_1,a_2))=(a_1,a_2)$ by the definition of greatest common divisor of one integer,  $(a_1,a_2)=((a_1,a_2))$. The $k=3$ case is the first lemma in this section (\ref{gcd_3case}).
 
 Assume that for all $2\leq k< n$, 
 	\[(a_1,\dots,a_k)=((a_1,a_2,a_3,\dots,a_{k-1}),a_k).\]
Let $d=(a_1,a_2,a_3,\dots,a_{k}),$
$e=((a_1,a_2,a_3,\dots,a_{k}),a_{k+1})=(d,a_{k+1}),$ and $f= (a_1,a_2,a_3,\dots,a_{k},a_{k+1}).$ We will show that $e\mid f$ and $f\mid e$. Since both $e$ and $f$ are positive, this will prove that $e=f$.

Note that $e\mid (a_1,a_2,a_3,\dots,a_{k})$ and $e\mid a_{k+1}$ by definition. 
Since  $(a_1,\dots,a_k)=((a_1,a_2,a_3,\dots,a_{k-1}),a_k)$ by the induction hypothesis, $e\mid(a_1,a_2,a_3,\dots,a_{k-1})$ and $e\mid a_k$ by Lemma \ref{gcd_trans}. Again, by the induction hypothesis, $(a_1,a_2,a_3,\dots,a_{k-1})=((a_1,a_2,a_3,\dots,a_{k-2}),a_{k-1}),$ so $e\mid a_{k-1}$ and $e\mid (a_1,a_2,a_3,\dots,a_{k-2})$ by Lemma \ref{gcd_trans}. Repeat this process until we get $(a_1,a_2,a_3)=((a_1,a_2),a_3)$, so $e\mid a_3$ and $e\mid (a_1,a_2)$ by Lemma \ref{gcd_trans}. Thus $e\mid a_1,a_2,\dots,a_{k+1}$ by repeated applications of Lemma \ref{gcd_trans}. By the generalized version of the Lemma \ref{gcd_mult} on Homework 3, $e\mid f.$

To show that $f\mid e,$ we note that $f\mid a_1,a_2,\dots, a_k,a_{k+1}$ by definition. Then $f\mid d$ by the generalized version of the Lemma \ref{gcd_mult} on Homework 3. Since $e=(d,a_k),$ we have that $f\mid e$ by Lemma \ref{gcd_mult}.
\end{proof} 
\end{document}

%%%%%%%%%%%%%%%%%%%%%%%%%%%
