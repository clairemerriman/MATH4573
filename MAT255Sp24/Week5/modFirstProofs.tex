\documentclass{ximera}
\usepackage{amssymb, latexsym, amsmath, amsthm, graphicx, amsthm,alltt,color, listings,multicol,xr-hyper,hyperref,aliascnt,enumitem}
\usepackage{xfrac}

\usepackage{parskip}
\usepackage[,margin=0.7in]{geometry}
\setlength{\textheight}{8.5in}

\usepackage{epstopdf}

\DeclareGraphicsExtensions{.eps}
\usepackage{tikz}


\usepackage{tkz-euclide}
%\usetkzobj{all}
\tikzstyle geometryDiagrams=[rounded corners=.5pt,ultra thick,color=black]
\colorlet{penColor}{black} % Color of a curve in a plot


\usepackage{subcaption}
\usepackage{float}
\usepackage{fancyhdr}
\usepackage{pdfpages}
\newcounter{includepdfpage}
\usepackage{makecell}


\usepackage{currfile}
\usepackage{xstring}




\graphicspath{  
{./otherDocuments/}
}

\author{Claire Merriman}
\newcommand{\classday}[1]{\def\classday{#1}}

%%%%%%%%%%%%%%%%%%%%%
% Counters and autoref for unnumbered environments
% Not needed??
%%%%%%%%%%%%%%%%%%%%%
\theoremstyle{plain}


\newtheorem*{namedthm}{Theorem}
\newcounter{thm}%makes pointer correct
\providecommand{\thmname}{Theorem}

\makeatletter
\NewDocumentEnvironment{thm*}{o}
 {%
  \IfValueTF{#1}
    {\namedthm[#1]\refstepcounter{thm}\def\@currentlabel{(#1)}}%
    {\namedthm}%
 }
 {%
  \endnamedthm
 }
\makeatother


\newtheorem*{namedprop}{Proposition}
\newcounter{prop}%makes pointer correct
\providecommand{\propname}{Proposition}

\makeatletter
\NewDocumentEnvironment{prop*}{o}
 {%
  \IfValueTF{#1}
    {\namedprop[#1]\refstepcounter{prop}\def\@currentlabel{(#1)}}%
    {\namedprop}%
 }
 {%
  \endnamedprop
 }
\makeatother

\newtheorem*{namedlem}{Lemma}
\newcounter{lem}%makes pointer correct
\providecommand{\lemname}{Lemma}

\makeatletter
\NewDocumentEnvironment{lem*}{o}
 {%
  \IfValueTF{#1}
    {\namedlem[#1]\refstepcounter{lem}\def\@currentlabel{(#1)}}%
    {\namedlem}%
 }
 {%
  \endnamedlem
 }
\makeatother

\newtheorem*{namedcor}{Corollary}
\newcounter{cor}%makes pointer correct
\providecommand{\corname}{Corollary}

\makeatletter
\NewDocumentEnvironment{cor*}{o}
 {%
  \IfValueTF{#1}
    {\namedcor[#1]\refstepcounter{cor}\def\@currentlabel{(#1)}}%
    {\namedcor}%
 }
 {%
  \endnamedcor
 }
\makeatother

\theoremstyle{definition}
\newtheorem*{annotation}{Annotation}
\newtheorem*{rubric}{Rubric}

\newtheorem*{innerrem}{Remark}
\newcounter{rem}%makes pointer correct
\providecommand{\remname}{Remark}

\makeatletter
\NewDocumentEnvironment{rem}{o}
 {%
  \IfValueTF{#1}
    {\innerrem[#1]\refstepcounter{rem}\def\@currentlabel{(#1)}}%
    {\innerrem}%
 }
 {%
  \endinnerrem
 }
\makeatother

\newtheorem*{innerdefn}{Definition}%%placeholder
\newcounter{defn}%makes pointer correct
\providecommand{\defnname}{Definition}

\makeatletter
\NewDocumentEnvironment{defn}{o}
 {%
  \IfValueTF{#1}
    {\innerdefn[#1]\refstepcounter{defn}\def\@currentlabel{(#1)}}%
    {\innerdefn}%
 }
 {%
  \endinnerdefn
 }
\makeatother

\newtheorem*{scratch}{Scratch Work}


\newtheorem*{namedconj}{Conjecture}
\newcounter{conj}%makes pointer correct
\providecommand{\conjname}{Conjecture}
\makeatletter
\NewDocumentEnvironment{conj}{o}
 {%
  \IfValueTF{#1}
    {\innerconj[#1]\refstepcounter{conj}\def\@currentlabel{(#1)}}%
    {\innerconj}%
 }
 {%
  \endinnerconj
 }
\makeatother

\newtheorem*{poll}{Poll question}
\newtheorem{tps}{Think-Pair-Share}[section]


\newenvironment{obj}{
	\textbf{Learning Objectives.} By the end of class, students will be able to:
		\begin{itemize}}
		{\!.\end{itemize}
		}

\newenvironment{pre}{
	\begin{description}
	}{
	\end{description}
}


\newcounter{ex}%makes pointer correct
\providecommand{\exname}{Homework Problem}
\newenvironment{ex}[1][2in]%
{%Env start code
\problemEnvironmentStart{#1}{Homework Problem}
\refstepcounter{ex}
}
{%Env end code
\problemEnvironmentEnd
}

\newcommand{\inlineAnswer}[2][2 cm]{
    \ifhandout{\pdfOnly{\rule{#1}{0.4pt}}}
    \else{\answer{#2}}
    \fi
}


\ifhandout
\newenvironment{shortAnswer}[1][
    \vfill]
        {% Begin then result
        #1
            \begin{freeResponse}
            }
    {% Environment Ending Code
    \end{freeResponse}
    }
\else
\newenvironment{shortAnswer}[1][]
        {\begin{freeResponse}
            }
    {% Environment Ending Code
    \end{freeResponse}
    }
\fi

\let\question\relax
\let\endquestion\relax

\newtheoremstyle{ExerciseStyle}{\topsep}{\topsep}%%% space between body and thm
		{}                      %%% Thm body font
		{}                              %%% Indent amount (empty = no indent)
		{\bfseries}            %%% Thm head font
		{}                              %%% Punctuation after thm head
		{3em}                           %%% Space after thm head
		{{#1}~\thmnumber{#2}\thmnote{ \bfseries(#3)}}%%% Thm head spec
\theoremstyle{ExerciseStyle}
\newtheorem{br}{In-class Problem}

\newenvironment{sketch}
 {\begin{proof}[Sketch of Proof]}
 {\end{proof}}


\newcommand{\gt}{>}
\newcommand{\lt}{<}
\newcommand{\N}{\mathbb N}
\newcommand{\Q}{\mathbb Q}
\newcommand{\Z}{\mathbb Z}
\newcommand{\C}{\mathbb C}
\newcommand{\R}{\mathbb R}
\renewcommand{\H}{\mathbb{H}}
\newcommand{\lcm}{\operatorname{lcm}}
\newcommand{\nequiv}{\not\equiv}
\newcommand{\ord}{\operatorname{ord}}
\newcommand{\ds}{\displaystyle}
\newcommand{\floor}[1]{\left\lfloor #1\right\rfloor}
\newcommand{\legendre}[2]{\left(\frac{#1}{#2}\right)}



%%%%%%%%%%%%



<<<<<<< Updated upstream
\title{More modular arithmetics}
=======
\title{Modular arithmetic facts}
>>>>>>> Stashed changes
\begin{document}
\begin{abstract}
\end{abstract}
\maketitle

%%%%%%%%%%%%%%%%%%%%%%%%%%

\begin{obj}
\item Prove basic facts about modular arithmetic.
%  \item Understand gaps in argument and writing of proof of Exercise 83. Give classmates useful feedback on their proofs.
\end{obj}

\begin{defn}[$a\equiv b\pmod{m}$]\label{defn:mod-equiv-all}
Let $a,b,m\in\Z$ with $m>0.$ From Friday, we have the following equivalent definitions of congruence modulo $m:$
\begin{enumerate}
 \item $a\equiv b \pmod m$ if and only if\footnote{all definitions are if and only if} $m\mid b-a$ (standard definition, generalizing even/odd based on divisibility)
 \item $a\equiv b \pmod m$ if and only if $a$ and $b$ have the same remainder with divided by $m.$ That is, That is, there exists unique $q_1,q_2,r\in\Z$ such that  $a=mq_1+r,\   b=mq_2+r,\  0\leq r<m.$ (definition generalizing even/odd based on remainder)
\item $a\equiv b\pmod m$ if and only if $a$ and $b$ differ by a multiple of $m.$ That is, $b=a+mk$ for some $k\in\Z.$ (arithmetic progression definition)
\end{enumerate}
\end{defn}

Different statements of the definition will be useful in different situations

\begin{proposition}\label{prop:equiv-arith}%[Restatement of Propositions 2.1, 2.4, and 2.5]
 Let $a,b,c,d,m\in\Z$ with $m>0,$ then:
\begin{enumerate}
\item $a\equiv b \pmod{m}$ and $b\equiv c \pmod{m}$ implies $a\equiv c \pmod{m}$
\item\label{equiv-add} $a\equiv b \pmod{m}$ and $c\equiv d \pmod{m}$ implies $a+c \equiv b+d \pmod{m}$ 
\item\label{equiv-multiply} $a\equiv b\pmod{m}$ and $c\equiv d \pmod{m}$ implies $ac\equiv bd \pmod{m}$.
\item $a\equiv b \pmod{m}$ and $d\mid m$, $d>0$ implies $a\equiv b \pmod{d}$
\item\label{equiv-upmod} $a\equiv b \pmod{m}$ implies $ac\equiv bc \pmod{mc}$ for $c>0$.
\end{enumerate}
\end{proposition}
\begin{proof}
  Let $a,b,c,d,m\in\Z$ with $m>0.$
  
\begin{enumerate}
  \item Assume $a\equiv b \pmod{m}$ and $b\equiv c \pmod{m}.$ Then using the second definition of equivalence, there exists $q_1,q_2,q_3,r\in\Z$ such that 
  \begin{align*}
    a&=mq_1+r, \qquad 0\leq r<m,\\
    b&=mq_2+r, \qquad 0\leq r<m,\\
    c&=mq_3+r, \qquad 0\leq r<m. 
  \end{align*}
  Thus, $a$ and $c$ have the same remainder when divided by $m,$ so $a\equiv c\pmod m.$

  \item[\ref{equiv-add}/\ref{equiv-multiply}] \addtocounter{enumi}{2}
  Assume $a\equiv b \pmod{m}$ and $c\equiv d \pmod{m}.$ Then by the third definition of equivalence, there exists $j,k\in\Z$ such that $b=a+mj$ and $d=c+mk.$ Thus, 
  \begin{align*}
    b+d&=a+c+m(j+k), \qquad\qquad\textnormal{and}\\
    bd &= ac+m(ak+cj+mjk).
  \end{align*}
  Thus, $a+c\equiv b+d\pmod m$ and $ac=bd\pmod m.$

  \item Assume $a\equiv b \pmod{m},$ and $d>0$ with $d\mid m.$ From the first definition of equivalence modulo $m,$ $m\mid b-a$. Since division is transitive, $d\mid b-a,$ so $a\equiv b\pmod d.$

  \item Assume $a\equiv b \pmod{m},$ and $c>0.$ From the third definition of equivalence modulo $m,$ there exists $k\in\Z$ such that $b=a+mk.$ Thus, $bc=ac+mck,$ so $ac\equiv bc \pmod{mc}.$ \qedhere
\end{enumerate}
\end{proof}

\begin{example}
Note that $2\equiv 5 \pmod 3$. Then $4\equiv 10 \pmod 3$ by \hyperref[equiv-multiply]{Proposition \ref{prop:equiv-arith}\ref{equiv-multiply}}, since $2\equiv 2\pmod 3.$ From part \ref{equiv-upmod}, $4\equiv 10 \pmod 6,$ but $2\not\equiv 5\pmod 6.$ 
\end{example}

\end{document}
