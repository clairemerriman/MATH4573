\documentclass{ximera}
<<<<<<< Updated upstream
\usepackage{amssymb, latexsym, amsmath, amsthm, graphicx, amsthm,alltt,color, listings,multicol,xr-hyper,hyperref,aliascnt,enumitem}
=======
\usepackage{amssymb, latexsym, amsmath, amsthm, graphicx, amsthm,alltt,color, listings,multicol,hyperref}
\usepackage[capitalise,nameinlink]{cleveref}
>>>>>>> Stashed changes
\usepackage{xfrac}

\usepackage{parskip}
\usepackage[,margin=0.7in]{geometry}
\setlength{\textheight}{8.5in}

\usepackage{epstopdf}

\DeclareGraphicsExtensions{.eps}
\usepackage{tikz}


\usepackage{tkz-euclide}
%\usetkzobj{all}
\tikzstyle geometryDiagrams=[rounded corners=.5pt,ultra thick,color=black]
\colorlet{penColor}{black} % Color of a curve in a plot


\usepackage{subcaption}
\usepackage{float}
\usepackage{fancyhdr}
\usepackage{pdfpages}
\newcounter{includepdfpage}
\usepackage{makecell}


\usepackage{currfile}
\usepackage{xstring}




\graphicspath{  
{./otherDocuments/}
}

\author{Claire Merriman}
\newcommand{\classday}[1]{\def\classday{#1}}

%%%%%%%%%%%%%%%%%%%%%
% Counters and autoref for unnumbered environments
% Not needed??
%%%%%%%%%%%%%%%%%%%%%
<<<<<<< Updated upstream
\theoremstyle{plain}


\newtheorem*{namedthm}{Theorem}
\newcounter{thm}%makes pointer correct
\providecommand{\thmname}{Theorem}
=======

\crefname{problem}{problem}{problems}


% \theoremstyle{plain}


% \newtheorem*{namedthm}{Theorem}
% \newcounter{thm}%makes pointer correct
% \providecommand{\thmname}{Theorem}
>>>>>>> Stashed changes

\makeatletter
\NewDocumentEnvironment{thm*}{o}
 {%
  \IfValueTF{#1}
    {\namedthm[#1]\refstepcounter{thm}\def\@currentlabel{(#1)}}%
    {\namedthm}%
 }
 {%
  \endnamedthm
 }
\makeatother


\newtheorem*{namedprop}{Proposition}
\newcounter{prop}%makes pointer correct
\providecommand{\propname}{Proposition}

\makeatletter
\NewDocumentEnvironment{prop*}{o}
 {%
  \IfValueTF{#1}
    {\namedprop[#1]\refstepcounter{prop}\def\@currentlabel{(#1)}}%
    {\namedprop}%
 }
 {%
  \endnamedprop
 }
\makeatother

\newtheorem*{namedlem}{Lemma}
\newcounter{lem}%makes pointer correct
\providecommand{\lemname}{Lemma}

\makeatletter
\NewDocumentEnvironment{lem*}{o}
 {%
  \IfValueTF{#1}
    {\namedlem[#1]\refstepcounter{lem}\def\@currentlabel{(#1)}}%
    {\namedlem}%
 }
 {%
  \endnamedlem
 }
\makeatother

\newtheorem*{namedcor}{Corollary}
\newcounter{cor}%makes pointer correct
\providecommand{\corname}{Corollary}

\makeatletter
\NewDocumentEnvironment{cor*}{o}
 {%
  \IfValueTF{#1}
    {\namedcor[#1]\refstepcounter{cor}\def\@currentlabel{(#1)}}%
    {\namedcor}%
 }
 {%
  \endnamedcor
 }
\makeatother

\theoremstyle{definition}
\newtheorem*{annotation}{Annotation}
\newtheorem*{rubric}{Rubric}

\newtheorem*{innerrem}{Remark}
\newcounter{rem}%makes pointer correct
\providecommand{\remname}{Remark}

\makeatletter
\NewDocumentEnvironment{rem}{o}
 {%
  \IfValueTF{#1}
    {\innerrem[#1]\refstepcounter{rem}\def\@currentlabel{(#1)}}%
    {\innerrem}%
 }
 {%
  \endinnerrem
 }
\makeatother

\newtheorem*{innerdefn}{Definition}%%placeholder
\newcounter{defn}%makes pointer correct
\providecommand{\defnname}{Definition}

\makeatletter
\NewDocumentEnvironment{defn}{o}
 {%
  \IfValueTF{#1}
    {\innerdefn[#1]\refstepcounter{defn}\def\@currentlabel{(#1)}}%
    {\innerdefn}%
 }
 {%
  \endinnerdefn
 }
\makeatother

\newtheorem*{scratch}{Scratch Work}


\newtheorem*{namedconj}{Conjecture}
\newcounter{conj}%makes pointer correct
\providecommand{\conjname}{Conjecture}
\makeatletter
\NewDocumentEnvironment{conj}{o}
 {%
  \IfValueTF{#1}
    {\innerconj[#1]\refstepcounter{conj}\def\@currentlabel{(#1)}}%
    {\innerconj}%
 }
 {%
  \endinnerconj
 }
\makeatother

\newtheorem*{poll}{Poll question}
\newtheorem{tps}{Think-Pair-Share}[section]


\newenvironment{obj}{
	\textbf{Learning Objectives.} By the end of class, students will be able to:
		\begin{itemize}}
		{\!.\end{itemize}
		}

<<<<<<< Updated upstream
\newenvironment{pre}{
	\begin{description}
	}{
	\end{description}
}
=======

\ifinstructornotes
\newenvironment{pre}
  {{\textbf Reading assignment:}
  \begin{description}
    }{
	\end{description}
  }
\else
\newenvironment{pre}{ 
  \begin{trivlist}
  \item[]}
  {\end{trivlist}}
\fi
>>>>>>> Stashed changes


\newcounter{ex}%makes pointer correct
\providecommand{\exname}{Homework Problem}
\newenvironment{ex}[1][2in]%
{%Env start code
\problemEnvironmentStart{#1}{Homework Problem}
\refstepcounter{ex}
}
{%Env end code
\problemEnvironmentEnd
}

\newcommand{\inlineAnswer}[2][2 cm]{
    \ifhandout{\pdfOnly{\rule{#1}{0.4pt}}}
    \else{\answer{#2}}
    \fi
}


\ifhandout
\newenvironment{shortAnswer}[1][
    \vfill]
        {% Begin then result
        #1
            \begin{freeResponse}
            }
    {% Environment Ending Code
    \end{freeResponse}
    }
\else
\newenvironment{shortAnswer}[1][]
        {\begin{freeResponse}
            }
    {% Environment Ending Code
    \end{freeResponse}
    }
\fi

\let\question\relax
\let\endquestion\relax

\newtheoremstyle{ExerciseStyle}{\topsep}{\topsep}%%% space between body and thm
		{}                      %%% Thm body font
		{}                              %%% Indent amount (empty = no indent)
		{\bfseries}            %%% Thm head font
		{}                              %%% Punctuation after thm head
		{3em}                           %%% Space after thm head
		{{#1}~\thmnumber{#2}\thmnote{ \bfseries(#3)}}%%% Thm head spec
\theoremstyle{ExerciseStyle}
\newtheorem{br}{In-class Problem}

\newenvironment{sketch}
 {\begin{proof}[Sketch of Proof]}
 {\end{proof}}


\newcommand{\gt}{>}
\newcommand{\lt}{<}
\newcommand{\N}{\mathbb N}
\newcommand{\Q}{\mathbb Q}
\newcommand{\Z}{\mathbb Z}
\newcommand{\C}{\mathbb C}
\newcommand{\R}{\mathbb R}
\renewcommand{\H}{\mathbb{H}}
\newcommand{\lcm}{\operatorname{lcm}}
\newcommand{\nequiv}{\not\equiv}
\newcommand{\ord}{\operatorname{ord}}
\newcommand{\ds}{\displaystyle}
\newcommand{\floor}[1]{\left\lfloor #1\right\rfloor}
\newcommand{\legendre}[2]{\left(\frac{#1}{#2}\right)}



%%%%%%%%%%%%



\title{Linear congruences in one variable}
\begin{document}
\begin{abstract}
\end{abstract}
\maketitle

%%%%%%%%%%%%%%%%%%%%%%%%%%
%%%%%%%%%%%%%%%%%%%%%%%%%%
\begin{obj}
	\item Prove when a linear congruence in one variable has a solution
	\item Find all solutions to a linear congruence given a particular solution
	\item Find the number of incongruent solutions to a linear congruence
\end{obj}

<<<<<<< Updated upstream
\begin{instructorNotes}
	\begin{pre} \item Paper 1 due
	\end{pre}
\end{instructorNotes}
=======

	\begin{pre} 
        \item Paper 1 due
	\end{pre}

>>>>>>> Stashed changes



\begin{remark}\label{rem:add-inverse}
	Let $a,b,m\in\Z$ with $m>0$. Every row/column of addition modulo $m$ contains $\{0,1,\dots,m-1\}$. 

	We can also say that $a+x\equiv b\pmod m$ always has a solution, since $x\equiv b-a\pmod m$.
\end{remark}

\begin{theorem}\label{thm:lin-cong-solutions}
	Let $a,b,m\in\Z$ with $m>0,$ and $d=(a,m)$. The linear congruence in one variable $ax\equiv b\pmod m$ has a solution if and only if $d\mid b$. When $d\mid b,$ there are exactly $d$ incongruent solutions modulo $m$ corresponding to the congruence classes \[x_0, x_0+\frac{m}{d},\dots,x_0+\frac{(d-1)m}{d} \pmod m.\]

	\begin{proof}
<<<<<<< Updated upstream
		Let $a,b,m\in\Z$ with $m>0$, and $d=(a,m)$. From the definition of congruence modulo $m$, $ax\equiv b \pmod m$ if and only if $m\mid (ax-b)$. That is, $ax\equiv b \pmod m$ if and only if $my=ax-b$ for some $y\in\Z$ from the definition of divisibility. Sine $ax-my=b$ is a linear Diophantine equation, \nameref{thm:linear-dioph} says solutions exist if and only if $(a,-m)=d\mid b.$
		
		In the case that solutions exist, let $x_0, y_0$ be a particular solution to the linear Diophantine equation. Then $x_0$ is also a solution to the linear congruence in one variable, since $a x_0-my_0=b,$ implies $ax_0\equiv b\pmod m$. From \nameref{thm:linear-dioph}, all solutions have the from $x=x_0+\frac{mn}{d}$ for all $n\in\Z.$  We need to show that these solutions are in exactly $d$ distinct congruence classes modulo $m.$
=======
		Let $a,b,m\in\Z$ with $m>0$, and $d=(a,m)$. From the definition of congruence modulo $m$, $ax\equiv b \pmod m$ if and only if $m\mid (ax-b)$. That is, $ax\equiv b \pmod m$ if and only if $my=ax-b$ for some $y\in\Z$ from the definition of divisibility. Sine $ax-my=b$ is a linear Diophantine equation, \cref{thm:linear-dioph} says solutions exist if and only if $(a,-m)=d\mid b.$
		
		In the case that solutions exist, let $x_0, y_0$ be a particular solution to the linear Diophantine equation. Then $x_0$ is also a solution to the linear congruence in one variable, since $a x_0-my_0=b,$ implies $ax_0\equiv b\pmod m$. From \cref{thm:linear-dioph}, all solutions have the from $x=x_0+\frac{mn}{d}$ for all $n\in\Z.$  We need to show that these solutions are in exactly $d$ distinct congruence classes modulo $m.$
>>>>>>> Stashed changes
		
		Consider the solutions $x_0+\frac{mi}{d}$ and $x_0+\frac{mj}{d}$ for some integers $i$ and $j.$ Then $x_0+\frac{mj}{d}\equiv x_0+\frac{mk}{d} \pmod m$ if and only if $m\mid \left(\frac{mi}{d}-\frac{mj}{d}\right).$ That is, if and only if there exists $k\in\Z$ such that $mk=\frac{mi}{d}-\frac{mj}{d}$. Rearranging this equation, we get that $x_0+\frac{mj}{d}\equiv x_0+\frac{mk}{d} \pmod m$ if and only if $dk=i-j.$ Thus, $i\equiv j\pmod$ by definition of equivalence modulo $d.$ Thus, the incongruent solutions to $ax\equiv b\pmod m$ are the congruence classes \[x_0, x_0+\frac{m}{d},\dots,x_0+\frac{(d-1)m}{d} \pmod m.\qedhere\]
	\end{proof}
\end{theorem}

\begin{example} 
	Let's consider several linear congruences modulo $12.$
	\begin{itemize}
		\item The linear congruence $2x\equiv 1\pmod{12}$ has no solutions, since $2\nmid1.$
		\item The linear congruence $8x\equiv b \pmod{12}$ has a solution if and only if $4\mid b$. Considering the least nonnegative residues, the options for $b$ are:
			\begin{itemize}
			\item $8x\equiv 0\pmod{12}.$ The incongruent solutions are $0,3,6,9\pmod{12}.$
			\item $8x\equiv 4\pmod{12}.$ The incongruent solutions are $2,5,8,11\pmod{12}.$ Notice we cannot divide across the equivalence, since $2x\equiv 1\pmod{12}$ has no solutions. 
			\item $8x\equiv 8\pmod{12}.$ The incongruent solutions are $1,4,7,10\pmod{12}.$ 
			\end{itemize}
		\item The linear congruence $5x\equiv 1 \pmod{12}$ has solution $x\equiv 5\pmod{12}.$ Since $(5,12)=1,$ the solution is unique.
		\item The linear congruence $5x\equiv 7 \pmod{12}$ has solution $x\equiv -1\equiv 11\pmod{12}.$ Since $(5,12)=1,$ the solution is unique. Note that instead of $12+5(-1)=7,$ we could have done 
		\[5(5x)\equiv 5(7)\equiv 11\pmod{12}.\]
	\end{itemize}
\end{example}




<<<<<<< Updated upstream
\begin{corollary}\label{cor:condition-invertible}[Corollary of \autoref{thm:lin-cong-solutions}]
=======
\begin{corollary}\label{cor:condition-invertible}[Corollary of \cref{thm:lin-cong-solutions}]
>>>>>>> Stashed changes
    Let $a,m\in\Z$ with $m>0.$ The linear congruence in one variable $ax\equiv 1\pmod{m}$ has a solution if and only if $(a,m)=1$. If $(a,m)=1,$ then the solution is unique modulo $m$. 
\end{corollary}

\begin{defn}[multiplicative inverse of $a$ modulo $m$]\label{defn:mult-inv} Let $a,m\in\Z$ with $m>0$ and $(a,m)=1.$
    We call the unique incongruent solution to $ax\equiv 1\pmod m$ the \emph{multiplicative inverse of $a$ modulo $m$}.
\end{defn}

\begin{example}\label{example:mult-inv}
    Examples of multiplicative inverses:
    \begin{itemize}
        \item $5(3)\equiv 1\pmod{7}$ so $3$ is the multiplicative inverse of $5$ modulo $7$ and $5$ is the multiplicative inverse of $3$ modulo $7$.
        \item $9(5)\equiv 1\pmod{11}$ so $5$ is the multiplicative inverse of $9$ modulo $11$ and $9$ is the multiplicative inverse of $5$ modulo $11$.
        \item $8(-4)\equiv 8(7)\equiv 1\pmod{11}$ so $7\equiv -4\pmod{11}$ is the multiplicative inverse of $8$ modulo $11$ and $8$ is the multiplicative inverse of $7\equiv -4\pmod{11}$ modulo $11$.
        \item $8(5)\equiv 1\pmod{13}$ so $5$ is the multiplicative inverse of $8$ modulo $13$ and $8$ is the multiplicative inverse of $5$ modulo $13$.
    \end{itemize}

    Example using multiplicative inverses:
    \begin{align*}
        6!& \equiv 6\ast 5\ast 4\ast 3\ast 2\ast 1\pmod{7}\\
          & \equiv 6\ast 5(3)\ast 4(2)\ast 1\pmod{7}\\
          & \equiv 6 \pmod{7}
    \end{align*}
\end{example}

\begin{tps}
    Find $10!\pmod{11}$ and $12!\pmod{13}.$ Is there a pattern?

    
    \begin{solution}
        \begin{align*}
            10! & \equiv 10\ast 9\ast 8\ast 7\ast 6\ast 5\ast 4\ast 3\ast 2\ast 1\pmod{11}\\
                & \equiv 10\ast 9(5)\ast 8(7)\ast 6(2)\ast 4(3)\ast 1\pmod{11}\\
                & \equiv 1\pmod{11}
        \end{align*}

        \begin{align*}
            12! & \equiv 12\ast 11\ast 10\ast 9\ast 8\ast 7\ast 6\ast 5\ast 4\ast 3\ast 2\ast 1\pmod{13}\\
                & \equiv 12\ast 11(6) \ast 10(4)\ast 9(3)\ast 8(5)\ast 7(2) \ast 1\pmod{13}\\
                & \equiv 1\pmod{13}
        \end{align*}

        For a prime $p,$ $(p-1)!\equiv 1\pmod{p}$.
    \end{solution}
\end{tps}
\begin{remark}
    We do need the condition that $p$ is prime. For example, $3!\equiv 2\pmod{4},$ and $8!\equiv 0\pmod{9}.$
\end{remark}
%%%%%%%%%%%%%%%%%%%%%%%%%%


\end{document}
