\documentclass{../ximera}
<<<<<<< Updated upstream
\usepackage{amssymb, latexsym, amsmath, amsthm, graphicx, amsthm,alltt,color, listings,multicol,xr-hyper,hyperref,aliascnt,enumitem}
=======
\usepackage{amssymb, latexsym, amsmath, amsthm, graphicx, amsthm,alltt,color, listings,multicol,hyperref}
\usepackage[capitalise,nameinlink]{cleveref}
>>>>>>> Stashed changes
\usepackage{xfrac}

\usepackage{parskip}
\usepackage[,margin=0.7in]{geometry}
\setlength{\textheight}{8.5in}

\usepackage{epstopdf}

\DeclareGraphicsExtensions{.eps}
\usepackage{tikz}


\usepackage{tkz-euclide}
%\usetkzobj{all}
\tikzstyle geometryDiagrams=[rounded corners=.5pt,ultra thick,color=black]
\colorlet{penColor}{black} % Color of a curve in a plot


\usepackage{subcaption}
\usepackage{float}
\usepackage{fancyhdr}
\usepackage{pdfpages}
\newcounter{includepdfpage}
\usepackage{makecell}


\usepackage{currfile}
\usepackage{xstring}




\graphicspath{  
{./otherDocuments/}
}

\author{Claire Merriman}
\newcommand{\classday}[1]{\def\classday{#1}}

%%%%%%%%%%%%%%%%%%%%%
% Counters and autoref for unnumbered environments
% Not needed??
%%%%%%%%%%%%%%%%%%%%%
<<<<<<< Updated upstream
\theoremstyle{plain}


\newtheorem*{namedthm}{Theorem}
\newcounter{thm}%makes pointer correct
\providecommand{\thmname}{Theorem}
=======

\crefname{problem}{problem}{problems}


% \theoremstyle{plain}


% \newtheorem*{namedthm}{Theorem}
% \newcounter{thm}%makes pointer correct
% \providecommand{\thmname}{Theorem}
>>>>>>> Stashed changes

\makeatletter
\NewDocumentEnvironment{thm*}{o}
 {%
  \IfValueTF{#1}
    {\namedthm[#1]\refstepcounter{thm}\def\@currentlabel{(#1)}}%
    {\namedthm}%
 }
 {%
  \endnamedthm
 }
\makeatother


\newtheorem*{namedprop}{Proposition}
\newcounter{prop}%makes pointer correct
\providecommand{\propname}{Proposition}

\makeatletter
\NewDocumentEnvironment{prop*}{o}
 {%
  \IfValueTF{#1}
    {\namedprop[#1]\refstepcounter{prop}\def\@currentlabel{(#1)}}%
    {\namedprop}%
 }
 {%
  \endnamedprop
 }
\makeatother

\newtheorem*{namedlem}{Lemma}
\newcounter{lem}%makes pointer correct
\providecommand{\lemname}{Lemma}

\makeatletter
\NewDocumentEnvironment{lem*}{o}
 {%
  \IfValueTF{#1}
    {\namedlem[#1]\refstepcounter{lem}\def\@currentlabel{(#1)}}%
    {\namedlem}%
 }
 {%
  \endnamedlem
 }
\makeatother

\newtheorem*{namedcor}{Corollary}
\newcounter{cor}%makes pointer correct
\providecommand{\corname}{Corollary}

\makeatletter
\NewDocumentEnvironment{cor*}{o}
 {%
  \IfValueTF{#1}
    {\namedcor[#1]\refstepcounter{cor}\def\@currentlabel{(#1)}}%
    {\namedcor}%
 }
 {%
  \endnamedcor
 }
\makeatother

\theoremstyle{definition}
\newtheorem*{annotation}{Annotation}
\newtheorem*{rubric}{Rubric}

\newtheorem*{innerrem}{Remark}
\newcounter{rem}%makes pointer correct
\providecommand{\remname}{Remark}

\makeatletter
\NewDocumentEnvironment{rem}{o}
 {%
  \IfValueTF{#1}
    {\innerrem[#1]\refstepcounter{rem}\def\@currentlabel{(#1)}}%
    {\innerrem}%
 }
 {%
  \endinnerrem
 }
\makeatother

\newtheorem*{innerdefn}{Definition}%%placeholder
\newcounter{defn}%makes pointer correct
\providecommand{\defnname}{Definition}

\makeatletter
\NewDocumentEnvironment{defn}{o}
 {%
  \IfValueTF{#1}
    {\innerdefn[#1]\refstepcounter{defn}\def\@currentlabel{(#1)}}%
    {\innerdefn}%
 }
 {%
  \endinnerdefn
 }
\makeatother

\newtheorem*{scratch}{Scratch Work}


\newtheorem*{namedconj}{Conjecture}
\newcounter{conj}%makes pointer correct
\providecommand{\conjname}{Conjecture}
\makeatletter
\NewDocumentEnvironment{conj}{o}
 {%
  \IfValueTF{#1}
    {\innerconj[#1]\refstepcounter{conj}\def\@currentlabel{(#1)}}%
    {\innerconj}%
 }
 {%
  \endinnerconj
 }
\makeatother

\newtheorem*{poll}{Poll question}
\newtheorem{tps}{Think-Pair-Share}[section]


\newenvironment{obj}{
	\textbf{Learning Objectives.} By the end of class, students will be able to:
		\begin{itemize}}
		{\!.\end{itemize}
		}

<<<<<<< Updated upstream
\newenvironment{pre}{
	\begin{description}
	}{
	\end{description}
}
=======

\ifinstructornotes
\newenvironment{pre}
  {{\textbf Reading assignment:}
  \begin{description}
    }{
	\end{description}
  }
\else
\newenvironment{pre}{ 
  \begin{trivlist}
  \item[]}
  {\end{trivlist}}
\fi
>>>>>>> Stashed changes


\newcounter{ex}%makes pointer correct
\providecommand{\exname}{Homework Problem}
\newenvironment{ex}[1][2in]%
{%Env start code
\problemEnvironmentStart{#1}{Homework Problem}
\refstepcounter{ex}
}
{%Env end code
\problemEnvironmentEnd
}

\newcommand{\inlineAnswer}[2][2 cm]{
    \ifhandout{\pdfOnly{\rule{#1}{0.4pt}}}
    \else{\answer{#2}}
    \fi
}


\ifhandout
\newenvironment{shortAnswer}[1][
    \vfill]
        {% Begin then result
        #1
            \begin{freeResponse}
            }
    {% Environment Ending Code
    \end{freeResponse}
    }
\else
\newenvironment{shortAnswer}[1][]
        {\begin{freeResponse}
            }
    {% Environment Ending Code
    \end{freeResponse}
    }
\fi

\let\question\relax
\let\endquestion\relax

\newtheoremstyle{ExerciseStyle}{\topsep}{\topsep}%%% space between body and thm
		{}                      %%% Thm body font
		{}                              %%% Indent amount (empty = no indent)
		{\bfseries}            %%% Thm head font
		{}                              %%% Punctuation after thm head
		{3em}                           %%% Space after thm head
		{{#1}~\thmnumber{#2}\thmnote{ \bfseries(#3)}}%%% Thm head spec
\theoremstyle{ExerciseStyle}
\newtheorem{br}{In-class Problem}

\newenvironment{sketch}
 {\begin{proof}[Sketch of Proof]}
 {\end{proof}}


\newcommand{\gt}{>}
\newcommand{\lt}{<}
\newcommand{\N}{\mathbb N}
\newcommand{\Q}{\mathbb Q}
\newcommand{\Z}{\mathbb Z}
\newcommand{\C}{\mathbb C}
\newcommand{\R}{\mathbb R}
\renewcommand{\H}{\mathbb{H}}
\newcommand{\lcm}{\operatorname{lcm}}
\newcommand{\nequiv}{\not\equiv}
\newcommand{\ord}{\operatorname{ord}}
\newcommand{\ds}{\displaystyle}
\newcommand{\floor}[1]{\left\lfloor #1\right\rfloor}
\newcommand{\legendre}[2]{\left(\frac{#1}{#2}\right)}



%%%%%%%%%%%%



\title{ Introduction and Divisibility}
\begin{document}
\begin{abstract}
\end{abstract}
\maketitle

%%%%%%%%%%%%%%%%%%%%%%%%%%
%%%%%%%%%%%%%%%%%%%%%%%%%%

\begin{obj}
  \item Understand the course structure
  \item Formally define even and odd
  \item Formally define ``divides"
  \item Complete basic algebraic proofs
\end{obj}

%%%%%%%%%%%%%%%%%%%%%%%%%%
\subsection{Introduction }% \instructorNotes{(15 minutes)}
%%%%%%%%%%%%%%%%%%%%%%%%%%

What is number theory?

Elementary number theory is the study of integers, especially the positive integers. A lot of the course focuses on prime numbers, which are the multiplicative building blocks of the integers. Another big topic in number theory is integer solutions to equations such as the Pythagorean triples $x^2+y^2=z^2$ or the generalization $x^n+y^n=z^n$. Proving that there are no integer solutions when $n>2$ was an open problem for close to 400 years.

The first part of this course reproves facts about divisibility and prime numbers that you are probably familiar with. There are two purposes to this: 1) formalizing definitions and 2) starting with the situation you understand before moving to the new material.

Go over syllabus highlights: Deadlines, make-up policy, in-class work, reading assignments.


%%%%%%%%%%%%%%%%%%%%%%%%%%
\subsection{Mathematical definitions, mathematical notation}% % \instructorNotes{(30 minutes)}
%%%%%%%%%%%%%%%%%%%%%%%%%%
\begin{defn}\label{defn:number-systems} 
  We will use the following number systems and abbreviations:
  \begin{itemize}
    \item The \emph{integers,} written $\Z$, is the set $\{\dots,-3,-2,-1,0,1,2,3,\dots\}$. 
    \item The \emph{natural numbers,} written $\N$. Most elementary number theory texts either define $\N$ to be the positive integers or avoid using $\N$. Some mathematicians include $0$ in $\N$.
    \item The \emph{real numbers,} written $\R$.
    \item The \emph{integers modulo $n$,} written $\Z_n$. We will define this set in Strayer Chapter 2, although Strayer does not use this notation.
  \end{itemize}
  We will also use the following notation:
  \begin{itemize}
    \item The symbol $\in$ means ``element of" or ``in." For example, $x\in\Z$ means ``$x$ is an element of the integers" or ``$x$ in the integers."
  \end{itemize}
\end{defn}

This first section will cover results in both Strayer and Ernst.

\begin{defn}[Ernst, Definition 2.1]\label{defn:even-odd-form}
  An integer $n$ is \emph{even} if $n=2k$ for some $k\in\Z$. 
  An integer $n$ is odd if $n=2k+1$ for some $k\in\Z$.
\end{defn}

Now, the preceding definition is standard in an introduction to proofs course, but it is not the only definition of even/odd. We also have the following definition that is closer to the definition you are probably used to:

\begin{defn}[Strayer, Definition 4]\label{defn:even-odd-divides}
  Let $n\in\Z$. Then $n$ is said to be \emph{even} if $2$ divides $n$ and $n$ is said to be \emph{odd} if $2$ does not divide $n$.
\end{defn}
Note that we need to define \emph{divides} in order to use Strayer's definition. We will formally prove that these definitions are \emph{equivalent,} but for now, let's use Ernst definition.
 
 
\begin{thm*}[Ernst, Theorem 2.2]\label{thm:even-sq}
  If $n$ is an even integer, then $n^2$ is even.
\end{thm*}

\begin{br}\label{br:sqr-evens}
  Prove this theorem.

  \begin{proof}
    If $n$ is an even integer, then by \nameref{defn:even-odd-form}, there is some $k\in\Z$ such that $n=2k$. Then \[n^2=(2k)^2=2(2k^2).\] Since $2(k^2)$ is an integer, we have written $n^2$ in the desired form. Thus, $n^2$ is even.
  \end{proof}
\end{br}
 
\begin{thm*}[Ernst, Theorem 2.3]\label{thm:sum-odd}
  The sum of two consecutive integers in odd.
\end{thm*}
For this problem, we need to figure out how to write two consecutive integers. 
\begin{proof}
  Let $n,n+1$ be two consecutive integers. Then their sum is $n+n+1=2n+1,$ which is odd by \nameref{defn:even-odd-form}.
\end{proof}

%%%%%%%%%%%%%%%%%%%%%%%%%%
\subsection{Divisibility}% % \instructorNotes{(5 minutes)}
%%%%%%%%%%%%%%%%%%%%%%%%%%

The goal of this chapter is to review basic facts about divisibility, get comfortable with the new notation, and solve some basic linear equations.

We will also use this material as an opportunity to get used to the course. 

\begin{defn}[$a$ divides $b$]\label{defn:divides}
 Let $a,b\in \Z$. The \emph{$a$ divides $b$}, denoted $a\mid b$,  if there exists an integer $c$ such that $b=ac$. If $a\mid b$, then $a$ is said to be a \emph{divisor} or \emph{factor of $b$}. The notation $a\nmid b$ means $a$ does not divide $b$.
\end{defn}

Note that 0 is not a divisor of any integer other than itself, since $b=0c$ implies $a=0$. Also all integers are divisors of 0, as odd as that sounds at first. This is because for any $a\in\Z$, $0=a0$. 


Reminding students about the reading for Friday.

%%%%%%%%%%%%%%%%%%%%%%%%%%
%%%%%%%%%%%%%%%%%%%%%%%%%%
%%%%%%%%%%%%%%%%%%%%%%%%%%


\end{document}
