\documentclass{../ximera}
\usepackage{amssymb, latexsym, amsmath, amsthm, graphicx, amsthm,alltt,color, listings,multicol,xr-hyper,hyperref,aliascnt,enumitem}
\usepackage{xfrac}

\usepackage{parskip}
\usepackage[,margin=0.7in]{geometry}
\setlength{\textheight}{8.5in}

\usepackage{epstopdf}

\DeclareGraphicsExtensions{.eps}
\usepackage{tikz}


\usepackage{tkz-euclide}
%\usetkzobj{all}
\tikzstyle geometryDiagrams=[rounded corners=.5pt,ultra thick,color=black]
\colorlet{penColor}{black} % Color of a curve in a plot


\usepackage{subcaption}
\usepackage{float}
\usepackage{fancyhdr}
\usepackage{pdfpages}
\newcounter{includepdfpage}
\usepackage{makecell}


\usepackage{currfile}
\usepackage{xstring}




\graphicspath{  
{./otherDocuments/}
}

\author{Claire Merriman}
\newcommand{\classday}[1]{\def\classday{#1}}

%%%%%%%%%%%%%%%%%%%%%
% Counters and autoref for unnumbered environments
% Not needed??
%%%%%%%%%%%%%%%%%%%%%
\theoremstyle{plain}


\newtheorem*{namedthm}{Theorem}
\newcounter{thm}%makes pointer correct
\providecommand{\thmname}{Theorem}

\makeatletter
\NewDocumentEnvironment{thm*}{o}
 {%
  \IfValueTF{#1}
    {\namedthm[#1]\refstepcounter{thm}\def\@currentlabel{(#1)}}%
    {\namedthm}%
 }
 {%
  \endnamedthm
 }
\makeatother


\newtheorem*{namedprop}{Proposition}
\newcounter{prop}%makes pointer correct
\providecommand{\propname}{Proposition}

\makeatletter
\NewDocumentEnvironment{prop*}{o}
 {%
  \IfValueTF{#1}
    {\namedprop[#1]\refstepcounter{prop}\def\@currentlabel{(#1)}}%
    {\namedprop}%
 }
 {%
  \endnamedprop
 }
\makeatother

\newtheorem*{namedlem}{Lemma}
\newcounter{lem}%makes pointer correct
\providecommand{\lemname}{Lemma}

\makeatletter
\NewDocumentEnvironment{lem*}{o}
 {%
  \IfValueTF{#1}
    {\namedlem[#1]\refstepcounter{lem}\def\@currentlabel{(#1)}}%
    {\namedlem}%
 }
 {%
  \endnamedlem
 }
\makeatother

\newtheorem*{namedcor}{Corollary}
\newcounter{cor}%makes pointer correct
\providecommand{\corname}{Corollary}

\makeatletter
\NewDocumentEnvironment{cor*}{o}
 {%
  \IfValueTF{#1}
    {\namedcor[#1]\refstepcounter{cor}\def\@currentlabel{(#1)}}%
    {\namedcor}%
 }
 {%
  \endnamedcor
 }
\makeatother

\theoremstyle{definition}
\newtheorem*{annotation}{Annotation}
\newtheorem*{rubric}{Rubric}

\newtheorem*{innerrem}{Remark}
\newcounter{rem}%makes pointer correct
\providecommand{\remname}{Remark}

\makeatletter
\NewDocumentEnvironment{rem}{o}
 {%
  \IfValueTF{#1}
    {\innerrem[#1]\refstepcounter{rem}\def\@currentlabel{(#1)}}%
    {\innerrem}%
 }
 {%
  \endinnerrem
 }
\makeatother

\newtheorem*{innerdefn}{Definition}%%placeholder
\newcounter{defn}%makes pointer correct
\providecommand{\defnname}{Definition}

\makeatletter
\NewDocumentEnvironment{defn}{o}
 {%
  \IfValueTF{#1}
    {\innerdefn[#1]\refstepcounter{defn}\def\@currentlabel{(#1)}}%
    {\innerdefn}%
 }
 {%
  \endinnerdefn
 }
\makeatother

\newtheorem*{scratch}{Scratch Work}


\newtheorem*{namedconj}{Conjecture}
\newcounter{conj}%makes pointer correct
\providecommand{\conjname}{Conjecture}
\makeatletter
\NewDocumentEnvironment{conj}{o}
 {%
  \IfValueTF{#1}
    {\innerconj[#1]\refstepcounter{conj}\def\@currentlabel{(#1)}}%
    {\innerconj}%
 }
 {%
  \endinnerconj
 }
\makeatother

\newtheorem*{poll}{Poll question}
\newtheorem{tps}{Think-Pair-Share}[section]


\newenvironment{obj}{
	\textbf{Learning Objectives.} By the end of class, students will be able to:
		\begin{itemize}}
		{\!.\end{itemize}
		}

\newenvironment{pre}{
	\begin{description}
	}{
	\end{description}
}


\newcounter{ex}%makes pointer correct
\providecommand{\exname}{Homework Problem}
\newenvironment{ex}[1][2in]%
{%Env start code
\problemEnvironmentStart{#1}{Homework Problem}
\refstepcounter{ex}
}
{%Env end code
\problemEnvironmentEnd
}

\newcommand{\inlineAnswer}[2][2 cm]{
    \ifhandout{\pdfOnly{\rule{#1}{0.4pt}}}
    \else{\answer{#2}}
    \fi
}


\ifhandout
\newenvironment{shortAnswer}[1][
    \vfill]
        {% Begin then result
        #1
            \begin{freeResponse}
            }
    {% Environment Ending Code
    \end{freeResponse}
    }
\else
\newenvironment{shortAnswer}[1][]
        {\begin{freeResponse}
            }
    {% Environment Ending Code
    \end{freeResponse}
    }
\fi

\let\question\relax
\let\endquestion\relax

\newtheoremstyle{ExerciseStyle}{\topsep}{\topsep}%%% space between body and thm
		{}                      %%% Thm body font
		{}                              %%% Indent amount (empty = no indent)
		{\bfseries}            %%% Thm head font
		{}                              %%% Punctuation after thm head
		{3em}                           %%% Space after thm head
		{{#1}~\thmnumber{#2}\thmnote{ \bfseries(#3)}}%%% Thm head spec
\theoremstyle{ExerciseStyle}
\newtheorem{br}{In-class Problem}

\newenvironment{sketch}
 {\begin{proof}[Sketch of Proof]}
 {\end{proof}}


\newcommand{\gt}{>}
\newcommand{\lt}{<}
\newcommand{\N}{\mathbb N}
\newcommand{\Q}{\mathbb Q}
\newcommand{\Z}{\mathbb Z}
\newcommand{\C}{\mathbb C}
\newcommand{\R}{\mathbb R}
\renewcommand{\H}{\mathbb{H}}
\newcommand{\lcm}{\operatorname{lcm}}
\newcommand{\nequiv}{\not\equiv}
\newcommand{\ord}{\operatorname{ord}}
\newcommand{\ds}{\displaystyle}
\newcommand{\floor}[1]{\left\lfloor #1\right\rfloor}
\newcommand{\legendre}[2]{\left(\frac{#1}{#2}\right)}



%%%%%%%%%%%%



\title{Tthe Fundamental Theorem of Arithmetic}
\begin{document}
\begin{abstract}
\end{abstract}
\maketitle

%%%%%%%%%%%%%%%%%%%%%%%%%%
%%%%%%%%%%%%%%%%%%%%%%%%%%

\begin{obj}
\item Prove the Fundamental Theorem of Arithmetic
\item  Prove $\sqrt{2}$ is irrational
\end{obj}

\begin{instructorNotes}
	\begin{pre}
		\item[Read] Strayer, Section 1.5 through \nameref{prop:form-lcm-gcd}
		\item[Turn in] 
		\begin{itemize}
		   \item Answer these questions about the proof of the Fundamental Theorem of Arithmetic (taken from \href{https://maa.org/node/121566}{Helping Undergraduates Learn to Read Mathematics}):
		   
		   \begin{itemize}
			   \item Can you write a brief outline (maybe 1/10 as long as the theorem) giving the logic of the argument -- proof by contradiction, induction on n, etc.? (This is KEY.)
			   \item What mathematical raw materials are used in the proof? (Do we need a lemma? Do we need a new definition? A powerful theorem? and do you recall how to prove it? Is the full generality of that theorem needed, or just a weak version?)
			   \item What does the proof tell you about why the theorem holds?
			   \item Where is each of the hypotheses used in the proof?
			   \item Can you think of other questions to ask yourself?
		   \end{itemize}
	   
	   \item Strayer states that the proof of \nameref{prop:form-lcm-gcd} is ``obvious from the Fundamental Theorem of Arithmetic and the definitions of $(a,b)$ and $[a,b]$." Is this true? If so, why? If not, fill in the gaps.
		\end{itemize}
	   
		
	   \begin{solution}
	   Answers to both questions will vary between students.
	   \end{solution}
	   \end{pre}
\end{instructorNotes}

%%%%%%%%%%%%%%%%%%%%%%%%%


\begin{thm*}[Fundamental Theorem of Arithmetic]\label{FTA}
	Every integer greater than one can be written in the form $p_1^{a_1}p_2^{a_2}\cdots p_r^{a_r}$ where the $p_i$ are distinct prime numbers and the $a_i$ are positive integers. This factorization into primes is unique up to the ordering of the terms.

	\begin{proof}
 		We will show that every integer $n$ greater than $1$ has a prime factorization. First, note that all primes are already in the desired form. We will use induction to show that every composite integer can be factored into the product of primes. When $n=4$, we can write $n=2^2$, so $4$ has the desired form.
 
		Assume that for all integers $k$ with $1<k<n$, $k$ can be written in the form  $p_1^{a_1}p_2^{a_2}\cdots p_r^{a_r}$ where the $p_i$ are distinct prime numbers and the $a_i$ are positive integers. If $n$ is prime, we are done, otherwise there exists $a,b\in\Z$ with $1<a,b<n$ such that $n=ab$. By the induction hypothesis, there exist primes $p_1,p_2,\dots,p_r,q_1,q_2,\dots,q_s$ and positive integers $a_1,a_2,\dots,a_r,b_1,b_2,\dots b_s$ such that $a=p_1^{a_1}p_2^{a_2}\cdots p_r^{a_r}$ and $b=q_1^{b_1}q_2^{a_2}\cdots q_s^{b_a}$. Then \[n=p_1^{a_1}p_2^{a_2}\cdots p_r^{a_r}q_1^{b_1}q_2^{a_2}\cdots q_s^{b_a}.\]
	\end{proof}
\end{thm*}

We will use an idea similar to the proof of the Fundamental Theorem of Arithmetic to proof the following:

\begin{br}
	\begin{prop*}
		$\sqrt{2}$ is irrational
	\end{prop*}

	As class, put the steps of the proof in order, then fill in the missing information.
\end{br}

Finally, work two groups. Each group will be assigned one of the following question.


\begin{br}
	Let $p$ be prime.
	\begin{enumerate}
		\item If $(a,b)=p$, what are the possible values of $(a^2,b)$? Of $(a^3,b)$? Of $(a^2,b^3)$?
		
		% \begin{solution}
		% 	If $(a,b)=p$, then there exist $j,k\in\Z$ such that $a=pj, b=pk$, and $p\nmid j$ or $p\nmid k$ (otherwise $(a,b)=p^2$). 
		% 	\[a^2=p^2j^2,\quad
		% 	a^3=p^3j^3,\quad
		% 	b^3=p^3k^3\]
		% 	Then $(a^2,b)$ is $p$ if $p\nmid k$ or $p^2$ if $p\mid k$; and
		% 	$(a^3,b)$ is $p$ if $p\nmid k,$ $p^2$ if $p\mid k$ and $p^2\nmid k,$ or $p^3$ if $p^2\mid k$. 
			
		% 	If $p\mid j,$ then $p\nmid k$ and   
		% 	$(a^2,b^3)=p^3$.
		% 	If $p\nmid j,$ then   
		% 	$(a^2,b^3)=p^2.$
		% \end{solution}
		\item If $(a,b)=p$ and $(b,p^3)=p^2$, find $(ab,p^4)$ and $(a+b,p^4)$.
		
		% \begin{solution}
		% 	There exists $j,k\in\Z$ such that $a=pj, b=p^2k,$ and $p\nmid k, p\nmid k$. 
		% 	Then $ab=p^3jk$ and $a+b=pj+p^2k=p(j+pk)$. Thus, $(ab,p^4)=p^3$ and $(a+b,p^4)=p.$
		% \end{solution}
	\end{enumerate}
\end{br}

%%%%%%%%%%%%%%%%%%%%%%%%%%


\end{document}
