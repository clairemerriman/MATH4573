\documentclass{../ximera}
\usepackage{amssymb, latexsym, amsmath, amsthm, graphicx, amsthm,alltt,color, listings,multicol,xr-hyper,hyperref,aliascnt,enumitem}
\usepackage{xfrac}

\usepackage{parskip}
\usepackage[,margin=0.7in]{geometry}
\setlength{\textheight}{8.5in}

\usepackage{epstopdf}

\DeclareGraphicsExtensions{.eps}
\usepackage{tikz}


\usepackage{tkz-euclide}
%\usetkzobj{all}
\tikzstyle geometryDiagrams=[rounded corners=.5pt,ultra thick,color=black]
\colorlet{penColor}{black} % Color of a curve in a plot


\usepackage{subcaption}
\usepackage{float}
\usepackage{fancyhdr}
\usepackage{pdfpages}
\newcounter{includepdfpage}
\usepackage{makecell}


\usepackage{currfile}
\usepackage{xstring}




\graphicspath{  
{./otherDocuments/}
}

\author{Claire Merriman}
\newcommand{\classday}[1]{\def\classday{#1}}

%%%%%%%%%%%%%%%%%%%%%
% Counters and autoref for unnumbered environments
% Not needed??
%%%%%%%%%%%%%%%%%%%%%
\theoremstyle{plain}


\newtheorem*{namedthm}{Theorem}
\newcounter{thm}%makes pointer correct
\providecommand{\thmname}{Theorem}

\makeatletter
\NewDocumentEnvironment{thm*}{o}
 {%
  \IfValueTF{#1}
    {\namedthm[#1]\refstepcounter{thm}\def\@currentlabel{(#1)}}%
    {\namedthm}%
 }
 {%
  \endnamedthm
 }
\makeatother


\newtheorem*{namedprop}{Proposition}
\newcounter{prop}%makes pointer correct
\providecommand{\propname}{Proposition}

\makeatletter
\NewDocumentEnvironment{prop*}{o}
 {%
  \IfValueTF{#1}
    {\namedprop[#1]\refstepcounter{prop}\def\@currentlabel{(#1)}}%
    {\namedprop}%
 }
 {%
  \endnamedprop
 }
\makeatother

\newtheorem*{namedlem}{Lemma}
\newcounter{lem}%makes pointer correct
\providecommand{\lemname}{Lemma}

\makeatletter
\NewDocumentEnvironment{lem*}{o}
 {%
  \IfValueTF{#1}
    {\namedlem[#1]\refstepcounter{lem}\def\@currentlabel{(#1)}}%
    {\namedlem}%
 }
 {%
  \endnamedlem
 }
\makeatother

\newtheorem*{namedcor}{Corollary}
\newcounter{cor}%makes pointer correct
\providecommand{\corname}{Corollary}

\makeatletter
\NewDocumentEnvironment{cor*}{o}
 {%
  \IfValueTF{#1}
    {\namedcor[#1]\refstepcounter{cor}\def\@currentlabel{(#1)}}%
    {\namedcor}%
 }
 {%
  \endnamedcor
 }
\makeatother

\theoremstyle{definition}
\newtheorem*{annotation}{Annotation}
\newtheorem*{rubric}{Rubric}

\newtheorem*{innerrem}{Remark}
\newcounter{rem}%makes pointer correct
\providecommand{\remname}{Remark}

\makeatletter
\NewDocumentEnvironment{rem}{o}
 {%
  \IfValueTF{#1}
    {\innerrem[#1]\refstepcounter{rem}\def\@currentlabel{(#1)}}%
    {\innerrem}%
 }
 {%
  \endinnerrem
 }
\makeatother

\newtheorem*{innerdefn}{Definition}%%placeholder
\newcounter{defn}%makes pointer correct
\providecommand{\defnname}{Definition}

\makeatletter
\NewDocumentEnvironment{defn}{o}
 {%
  \IfValueTF{#1}
    {\innerdefn[#1]\refstepcounter{defn}\def\@currentlabel{(#1)}}%
    {\innerdefn}%
 }
 {%
  \endinnerdefn
 }
\makeatother

\newtheorem*{scratch}{Scratch Work}


\newtheorem*{namedconj}{Conjecture}
\newcounter{conj}%makes pointer correct
\providecommand{\conjname}{Conjecture}
\makeatletter
\NewDocumentEnvironment{conj}{o}
 {%
  \IfValueTF{#1}
    {\innerconj[#1]\refstepcounter{conj}\def\@currentlabel{(#1)}}%
    {\innerconj}%
 }
 {%
  \endinnerconj
 }
\makeatother

\newtheorem*{poll}{Poll question}
\newtheorem{tps}{Think-Pair-Share}[section]


\newenvironment{obj}{
	\textbf{Learning Objectives.} By the end of class, students will be able to:
		\begin{itemize}}
		{\!.\end{itemize}
		}

\newenvironment{pre}{
	\begin{description}
	}{
	\end{description}
}


\newcounter{ex}%makes pointer correct
\providecommand{\exname}{Homework Problem}
\newenvironment{ex}[1][2in]%
{%Env start code
\problemEnvironmentStart{#1}{Homework Problem}
\refstepcounter{ex}
}
{%Env end code
\problemEnvironmentEnd
}

\newcommand{\inlineAnswer}[2][2 cm]{
    \ifhandout{\pdfOnly{\rule{#1}{0.4pt}}}
    \else{\answer{#2}}
    \fi
}


\ifhandout
\newenvironment{shortAnswer}[1][
    \vfill]
        {% Begin then result
        #1
            \begin{freeResponse}
            }
    {% Environment Ending Code
    \end{freeResponse}
    }
\else
\newenvironment{shortAnswer}[1][]
        {\begin{freeResponse}
            }
    {% Environment Ending Code
    \end{freeResponse}
    }
\fi

\let\question\relax
\let\endquestion\relax

\newtheoremstyle{ExerciseStyle}{\topsep}{\topsep}%%% space between body and thm
		{}                      %%% Thm body font
		{}                              %%% Indent amount (empty = no indent)
		{\bfseries}            %%% Thm head font
		{}                              %%% Punctuation after thm head
		{3em}                           %%% Space after thm head
		{{#1}~\thmnumber{#2}\thmnote{ \bfseries(#3)}}%%% Thm head spec
\theoremstyle{ExerciseStyle}
\newtheorem{br}{In-class Problem}

\newenvironment{sketch}
 {\begin{proof}[Sketch of Proof]}
 {\end{proof}}


\newcommand{\gt}{>}
\newcommand{\lt}{<}
\newcommand{\N}{\mathbb N}
\newcommand{\Q}{\mathbb Q}
\newcommand{\Z}{\mathbb Z}
\newcommand{\C}{\mathbb C}
\newcommand{\R}{\mathbb R}
\renewcommand{\H}{\mathbb{H}}
\newcommand{\lcm}{\operatorname{lcm}}
\newcommand{\nequiv}{\not\equiv}
\newcommand{\ord}{\operatorname{ord}}
\newcommand{\ds}{\displaystyle}
\newcommand{\floor}[1]{\left\lfloor #1\right\rfloor}
\newcommand{\legendre}[2]{\left(\frac{#1}{#2}\right)}



%%%%%%%%%%%%



\title{Linear Diophantine Equations}
\begin{document}
\begin{abstract}
\end{abstract}
\maketitle

%%%%From Spring 2020. Already know the online version works!!

\begin{definition}
    A \emph{Diophantine equation} is any equation in one or more variables to be solved in the integers.
   \end{definition}
    
    
   \begin{definition}
    Let $a_1,a_2,\dots,a_n,b\in\mathbb{Z}$ with $a_1,a_2,\dots,a_n$ not zero. A Diophantine equation of the form \[a_1x_1+a_2x_2+\cdots+a_nx_n=b\] is a \emph{linear Diophantine equation in the $n$ variable $x_1,\dots,x_n$}.
   \end{definition}
    
   The participation assignment classifies linear Diophantine equations in one variable.
    
   The question of whether there are solutions to Diophantine equations becomes harder when there is more than one variable. Then next step is to classify Diophantine equations in two variables.
    
\begin{theorem}\label{thm:linear-dioph}
    Let $ax+by=c$ be a linear Diophantine equation in the variables $x$ and $y$. Let $d=(a,b)$. If $d\nmid c$, then the equation has no solutions; if $d\mid c$, then the equation has infinitely many solutions. Furthermore, if $x_0,y_0$ is a particular solution of the equation, then all solution are given by $x=x_0+\frac{b}{d}n$ and $y=y_0-\frac{a}{d}n$ where $n\in\mathbb{Z}$.
   
    \begin{proof}
        Since $d\mid a,d\mid b$, we have that $d\mid\answer{c}$. So, if $d\nmid c$, then the given linear Diophantine equation has no solutions.
     
        Assume that $d\mid c$. Then, there exists $r,s\in\mathbb{Z}$ such that \[d=(a,b)=ar+bs.\] Furthermore, $d\mid c$ implies $c=de$ for some $e\in\mathbb{Z}$. Then \[c=de=(ar+bs)e=a(re)+b(se).\]
        Thus, $x=re$ and $y=se$ are integer solutions.
     
        Let $x_0,y_0$ be a particular solution to $ax+by=c$ Then, if $n\in\mathbb{Z}, x=x_0+\frac{b}{d}n$ and $y=y_0-\frac{a}{d}n$, \[ax+by=a(x_0+\frac{b}{d}n)+b(y_0-\frac{a}{d}n)=ax_0+\frac{abn}{d}+by_0-\frac{abn}{d}=c.\] We now need to show that every solution has this form. Let $x$ and $y$ be any solution to $ax+by=c$. Then \[(ax+by)-(ax_0+by_0)=c-c=0.\] Rearranging, we get \[a(x-x_0)=b(y_0-y).\] Dividing both sides by $d$ gives \[\frac{a}{d}(x-x_0)=\frac{b}{d}(y_0-y).\] Now $\frac{b}{d}\mid \frac{a}{d}(x-x_0)$ and $(\frac{a}{d},\frac{b}{d})=1$, so $\frac{b}{d}\mid x-x_0$. Thus, $x-x_0=\frac{b}{d}n$ for some $n\in\mathbb{Z}$. The proof for $y$ is similar.
    \end{proof}
\end{theorem}
    
   \begin{example}
   Is $24x+60y=15$ is solvable?
   \begin{multipleChoice}
    \choice {Yes}
    \choice[correct] {No}
   \end{multipleChoice}
   \end{example}
    
   \begin{example}
   Find all solutions to $803x+154y=11$.
    
   Using the Euclidean Algorithm, we find:
     
   \begin{align*}
    803&=154*\answer{5}+\answer{33}\\
    154&=\answer{33}*\answer{4}+\answer{22}\\
    \answer{33}&=\answer{22}*1+\answer{11}
   \end{align*}
   Thus
   \begin{align*}
    (803,154)&=\answer{33}-\answer{22}\\
    &=\answer{33}-(154-\answer{33}*\answer{4})=\answer{33}*\answer{5}-154\\
    &=(803-154*\answer{5})*\answer{5}-154=803*\answer{5}-154*\answer{26}
   \end{align*}
    
   Thus, all solutions to the Diophantine equation have the form $x=\answer{5}+\frac{\answer{154}}{\answer{11}}n$ and $y=\answer{-26}-\frac{\answer{803}}{\answer{11}}n$.
   \end{example}
    
   \begin{example}
        There is a famous riddle about Diophantus: ``God gave him his boyhood one-sixth of his life, One twelfth more as youth while whiskers grew rife; And then yet one-seventh ere marriage begun; In five years there came a bouncing new son. Alas, the dear child of master and sage After attaining half the measure of his father's life chill fate took him. After consoling his fate by the science of numbers for four years, he ended his life."
     
        That is: Diophantus's childhood was $1/6^{th}$ of his life, adolescence was $1/12^{th}$ of his life, after another $1/7^{th}$ of his life he married, his son was born 5 years after he married, his son then died at half the age that Diophantus died, and 4 years later Diophantus died.
     
        The Diophantine equation that let's us solve this riddle is: 
            \[
                x=\frac{x}{6}+\frac{x}{12}+\frac{x}{7}+5+\frac{x}{2}+4.
            \] 
        Then, Diophantus's childhood was $\answer{14}$ years, his adolescence was $\answer{7}$ years, he married when he was $\answer{33}$, his son was born when he was $\answer{38}$ and died $\answer{42}$ years later, then Diophantus died when he was $\answer{84}$.
   \end{example}
   
\end{document}