\documentclass{ximera}
\usepackage{amssymb, latexsym, amsmath, amsthm, graphicx, amsthm,alltt,color, listings,multicol,xr-hyper,hyperref,aliascnt,enumitem}
\usepackage{xfrac}

\usepackage{parskip}
\usepackage[,margin=0.7in]{geometry}
\setlength{\textheight}{8.5in}

\usepackage{epstopdf}

\DeclareGraphicsExtensions{.eps}
\usepackage{tikz}


\usepackage{tkz-euclide}
%\usetkzobj{all}
\tikzstyle geometryDiagrams=[rounded corners=.5pt,ultra thick,color=black]
\colorlet{penColor}{black} % Color of a curve in a plot


\usepackage{subcaption}
\usepackage{float}
\usepackage{fancyhdr}
\usepackage{pdfpages}
\newcounter{includepdfpage}
\usepackage{makecell}


\usepackage{currfile}
\usepackage{xstring}




\graphicspath{  
{./otherDocuments/}
}

\author{Claire Merriman}
\newcommand{\classday}[1]{\def\classday{#1}}

%%%%%%%%%%%%%%%%%%%%%
% Counters and autoref for unnumbered environments
% Not needed??
%%%%%%%%%%%%%%%%%%%%%
\theoremstyle{plain}


\newtheorem*{namedthm}{Theorem}
\newcounter{thm}%makes pointer correct
\providecommand{\thmname}{Theorem}

\makeatletter
\NewDocumentEnvironment{thm*}{o}
 {%
  \IfValueTF{#1}
    {\namedthm[#1]\refstepcounter{thm}\def\@currentlabel{(#1)}}%
    {\namedthm}%
 }
 {%
  \endnamedthm
 }
\makeatother


\newtheorem*{namedprop}{Proposition}
\newcounter{prop}%makes pointer correct
\providecommand{\propname}{Proposition}

\makeatletter
\NewDocumentEnvironment{prop*}{o}
 {%
  \IfValueTF{#1}
    {\namedprop[#1]\refstepcounter{prop}\def\@currentlabel{(#1)}}%
    {\namedprop}%
 }
 {%
  \endnamedprop
 }
\makeatother

\newtheorem*{namedlem}{Lemma}
\newcounter{lem}%makes pointer correct
\providecommand{\lemname}{Lemma}

\makeatletter
\NewDocumentEnvironment{lem*}{o}
 {%
  \IfValueTF{#1}
    {\namedlem[#1]\refstepcounter{lem}\def\@currentlabel{(#1)}}%
    {\namedlem}%
 }
 {%
  \endnamedlem
 }
\makeatother

\newtheorem*{namedcor}{Corollary}
\newcounter{cor}%makes pointer correct
\providecommand{\corname}{Corollary}

\makeatletter
\NewDocumentEnvironment{cor*}{o}
 {%
  \IfValueTF{#1}
    {\namedcor[#1]\refstepcounter{cor}\def\@currentlabel{(#1)}}%
    {\namedcor}%
 }
 {%
  \endnamedcor
 }
\makeatother

\theoremstyle{definition}
\newtheorem*{annotation}{Annotation}
\newtheorem*{rubric}{Rubric}

\newtheorem*{innerrem}{Remark}
\newcounter{rem}%makes pointer correct
\providecommand{\remname}{Remark}

\makeatletter
\NewDocumentEnvironment{rem}{o}
 {%
  \IfValueTF{#1}
    {\innerrem[#1]\refstepcounter{rem}\def\@currentlabel{(#1)}}%
    {\innerrem}%
 }
 {%
  \endinnerrem
 }
\makeatother

\newtheorem*{innerdefn}{Definition}%%placeholder
\newcounter{defn}%makes pointer correct
\providecommand{\defnname}{Definition}

\makeatletter
\NewDocumentEnvironment{defn}{o}
 {%
  \IfValueTF{#1}
    {\innerdefn[#1]\refstepcounter{defn}\def\@currentlabel{(#1)}}%
    {\innerdefn}%
 }
 {%
  \endinnerdefn
 }
\makeatother

\newtheorem*{scratch}{Scratch Work}


\newtheorem*{namedconj}{Conjecture}
\newcounter{conj}%makes pointer correct
\providecommand{\conjname}{Conjecture}
\makeatletter
\NewDocumentEnvironment{conj}{o}
 {%
  \IfValueTF{#1}
    {\innerconj[#1]\refstepcounter{conj}\def\@currentlabel{(#1)}}%
    {\innerconj}%
 }
 {%
  \endinnerconj
 }
\makeatother

\newtheorem*{poll}{Poll question}
\newtheorem{tps}{Think-Pair-Share}[section]


\newenvironment{obj}{
	\textbf{Learning Objectives.} By the end of class, students will be able to:
		\begin{itemize}}
		{\!.\end{itemize}
		}

\newenvironment{pre}{
	\begin{description}
	}{
	\end{description}
}


\newcounter{ex}%makes pointer correct
\providecommand{\exname}{Homework Problem}
\newenvironment{ex}[1][2in]%
{%Env start code
\problemEnvironmentStart{#1}{Homework Problem}
\refstepcounter{ex}
}
{%Env end code
\problemEnvironmentEnd
}

\newcommand{\inlineAnswer}[2][2 cm]{
    \ifhandout{\pdfOnly{\rule{#1}{0.4pt}}}
    \else{\answer{#2}}
    \fi
}


\ifhandout
\newenvironment{shortAnswer}[1][
    \vfill]
        {% Begin then result
        #1
            \begin{freeResponse}
            }
    {% Environment Ending Code
    \end{freeResponse}
    }
\else
\newenvironment{shortAnswer}[1][]
        {\begin{freeResponse}
            }
    {% Environment Ending Code
    \end{freeResponse}
    }
\fi

\let\question\relax
\let\endquestion\relax

\newtheoremstyle{ExerciseStyle}{\topsep}{\topsep}%%% space between body and thm
		{}                      %%% Thm body font
		{}                              %%% Indent amount (empty = no indent)
		{\bfseries}            %%% Thm head font
		{}                              %%% Punctuation after thm head
		{3em}                           %%% Space after thm head
		{{#1}~\thmnumber{#2}\thmnote{ \bfseries(#3)}}%%% Thm head spec
\theoremstyle{ExerciseStyle}
\newtheorem{br}{In-class Problem}

\newenvironment{sketch}
 {\begin{proof}[Sketch of Proof]}
 {\end{proof}}


\newcommand{\gt}{>}
\newcommand{\lt}{<}
\newcommand{\N}{\mathbb N}
\newcommand{\Q}{\mathbb Q}
\newcommand{\Z}{\mathbb Z}
\newcommand{\C}{\mathbb C}
\newcommand{\R}{\mathbb R}
\renewcommand{\H}{\mathbb{H}}
\newcommand{\lcm}{\operatorname{lcm}}
\newcommand{\nequiv}{\not\equiv}
\newcommand{\ord}{\operatorname{ord}}
\newcommand{\ds}{\displaystyle}
\newcommand{\floor}[1]{\left\lfloor #1\right\rfloor}
\newcommand{\legendre}[2]{\left(\frac{#1}{#2}\right)}



%%%%%%%%%%%%



\title{The Euclidean Algorithm}
\begin{document}
\begin{abstract}
\end{abstract}
\maketitle

%%%%%%%%%%%%%%%%%%%%%%%%%%

\begin{obj}
	\item Prove the Euclidean Algorithm halts and generates the greatest common divisor of two positive integers
	\item Use the Euclidean Algorithm to findd the greate common divisor of two integers
	\item  Use the (extended) \nameref{euclid-alg} to write $(a,b)$ as a linear combination of $a$ and $b$
\end{obj}

%%%%%%%%%%%%%%%%%%%%%%%%%%
 % \subsection{Example of proof like the floor function (30 minutes)}
% %%%%%%%%%%%%%%%%%%%%%%%%%%
% I did not go over the entire example in class. This also took the place of in class problems.

% Let's explicitly think about the floor function as a function. That is, $f(x):\mathbb{R}\to\mathbb{Z}$ (a function from the real numbers to the integers). Here is a restatement of the homework problem:

% \begin{prob}
% For each of the following equations, find a domain for $f(x)=\floor{ x }$ make the statement true. Prove your statement. 
% 	\begin{enumerate}
%  		\item $f( x ) + f( x ) =f( 2x)$
% 		\item $f( x + 3 )  = 3 +f( x)$
% 		\item $f( x +3 ) = 	3 + x$
% 	\end{enumerate} 
% \end{prob}

% Here is a similar problem with proofs where $g:\mathbb{R}\to\mathbb{R}$. Note that the scratch work is one way to think about solving the problem but would not be included in the homework writeup.

% \begin{example}
%  For each of the following equations, find the full domain for $g(x)=3\sin(\pi x)$ that
%  makes the statement true. Prove that the equation is always true on this domain. 
% 	\begin{enumerate}
%  		\item $g( x ) + g( x ) =g( 2x)$
		
% 		\begin{scratch}
% 		We want to find a restriction such that $3\sin(\pi x)+3\sin(\pi x)=3\sin(2\pi x)$. Using the double angle formula, we get \[3\sin(2\pi x)=6\sin(\pi x)\cos(2\pi x).\] This means we are actually looking for 
% 			\begin{align*}
%  				6\sin(\pi x)\cos(2\pi x)&=6\sin(\pi x)\\
% 				\cos(2\pi x)&=1.
% 			\end{align*}
% 		\end{scratch}

% 		\begin{solution}
%  		If $x\in\mathbb{Z}$, then $g(x)+g(x)=g(2 x)$.
 
% 		\begin{proof}
%  		Let $x\in\mathbb{Z}$. Then $g(x)+g(x)=3\sin(\pi x)+3\sin(\pi x)=0$ and $g(2x)=3\sin(2\pi x)=0$. Thus, $g(x)+g(x)=g(2 x)$.
% 		\end{proof}
% 		\end{solution}

% 		\item $g( x + 3 )  = 3 +g( x)$
		
% 		\begin{scratch}
%  			We want to find a restriction such that $3\sin(\pi (x+3))=3+3\sin(\pi x)$.
% 			Using the angle addition formula, we get 
% 			\[3\sin(\pi x+3\pi))=3\sin(\pi x)\cos(3\pi) +3\cos(\pi x)\sin(3\pi)=-3\sin(\pi x).\] This means we are actually looking for 
% 			\begin{align*}
%  				-3\sin(\pi x)&=3\sin(\pi x)+3\\
% 				-1&=2\sin(\pi x)\\.
% 			\end{align*}
% 		\end{scratch}
		
		
% 		\begin{solution}
%  		If $x=2k+\frac{7}{6}$ or $x=2k-\frac{1}{6}$ for some $k\in\mathbb{Z}$, then $g (x+3)=g(x)+3.$
		
% 		\begin{proof}
		
% 		\begin{description} Note that $g(x+3)=3\sin(\pi x+3\pi)=-3\sin(\pi x)$ by the angle addition formula. We will consider two cases:
%  			\item{Case 1:} Let $x=2k+\frac{7}{6}$ for some $k\in\mathbb{Z}$. Then 			
% 			\begin{align*}
%  			g(x+3)&=-3\sin(\pi x)\\
% 			&=-3\sin(2k\pi+\frac{7\pi}{6})\\
% 			&=\frac{3}{2},
% 			\end{align*}
% 			and $g(x)+3=3\sin(2k\pi+\frac{7\pi}{6})+3=\frac{3}{2}$.
% 			\item{Case 2:} Let $x=2k-\frac{1}{6}$ for some $k\in\mathbb{Z}$. Then 			
% 			\begin{align*}
%  			g(x+3)&=-3\sin(\pi x)\\
% 			&=-3\sin(2k\pi-\frac{\pi}{6})\\
% 			&=\frac{3}{2},
% 			\end{align*}
% 			and $g(x)+3=3\sin(2k\pi-\frac{\pi}{6})+3=\frac{3}{2}$.
% 		\end{description}
% 		Thus, $g(x+3)=g(x)+3$ when $x=2k+\frac{7}{6}$ or $x=2k-\frac{1}{6}$ for some $k\in\Z$.
% 		\end{proof}
% 		\end{solution}
% 		\item Let $h(x)=x^2-1$. For each of the following equations, find the full domain for $h(x)$ that makes the statement true. Prove your statement.
		
% 		\begin{enumerate}
% 			\item $h(x+3)=h(x)+3$
% 			\item $h(x+3)=x+3$
% 		\end{enumerate}


% 		\begin{scratch}
% 			First, $h(x+3)=(x+3)^2-1=x^2+6x+8$.
			
% 			For (a) 
% 			\begin{align*}
% 				x^2+6x+8&= x^2 -1+3\\
% 				6x& = -6
% 			\end{align*}
% 			For (b)
% 			\begin{align*}
% 				x^2+6x+8&= x+3\\
% 				x^2 +5x+5& = 0\\
% 				x&=\frac{-5\pm\sqrt{5}}{2}
% 			\end{align*}
% 		\end{scratch}
		
% 		\begin{solution}
			
% 			\begin{enumerate}
% 				\item For $h(x)=x^2-1,$ $h(x+3)=h(x)+3$ if and only if $x=-1$.
% 				\begin{proof}
% 					Let $h(x)=x^2-1$ and $x=-1$. Then $h(x+3)=3=0+3=h(x)+3$.

% 					To prove that this is the only such $x$, let $h(x+3)=h(x)+3$. Then $x^2+6x+8=x^2+2$, which simplifies to $x=-1$.
% 				\end{proof}
% 				\item For $h(x)=x^2-1,$ $h(x+3)=x+3$ if and only if $x=\frac{-5\pm\sqrt{5}}{2}$.
% 				\begin{proof}
% 					Let $h(x)=x^2-1$. First we consider the case $x=\frac{-5+\sqrt{5}}{2}$. Then $h(x+3)=\frac{1+\sqrt{5}}{2}=
% 					\frac{-5+\sqrt{5}}{2}+3=x+3$. 
					
% 					Next, we consider the case $x=\frac{-5-\sqrt{5}}{2}$. Then $h(x+3)=\frac{1-\sqrt{5}}{2}=
% 					\frac{-5-\sqrt{5}}{2}+3=x+3$.

% 					To prove that these are the only such $x$, let $h(x+3)=x+3$. Then $x^2+6x+8=x+3$, which  hasthe solutions  $x=\frac{-5\pm\sqrt{5}}{2}$.
% 				\end{proof}
% 			\end{enumerate}
% 		\end{solution}
% 	\end{enumerate}
% \end{example}

% %%%%%%%%%%%%%%%%%%%%%%%%%%


Typically by \emph{Euclidean Algorithm}, we mean  both the algorithm and the theorem that the algorithm always generates the greatest common divisor of two (positive) integers.


\begin{thm*}[Euclidean algorithm]\label{euclid-alg}
	Let $a,b\in\Z$ with $a\geq b>0$. By the \nameref{div-alg}, there exist $q_1,r_1\in\Z$ such that 
	\[a=b q_1+r_1,\quad 0\leq r_1<b.\]
	If $r_1>0$, there exist $q_2,r_2\in\Z$ such that 
	\[b=r_1 q_2+r_2,\quad 0\leq r_2<r_1.\]
	If $r_2>0$, there exist $q_3,r_3\in\Z$ such that 
	\[r_1=r_2 q_3+r_3,\quad 0\leq r_3<r_2.\]
	
	Continuing this process, $r_n=0$ for some $n$. If $n>1$, then $\gcd(a,b)=r_{n-1}$. If $n=1$, then $\gcd(a,b)=b$.
\begin{proof}
	 Note that $r_1>r_2>r_3>\dots\geq0$ by construction. If the sequence did not stop, then we would have an infinite, decreasing sequence of positive integers, which is not possible. Thus, $r_n=0$ for some $n$. 
	 
	 When $n=1$, $a=bq+0$ and $\gcd(a,b)=b$.
	 
	 \nameref{lem:gcd-remainders} states that for $a=bq_1+r_1$, $\gcd(a,b)=\gcd(b,r_1)$. This is because any common divisor of $a$ and $b$  is also a divisor of $r_1=a-bq_1$. 
	 
	 If $n>1$, then by repeated application of the \nameref{lem:gcd-remainders}, we have 
	 \[\gcd(a,b)=\gcd(b,r_1)=\gcd(r_1,r_2)=\cdots=\gcd(r_{n-2},r_{n-1})\]
	 Then $r_{n-2}=r_{n-1} q_n+0$. Thus $\gcd(r_{n-2},r_{n-1})=r_{n-1}$.
\end{proof}
\end{thm*}

When using the \nameref{euclid-alg}, it can be tricky to keep track of what is happening. Doing a lot of examples can help.
	


Work in pairs to answer the following. Each pair will be assigned parts the following question.

\begin{br}
Find the greatest common divisors of the pairs of integers below and write the greatest common divisor as a linear combination of the integers.
\begin{enumerate}
	\item $(21,28)$
	
	\begin{solution}
		By inspection: $28-21=7$.

		Using the \nameref{euclid-alg}:
		$a=28,b=21$
		\begin{align*}
			28 & = 21(1)+7 &q_1=1,r_1=7 &&7=21(1)+28(-1)\\
			21 & = 7(3) +0 & q_2=3, r_2=0
		\end{align*}
		so $28+(-1)21=7=(28,21)$
	\end{solution}


	\item $(32,56)$
	 \begin{solution}
	 	Using the \nameref{euclid-alg}:
	 	$a=56,b=32$
	 	\begin{align*}
	 		56 & = 32(1)+24 &q_1=1,r_1=24 &&24=56(1)+32(-1)\\
	 		32 & = 24(1) +8 & q_2=1, r_2=8 &&8=32(1)+24(-1)=32(1)+(56(1)+32(-1))(-1)=32(2)+56(-1)\\
	 		32&=8(4)+0 & q_3=4, r_3=0.
	 	\end{align*}
	 	so $56(-1)+32(2)=8=(56,32)$
	 \end{solution}

	
	\item $(0,113)$
	 \begin{solution}
	 	Since $0=113(0)$, $(0,113)=113=0(0)=113(1)$.
	 \end{solution}
\end{enumerate}


\begin{enumerate}[label=54(\alph*)]
	\setcounter{enumi}{1}
	\item $(78,708)$
	 \begin{solution}
	 	Using the \nameref{euclid-alg}:
	 	$a=708,b=78$
	 	\begin{align*}
	 		708 & = 78(9)+6 &q_1=9,r_1=6 &&6=708(1)+78(-9)\\
	 		78 & = 6(13) +0 & q_2=13, r_2=0.
	 	\end{align*}
	 	so $708(1)+78(-6)=6=(78,708)$
	 \end{solution}
\end{enumerate}
\end{br}



\end{document}
