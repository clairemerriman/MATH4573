\documentclass{ximera}
\usepackage{amssymb, latexsym, amsmath, amsthm, graphicx, amsthm,alltt,color, listings,multicol,xr-hyper,hyperref,aliascnt,enumitem}
\usepackage{xfrac}

\usepackage{parskip}
\usepackage[,margin=0.7in]{geometry}
\setlength{\textheight}{8.5in}

\usepackage{epstopdf}

\DeclareGraphicsExtensions{.eps}
\usepackage{tikz}


\usepackage{tkz-euclide}
%\usetkzobj{all}
\tikzstyle geometryDiagrams=[rounded corners=.5pt,ultra thick,color=black]
\colorlet{penColor}{black} % Color of a curve in a plot


\usepackage{subcaption}
\usepackage{float}
\usepackage{fancyhdr}
\usepackage{pdfpages}
\newcounter{includepdfpage}
\usepackage{makecell}


\usepackage{currfile}
\usepackage{xstring}




\graphicspath{  
{./otherDocuments/}
}

\author{Claire Merriman}
\newcommand{\classday}[1]{\def\classday{#1}}

%%%%%%%%%%%%%%%%%%%%%
% Counters and autoref for unnumbered environments
% Not needed??
%%%%%%%%%%%%%%%%%%%%%
\theoremstyle{plain}


\newtheorem*{namedthm}{Theorem}
\newcounter{thm}%makes pointer correct
\providecommand{\thmname}{Theorem}

\makeatletter
\NewDocumentEnvironment{thm*}{o}
 {%
  \IfValueTF{#1}
    {\namedthm[#1]\refstepcounter{thm}\def\@currentlabel{(#1)}}%
    {\namedthm}%
 }
 {%
  \endnamedthm
 }
\makeatother


\newtheorem*{namedprop}{Proposition}
\newcounter{prop}%makes pointer correct
\providecommand{\propname}{Proposition}

\makeatletter
\NewDocumentEnvironment{prop*}{o}
 {%
  \IfValueTF{#1}
    {\namedprop[#1]\refstepcounter{prop}\def\@currentlabel{(#1)}}%
    {\namedprop}%
 }
 {%
  \endnamedprop
 }
\makeatother

\newtheorem*{namedlem}{Lemma}
\newcounter{lem}%makes pointer correct
\providecommand{\lemname}{Lemma}

\makeatletter
\NewDocumentEnvironment{lem*}{o}
 {%
  \IfValueTF{#1}
    {\namedlem[#1]\refstepcounter{lem}\def\@currentlabel{(#1)}}%
    {\namedlem}%
 }
 {%
  \endnamedlem
 }
\makeatother

\newtheorem*{namedcor}{Corollary}
\newcounter{cor}%makes pointer correct
\providecommand{\corname}{Corollary}

\makeatletter
\NewDocumentEnvironment{cor*}{o}
 {%
  \IfValueTF{#1}
    {\namedcor[#1]\refstepcounter{cor}\def\@currentlabel{(#1)}}%
    {\namedcor}%
 }
 {%
  \endnamedcor
 }
\makeatother

\theoremstyle{definition}
\newtheorem*{annotation}{Annotation}
\newtheorem*{rubric}{Rubric}

\newtheorem*{innerrem}{Remark}
\newcounter{rem}%makes pointer correct
\providecommand{\remname}{Remark}

\makeatletter
\NewDocumentEnvironment{rem}{o}
 {%
  \IfValueTF{#1}
    {\innerrem[#1]\refstepcounter{rem}\def\@currentlabel{(#1)}}%
    {\innerrem}%
 }
 {%
  \endinnerrem
 }
\makeatother

\newtheorem*{innerdefn}{Definition}%%placeholder
\newcounter{defn}%makes pointer correct
\providecommand{\defnname}{Definition}

\makeatletter
\NewDocumentEnvironment{defn}{o}
 {%
  \IfValueTF{#1}
    {\innerdefn[#1]\refstepcounter{defn}\def\@currentlabel{(#1)}}%
    {\innerdefn}%
 }
 {%
  \endinnerdefn
 }
\makeatother

\newtheorem*{scratch}{Scratch Work}


\newtheorem*{namedconj}{Conjecture}
\newcounter{conj}%makes pointer correct
\providecommand{\conjname}{Conjecture}
\makeatletter
\NewDocumentEnvironment{conj}{o}
 {%
  \IfValueTF{#1}
    {\innerconj[#1]\refstepcounter{conj}\def\@currentlabel{(#1)}}%
    {\innerconj}%
 }
 {%
  \endinnerconj
 }
\makeatother

\newtheorem*{poll}{Poll question}
\newtheorem{tps}{Think-Pair-Share}[section]


\newenvironment{obj}{
	\textbf{Learning Objectives.} By the end of class, students will be able to:
		\begin{itemize}}
		{\!.\end{itemize}
		}

\newenvironment{pre}{
	\begin{description}
	}{
	\end{description}
}


\newcounter{ex}%makes pointer correct
\providecommand{\exname}{Homework Problem}
\newenvironment{ex}[1][2in]%
{%Env start code
\problemEnvironmentStart{#1}{Homework Problem}
\refstepcounter{ex}
}
{%Env end code
\problemEnvironmentEnd
}

\newcommand{\inlineAnswer}[2][2 cm]{
    \ifhandout{\pdfOnly{\rule{#1}{0.4pt}}}
    \else{\answer{#2}}
    \fi
}


\ifhandout
\newenvironment{shortAnswer}[1][
    \vfill]
        {% Begin then result
        #1
            \begin{freeResponse}
            }
    {% Environment Ending Code
    \end{freeResponse}
    }
\else
\newenvironment{shortAnswer}[1][]
        {\begin{freeResponse}
            }
    {% Environment Ending Code
    \end{freeResponse}
    }
\fi

\let\question\relax
\let\endquestion\relax

\newtheoremstyle{ExerciseStyle}{\topsep}{\topsep}%%% space between body and thm
		{}                      %%% Thm body font
		{}                              %%% Indent amount (empty = no indent)
		{\bfseries}            %%% Thm head font
		{}                              %%% Punctuation after thm head
		{3em}                           %%% Space after thm head
		{{#1}~\thmnumber{#2}\thmnote{ \bfseries(#3)}}%%% Thm head spec
\theoremstyle{ExerciseStyle}
\newtheorem{br}{In-class Problem}

\newenvironment{sketch}
 {\begin{proof}[Sketch of Proof]}
 {\end{proof}}


\newcommand{\gt}{>}
\newcommand{\lt}{<}
\newcommand{\N}{\mathbb N}
\newcommand{\Q}{\mathbb Q}
\newcommand{\Z}{\mathbb Z}
\newcommand{\C}{\mathbb C}
\newcommand{\R}{\mathbb R}
\renewcommand{\H}{\mathbb{H}}
\newcommand{\lcm}{\operatorname{lcm}}
\newcommand{\nequiv}{\not\equiv}
\newcommand{\ord}{\operatorname{ord}}
\newcommand{\ds}{\displaystyle}
\newcommand{\floor}[1]{\left\lfloor #1\right\rfloor}
\newcommand{\legendre}[2]{\left(\frac{#1}{#2}\right)}



%%%%%%%%%%%%



\title{Greatest Common Divisors and Diophantine Equations}
\begin{document}
\begin{abstract}
\end{abstract}
\maketitle

%%%%%%%%%%%%%%%%%%%%%%%%%%
%%%%%%%%%%%%%%%%%%%%%%%%%%

\begin{obj}
	\item Prove the formula for integer solutions to $ax+by=c$.
	\item State when integer solution exist for $a_1x_1+\cdots+a_kx_k=c$.
\end{obj}


\begin{instructorNotes}
	\begin{pre}
		\item[Read] Strayer, Section 6.1
		\item[Turn in] Exercise 2a.
		Find all integer solutions to $18x+28y=10$
	   % \begin{solution}
	   % Notice that $18(-1)+28(1)=10$. This can be done by inspection or the \nameref{euclid-alg}. Then our specific solution is $x_0=-1, y_0=1$ and all solutions have the form \[x=-1+\frac{28n}{2}=-1+14n,\quad y=1-\frac{18n}{2}=1-9n \qquad \textnormal{for all $n\in\Z$.}\]
	   % \end{solution}
	   \end{pre}
\end{instructorNotes}


\begin{lemma}\label{lem:gcd_3case}
	Let $a,b,c\in\Z,$ with $a\neq 0$. Then $(a,b,c)=((a,b),c).$

 	\begin{proof}
 		Let $a,b,c\in\Z,$ with $a\neq 0$. Define $d=(a,b,c)$ and $e=((a,b),c).$ We will show that $d\mid e$ and $e\mid d$. Since the greatest common divisor is positive, we can conclude that $d=e$\footnote{This is not true in general and a common mistake. In general $d=\pm e$}.

 		Since $d=(a,b,c),$ we know $d\mid a, d\mid b,$ and $d\mid c$. By Lemma \ref{lem:gcd_mult}, which we are about to prove, $d\mid (a,b)$. Thus, $d$ is a common divisor of $(a,b)$ and $c$, so $d\mid e$.

 		Since $e=((a,b),c)$, $e\mid  (a,b)$ and $e\mid c$. Since $e\mid (a,b),$ we know $e\mid a$ and $e\mid b$ by Lemma \ref{lem:gcd_mult}. Thus, $e$ is a common divides of $a,b$ and $c$
 	\end{proof}
\end{lemma}


\begin{lemma}\label{lem:gcd_mult}
	Let $a,b\in\Z,$ not both zero. Then any  common divisor of $a$ and $b$ divides the greatest common divisor.

 	\begin{proof}
 		Let $a,b\in\Z,$ not both zero. By \nameref{Bezout}, $(a,b)=am+bn$ for some $n,m\in\Z$. Thus, $d\mid (a,b)$ by linear combination.
 	\end{proof}
\end{lemma}


\begin{lemma}\label{lem:gcd_trans}
 	Let $a,b\in\Z,$ not both zero. Then any divisor of $(a,b)$ is a common divisor of $a$ and $b$.

	\begin{proof}
 		Let $c$ be a divisor of $(a,b)$. Since $(a,b)\mid a$ and $(a,b )\mid b,$ then $c\mid a$ and $c\mid b$ by transitivity.
	\end{proof}
\end{lemma}

%\begin{proposition}
%	Let $a,b,c\in\Z,$ at least one not equal to zero, and $d=(a,b,c)$. Then there exists integers $x,y,z$ such that \[ax+by+cz=d.\]
%\end{proposition}


\begin{proposition}
 	Let $a_1,\dots,a_n\in\Z$ with $a_1\neq 0$.  Then 
		\[
			(a_1,\dots,a_n)
			=((a_1,a_2,a_3,\dots,a_{n-1}),a_n).
		\]
	\begin{proof}
 		Let $k=2$. The since $((a_1,a_2))=(a_1,a_2)$ by the definition of \nameref{defn:gcd} of one integer,  $(a_1,a_2)=((a_1,a_2))$. The $k=3$ case is the first lemma in this section (\ref{lem:gcd_3case}).
 
 		Assume that for all $2\leq k< n$, 
 			\[
				(a_1,\dots,a_k)
				=((a_1,a_2,a_3,\dots,a_{k-1}),a_k).
			\]
		Let $d=(a_1,a_2,a_3,\dots,a_{k}),$
		$e=((a_1,a_2,a_3,\dots,a_{k}),a_{k+1})=(d,a_{k+1}),$ and $f= (a_1,a_2,a_3,\dots,a_{k},a_{k+1}).$ We will show that $e\mid f$ and $f\mid e$. Since both $e$ and $f$ are positive, this will prove that $e=f$.

		Note that $e\mid (a_1,a_2,a_3,\dots,a_{k})$ and $e\mid a_{k+1}$ by definition. 
		Since $(a_1,\dots,a_k)=((a_1,a_2,a_3,\dots,a_{k-1}),a_k)$ by the induction hypothesis, $e\mid(a_1,a_2,a_3,\dots,a_{k-1})$ and $e\mid a_k$ by Lemma \ref{lem:gcd_trans}. Again, by the induction hypothesis, $(a_1,a_2,a_3,\dots,a_{k-1})=((a_1,a_2,a_3,\dots,a_{k-2}),a_{k-1}),$ so $e\mid a_{k-1}$ and $e\mid (a_1,a_2,a_3,\dots,a_{k-2})$ by Lemma \ref{lem:gcd_trans}. Repeat this process until we get $(a_1,a_2,a_3)=((a_1,a_2),a_3)$, so $e\mid a_3$ and $e\mid (a_1,a_2)$ by Lemma \ref{lem:gcd_trans}. Thus $e\mid a_1,a_2,\dots,a_{k+1}$ by repeated applications of Lemma \ref{lem:gcd_trans}. By the generalized version of the Lemma \ref{lem:gcd_mult} on Homework 3, $e\mid f.$

		To show that $f\mid e,$ we note that $f\mid a_1,a_2,\dots, a_k,a_{k+1}$ by definition. Then $f\mid d$ by the generalized version of the Lemma \ref{lem:gcd_mult} on Homework 3. Since $e=(d,a_k),$ we have that $f\mid e$ by Lemma \ref{lem:gcd_mult}.
	\end{proof}
\end{proposition}



\end{document}
