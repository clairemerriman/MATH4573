\documentclass[letterpaper, 11 pt]{ximera}
\usepackage{amssymb, latexsym, amsmath, amsthm, graphicx, amsthm,alltt,color, listings,multicol,hyperref,enumitem,tikz}
\usepackage{xfrac}

\usepackage{parskip}
\usepackage{graphicx}
\usepackage[,margin=0.7in]{geometry}
\setlength{\textheight}{8.5in}
 
\usepackage{tkz-euclide}
%\usetkzobj{all}
\tikzstyle geometryDiagrams=[rounded corners=.5pt,ultra thick,color=black]
\colorlet{penColor}{black} % Color of a curve in a plot


\usepackage{subcaption}
\usepackage{float}
\usepackage{fancyhdr}
\usepackage{pdfpages}
\newcounter{includepdfpage}


\newcommand{\semester}{%
  \ifcase\month
  \or Spring %1
  \or Spring %2
  \or Spring %3
  \or Spring %4
  \or Spring  %5
  \or Fall %8
  \or Fall %9
  \or Fall %10
  \or Fall %11
  \or Fall %12
  \fi
}
\usepackage{currfile}
\usepackage{xstring}



\lhead{\large{Number Theory: MAT-255}}
%Put your Document Title (Camp: Topic) Here
\chead{}
\rhead{\semester 24}
\lfoot{}
\cfoot{}
\rfoot{Page \thepage}
\renewcommand\headrulewidth{0pt}
\renewcommand\footrulewidth{0pt}

\headheight 50pt
\headsep 30pt

\author{Spring 2024}
\date{}
 \theoremstyle{plain}
 
\newtheorem{thm}{Theorem}%[section] % reset theorem numbering for each section
\newtheorem*{thm*}{Theorem}
\newtheorem{prop}[thm]{Proposition}
\newtheorem*{prop*}{Proposition}
\newtheorem{lem}[thm]{Lemma}
\newtheorem*{lem*}{Lemma}
\newtheorem{cor}[thm]{Corollary}
\newtheorem*{cor*}{Corollary}

\theoremstyle{definition}
\newtheorem*{rem}{Remark}
\newtheorem*{defn}{Definition}
\newtheorem{defn_num}{Definition}
\newtheorem*{ex}{Exercise}
\newtheorem*{scratch}{Scratch Work}

%\newtheorem*{focus}{Central Focus}
%\newtheorem*{obj}{Objectives}

\newtheorem*{conj}{Conjecture}

\newtheorem*{poll}{Poll question}
\newtheorem{tps}{Think-Pair-Share}[section]
%\newtheorem{br}{In-class Problem}[section]
\newtheorem*{cs}{Crowd Sourced Proof}

\newlist{checklist}{itemize}{2}
\setlist[checklist]{label=$\square$}

\newenvironment{obj}{
	\textbf{Learning Objectives.} By the end of class, students will be able to:
		\begin{itemize}}
		{\!.\end{itemize}
		}

\newenvironment{pre}{
	\begin{description}
	}{
	\end{description}
}


\newenvironment{br}[1][2in]%
{%Env start code
\problemEnvironmentStart{#1}{In-class Problem}
}
{%Env end code
\problemEnvironmentEnd
}


%\newenvironment{solution}
%  {\begin{proof}[Solution]}
%  {\end{proof}}
%\newenvironment{hint}
%  {\begin{proof}[Hint]}
%  {\end{proof}}

\newcommand{\gt}{>}
\newcommand{\lt}{<}
\newcommand{\N}{\mathbb N}
\newcommand{\Q}{\mathbb Q}
\newcommand{\Z}{\mathbb Z}
\newcommand{\C}{\mathbb C}
\newcommand{\R}{\mathbb R}
\renewcommand{\H}{\mathbb{H}}
\newcommand{\lcm}{\operatorname{lcm}}
\newcommand{\nequiv}{\not\equiv}
\newcommand{\ord}{\operatorname{ord}}
\newcommand{\ds}{\displaystyle}

\usepackage{makecell}

%Imports for cross references
\externaldocument{otherResults}
\externaldocument{Week1Spring24}
\externaldocument{Week2Spring24}
\externaldocument{Week3Spring24}
\externaldocument{Week4Spring24}
\externaldocument{Week5Spring24}
\externaldocument{Week6Spring24}
\externaldocument{Week7Spring24}
\externaldocument{Week8Spring24}
\externaldocument{Week9Spring24}
\externaldocument{Week10Spring24}




\StrBetween*[1,1]{\currfilename}{Week}{Sp}[\week]

\title{Week \week--MAT-255 Number Theory}

\begin{document}
\section{Monday, April 1: Gauss's Lemma and Practice}
%%%%%%%%%%%%%%%%%%%%%%%%%%

\begin{obj}
    \item Find the Legendre symbol using \nameref{lem:gauss}
	\item Find the Legendre symbol using several different methods
\end{obj}


\begin{pre}
    \item[Reading] None
\end{pre}

%%%%%%%%%%%%%%%%%%%%%%%%%%
\subsection{Statement of Guass's Lemma (20 minutes)}
%%%%%%%%%%%%%%%%%%%%%%%%%%
\begin{lemma}[Gauss's Lemma]\label{lem:gauss}
	Let $p$ be an odd prime number and like $a\in\Z$ with $p\nmid a$. Let $n$ be the number of least positive residues of the integers $a,2a,3a,\dots,\frac{p-1}{a}$ modulo $p$ that are greater than $\frac{p}{2}.$ Then \[\legendre{a}{p}=(-1)^n.\]
\end{lemma}

\begin{example}
	Find $\legendre{6}{11}$ 
		
	\begin{enumerate}
		\item Using \nameref{lem:gauss}
		\begin{solution}
			Note that $\frac{11-1}{2}=5$.

			First, we list $6,2(6),3(6),4(6),5(6)$ and find the least nonnegative residues modulo $11:$
			\[ 6,\ 2(6)\equiv 1\pmod{11},\ 3(6)\equiv 7\pmod{11},
			\ 4(6)\equiv 2\pmod{11},\ 5(6)\equiv 8\pmod{11}. \]

			Now we count $n=3$ of the least nonnegative residues modulo $11$ are greater than $\frac{11}{2}=5.5$

			Thus, $\legendre{6}{11}=(-1)^3=-1.$
		\end{solution}

		\item Factoring and using \hyperref[quad-rec-standard-form]{quadratic reciprocity}
			
		\begin{solution}
			Using \nameref{prop:legendre-facts} and the fact that $2\equiv -9\pmod{11},$ 
			\begin{align*}
				\legendre{6}{11}&=\legendre{2}{11}\legendre{3}{11}
					=\legendre{-9}{11}\legendre{3}{11}=\legendre{-1}{11}\legendre{9}{11}\legendre{3}{11}=\legendre{-1}{11}(1)\legendre{3}{11}
			\end{align*}
			Since $11\equiv 3\pmod{4},$ $\legendre{-1}{11}=-1$ by \nameref{thm:residue-neg1} and $\legendre{3}{11}=-\legendre{11}{3}$. Thus, 
			\begin{align*}
				\legendre{6}{11}&=\legendre{-1}{11}\legendre{3}{11}=(-1)(-1)\legendre{11}{3}=\legendre{-1}{3}=-1.
			\end{align*}
		\end{solution}
	\end{enumerate}
\end{example}


\begin{example}
	Find $\legendre{-11}{7}$
		
	\begin{enumerate}
		\item Using \nameref{lem:gauss}
			
		\begin{solution}
			Since $\frac{7-1}{2}=3,$ we need to find the least nonnegative residues of $-11,2(-11),3(-11)$ modulo $7.$ These are \[-11\equiv 3\pmod{7},\ 2(-11)\equiv 6\pmod{7},\ 3(-11)\equiv 2\pmod{7}.\]
			Then $n=1$ is greater than $\frac{7}{2}=3.5$ and $\legendre{-11}{7}=(-1)^1=-1.$
		\end{solution}

		\item By reducing modulo $7$ then using \nameref{lem:gauss}
			
		\begin{solution}
			By \hyperref[legendre-respects-mod]{Theorem 4.5(b)} $\legendre{-11}{7}=\legendre{3}{11}.$
			Since $\frac{7-1}{2}=3,$ we need to find the least nonnegative residues of $3,2(3),3(3)$ modulo $7.$ These are \[3\pmod{7},\ 6\pmod{7},\ 3(3)\equiv 2\pmod{7}.\]
			Then $n=1$ is greater than $\frac{7}{2}=3.5$ and $\legendre{-11}{7}=(-1)^1=-1.$
		\end{solution}

		\item By reducing modulo $7$ and using \hyperref[quad-rec-useful-form]{quadratic reciprocity}
			
		\begin{solution}
			By \hyperref[legendre-respects-mod]{Theorem 4.5(b)} $\legendre{-11}{7}=\legendre{3}{11}.$ Since $11\equiv 3\equiv 3\pmod{4},$ $\legendre{3}{11}=-\legendre{11}{3}$ By \hyperref[legendre-respects-mod]{Theorem 4.5(b)} $-\legendre{11}{3}=-\legendre{-1}{3}=1$ using \nameref{thm:residue-neg1}.
		\end{solution}
	\end{enumerate}
\end{example}

%%%%%%%%%%%%%%%%%%%%%%%%%%
\subsection{Practice Problems (30 minutes)}
%%%%%%%%%%%%%%%%%%%%%%%%%%
We can combine these results to find the Legendre symbol many different ways.

\begin{br}
	Use the following methods to find $\legendre{-6}{11}$:
 
	\begin{enumerate}
		\item Euler's Criterion, from March 22: 
			\begin{solution}
 				$\left(\frac{-6}{11}\right)\equiv (-6)^{(11-1)/2}\equiv (-6)^{5}\pmod{11}$ By Euler's Criterion. Then
					\[
						(-6)^{5}\equiv ((6)^{2})^{2}(-6)\equiv 3^2(-6)\equiv -54 \equiv 1\pmod{11}\qedhere
					\]
			\end{solution}
		
			\item Factor into $\legendre{-6}{11}=\legendre{-1}{11}\legendre{2}{11}\legendre{3}{11}=(\answer{-1})\legendre{2}{11}\legendre{3}{11}$. From here, we will explore the various was to find $\legendre{2}{11}$ and $\legendre{3}{11}$.
		
			\begin{enumerate}
 				\item Find $\legendre{2}{11}$ 
				 \begin{itemize}
					\item Using \nameref{thm:euler-quads}.
						\begin{solution} 
							From  \nameref{thm:euler-quads}, 
								\[
									\legendre{2}{11}\equiv 2^{(11-1)/2}\equiv 32\equiv -1\pmod{11}.
								\]
						\end{solution}
					
			
					\item Using \nameref{lem:gauss}.
				
						\begin{solution} 
							First, find the least nonnegative residues of $2, 2(2), 3(2),4(2), 5(2)$ modulo $11.$ 
							These are \[2,4,6,8,10,\] 
							and $n=\answer{3}$ are greater than $\frac{11}{2}.$ Thus, by \nameref{lem:gauss}, \[\legendre{2}{11}=(-1)^{\answer{3}}=\answer{3}.\qedhere\]
						\end{solution}
					\end{itemize}
				
				\item Find $\legendre{3}{11}$
					\begin{itemize}
						\item Using \nameref{thm:euler-quads}.
							\begin{solution} 
								From  \nameref{thm:euler-quads}, 
									\[
										\legendre{3}{11}\equiv 3^{(11-1)/2}
										\equiv (-2)^2(3)\equiv 
										1\pmod{11}.\qedhere
							\]
					\end{solution}
						
			
						\item Using \hyperref[quad-rec-useful-form]{quadratic reciprocity}
							\begin{solution}
								Since $11\equiv 3\pmod{4},$ $\legendre{3}{11}=-\legendre{11}{3}=-\legendre{2}{3}=1.$
			 				\end{solution}
							

						\item Using \nameref{lem:gauss}.
							\begin{solution} 
								First, find the least nonnegative residues of $3, 2(3), 3(3),4(3), 5(3)$ modulo $11.$ 
								These are \[\answer{3,6,9,1,4}\] 
					
								
								and $n=\answer{2}$ are greater than $\frac{11}{2}.$ Thus, by \nameref{lem:gauss}, \[\legendre{3}{11}=(-1)^{\answer{2}}=\answer{2}.\qedhere\]
							\end{solution}
					\end{itemize}


			\end{enumerate}
			Thus, $\legendre{-6}{11}=$

			\item Use that $-6\equiv 5\pmod{11},$ so $\legendre{-6}{11}=\legendre{5}{11}.$ Then find $\legendre{5}{11}$ using \nameref{thm:euler-quads}.
			
			\begin{enumerate}
				\item Using \nameref{thm:euler-quads}.
					\begin{solution} 
						From  \nameref{thm:euler-quads}, 
							\[
								\legendre{5}{11}\equiv 5^{(11-1)/2}									
								\equiv (3)^2(5)\equiv 
								1\pmod{11}.\qedhere
							\]
					\end{solution}
				
	
				\item Using \hyperref[quad-rec-useful-form]{quadratic reciprocity}
							\begin{solution}
								Since $5\equiv 1\pmod{4},$ $\legendre{5}{11}=\legendre{11}{5}=\legendre{1}{5}=1.$
			 				\end{solution}
					

				\item Using \nameref{lem:gauss}.
					\begin{solution} 
						First, find the least nonnegative residues of $5, 2(5), 3(5),4(5), 5(5)$ modulo $11.$ 
						These are \[\answer{5,10,4,9,2},\] 
			
						
						and $n=\answer{2}$ are greater than $\frac{11}{2}.$ Thus, by \nameref{lem:gauss}, \[\legendre{5}{11}=(-1)^{\answer{2}}=\answer{2}.\qedhere\]
					\end{solution}
			\end{enumerate}


\end{enumerate}
\end{br}


\begin{br}
	Now we will examine the Legendre symbol of $2$ using Gauss's Lemma. First, note that $2,2(2),3(2),\dots,2(\frac{p-1}{2})$ are already least nonnegative residues modulo $p.$ It will be slightly easier to count how many are \emph{less than} $\frac{p}{2},$ then subtract from the total number, $\frac{p-1}{2}.$

	Let $k\in\Z$ with $1\leq k\leq \frac{p-1}{2}.$ Then $2k< \frac{p}{2}$ if and only if $k <\answer{\frac{p}{4}}.$ Thus, $\frac{p-1}{2}-\lfloor\answer{\frac{p}{4}}\rfloor$ of $2,2(2),3(2),\dots,2(\frac{p-1}{2})$ are greater than $\frac{p}{2}.$ (\emph{Hint:} The two blanks should be the same, and also go in the blanks in the table headers)

	Now complete this table

	\begin{tabular}{c|cclc}
		$p$ & $\lfloor\answer{\frac{p}{4}}\rfloor$ & $\frac{p-1}{2}-\lfloor\answer{\frac{p}{4}}\rfloor$ & $2,2(2),3(2),\dots,2(\frac{p-1}{2})$ & $\legendre{2}{p}$\\\hline
		$3$ & $\answer{0}$	& $\answer{1}$
			& \makecell{Less than $\tfrac{3}{2}:$ $\answer{N/A}$
			\\Greater than $\tfrac{3}{2}:$ $\answer{2}$}
			& $\answer{(-1)^1=-1}$\\\hline
		$5$ & $\answer{1}$	& $\answer{1}$
			& \makecell{Less than $\tfrac{5}{2}:$ $\answer{2}$
			\\Greater than $\tfrac{5}{2}:$ $\answer{4}$}
			& $\answer{(-1)^1=-1}$\\\hline
		$7$ & $\answer{1}$	& $\answer{2}$ 
			& \makecell{Less than $\tfrac{7}{2}:$ $\answer{2}$
			\\Greater than $\tfrac{7}{2}:$ $\answer{4,6}$}
			& $\answer{(-1)^2=1}$\\\hline
		$11$ & $\answer{2}$	& $\answer{3}$
			& \makecell{Less than $\tfrac{11}{2}:$ $\answer{2,4}$
			\\Greater than $\tfrac{11}{2}:$ $\answer{6,8,10}$}
			& $\answer{(-1)^3=-1}$\\\hline
		$13$ & $\answer{3}$	& $\answer{3}$
			& \makecell{Less than $\tfrac{13}{2}:$ $\answer{2,4,6}$
			\\Greater than $\tfrac{13}{2}:$ $\answer{8,10,12}$}
			& $\answer{(-1)^3=-1}$\\\hline
		$17$ & $\answer{4}$	& $\answer{4}$
			& \makecell{Less than $\tfrac{17}{2}:$ $\answer{2,4,6,8}$
			\\Greater than $\tfrac{17}{2}:$ $\answer{10,12,14,16}$}
			& $\answer{(-1)^4=1}$\\\hline
		$19$ & $\answer{4}$	& $\answer{5}$
			& \makecell{Less than $\tfrac{19}{2}:$ $\answer{2,4,6,8}$
		\\Greater than $\tfrac{17}{2}:$ $\answer{10,12,14,16,18}$}
		& $\answer{(-1)^5=-1}$\\\hline
		$23$ & $\answer{5}$	& $\answer{6}$
			& \makecell{Less than $\tfrac{17}{2}:$ $\answer{2,4,6,8,10}$
			\\Greater than $\tfrac{17}{2}:$ $\answer{12,14,16,18,20,22}$}
			& $\answer{(-1)^6=1}$\\\hline
	\end{tabular}
\end{br}

\section{Wednesday, April 3: Proving Gauss's Lemma and the Quadratic Residue of $2$}
\end{document}