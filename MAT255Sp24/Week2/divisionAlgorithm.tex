\documentclass{ximera}
\usepackage{amssymb, latexsym, amsmath, amsthm, graphicx, amsthm,alltt,color, listings,multicol,xr-hyper,hyperref,aliascnt,enumitem}
\usepackage{xfrac}

\usepackage{parskip}
\usepackage[,margin=0.7in]{geometry}
\setlength{\textheight}{8.5in}

\usepackage{epstopdf}

\DeclareGraphicsExtensions{.eps}
\usepackage{tikz}


\usepackage{tkz-euclide}
%\usetkzobj{all}
\tikzstyle geometryDiagrams=[rounded corners=.5pt,ultra thick,color=black]
\colorlet{penColor}{black} % Color of a curve in a plot


\usepackage{subcaption}
\usepackage{float}
\usepackage{fancyhdr}
\usepackage{pdfpages}
\newcounter{includepdfpage}
\usepackage{makecell}


\usepackage{currfile}
\usepackage{xstring}




\graphicspath{  
{./otherDocuments/}
}

\author{Claire Merriman}
\newcommand{\classday}[1]{\def\classday{#1}}

%%%%%%%%%%%%%%%%%%%%%
% Counters and autoref for unnumbered environments
% Not needed??
%%%%%%%%%%%%%%%%%%%%%
\theoremstyle{plain}


\newtheorem*{namedthm}{Theorem}
\newcounter{thm}%makes pointer correct
\providecommand{\thmname}{Theorem}

\makeatletter
\NewDocumentEnvironment{thm*}{o}
 {%
  \IfValueTF{#1}
    {\namedthm[#1]\refstepcounter{thm}\def\@currentlabel{(#1)}}%
    {\namedthm}%
 }
 {%
  \endnamedthm
 }
\makeatother


\newtheorem*{namedprop}{Proposition}
\newcounter{prop}%makes pointer correct
\providecommand{\propname}{Proposition}

\makeatletter
\NewDocumentEnvironment{prop*}{o}
 {%
  \IfValueTF{#1}
    {\namedprop[#1]\refstepcounter{prop}\def\@currentlabel{(#1)}}%
    {\namedprop}%
 }
 {%
  \endnamedprop
 }
\makeatother

\newtheorem*{namedlem}{Lemma}
\newcounter{lem}%makes pointer correct
\providecommand{\lemname}{Lemma}

\makeatletter
\NewDocumentEnvironment{lem*}{o}
 {%
  \IfValueTF{#1}
    {\namedlem[#1]\refstepcounter{lem}\def\@currentlabel{(#1)}}%
    {\namedlem}%
 }
 {%
  \endnamedlem
 }
\makeatother

\newtheorem*{namedcor}{Corollary}
\newcounter{cor}%makes pointer correct
\providecommand{\corname}{Corollary}

\makeatletter
\NewDocumentEnvironment{cor*}{o}
 {%
  \IfValueTF{#1}
    {\namedcor[#1]\refstepcounter{cor}\def\@currentlabel{(#1)}}%
    {\namedcor}%
 }
 {%
  \endnamedcor
 }
\makeatother

\theoremstyle{definition}
\newtheorem*{annotation}{Annotation}
\newtheorem*{rubric}{Rubric}

\newtheorem*{innerrem}{Remark}
\newcounter{rem}%makes pointer correct
\providecommand{\remname}{Remark}

\makeatletter
\NewDocumentEnvironment{rem}{o}
 {%
  \IfValueTF{#1}
    {\innerrem[#1]\refstepcounter{rem}\def\@currentlabel{(#1)}}%
    {\innerrem}%
 }
 {%
  \endinnerrem
 }
\makeatother

\newtheorem*{innerdefn}{Definition}%%placeholder
\newcounter{defn}%makes pointer correct
\providecommand{\defnname}{Definition}

\makeatletter
\NewDocumentEnvironment{defn}{o}
 {%
  \IfValueTF{#1}
    {\innerdefn[#1]\refstepcounter{defn}\def\@currentlabel{(#1)}}%
    {\innerdefn}%
 }
 {%
  \endinnerdefn
 }
\makeatother

\newtheorem*{scratch}{Scratch Work}


\newtheorem*{namedconj}{Conjecture}
\newcounter{conj}%makes pointer correct
\providecommand{\conjname}{Conjecture}
\makeatletter
\NewDocumentEnvironment{conj}{o}
 {%
  \IfValueTF{#1}
    {\innerconj[#1]\refstepcounter{conj}\def\@currentlabel{(#1)}}%
    {\innerconj}%
 }
 {%
  \endinnerconj
 }
\makeatother

\newtheorem*{poll}{Poll question}
\newtheorem{tps}{Think-Pair-Share}[section]


\newenvironment{obj}{
	\textbf{Learning Objectives.} By the end of class, students will be able to:
		\begin{itemize}}
		{\!.\end{itemize}
		}

\newenvironment{pre}{
	\begin{description}
	}{
	\end{description}
}


\newcounter{ex}%makes pointer correct
\providecommand{\exname}{Homework Problem}
\newenvironment{ex}[1][2in]%
{%Env start code
\problemEnvironmentStart{#1}{Homework Problem}
\refstepcounter{ex}
}
{%Env end code
\problemEnvironmentEnd
}

\newcommand{\inlineAnswer}[2][2 cm]{
    \ifhandout{\pdfOnly{\rule{#1}{0.4pt}}}
    \else{\answer{#2}}
    \fi
}


\ifhandout
\newenvironment{shortAnswer}[1][
    \vfill]
        {% Begin then result
        #1
            \begin{freeResponse}
            }
    {% Environment Ending Code
    \end{freeResponse}
    }
\else
\newenvironment{shortAnswer}[1][]
        {\begin{freeResponse}
            }
    {% Environment Ending Code
    \end{freeResponse}
    }
\fi

\let\question\relax
\let\endquestion\relax

\newtheoremstyle{ExerciseStyle}{\topsep}{\topsep}%%% space between body and thm
		{}                      %%% Thm body font
		{}                              %%% Indent amount (empty = no indent)
		{\bfseries}            %%% Thm head font
		{}                              %%% Punctuation after thm head
		{3em}                           %%% Space after thm head
		{{#1}~\thmnumber{#2}\thmnote{ \bfseries(#3)}}%%% Thm head spec
\theoremstyle{ExerciseStyle}
\newtheorem{br}{In-class Problem}

\newenvironment{sketch}
 {\begin{proof}[Sketch of Proof]}
 {\end{proof}}


\newcommand{\gt}{>}
\newcommand{\lt}{<}
\newcommand{\N}{\mathbb N}
\newcommand{\Q}{\mathbb Q}
\newcommand{\Z}{\mathbb Z}
\newcommand{\C}{\mathbb C}
\newcommand{\R}{\mathbb R}
\renewcommand{\H}{\mathbb{H}}
\newcommand{\lcm}{\operatorname{lcm}}
\newcommand{\nequiv}{\not\equiv}
\newcommand{\ord}{\operatorname{ord}}
\newcommand{\ds}{\displaystyle}
\newcommand{\floor}[1]{\left\lfloor #1\right\rfloor}
\newcommand{\legendre}[2]{\left(\frac{#1}{#2}\right)}



%%%%%%%%%%%%



\title{The Division Algorithm}
\begin{document}
\begin{abstract}
\end{abstract}
\maketitle

%%%%%%%%%%%%%%%%%%%%%%%%%%
%%%%%%%%%%%%%%%%%%%%%%%%%%

\begin{obj}
  \item Prove existence and uniqueness for the Division Algorithm
  \item Prove existence and uniqueness for the general Division Algorithm
\end{obj}

\begin{pre}
  \item[Read] Ernst \href{https://danaernst.com/IBL-IntroToProof/pretext/sec_Intro_to_Logic.html}{Section 2.2} and \href{https://danaernst.com/IBL-IntroToProof/pretext/sec_Introduction_to_Quantification.html}{Section 2.4}
 
  \item[Turn in] Ernst, Problem 2.59 and 2.64
\end{pre}

\begin{instructorNotes}
  Go over reading assignment at the start of class.
\end{instructorNotes}


This section introduces the division algorithm, which will come up repeatedly throughout the semester, as well as the definition of divisors from last class.

First, let's define a \emph{lemma.} A lemma is a minor result whose sole purpose is to help in proving a theorem, although some famous named lemmas have become important results in their own right.

\begin{defn}[greatest integer (floor) function]\label{defn:floor}
  Let $x\in\R$. The \emph{greatest integer function of $x$,} denoted $[x]$ or $\floor{x}$, is the greatest integer less than or equal to $x$.
\end{defn}

\begin{lemma}\label{lem:floor-inter}
  Let $x\in\R$. Then $x-1<[x]\leq x$.
  
  \begin{proof}
    By the definition of the floor function, $[x]\leq x$. 

    To prove that $x-1<[x],$ we proceed by contradiction. Assume that $x-1\geq [x]$ (the negation of $x-1<[x]$). Then, $x\geq [x]+1$. This contradicts the assumption that $[x]$ is the greatest integer \emph{less than or equal to} $x$. Thus, $x-1<[x].$
  \end{proof}
\end{lemma}

\begin{thm*}[Division Algorithm]\label{div-alg}
 Let $a,b\in\Z$ with $b>0$. Then there exists a unique $q,r\in\Z$ such that \[a=bq+r, \quad 0\leq r <b.\]
\end{thm*}

Before proving this theorem, let's think about division with remainders, ie long division. The quotient $q$ should be the largest integer such that $bq\leq a$. If we divide both sides by $b$, we have $q\leq\frac{a}{b}$. We have a function to find the greatest integer less than or equal to $\frac{a}{b},$ namely $q=\floor{\frac{a}{b}}$. If we rearrange the equation $a=bq+r,$ we gave $r=a-bq$. This is our scratch work for existence.

\begin{proof} Let $a,b\in\Z$ with $b>0$.
Define $q=\floor{\frac{a}{b}}$ and $r=a-b\floor{\frac{a}{b}}$. Then $a=bq+r$ by rearranging the equation. 
Now we need to show $0\leq r<b$. 

Since $x-1<\floor{ x}\leq x$ by Lemma \ref{lem:floor-inter}, we have \[\frac{a}{b}-1<\floor{\frac{a}{b}}\leq\frac{a}{b}.\] 
Multiplying all terms by $-b$, we get 
 \[-a+b>-b\floor{\frac{a}{b}}\geq-a.\]
 Adding $a$ to every term gives \[b>a-b\floor{\frac{a}{b}}\geq 0.\] 
By the definition of $r$, we have shown $0\leq r <b$.

Finally, we need to show that $q$ and $r$ are unique.
Assume there exist $q_1,q_2,r_1,r_2\in\Z$ with \[a=bq_1+r_1, \quad 0\leq r_1<b\]
 \[a=bq_2+r_2, \quad 0\leq r_2<b.\]
 We need to show $q_1=q_2$ and $r_1=r_2$. We can subtract the two equations from each other. 
 
\begin{align*}
  a&=bq_1+r_1, \\
\underline{ -(a}&\underline{=bq_2+r_2)}, \\
 0&=bq_1+r_1-bq_2-r_2=b(q_1-q_2)+(r_1-r_2) . 
\end{align*}

Rearranging, we get $b(q_1-q_2)=r_2-r_1$. Thus, $b\mid r_2-r_1$. From rearranging the inequalities:
\begin{align*}
 & 0\leq r_2<b\\
- & \underline{b< -r_1\leq 0}\\
 -&b<r_2-r_1<b.
\end{align*}
Thus, the only way $b\mid r_2-r_1$ is that $r_2-r_1=0$ and thus $r_1=r_2$. Now, $0=b(q_1-q_2)+(r_1-r_2)$ becomes $0=b(q_1-q_2)$. Since we assumed $b>0$, we have that $q_1-q_2=0$. 
\end{proof}



\begin{br}
 Use the \nameref{div-alg} on $a=47, b=6$ and $a=281, b=13$.
\end{br}
\begin{solution}
For $a=47, b=6$, we have that $a=(7)6+5, q=7, r=5$.
For $a=281, b=13$, we have that $a=(21)13+8, q=21, r=8$.
\end{solution}

\begin{corollary}
 Let $a,b\in\Z$ with $b\neq0$. Then there exists a unique $q,r\in\Z$ such that \[a=bq+r, \quad 0\leq r <|b|.\]
\end{corollary}
One proof method is using an existing proof as a guide.

\begin{br} Let $a$ and $b$ be nonzero integers. Prove that there exists a unique $q,r\in\Z$ such that 
  \[a=bq+r, \quad 0\leq r <|b|.\]
  \begin{enumerate}
    	\item Use the \nameref{div-alg} to prove this statement as a corollary. That is, use the \emph{conclusion} of the \nameref{div-alg} as part of the proof.  Use the following outline:
    	\begin{enumerate}
		\item  Let $a$ and $b$ be nonzero integers. Since $|b|>0$, the \nameref{div-alg} says that there exist unique $p,s\in\Z$ such that 
		\begin{prompt}
			$\answer{a=p|b|+s}$
		\end{prompt} and 
		\begin{prompt}
			$\answer{0\leq s<|b|}$.
		\end{prompt}
      		\item There are two cases:
      		\begin{enumerate}
        			\item When 
					\begin{prompt}
						$\answer{b>0}$
					\end{prompt}, the conditions are already met and 
					\begin{prompt}
						$r=\answer{s}$ and $q=answer{b}$.
					\end{prompt}
        			\item Otherwise, 
					\begin{prompt}
						$\answer{b<0}$
					\end{prompt}, \begin{prompt}
						$r=\answer{s}$ and $q=answer{-b}$.
					\end{prompt}.
      		\end{enumerate}
      		\item Since both cases used that the $p,s$ are unique, then $q,r$ are also unique
	\end{enumerate}
    	\item Use the \emph{proof} of the \nameref{div-alg} as a template to prove this statement. That is, repeat the steps, adjusting as necessary, but do not use the conclusion.
    	\begin{enumerate}
    		\item In the proof of the \nameref{div-alg}, we let $q=\floor{\frac{a}{b}}$. Here we have two cases:
    		\begin{enumerate}
      			\item When 
				\begin{prompt}
					$\answer{b>0}$
				\end{prompt}, 
				\begin{prompt}
					$q=\answer{\floor{\frac{a}{b}}}$ and $r=\answer{a-bq}.$
				\end{prompt}

      			\item When 
				\begin{prompt}
					$\answer{b<0}$
				\end{prompt}, 
				\begin{prompt}
					$q=\answer{-\floor{\frac{a}{b}}}$ and $r=\answer{a-bq}.$
				\end{prompt}
		\end{enumerate}
    		\item Follow the steps of the \emph{proof} of the \nameref{div-alg} to finish the proof.
    	\end{enumerate}

\end{enumerate}


\end{br}
%%%%%%%%%%%%%%%%%%%%%%%%%%
%%%%%%%%%%%%%%%%%%%%%%%%%%


\end{document}
