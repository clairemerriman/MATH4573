\documentclass{ximera}
\usepackage{amssymb, latexsym, amsmath, amsthm, graphicx, amsthm,alltt,color, listings,multicol,xr-hyper,hyperref,aliascnt,enumitem}
\usepackage{xfrac}

\usepackage{parskip}
\usepackage[,margin=0.7in]{geometry}
\setlength{\textheight}{8.5in}

\usepackage{epstopdf}

\DeclareGraphicsExtensions{.eps}
\usepackage{tikz}


\usepackage{tkz-euclide}
%\usetkzobj{all}
\tikzstyle geometryDiagrams=[rounded corners=.5pt,ultra thick,color=black]
\colorlet{penColor}{black} % Color of a curve in a plot


\usepackage{subcaption}
\usepackage{float}
\usepackage{fancyhdr}
\usepackage{pdfpages}
\newcounter{includepdfpage}
\usepackage{makecell}


\usepackage{currfile}
\usepackage{xstring}




\graphicspath{  
{./otherDocuments/}
}

\author{Claire Merriman}
\newcommand{\classday}[1]{\def\classday{#1}}

%%%%%%%%%%%%%%%%%%%%%
% Counters and autoref for unnumbered environments
% Not needed??
%%%%%%%%%%%%%%%%%%%%%
\theoremstyle{plain}


\newtheorem*{namedthm}{Theorem}
\newcounter{thm}%makes pointer correct
\providecommand{\thmname}{Theorem}

\makeatletter
\NewDocumentEnvironment{thm*}{o}
 {%
  \IfValueTF{#1}
    {\namedthm[#1]\refstepcounter{thm}\def\@currentlabel{(#1)}}%
    {\namedthm}%
 }
 {%
  \endnamedthm
 }
\makeatother


\newtheorem*{namedprop}{Proposition}
\newcounter{prop}%makes pointer correct
\providecommand{\propname}{Proposition}

\makeatletter
\NewDocumentEnvironment{prop*}{o}
 {%
  \IfValueTF{#1}
    {\namedprop[#1]\refstepcounter{prop}\def\@currentlabel{(#1)}}%
    {\namedprop}%
 }
 {%
  \endnamedprop
 }
\makeatother

\newtheorem*{namedlem}{Lemma}
\newcounter{lem}%makes pointer correct
\providecommand{\lemname}{Lemma}

\makeatletter
\NewDocumentEnvironment{lem*}{o}
 {%
  \IfValueTF{#1}
    {\namedlem[#1]\refstepcounter{lem}\def\@currentlabel{(#1)}}%
    {\namedlem}%
 }
 {%
  \endnamedlem
 }
\makeatother

\newtheorem*{namedcor}{Corollary}
\newcounter{cor}%makes pointer correct
\providecommand{\corname}{Corollary}

\makeatletter
\NewDocumentEnvironment{cor*}{o}
 {%
  \IfValueTF{#1}
    {\namedcor[#1]\refstepcounter{cor}\def\@currentlabel{(#1)}}%
    {\namedcor}%
 }
 {%
  \endnamedcor
 }
\makeatother

\theoremstyle{definition}
\newtheorem*{annotation}{Annotation}
\newtheorem*{rubric}{Rubric}

\newtheorem*{innerrem}{Remark}
\newcounter{rem}%makes pointer correct
\providecommand{\remname}{Remark}

\makeatletter
\NewDocumentEnvironment{rem}{o}
 {%
  \IfValueTF{#1}
    {\innerrem[#1]\refstepcounter{rem}\def\@currentlabel{(#1)}}%
    {\innerrem}%
 }
 {%
  \endinnerrem
 }
\makeatother

\newtheorem*{innerdefn}{Definition}%%placeholder
\newcounter{defn}%makes pointer correct
\providecommand{\defnname}{Definition}

\makeatletter
\NewDocumentEnvironment{defn}{o}
 {%
  \IfValueTF{#1}
    {\innerdefn[#1]\refstepcounter{defn}\def\@currentlabel{(#1)}}%
    {\innerdefn}%
 }
 {%
  \endinnerdefn
 }
\makeatother

\newtheorem*{scratch}{Scratch Work}


\newtheorem*{namedconj}{Conjecture}
\newcounter{conj}%makes pointer correct
\providecommand{\conjname}{Conjecture}
\makeatletter
\NewDocumentEnvironment{conj}{o}
 {%
  \IfValueTF{#1}
    {\innerconj[#1]\refstepcounter{conj}\def\@currentlabel{(#1)}}%
    {\innerconj}%
 }
 {%
  \endinnerconj
 }
\makeatother

\newtheorem*{poll}{Poll question}
\newtheorem{tps}{Think-Pair-Share}[section]


\newenvironment{obj}{
	\textbf{Learning Objectives.} By the end of class, students will be able to:
		\begin{itemize}}
		{\!.\end{itemize}
		}

\newenvironment{pre}{
	\begin{description}
	}{
	\end{description}
}


\newcounter{ex}%makes pointer correct
\providecommand{\exname}{Homework Problem}
\newenvironment{ex}[1][2in]%
{%Env start code
\problemEnvironmentStart{#1}{Homework Problem}
\refstepcounter{ex}
}
{%Env end code
\problemEnvironmentEnd
}

\newcommand{\inlineAnswer}[2][2 cm]{
    \ifhandout{\pdfOnly{\rule{#1}{0.4pt}}}
    \else{\answer{#2}}
    \fi
}


\ifhandout
\newenvironment{shortAnswer}[1][
    \vfill]
        {% Begin then result
        #1
            \begin{freeResponse}
            }
    {% Environment Ending Code
    \end{freeResponse}
    }
\else
\newenvironment{shortAnswer}[1][]
        {\begin{freeResponse}
            }
    {% Environment Ending Code
    \end{freeResponse}
    }
\fi

\let\question\relax
\let\endquestion\relax

\newtheoremstyle{ExerciseStyle}{\topsep}{\topsep}%%% space between body and thm
		{}                      %%% Thm body font
		{}                              %%% Indent amount (empty = no indent)
		{\bfseries}            %%% Thm head font
		{}                              %%% Punctuation after thm head
		{3em}                           %%% Space after thm head
		{{#1}~\thmnumber{#2}\thmnote{ \bfseries(#3)}}%%% Thm head spec
\theoremstyle{ExerciseStyle}
\newtheorem{br}{In-class Problem}

\newenvironment{sketch}
 {\begin{proof}[Sketch of Proof]}
 {\end{proof}}


\newcommand{\gt}{>}
\newcommand{\lt}{<}
\newcommand{\N}{\mathbb N}
\newcommand{\Q}{\mathbb Q}
\newcommand{\Z}{\mathbb Z}
\newcommand{\C}{\mathbb C}
\newcommand{\R}{\mathbb R}
\renewcommand{\H}{\mathbb{H}}
\newcommand{\lcm}{\operatorname{lcm}}
\newcommand{\nequiv}{\not\equiv}
\newcommand{\ord}{\operatorname{ord}}
\newcommand{\ds}{\displaystyle}
\newcommand{\floor}[1]{\left\lfloor #1\right\rfloor}
\newcommand{\legendre}[2]{\left(\frac{#1}{#2}\right)}



%%%%%%%%%%%%



\title{Primes}
\begin{document}
\begin{abstract}
\end{abstract}
\maketitle

%%%%%%%%%%%%%%%%%%%%%%%%%%
%%%%%%%%%%%%%%%%%%%%%%%%%%

\begin{obj}
\item  Prove every integer greater than 1 has a prime divisor.
\item  Prove that there are infinitely many prime numbers
\end{obj}


\begin{instructorNotes}

\begin{pre}
 \item[Read] Strayer, Section 1.2
 \item[Turn in] 
\begin{itemize}
 \item The proof method for Euclid's infinitude of primes is an important method. Summarize this method in your own words.
 
\begin{solution}
 Summaries will vary
\end{solution}
 \item Identify any other new proof methods in this section
 
\begin{solution}
 Proof by construction may be new to some students. Students also identified: 
\begin{itemize}
 \item Introducing a variable to aid in proof
 \item Without loss of generality
 
\end{itemize}
\end{solution}
 \item Exercise 22. Prove that 2 is the only even prime number.
 
\begin{solution}
 Assume that there exists another even prime number, call it $p$. Then there exists $2\mid p$ by the definition of even, but that implies that $p=2$ by the definition of prime. Thus, $2$ is the only even prime number.
\end{solution}
\end{itemize}
\end{pre}
  
\end{instructorNotes}
%%%%%%%%%%%%%%%%%%%%%%%%%



\begin{defn}[prime and composite]
An integer $p>1$ is \emph{prime} if the only positive divisors of $p$ are $1$ and itself. An integer $n$ which is not prime is \emph{composite}. 
\end{defn}

Why is $1$ not prime?

%It has to do with units and invertibility. The number $1$ holds a special place. It is the multiplicative identity, i.e., anything multiplied by $1$ is just that thing again. Something is said to be invertible in a ``group" (more on that later) if there exist something, which, when multiplied to it, gives you $1$. How many invertible elements are there among the integers? Just two. $1$ and $-1$. And that's the key. If we want to extend our results from positive integers to non-zero integers, we often just need to take into account $\pm 1$. That sounds obvious, but it turns out to be surprisingly critical and yet non-intuitive when we start moving from real integers to complex ones.

\begin{lemma}\label{lem:prime-composite}
 Every integer greater than 1 has a prime divisor.
\end{lemma}

We will not go over this proof in class.

\begin{proof} 
 Assume by contradiction that there exists $n\in\Z$ greater than 1 with no prime divisor. By the \nameref{well-order}, we may assume $n$ is the least such integer. By definition, $n\mid n$, so $n$ is not prime. Thus, $n$ is composite and there exists $a,b\in\Z$ such that $n=ab$ and $1<a<n$, $1<b<n$. Since $a<n$, then it has a prime divisor $p$. But since $p\mid a$ and $p\mid n$, $p\mid n$. This contradicts our assumption, so no such integer exists.
\end{proof}


\begin{theorem}[Euclid's Infinitude of Primes]\label{thm:inf-primes}(Theorem 1.6)
 There are infinitely many prime numbers.
\end{theorem}
\begin{proof}
 Assume by way of contradiction, that there are only finitely many prime numbers, so $p_1,p_2,\dots,p_n$. Consider the number $N=p_1p_2\cdots p_n +1$. Now $N$ has a prime divisor, say, $p$, by \nameref{lem:prime-composite}. So $p=p_i$ for some $i$, $i=1,2,\dots,n$. Then $p\mid N-p_1p_2\dots p_n$, which implies that $p\mid 1$, a contradiction. Hence, there are infinitely many prime numbers.
\end{proof}

One famous open problem is the Twin Primes Conjecture. A \emph{conjecture} is a proposition you (or in this case, the mathematical community) believe to be true, but have not proven.

\begin{conjecture}[Twin Prime Conjecture]\label{conj:twin-primes}
  There are infinitely many prime number $p$ for which $p+2$ is also prime number.
\end{conjecture}

Another important fact is there are arbitrarily large sequences of composite numbers. Put another way, there are arbitrarily large gaps in the primes. Another important proof method, which is a \emph{constructive proof}:

\begin{theorem}\label{prop:gaps-primes}
 For any positive integer $n$, there are at least $n$ consecutive positive integers.
\end{theorem}
\begin{proof}
 Given the positive integer $n$, consider the $n$ consecutive positive integers \[(n+1)!+2, (n+1)!+3,\dots, (n+1)!+n+1.\]
 Let $i$ be a positive integer such that $2\leq i\leq n+1$. Since $i\mid (n+1)!$ and $i\mid i$, we have \[i\mid(n+1)! +i,\quad 2\leq i\leq n+1\] by linear combination (\nameref{lem:linear-combo}). So each of the $n$ consecutive positive integers is composite.
\end{proof}

  \begin{br} Let $n$ be a positive integer with $n\neq 1$. Prove that if $n^2+1$ is prime, then $n^2+1$ can be written in the form $4k+1$ with $k\in\Z$.
  \begin{solution}
   Assume that $n$ is a positive integer, $n\neq 1,$ and $n^2+1$ is prime. If $n$ is odd, then $n^2$ is odd, which would imply $n^2+1=2,$ the only even prime. However, $n\neq 1$ by assumption. Thus, $n$ is even. 

    By definition of even, there exists $j\in\Z$ such that $n=2k$ and $n^2=4j^2$. Thus, $n^2+1=4k+1$ when $k=j^2.$
  \end{solution}
\end{br}

  \begin{br}
    Prove or disprove the following conjecture, which is similar to \nameref{conj:twin-primes}:

    \begin{conjecture}
      There are infinitely many prime number $p$ for which $p+2$ and $p+4$ are also prime numbers.
    \end{conjecture}
    \end{br}



%%%%%%%%%%%%%%%%%%%%%%%%%%


\end{document}
