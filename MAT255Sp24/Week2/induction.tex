\documentclass{../ximera}
<<<<<<< Updated upstream
\usepackage{amssymb, latexsym, amsmath, amsthm, graphicx, amsthm,alltt,color, listings,multicol,xr-hyper,hyperref,aliascnt,enumitem}
=======
\usepackage{amssymb, latexsym, amsmath, amsthm, graphicx, amsthm,alltt,color, listings,multicol,hyperref}
\usepackage[capitalise,nameinlink]{cleveref}
>>>>>>> Stashed changes
\usepackage{xfrac}

\usepackage{parskip}
\usepackage[,margin=0.7in]{geometry}
\setlength{\textheight}{8.5in}

\usepackage{epstopdf}

\DeclareGraphicsExtensions{.eps}
\usepackage{tikz}


\usepackage{tkz-euclide}
%\usetkzobj{all}
\tikzstyle geometryDiagrams=[rounded corners=.5pt,ultra thick,color=black]
\colorlet{penColor}{black} % Color of a curve in a plot


\usepackage{subcaption}
\usepackage{float}
\usepackage{fancyhdr}
\usepackage{pdfpages}
\newcounter{includepdfpage}
\usepackage{makecell}


\usepackage{currfile}
\usepackage{xstring}




\graphicspath{  
{./otherDocuments/}
}

\author{Claire Merriman}
\newcommand{\classday}[1]{\def\classday{#1}}

%%%%%%%%%%%%%%%%%%%%%
% Counters and autoref for unnumbered environments
% Not needed??
%%%%%%%%%%%%%%%%%%%%%
<<<<<<< Updated upstream
\theoremstyle{plain}


\newtheorem*{namedthm}{Theorem}
\newcounter{thm}%makes pointer correct
\providecommand{\thmname}{Theorem}
=======

\crefname{problem}{problem}{problems}


% \theoremstyle{plain}


% \newtheorem*{namedthm}{Theorem}
% \newcounter{thm}%makes pointer correct
% \providecommand{\thmname}{Theorem}
>>>>>>> Stashed changes

\makeatletter
\NewDocumentEnvironment{thm*}{o}
 {%
  \IfValueTF{#1}
    {\namedthm[#1]\refstepcounter{thm}\def\@currentlabel{(#1)}}%
    {\namedthm}%
 }
 {%
  \endnamedthm
 }
\makeatother


\newtheorem*{namedprop}{Proposition}
\newcounter{prop}%makes pointer correct
\providecommand{\propname}{Proposition}

\makeatletter
\NewDocumentEnvironment{prop*}{o}
 {%
  \IfValueTF{#1}
    {\namedprop[#1]\refstepcounter{prop}\def\@currentlabel{(#1)}}%
    {\namedprop}%
 }
 {%
  \endnamedprop
 }
\makeatother

\newtheorem*{namedlem}{Lemma}
\newcounter{lem}%makes pointer correct
\providecommand{\lemname}{Lemma}

\makeatletter
\NewDocumentEnvironment{lem*}{o}
 {%
  \IfValueTF{#1}
    {\namedlem[#1]\refstepcounter{lem}\def\@currentlabel{(#1)}}%
    {\namedlem}%
 }
 {%
  \endnamedlem
 }
\makeatother

\newtheorem*{namedcor}{Corollary}
\newcounter{cor}%makes pointer correct
\providecommand{\corname}{Corollary}

\makeatletter
\NewDocumentEnvironment{cor*}{o}
 {%
  \IfValueTF{#1}
    {\namedcor[#1]\refstepcounter{cor}\def\@currentlabel{(#1)}}%
    {\namedcor}%
 }
 {%
  \endnamedcor
 }
\makeatother

\theoremstyle{definition}
\newtheorem*{annotation}{Annotation}
\newtheorem*{rubric}{Rubric}

\newtheorem*{innerrem}{Remark}
\newcounter{rem}%makes pointer correct
\providecommand{\remname}{Remark}

\makeatletter
\NewDocumentEnvironment{rem}{o}
 {%
  \IfValueTF{#1}
    {\innerrem[#1]\refstepcounter{rem}\def\@currentlabel{(#1)}}%
    {\innerrem}%
 }
 {%
  \endinnerrem
 }
\makeatother

\newtheorem*{innerdefn}{Definition}%%placeholder
\newcounter{defn}%makes pointer correct
\providecommand{\defnname}{Definition}

\makeatletter
\NewDocumentEnvironment{defn}{o}
 {%
  \IfValueTF{#1}
    {\innerdefn[#1]\refstepcounter{defn}\def\@currentlabel{(#1)}}%
    {\innerdefn}%
 }
 {%
  \endinnerdefn
 }
\makeatother

\newtheorem*{scratch}{Scratch Work}


\newtheorem*{namedconj}{Conjecture}
\newcounter{conj}%makes pointer correct
\providecommand{\conjname}{Conjecture}
\makeatletter
\NewDocumentEnvironment{conj}{o}
 {%
  \IfValueTF{#1}
    {\innerconj[#1]\refstepcounter{conj}\def\@currentlabel{(#1)}}%
    {\innerconj}%
 }
 {%
  \endinnerconj
 }
\makeatother

\newtheorem*{poll}{Poll question}
\newtheorem{tps}{Think-Pair-Share}[section]


\newenvironment{obj}{
	\textbf{Learning Objectives.} By the end of class, students will be able to:
		\begin{itemize}}
		{\!.\end{itemize}
		}

<<<<<<< Updated upstream
\newenvironment{pre}{
	\begin{description}
	}{
	\end{description}
}
=======

\ifinstructornotes
\newenvironment{pre}
  {{\textbf Reading assignment:}
  \begin{description}
    }{
	\end{description}
  }
\else
\newenvironment{pre}{ 
  \begin{trivlist}
  \item[]}
  {\end{trivlist}}
\fi
>>>>>>> Stashed changes


\newcounter{ex}%makes pointer correct
\providecommand{\exname}{Homework Problem}
\newenvironment{ex}[1][2in]%
{%Env start code
\problemEnvironmentStart{#1}{Homework Problem}
\refstepcounter{ex}
}
{%Env end code
\problemEnvironmentEnd
}

\newcommand{\inlineAnswer}[2][2 cm]{
    \ifhandout{\pdfOnly{\rule{#1}{0.4pt}}}
    \else{\answer{#2}}
    \fi
}


\ifhandout
\newenvironment{shortAnswer}[1][
    \vfill]
        {% Begin then result
        #1
            \begin{freeResponse}
            }
    {% Environment Ending Code
    \end{freeResponse}
    }
\else
\newenvironment{shortAnswer}[1][]
        {\begin{freeResponse}
            }
    {% Environment Ending Code
    \end{freeResponse}
    }
\fi

\let\question\relax
\let\endquestion\relax

\newtheoremstyle{ExerciseStyle}{\topsep}{\topsep}%%% space between body and thm
		{}                      %%% Thm body font
		{}                              %%% Indent amount (empty = no indent)
		{\bfseries}            %%% Thm head font
		{}                              %%% Punctuation after thm head
		{3em}                           %%% Space after thm head
		{{#1}~\thmnumber{#2}\thmnote{ \bfseries(#3)}}%%% Thm head spec
\theoremstyle{ExerciseStyle}
\newtheorem{br}{In-class Problem}

\newenvironment{sketch}
 {\begin{proof}[Sketch of Proof]}
 {\end{proof}}


\newcommand{\gt}{>}
\newcommand{\lt}{<}
\newcommand{\N}{\mathbb N}
\newcommand{\Q}{\mathbb Q}
\newcommand{\Z}{\mathbb Z}
\newcommand{\C}{\mathbb C}
\newcommand{\R}{\mathbb R}
\renewcommand{\H}{\mathbb{H}}
\newcommand{\lcm}{\operatorname{lcm}}
\newcommand{\nequiv}{\not\equiv}
\newcommand{\ord}{\operatorname{ord}}
\newcommand{\ds}{\displaystyle}
\newcommand{\floor}[1]{\left\lfloor #1\right\rfloor}
\newcommand{\legendre}[2]{\left(\frac{#1}{#2}\right)}



%%%%%%%%%%%%



\title{Induction}
\begin{document}
\begin{abstract}
\end{abstract}
\maketitle

%%%%%%%%%%%%%%%%%%%%%%%%%%
%%%%%%%%%%%%%%%%%%%%%%%%%%

\begin{obj}
\item Construct a proof by induction
\end{obj}

The following reading assignment covers the basics on proof by induction:

  \begin{pre}
    \item[Read] Strayer Appendix A.1: The First Principle of Mathematical Induction  or Ernst \href{https://danaernst.com/IBL-IntroToProof/pretext/sec_Intro_to_Induction.html}{Section 4.1} and \href{https://danaernst.com/IBL-IntroToProof/pretext/sec_More_on_Induction.html}{Section 4.2}
   
    \item[Turn in] Strayer Exercise Set A, Exercise 1a. If $n$ is a positive integer, then 
      \[1^2+2^2+3^2+\cdots+n^2=\frac{n(n+1)(2n+1)}{6}.\]
  
  \end{pre}


\begin{br}
Theorems in Ernst \href{https://danaernst.com/IBL-IntroToProof/pretext/sec_Intro_to_Induction.html}{Section 4.1} 
 

\begin{thm*}[Ernst Theorem 4.5]
For all $n\in\mathbb{N}$, 3 divides $4^{n}-1$.
\end{thm*}
\begin{solution}
We proceed by induction.  When $n=1,$ $3\mid 4^n-1=3$. Thus, the statement is true for $n=1.$

Now assume $k\geq 1$ and the desired statement is true for $n=k$. Then the induction hypothesis is \[3\mid 4^k-1.\]
By the definition of \nameref{defn:divides}, there exists $m\in\Z$ such that $3m=4^k-1.$ In other words, $3m+1=4^k$. Multiplying both sides by $4$ gives $12m+4=4^{k+1}$. Rewriting this equation gives $3(4m+1)=4^{k+1}-1$. Thus, $3\mid 4^{k+1}-1$, and the desired statement is true for $n=k+1$. By the (first) principle of mathematical induction, the statement is true for all positive integers, and the proof is complete.
\end{solution}

 \begin{thm*}[Ernst Theorem 4.7]
 Let $p_{1}, p_{2}, \ldots, p_{n}$ be $n$ distinct points arranged on a circle.  Then the number of line segments joining all pairs of points is $\frac{n^{2}-n}{2}$.
 \end{thm*}
\begin{solution}
 We proceed by induction. When $n=1$, there is only one point, so there are no lines connecting pairs of points. Additionally, $\frac{1^2-1}{2}=0$.\footnote{Alternately, you could use $n=2$ for the base case. Then there is one line connecting the only pair of points and $\frac{2^2-2}{2}=1$}
 
 Now assume $k\geq 1$ and the desired statement is true for $n=k$. Then the induction hypothesis is for $k$ distinct points arranged in a circle, the number of line segments joining all pairs of points is $\frac{k^{2}-k}{2}$. Adding a $(k+1)^{st}$ point on the circle will add an additional $k$ line segments joining pairs of points, one for each existing point. Note that 
 \[ 
 	\frac{k^{2}-k}{2}+k=\frac{k^{2}+k}{2}=
	\frac{k^2+k+k+1-(k+1)}{2}=\frac{(k+1)^{2}-(k+1)}{2}\qedhere
 \]
\end{solution}
\end{br}

\begin{br}
[Strayer Exercise 1]. Use the first principle of mathematical induction to prove each statement.
  \begin{enumerate}
    \item If $n$ is a positive integer, then 
  \[1^3+2^3+3^3+\cdots+n^3=\frac{n^2(n+1)^2}{4}.\]
  
    \item If $n$ is an integer with $n\geq 5,$ then \[2^n>n^2.\]
 
  \end{enumerate}
\end{br}







%\begin{br} Let $a_1,a_2,\dots,a_n\in\Z$ with $a_1\neq 0$. Prove that \[\gcd(a_1,\dots,a_n)=\gcd(\gcd(a_1,a_2,a_3,\dots,a_{n-1}),a_n).\]
%\end{br}



\end{document}
