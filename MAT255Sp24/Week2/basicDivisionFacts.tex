\documentclass{ximera}
\usepackage{amssymb, latexsym, amsmath, amsthm, graphicx, amsthm,alltt,color, listings,multicol,xr-hyper,hyperref,aliascnt,enumitem}
\usepackage{xfrac}

\usepackage{parskip}
\usepackage[,margin=0.7in]{geometry}
\setlength{\textheight}{8.5in}

\usepackage{epstopdf}

\DeclareGraphicsExtensions{.eps}
\usepackage{tikz}


\usepackage{tkz-euclide}
%\usetkzobj{all}
\tikzstyle geometryDiagrams=[rounded corners=.5pt,ultra thick,color=black]
\colorlet{penColor}{black} % Color of a curve in a plot


\usepackage{subcaption}
\usepackage{float}
\usepackage{fancyhdr}
\usepackage{pdfpages}
\newcounter{includepdfpage}
\usepackage{makecell}


\usepackage{currfile}
\usepackage{xstring}




\graphicspath{  
{./otherDocuments/}
}

\author{Claire Merriman}
\newcommand{\classday}[1]{\def\classday{#1}}

%%%%%%%%%%%%%%%%%%%%%
% Counters and autoref for unnumbered environments
% Not needed??
%%%%%%%%%%%%%%%%%%%%%
\theoremstyle{plain}


\newtheorem*{namedthm}{Theorem}
\newcounter{thm}%makes pointer correct
\providecommand{\thmname}{Theorem}

\makeatletter
\NewDocumentEnvironment{thm*}{o}
 {%
  \IfValueTF{#1}
    {\namedthm[#1]\refstepcounter{thm}\def\@currentlabel{(#1)}}%
    {\namedthm}%
 }
 {%
  \endnamedthm
 }
\makeatother


\newtheorem*{namedprop}{Proposition}
\newcounter{prop}%makes pointer correct
\providecommand{\propname}{Proposition}

\makeatletter
\NewDocumentEnvironment{prop*}{o}
 {%
  \IfValueTF{#1}
    {\namedprop[#1]\refstepcounter{prop}\def\@currentlabel{(#1)}}%
    {\namedprop}%
 }
 {%
  \endnamedprop
 }
\makeatother

\newtheorem*{namedlem}{Lemma}
\newcounter{lem}%makes pointer correct
\providecommand{\lemname}{Lemma}

\makeatletter
\NewDocumentEnvironment{lem*}{o}
 {%
  \IfValueTF{#1}
    {\namedlem[#1]\refstepcounter{lem}\def\@currentlabel{(#1)}}%
    {\namedlem}%
 }
 {%
  \endnamedlem
 }
\makeatother

\newtheorem*{namedcor}{Corollary}
\newcounter{cor}%makes pointer correct
\providecommand{\corname}{Corollary}

\makeatletter
\NewDocumentEnvironment{cor*}{o}
 {%
  \IfValueTF{#1}
    {\namedcor[#1]\refstepcounter{cor}\def\@currentlabel{(#1)}}%
    {\namedcor}%
 }
 {%
  \endnamedcor
 }
\makeatother

\theoremstyle{definition}
\newtheorem*{annotation}{Annotation}
\newtheorem*{rubric}{Rubric}

\newtheorem*{innerrem}{Remark}
\newcounter{rem}%makes pointer correct
\providecommand{\remname}{Remark}

\makeatletter
\NewDocumentEnvironment{rem}{o}
 {%
  \IfValueTF{#1}
    {\innerrem[#1]\refstepcounter{rem}\def\@currentlabel{(#1)}}%
    {\innerrem}%
 }
 {%
  \endinnerrem
 }
\makeatother

\newtheorem*{innerdefn}{Definition}%%placeholder
\newcounter{defn}%makes pointer correct
\providecommand{\defnname}{Definition}

\makeatletter
\NewDocumentEnvironment{defn}{o}
 {%
  \IfValueTF{#1}
    {\innerdefn[#1]\refstepcounter{defn}\def\@currentlabel{(#1)}}%
    {\innerdefn}%
 }
 {%
  \endinnerdefn
 }
\makeatother

\newtheorem*{scratch}{Scratch Work}


\newtheorem*{namedconj}{Conjecture}
\newcounter{conj}%makes pointer correct
\providecommand{\conjname}{Conjecture}
\makeatletter
\NewDocumentEnvironment{conj}{o}
 {%
  \IfValueTF{#1}
    {\innerconj[#1]\refstepcounter{conj}\def\@currentlabel{(#1)}}%
    {\innerconj}%
 }
 {%
  \endinnerconj
 }
\makeatother

\newtheorem*{poll}{Poll question}
\newtheorem{tps}{Think-Pair-Share}[section]


\newenvironment{obj}{
	\textbf{Learning Objectives.} By the end of class, students will be able to:
		\begin{itemize}}
		{\!.\end{itemize}
		}

\newenvironment{pre}{
	\begin{description}
	}{
	\end{description}
}


\newcounter{ex}%makes pointer correct
\providecommand{\exname}{Homework Problem}
\newenvironment{ex}[1][2in]%
{%Env start code
\problemEnvironmentStart{#1}{Homework Problem}
\refstepcounter{ex}
}
{%Env end code
\problemEnvironmentEnd
}

\newcommand{\inlineAnswer}[2][2 cm]{
    \ifhandout{\pdfOnly{\rule{#1}{0.4pt}}}
    \else{\answer{#2}}
    \fi
}


\ifhandout
\newenvironment{shortAnswer}[1][
    \vfill]
        {% Begin then result
        #1
            \begin{freeResponse}
            }
    {% Environment Ending Code
    \end{freeResponse}
    }
\else
\newenvironment{shortAnswer}[1][]
        {\begin{freeResponse}
            }
    {% Environment Ending Code
    \end{freeResponse}
    }
\fi

\let\question\relax
\let\endquestion\relax

\newtheoremstyle{ExerciseStyle}{\topsep}{\topsep}%%% space between body and thm
		{}                      %%% Thm body font
		{}                              %%% Indent amount (empty = no indent)
		{\bfseries}            %%% Thm head font
		{}                              %%% Punctuation after thm head
		{3em}                           %%% Space after thm head
		{{#1}~\thmnumber{#2}\thmnote{ \bfseries(#3)}}%%% Thm head spec
\theoremstyle{ExerciseStyle}
\newtheorem{br}{In-class Problem}

\newenvironment{sketch}
 {\begin{proof}[Sketch of Proof]}
 {\end{proof}}


\newcommand{\gt}{>}
\newcommand{\lt}{<}
\newcommand{\N}{\mathbb N}
\newcommand{\Q}{\mathbb Q}
\newcommand{\Z}{\mathbb Z}
\newcommand{\C}{\mathbb C}
\newcommand{\R}{\mathbb R}
\renewcommand{\H}{\mathbb{H}}
\newcommand{\lcm}{\operatorname{lcm}}
\newcommand{\nequiv}{\not\equiv}
\newcommand{\ord}{\operatorname{ord}}
\newcommand{\ds}{\displaystyle}
\newcommand{\floor}[1]{\left\lfloor #1\right\rfloor}
\newcommand{\legendre}[2]{\left(\frac{#1}{#2}\right)}



%%%%%%%%%%%%



\title{Divisibility}
\begin{document}
\begin{abstract}
\end{abstract}
\maketitle

%%%%%%%%%%%%%%%%%%%%%%%%%%

\begin{obj}
  \item Define ``divisible" and ``factor"
  \item Prove basic facts about divisibility
\end{obj}
 

\begin{pre}
  \item[Read:]  Read Ernst  \href{https://danaernst.com/IBL-IntroToProof/pretext/chap_intro.html}{Chapter 1} and \href{https://danaernst.com/IBL-IntroToProof/pretext/sec_baby_number_theory.html}{Section 2.1}. Also read Strayer Introduction and Section 1.1 through the proof of Proposition 1.2 (that is, pages 1-5).
  
  \item[Turn in:] From Ernst: Problem 2.6 and 2.8
\end{pre}


\begin{defn}[$a$ divides $b$]\label{defn:divides}
  Let $a,b\in \Z$. The \emph{$a$ divides $b$}, denoted $a\mid b$,  if there exists an integer $c$ such that $b=ac$. 
  If $a\mid b$, then $a$ is said to be a \emph{divisor} or \emph{factor of $b$}. The notation $a\nmid b$ means $a$ does not divide $b$.
\end{defn}

Note that $0$ is not a divisor of any integer other than itself, since $b=0c$ implies $a=0$. Also all integers are divisors of $0,$ as weird as that sounds at first. This is because for any $a\in\Z$, $0=a0$.

\begin{proposition}\label{prop:div-trans}
Let $a,b\in\Z$. If $a\mid b$ and $b \mid c$, then $a\mid c$.
\end{proposition}

Since this is the first result in the chapter, the only tool we have is the definition of ``$a\mid b$". 

\begin{proof}
  Since $a\mid b$ and $b \mid c$, there exist $d,e\in\Z$ such that $b=ae$ and $c=bf$. Combining these, we see \[c=bf=(ae)f=a(ef),\] so $a\mid c$.
\end{proof}

This means that division is \emph{transitive}. 


\begin{proposition}\label{prop:linear-combo}
  Let $a,b,c,m,n\in\Z$.
  If $c\mid a$ and $c\mid b$ then $c\mid ma+nb$.

  \begin{proof}
    Let $a,b,c,m,n\in\Z$ such that $c\mid a$ and $c\mid b$. Then by definition of divisibility, there exists $j,k\in\Z$ such that $cj=a$ and $ck=b$. Thus, \[ma+nb=m(cj)+n(ck)=c(mj+nk).\] Therefore, $c\mid ma+nb$ by definition.
  \end{proof}
\end{proposition}

\begin{defn}
The expression $ma+nb$ in \cref{prop:linear-combo} is called \emph{an (integral) linear combination of $a$ and $b$.}
\end{defn}
\cref{prop:linear-combo} says that an integer dividing each of two integers also divides any integral linear combination of those integers. This fact will be extremely valuable in establishing theoretical results. But first, let's get some more practice with proof writing

Break into three groups. Using the proofs of \cref{prop:div-trans} and \cref{prop:linear-combo} as examples, prove the following facts. Each group will prove one part.

\begin{br}\label{divisfacts}
  Prove or disprove the following statements.
  \begin{enumerate}
    \item If $a,b,c,$ and $d$ are integers such that if $a\mid b$ and $c\mid d$, then $a+c\mid b+d$.
    \item If $a,b,c,$ and $d$ are integers such that if $a\mid b$ and $c\mid d$, then $ac\mid bd$.
    \item If $a,b,$ and $c$ are integers such that if $a\nmid b$ and $b\nmid c$, then $a\nmid c$.
  \end{enumerate}
  
  \begin{solution}
    Problem on Homework.
  \end{solution}
\end{br}
\end{document}
