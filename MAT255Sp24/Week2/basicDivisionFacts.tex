\documentclass{../ximera}
<<<<<<< Updated upstream
\usepackage{amssymb, latexsym, amsmath, amsthm, graphicx, amsthm,alltt,color, listings,multicol,xr-hyper,hyperref,aliascnt,enumitem}
=======
\usepackage{amssymb, latexsym, amsmath, amsthm, graphicx, amsthm,alltt,color, listings,multicol,hyperref}
\usepackage[capitalise,nameinlink]{cleveref}
>>>>>>> Stashed changes
\usepackage{xfrac}

\usepackage{parskip}
\usepackage[,margin=0.7in]{geometry}
\setlength{\textheight}{8.5in}

\usepackage{epstopdf}

\DeclareGraphicsExtensions{.eps}
\usepackage{tikz}


\usepackage{tkz-euclide}
%\usetkzobj{all}
\tikzstyle geometryDiagrams=[rounded corners=.5pt,ultra thick,color=black]
\colorlet{penColor}{black} % Color of a curve in a plot


\usepackage{subcaption}
\usepackage{float}
\usepackage{fancyhdr}
\usepackage{pdfpages}
\newcounter{includepdfpage}
\usepackage{makecell}


\usepackage{currfile}
\usepackage{xstring}




\graphicspath{  
{./otherDocuments/}
}

\author{Claire Merriman}
\newcommand{\classday}[1]{\def\classday{#1}}

%%%%%%%%%%%%%%%%%%%%%
% Counters and autoref for unnumbered environments
% Not needed??
%%%%%%%%%%%%%%%%%%%%%
<<<<<<< Updated upstream
\theoremstyle{plain}


\newtheorem*{namedthm}{Theorem}
\newcounter{thm}%makes pointer correct
\providecommand{\thmname}{Theorem}
=======

\crefname{problem}{problem}{problems}


% \theoremstyle{plain}


% \newtheorem*{namedthm}{Theorem}
% \newcounter{thm}%makes pointer correct
% \providecommand{\thmname}{Theorem}
>>>>>>> Stashed changes

\makeatletter
\NewDocumentEnvironment{thm*}{o}
 {%
  \IfValueTF{#1}
    {\namedthm[#1]\refstepcounter{thm}\def\@currentlabel{(#1)}}%
    {\namedthm}%
 }
 {%
  \endnamedthm
 }
\makeatother


\newtheorem*{namedprop}{Proposition}
\newcounter{prop}%makes pointer correct
\providecommand{\propname}{Proposition}

\makeatletter
\NewDocumentEnvironment{prop*}{o}
 {%
  \IfValueTF{#1}
    {\namedprop[#1]\refstepcounter{prop}\def\@currentlabel{(#1)}}%
    {\namedprop}%
 }
 {%
  \endnamedprop
 }
\makeatother

\newtheorem*{namedlem}{Lemma}
\newcounter{lem}%makes pointer correct
\providecommand{\lemname}{Lemma}

\makeatletter
\NewDocumentEnvironment{lem*}{o}
 {%
  \IfValueTF{#1}
    {\namedlem[#1]\refstepcounter{lem}\def\@currentlabel{(#1)}}%
    {\namedlem}%
 }
 {%
  \endnamedlem
 }
\makeatother

\newtheorem*{namedcor}{Corollary}
\newcounter{cor}%makes pointer correct
\providecommand{\corname}{Corollary}

\makeatletter
\NewDocumentEnvironment{cor*}{o}
 {%
  \IfValueTF{#1}
    {\namedcor[#1]\refstepcounter{cor}\def\@currentlabel{(#1)}}%
    {\namedcor}%
 }
 {%
  \endnamedcor
 }
\makeatother

\theoremstyle{definition}
\newtheorem*{annotation}{Annotation}
\newtheorem*{rubric}{Rubric}

\newtheorem*{innerrem}{Remark}
\newcounter{rem}%makes pointer correct
\providecommand{\remname}{Remark}

\makeatletter
\NewDocumentEnvironment{rem}{o}
 {%
  \IfValueTF{#1}
    {\innerrem[#1]\refstepcounter{rem}\def\@currentlabel{(#1)}}%
    {\innerrem}%
 }
 {%
  \endinnerrem
 }
\makeatother

\newtheorem*{innerdefn}{Definition}%%placeholder
\newcounter{defn}%makes pointer correct
\providecommand{\defnname}{Definition}

\makeatletter
\NewDocumentEnvironment{defn}{o}
 {%
  \IfValueTF{#1}
    {\innerdefn[#1]\refstepcounter{defn}\def\@currentlabel{(#1)}}%
    {\innerdefn}%
 }
 {%
  \endinnerdefn
 }
\makeatother

\newtheorem*{scratch}{Scratch Work}


\newtheorem*{namedconj}{Conjecture}
\newcounter{conj}%makes pointer correct
\providecommand{\conjname}{Conjecture}
\makeatletter
\NewDocumentEnvironment{conj}{o}
 {%
  \IfValueTF{#1}
    {\innerconj[#1]\refstepcounter{conj}\def\@currentlabel{(#1)}}%
    {\innerconj}%
 }
 {%
  \endinnerconj
 }
\makeatother

\newtheorem*{poll}{Poll question}
\newtheorem{tps}{Think-Pair-Share}[section]


\newenvironment{obj}{
	\textbf{Learning Objectives.} By the end of class, students will be able to:
		\begin{itemize}}
		{\!.\end{itemize}
		}

<<<<<<< Updated upstream
\newenvironment{pre}{
	\begin{description}
	}{
	\end{description}
}
=======

\ifinstructornotes
\newenvironment{pre}
  {{\textbf Reading assignment:}
  \begin{description}
    }{
	\end{description}
  }
\else
\newenvironment{pre}{ 
  \begin{trivlist}
  \item[]}
  {\end{trivlist}}
\fi
>>>>>>> Stashed changes


\newcounter{ex}%makes pointer correct
\providecommand{\exname}{Homework Problem}
\newenvironment{ex}[1][2in]%
{%Env start code
\problemEnvironmentStart{#1}{Homework Problem}
\refstepcounter{ex}
}
{%Env end code
\problemEnvironmentEnd
}

\newcommand{\inlineAnswer}[2][2 cm]{
    \ifhandout{\pdfOnly{\rule{#1}{0.4pt}}}
    \else{\answer{#2}}
    \fi
}


\ifhandout
\newenvironment{shortAnswer}[1][
    \vfill]
        {% Begin then result
        #1
            \begin{freeResponse}
            }
    {% Environment Ending Code
    \end{freeResponse}
    }
\else
\newenvironment{shortAnswer}[1][]
        {\begin{freeResponse}
            }
    {% Environment Ending Code
    \end{freeResponse}
    }
\fi

\let\question\relax
\let\endquestion\relax

\newtheoremstyle{ExerciseStyle}{\topsep}{\topsep}%%% space between body and thm
		{}                      %%% Thm body font
		{}                              %%% Indent amount (empty = no indent)
		{\bfseries}            %%% Thm head font
		{}                              %%% Punctuation after thm head
		{3em}                           %%% Space after thm head
		{{#1}~\thmnumber{#2}\thmnote{ \bfseries(#3)}}%%% Thm head spec
\theoremstyle{ExerciseStyle}
\newtheorem{br}{In-class Problem}

\newenvironment{sketch}
 {\begin{proof}[Sketch of Proof]}
 {\end{proof}}


\newcommand{\gt}{>}
\newcommand{\lt}{<}
\newcommand{\N}{\mathbb N}
\newcommand{\Q}{\mathbb Q}
\newcommand{\Z}{\mathbb Z}
\newcommand{\C}{\mathbb C}
\newcommand{\R}{\mathbb R}
\renewcommand{\H}{\mathbb{H}}
\newcommand{\lcm}{\operatorname{lcm}}
\newcommand{\nequiv}{\not\equiv}
\newcommand{\ord}{\operatorname{ord}}
\newcommand{\ds}{\displaystyle}
\newcommand{\floor}[1]{\left\lfloor #1\right\rfloor}
\newcommand{\legendre}[2]{\left(\frac{#1}{#2}\right)}



%%%%%%%%%%%%



\title{Divisibility practice}
\begin{document}
\begin{abstract}
\end{abstract}
\maketitle

%%%%%%%%%%%%%%%%%%%%%%%%%%

\begin{obj}
  \item Prove facts about divisibility
  \item Prove basic mathematical statements using definitions and direct proof
  \item Use truth tables to understand compound propositions
  \item Prove statements by contradiction
  \item Use the greatest integer function
\end{obj}
 

\begin{instructorNotes}
  \begin{pre}
    \item[Reading]  Read Ernst  \href{https://danaernst.com/IBL-IntroToProof/pretext/chap_intro.html}{Chapter 1} and \href{https://danaernst.com/IBL-IntroToProof/pretext/sec_baby_number_theory.html}{Section 2.1}. Also read Strayer Introduction and Section 1.1 through the proof of Proposition 1.2 (that is, pages 1-5).
  
    \item[Turn in:] From Ernst: Problem 2.6 and 2.8
  \end{pre}
\end{instructorNotes}




 

%%%%%%%%%%%%%%%%%%%%%%%%%%
\subsection{Divisibility practice}
%%%%%%%%%%%%%%%%%%%%%%%%%%

\begin{proposition}
Let $a,b\in\Z$. If $a\mid b$ and $b \mid c$, then $a\mid c$.
\end{proposition}

Since this is the first result in the chapter, the only tool we have is the definition of ``$a\mid b$". 

\begin{proof}
  Since $a\mid b$ and $b \mid c$, there exist $d,e\in\Z$ such that $b=ae$ and $c=bf$. Combining these, we see \[c=bf=(ae)f=a(ef),\] so $a\mid c$.
\end{proof}

This means that division is \emph{transitive}. 


\begin{proposition}
  Let $a,b,c,m,n\in\Z$.
  If $c\mid a$ and $c\mid b$ then $c\mid ma+nb$.

  \begin{proof}
    Let $a,b,c,m,n\in\Z$ such that $c\mid a$ and $c\mid b$. Then by definition of divisibility, there exists $j,k\in\Z$ such that $cj=a$ and $ck=b$. Thus, \[ma+nb=m(cj)+n(ck)=c(mj+nk).\] Therefore, $c\mid ma+nb$ by definition.
  \end{proof}
\end{proposition}

\begin{defn}
The expression $ma+nb$ in Proposition 1.2 is called \emph{an (integral) linear combination of $a$ and $b$.}
\end{defn}
Proposition 1.2 says that an integer dividing each of two integers also divides any integral linear combination of those integers. This fact will be extremely valuable in establishing theoretical results. But first, let's get some more practice with proof writing

Break into three groups. Using the proofs of Propositions 1.1 and 1.2 as examples, prove the following facts. Each group will prove one part.

\begin{br}[Exercise Set 1.1, Exercise 5]\label{divisfacts}
Prove or disprove the following statements.
\begin{enumerate}[label=(\alph*)]
\item If $a,b,c,$ and $d$ are integers such that if $a\mid b$ and $c\mid d$, then $a+c\mid b+d$.
\item If $a,b,c,$ and $d$ are integers such that if $a\mid b$ and $c\mid d$, then $ac\mid bd$.
\item If $a,b,$ and $c$ are integers such that if $a\nmid b$ and $b\nmid c$, then $a\nmid c$.
\end{enumerate}
\end{br}
\begin{solution}
Problem on Homework 1.
\end{solution}

\subsection{Logic, proof by contradiction, and biconditionals} 

We will begin by working through Ernst \href{https://danaernst.com/IBL-IntroToProof/pretext/sec_Intro_to_Logic.html}{Section 2.2} through Example 2.21. Discuss Problem 2.17 as a class. Problem 2.17 is also provided below:


\begin{br}
  Determine whether each of the following is a proposition. Explain your reasoning.

  \begin{itemize}
    \item All cars are red.
    \item Every person whose name begins with J has the name Joe.
    \item $x^2=4.$
    \item There exists a real number $x$ such that $x^2=4$.
    \item For all real numbers $x,$ $x^2=4.$
    \item $\sqrt{2}$ is an irrational number.
    \item $p$ is prime.
    \item Is it raining?
    \item It will rain tomorrow.
    \item Led Zeppelin is the best band of all time.
  \end{itemize}
\end{br}



\begin{br}
  Construct a truth table for $A\Rightarrow B, \neg (A\Rightarrow B)$ and $A\land \neg B$

  \begin{solution}
 
    \begin{tabular}{c|c|c|c|c}
      $A$ 	& $B$	& $A\Rightarrow B$ 	& $\neg (A\Rightarrow B)$ & $A\land \neg B$\\\hline
      T 	& T		& T 				& F					& F	\\
      T 	& F 		& F 				& T					& T\\
      F 	& T 		& T 				& F					& F\\
      F 	& F 		& T 				& F					& F\\
    \end{tabular}
  \end{solution}
\end{br}
This is the basis for \emph{proof by contradiction.} We assume both $A$ and $\neg B$, and proceed until we get a contradiction. That is, $A$ and $\neg B$ cannot both be true.

\begin{defn}[Proof by contradiction]\label{proof-contradiction}
  Let $A$ and $B$ be propositions. To prove $A$ implies $B$ by contradiction, first assume the $B$ is false. Then work through logical steps until you conclude $\neg A \land A$.
\end{defn}

All definitions are `biconditionals but we normally only write the ``if."

We say that two definitions are \emph{equivalent} if definition A is true if and only if definition B is true. 
%\begin{br}
%Prove that our two definitions of even are equivalent.
%\end{br}
%
%\begin{proposition}
% Let $n\in\Z$. Then there is some $k\in\Z$ such that $n=2k$ if and only if $2\mid n$.
%\end{proposition}
%%\begin{proof}
%% $(\Rightarrow)$  Let $n\in\Z$. Assume that there is some $n\in\Z$ such that $n=2k$. Then 
%% Thus, $2\mid n$ by definition.
%% 
%%  $(\Leftarrow)$  Let $n\in\Z$. Assume that $2\mid n$. Then, there is some $k\in\Z$ such that $n=2k$ by the definition of $2\mid n$.
%%\end{proof}


\end{document}
