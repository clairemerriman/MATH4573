\documentclass{ximera}
\usepackage{amssymb, latexsym, amsmath, amsthm, graphicx, amsthm,alltt,color, listings,multicol,xr-hyper,hyperref,aliascnt,enumitem}
\usepackage{xfrac}

\usepackage{parskip}
\usepackage[,margin=0.7in]{geometry}
\setlength{\textheight}{8.5in}

\usepackage{epstopdf}

\DeclareGraphicsExtensions{.eps}
\usepackage{tikz}


\usepackage{tkz-euclide}
%\usetkzobj{all}
\tikzstyle geometryDiagrams=[rounded corners=.5pt,ultra thick,color=black]
\colorlet{penColor}{black} % Color of a curve in a plot


\usepackage{subcaption}
\usepackage{float}
\usepackage{fancyhdr}
\usepackage{pdfpages}
\newcounter{includepdfpage}
\usepackage{makecell}


\usepackage{currfile}
\usepackage{xstring}




\graphicspath{  
{./otherDocuments/}
}

\author{Claire Merriman}
\newcommand{\classday}[1]{\def\classday{#1}}

%%%%%%%%%%%%%%%%%%%%%
% Counters and autoref for unnumbered environments
% Not needed??
%%%%%%%%%%%%%%%%%%%%%
\theoremstyle{plain}


\newtheorem*{namedthm}{Theorem}
\newcounter{thm}%makes pointer correct
\providecommand{\thmname}{Theorem}

\makeatletter
\NewDocumentEnvironment{thm*}{o}
 {%
  \IfValueTF{#1}
    {\namedthm[#1]\refstepcounter{thm}\def\@currentlabel{(#1)}}%
    {\namedthm}%
 }
 {%
  \endnamedthm
 }
\makeatother


\newtheorem*{namedprop}{Proposition}
\newcounter{prop}%makes pointer correct
\providecommand{\propname}{Proposition}

\makeatletter
\NewDocumentEnvironment{prop*}{o}
 {%
  \IfValueTF{#1}
    {\namedprop[#1]\refstepcounter{prop}\def\@currentlabel{(#1)}}%
    {\namedprop}%
 }
 {%
  \endnamedprop
 }
\makeatother

\newtheorem*{namedlem}{Lemma}
\newcounter{lem}%makes pointer correct
\providecommand{\lemname}{Lemma}

\makeatletter
\NewDocumentEnvironment{lem*}{o}
 {%
  \IfValueTF{#1}
    {\namedlem[#1]\refstepcounter{lem}\def\@currentlabel{(#1)}}%
    {\namedlem}%
 }
 {%
  \endnamedlem
 }
\makeatother

\newtheorem*{namedcor}{Corollary}
\newcounter{cor}%makes pointer correct
\providecommand{\corname}{Corollary}

\makeatletter
\NewDocumentEnvironment{cor*}{o}
 {%
  \IfValueTF{#1}
    {\namedcor[#1]\refstepcounter{cor}\def\@currentlabel{(#1)}}%
    {\namedcor}%
 }
 {%
  \endnamedcor
 }
\makeatother

\theoremstyle{definition}
\newtheorem*{annotation}{Annotation}
\newtheorem*{rubric}{Rubric}

\newtheorem*{innerrem}{Remark}
\newcounter{rem}%makes pointer correct
\providecommand{\remname}{Remark}

\makeatletter
\NewDocumentEnvironment{rem}{o}
 {%
  \IfValueTF{#1}
    {\innerrem[#1]\refstepcounter{rem}\def\@currentlabel{(#1)}}%
    {\innerrem}%
 }
 {%
  \endinnerrem
 }
\makeatother

\newtheorem*{innerdefn}{Definition}%%placeholder
\newcounter{defn}%makes pointer correct
\providecommand{\defnname}{Definition}

\makeatletter
\NewDocumentEnvironment{defn}{o}
 {%
  \IfValueTF{#1}
    {\innerdefn[#1]\refstepcounter{defn}\def\@currentlabel{(#1)}}%
    {\innerdefn}%
 }
 {%
  \endinnerdefn
 }
\makeatother

\newtheorem*{scratch}{Scratch Work}


\newtheorem*{namedconj}{Conjecture}
\newcounter{conj}%makes pointer correct
\providecommand{\conjname}{Conjecture}
\makeatletter
\NewDocumentEnvironment{conj}{o}
 {%
  \IfValueTF{#1}
    {\innerconj[#1]\refstepcounter{conj}\def\@currentlabel{(#1)}}%
    {\innerconj}%
 }
 {%
  \endinnerconj
 }
\makeatother

\newtheorem*{poll}{Poll question}
\newtheorem{tps}{Think-Pair-Share}[section]


\newenvironment{obj}{
	\textbf{Learning Objectives.} By the end of class, students will be able to:
		\begin{itemize}}
		{\!.\end{itemize}
		}

\newenvironment{pre}{
	\begin{description}
	}{
	\end{description}
}


\newcounter{ex}%makes pointer correct
\providecommand{\exname}{Homework Problem}
\newenvironment{ex}[1][2in]%
{%Env start code
\problemEnvironmentStart{#1}{Homework Problem}
\refstepcounter{ex}
}
{%Env end code
\problemEnvironmentEnd
}

\newcommand{\inlineAnswer}[2][2 cm]{
    \ifhandout{\pdfOnly{\rule{#1}{0.4pt}}}
    \else{\answer{#2}}
    \fi
}


\ifhandout
\newenvironment{shortAnswer}[1][
    \vfill]
        {% Begin then result
        #1
            \begin{freeResponse}
            }
    {% Environment Ending Code
    \end{freeResponse}
    }
\else
\newenvironment{shortAnswer}[1][]
        {\begin{freeResponse}
            }
    {% Environment Ending Code
    \end{freeResponse}
    }
\fi

\let\question\relax
\let\endquestion\relax

\newtheoremstyle{ExerciseStyle}{\topsep}{\topsep}%%% space between body and thm
		{}                      %%% Thm body font
		{}                              %%% Indent amount (empty = no indent)
		{\bfseries}            %%% Thm head font
		{}                              %%% Punctuation after thm head
		{3em}                           %%% Space after thm head
		{{#1}~\thmnumber{#2}\thmnote{ \bfseries(#3)}}%%% Thm head spec
\theoremstyle{ExerciseStyle}
\newtheorem{br}{In-class Problem}

\newenvironment{sketch}
 {\begin{proof}[Sketch of Proof]}
 {\end{proof}}


\newcommand{\gt}{>}
\newcommand{\lt}{<}
\newcommand{\N}{\mathbb N}
\newcommand{\Q}{\mathbb Q}
\newcommand{\Z}{\mathbb Z}
\newcommand{\C}{\mathbb C}
\newcommand{\R}{\mathbb R}
\renewcommand{\H}{\mathbb{H}}
\newcommand{\lcm}{\operatorname{lcm}}
\newcommand{\nequiv}{\not\equiv}
\newcommand{\ord}{\operatorname{ord}}
\newcommand{\ds}{\displaystyle}
\newcommand{\floor}[1]{\left\lfloor #1\right\rfloor}
\newcommand{\legendre}[2]{\left(\frac{#1}{#2}\right)}



%%%%%%%%%%%%



\title{The $\phi-$ function and a preview of primitive roots}
\begin{document}
\begin{abstract}
\end{abstract}
\maketitle

%%%%%%%%%%%%%%%%%%%%%%%%%%

\begin{obj}
  \item Prove that $\phi(m)\phi(n)=\phi(mn)$ when $(m,n)=1$
\end{obj}


\begin{pre}
    \item[Reading] None
    \item[Turn In] Paper 2
\end{pre}

%%%%%%%%%%%%%%%%%%%%%%%%%%
\subsection{Quiz (10 min)}
%%%%%%%%%%%%%%%%%%%%%%%%%%

%%%%%%%%%%%%%%%%%%%%%%%%%%
\subsection*{The Euler $\phi$-function (20 min)}
%%%%%%%%%%%%%%%%%%%%%%%%%%

We will use \autoref{br:multiplicative-proof} as an outline to prove 

\begin{theorem}[Theorem 3.2]\label{thm:phi-multiplicative}
  Let $m$ and $n$ be positive integers where $(m,n)=1$. Then $\phi(mn)=\phi(m)\phi(n).$
\end{theorem}
maybe works?

\begin{proof} 
  First, we note that an integer $a$ is relatively prime to $mn$ if and only if it is relatively prime to $m$ and $n,$ since $m$ and $n$ (together) have the same prime divisors as $mn$.

  
  We can partition the positive integers less that $mn$ into 
    \begin{alignat*}{8}
      &0  & \equiv & m  & \equiv & 2m & \equiv &\cdots  & \equiv & (n-1)m\pmod{m}\\
      &1  & \equiv & m + 1  & \equiv & 2m + 1 & \equiv &\cdots  & \equiv & (n-1)m + 1\pmod{m}\\
      &2  & \equiv & m + 2  & \equiv & 2m + 2 & \equiv & \cdots  & \equiv & (n-1)m + 2\pmod{m}\\
      & \vdots & & \vdots& & \vdots & & & &  \vdots\\
      &m-1  & \equiv & 2m - 1  & \equiv & 3m -1 & \equiv & \cdots  & \equiv & nm - 1\pmod{m}
    \end{alignat*}
 
    For any $b$ in the range $0,1,2,\dots,m-1$, define $s_b$ to be the number of integers $a$ in the range $0,1,2,\dots, mn-1$ such that $a\equiv b \pmod{m}$ and $\gcd(a,mn)=1$. For each equivalence class $b,$ $\gcd(b,m)\mid km +b$ by linear combination. Thus, $s_b={0}$ if $(b,m)>1.$ If $\gcd(b,m)=1$, the arithmetic progression, \{b, m + b, 2m + b, \dots, (n-1)m+ b\} contains $n$ elements. By \autoref{br:arithmetic-progression}, the arithmetic progression is a \nameref{defn:complete-residue} modulo $n,$ so $\phi(n)$ elements are relatively prime to $n$ and thus $mn.$
 
    Thus, can see that when $(b,m)=1$, $s_b=\phi(n)$ and when $(b,m)>1$, $s_b={0}$.
 
    Since all of the positive integers less than or equal to $mn$ is in exactly one of the congruence classes above and t he $s_i$ count how many integers in each congruence class are relatively prime to $mn$, $phi(mn)=s_0+s_1+\cdots +s_{m-1}.$
 
    Since $\phi(m)$ of the $s_i=\phi(n)$ and the rest are 0, $\phi(mn)=s_0+s_1+\cdots +s_{m-1}=\phi(m)\phi(n).$
 
\end{proof}


\begin{br}\label{br:arithmetic-progression}
  Complete the proof of \nameref{thm:phi-multiplicative} by proving that, if $m, n,$ and $i$ are positive integers with ($m, n) = (m, i) = 1,$ then the integers $i, m + i, 2m +i,..., (n - 1)m +i$ form a complete system of residues modulo $n.$
\end{br}


\end{document}
