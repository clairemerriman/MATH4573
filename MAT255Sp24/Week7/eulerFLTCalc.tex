\documentclass{ximera}
<<<<<<< Updated upstream
\usepackage{amssymb, latexsym, amsmath, amsthm, graphicx, amsthm,alltt,color, listings,multicol,xr-hyper,hyperref,aliascnt,enumitem}
=======
\usepackage{amssymb, latexsym, amsmath, amsthm, graphicx, amsthm,alltt,color, listings,multicol,hyperref}
\usepackage[capitalise,nameinlink]{cleveref}
>>>>>>> Stashed changes
\usepackage{xfrac}

\usepackage{parskip}
\usepackage[,margin=0.7in]{geometry}
\setlength{\textheight}{8.5in}

\usepackage{epstopdf}

\DeclareGraphicsExtensions{.eps}
\usepackage{tikz}


\usepackage{tkz-euclide}
%\usetkzobj{all}
\tikzstyle geometryDiagrams=[rounded corners=.5pt,ultra thick,color=black]
\colorlet{penColor}{black} % Color of a curve in a plot


\usepackage{subcaption}
\usepackage{float}
\usepackage{fancyhdr}
\usepackage{pdfpages}
\newcounter{includepdfpage}
\usepackage{makecell}


\usepackage{currfile}
\usepackage{xstring}




\graphicspath{  
{./otherDocuments/}
}

\author{Claire Merriman}
\newcommand{\classday}[1]{\def\classday{#1}}

%%%%%%%%%%%%%%%%%%%%%
% Counters and autoref for unnumbered environments
% Not needed??
%%%%%%%%%%%%%%%%%%%%%
<<<<<<< Updated upstream
\theoremstyle{plain}


\newtheorem*{namedthm}{Theorem}
\newcounter{thm}%makes pointer correct
\providecommand{\thmname}{Theorem}
=======

\crefname{problem}{problem}{problems}


% \theoremstyle{plain}


% \newtheorem*{namedthm}{Theorem}
% \newcounter{thm}%makes pointer correct
% \providecommand{\thmname}{Theorem}
>>>>>>> Stashed changes

\makeatletter
\NewDocumentEnvironment{thm*}{o}
 {%
  \IfValueTF{#1}
    {\namedthm[#1]\refstepcounter{thm}\def\@currentlabel{(#1)}}%
    {\namedthm}%
 }
 {%
  \endnamedthm
 }
\makeatother


\newtheorem*{namedprop}{Proposition}
\newcounter{prop}%makes pointer correct
\providecommand{\propname}{Proposition}

\makeatletter
\NewDocumentEnvironment{prop*}{o}
 {%
  \IfValueTF{#1}
    {\namedprop[#1]\refstepcounter{prop}\def\@currentlabel{(#1)}}%
    {\namedprop}%
 }
 {%
  \endnamedprop
 }
\makeatother

\newtheorem*{namedlem}{Lemma}
\newcounter{lem}%makes pointer correct
\providecommand{\lemname}{Lemma}

\makeatletter
\NewDocumentEnvironment{lem*}{o}
 {%
  \IfValueTF{#1}
    {\namedlem[#1]\refstepcounter{lem}\def\@currentlabel{(#1)}}%
    {\namedlem}%
 }
 {%
  \endnamedlem
 }
\makeatother

\newtheorem*{namedcor}{Corollary}
\newcounter{cor}%makes pointer correct
\providecommand{\corname}{Corollary}

\makeatletter
\NewDocumentEnvironment{cor*}{o}
 {%
  \IfValueTF{#1}
    {\namedcor[#1]\refstepcounter{cor}\def\@currentlabel{(#1)}}%
    {\namedcor}%
 }
 {%
  \endnamedcor
 }
\makeatother

\theoremstyle{definition}
\newtheorem*{annotation}{Annotation}
\newtheorem*{rubric}{Rubric}

\newtheorem*{innerrem}{Remark}
\newcounter{rem}%makes pointer correct
\providecommand{\remname}{Remark}

\makeatletter
\NewDocumentEnvironment{rem}{o}
 {%
  \IfValueTF{#1}
    {\innerrem[#1]\refstepcounter{rem}\def\@currentlabel{(#1)}}%
    {\innerrem}%
 }
 {%
  \endinnerrem
 }
\makeatother

\newtheorem*{innerdefn}{Definition}%%placeholder
\newcounter{defn}%makes pointer correct
\providecommand{\defnname}{Definition}

\makeatletter
\NewDocumentEnvironment{defn}{o}
 {%
  \IfValueTF{#1}
    {\innerdefn[#1]\refstepcounter{defn}\def\@currentlabel{(#1)}}%
    {\innerdefn}%
 }
 {%
  \endinnerdefn
 }
\makeatother

\newtheorem*{scratch}{Scratch Work}


\newtheorem*{namedconj}{Conjecture}
\newcounter{conj}%makes pointer correct
\providecommand{\conjname}{Conjecture}
\makeatletter
\NewDocumentEnvironment{conj}{o}
 {%
  \IfValueTF{#1}
    {\innerconj[#1]\refstepcounter{conj}\def\@currentlabel{(#1)}}%
    {\innerconj}%
 }
 {%
  \endinnerconj
 }
\makeatother

\newtheorem*{poll}{Poll question}
\newtheorem{tps}{Think-Pair-Share}[section]


\newenvironment{obj}{
	\textbf{Learning Objectives.} By the end of class, students will be able to:
		\begin{itemize}}
		{\!.\end{itemize}
		}

<<<<<<< Updated upstream
\newenvironment{pre}{
	\begin{description}
	}{
	\end{description}
}
=======

\ifinstructornotes
\newenvironment{pre}
  {{\textbf Reading assignment:}
  \begin{description}
    }{
	\end{description}
  }
\else
\newenvironment{pre}{ 
  \begin{trivlist}
  \item[]}
  {\end{trivlist}}
\fi
>>>>>>> Stashed changes


\newcounter{ex}%makes pointer correct
\providecommand{\exname}{Homework Problem}
\newenvironment{ex}[1][2in]%
{%Env start code
\problemEnvironmentStart{#1}{Homework Problem}
\refstepcounter{ex}
}
{%Env end code
\problemEnvironmentEnd
}

\newcommand{\inlineAnswer}[2][2 cm]{
    \ifhandout{\pdfOnly{\rule{#1}{0.4pt}}}
    \else{\answer{#2}}
    \fi
}


\ifhandout
\newenvironment{shortAnswer}[1][
    \vfill]
        {% Begin then result
        #1
            \begin{freeResponse}
            }
    {% Environment Ending Code
    \end{freeResponse}
    }
\else
\newenvironment{shortAnswer}[1][]
        {\begin{freeResponse}
            }
    {% Environment Ending Code
    \end{freeResponse}
    }
\fi

\let\question\relax
\let\endquestion\relax

\newtheoremstyle{ExerciseStyle}{\topsep}{\topsep}%%% space between body and thm
		{}                      %%% Thm body font
		{}                              %%% Indent amount (empty = no indent)
		{\bfseries}            %%% Thm head font
		{}                              %%% Punctuation after thm head
		{3em}                           %%% Space after thm head
		{{#1}~\thmnumber{#2}\thmnote{ \bfseries(#3)}}%%% Thm head spec
\theoremstyle{ExerciseStyle}
\newtheorem{br}{In-class Problem}

\newenvironment{sketch}
 {\begin{proof}[Sketch of Proof]}
 {\end{proof}}


\newcommand{\gt}{>}
\newcommand{\lt}{<}
\newcommand{\N}{\mathbb N}
\newcommand{\Q}{\mathbb Q}
\newcommand{\Z}{\mathbb Z}
\newcommand{\C}{\mathbb C}
\newcommand{\R}{\mathbb R}
\renewcommand{\H}{\mathbb{H}}
\newcommand{\lcm}{\operatorname{lcm}}
\newcommand{\nequiv}{\not\equiv}
\newcommand{\ord}{\operatorname{ord}}
\newcommand{\ds}{\displaystyle}
\newcommand{\floor}[1]{\left\lfloor #1\right\rfloor}
\newcommand{\legendre}[2]{\left(\frac{#1}{#2}\right)}



%%%%%%%%%%%%



\title{Calculations with Fermat's Little Theorem and Eluer's Theorem}
\begin{document}
\begin{abstract}
\end{abstract}
\maketitle

%%%%%%%%%%%%%%%%%%%%%%%%%%

\begin{obj}
  \item Use Fermat's Little Theorem to find the least nonnegative residue modulo a prime
  \item Use Euler's Theorem to find the least nonnegative residue modulo a composite
\end{obj}


\subsection*{Finding least nonnegative residue Fermat's Little Theorem}
%%%%%%%%%%%%%%%%%%%%%%%%%%
\begin{example}
  \begin{enumerate}
    \item  Find the least nonnegative residue of $29^{202}$ modulo $13$. 
    
    First, note that $29\equiv 3\pmod{13}$ and $202=12(10)+82=12(10)+12(6)+10=12(16)+10.$  Thus,
    \[29^{202}\equiv3^{202}\equiv (3^{12})^{16}  3^{10}\equiv 1^{16} 3^{10}\pmod{13}\]
    From here,  we have two  options: 
    \begin{description}
      \item[Keep reducing:] For this problem, this is the easier method: 
        \[3^{10}\equiv (3^3)^3 3\equiv (27)^3 3\equiv 3\pmod{13}.\]
      \item[Find inverse:] Note that $3^{12}\equiv 1\pmod{13},$ so $3^{10}$ is the multiplicative inverse of $3^2\equiv 9\pmod{13}.$ Since $9(3)\equiv 1\pmod{13},$ $3^{10}\equiv 1\pmod{13}.$
    \end{description}

    \item Find the least nonnegative residue of $71^{71}$ modulo $17$. 

    First, note that $71\equiv 3\pmod{17}$ and $71=8(8)+7.$  Thus,
    \[71^{71}\equiv3^{71}\equiv (3^{8})^{8}  3^{7}\equiv 1^{8} 3^{7}\pmod{17}\]
    Then \[3^7\equiv 3^{3}(3^3)(3)\equiv 10(10)(3)\equiv 10(-4)\equiv -6\equiv 11\pmod{17}.\]
  \end{enumerate}
\end{example}

\begin{corollary}\label{cor:inv-fermat}
  Let $p$ be a prime. If $a\in\Z$ with $p\nmid a,$ then $a^{p-2}$  is the multiplicative inverse of $a$ modulo $p$.
\end{corollary}

\begin{tps} Prove:
  Let $p$ be a prime. If $a,k\in\Z$ with $p\nmid a$ and $0\leq k<p,$ then $a^{p-k}$  is the multiplicative inverse of $a^k$ modulo $p$.
  
  
  \begin{proof}
    Let $p$ be a prime. If $a\in\Z$ with $p\nmid a,$ then by Fermat's Little Theorem, $a^{p-1}\equiv 1\pmod{p.}$ If $k\in\Z$ with $0\leq k<p,$ then $a^{p-1}=a^{p-k}a^k.$ Thus, $a^{p-k}a^k\equiv 1\pmod{p}.$
  \end{proof}
\end{tps}

\begin{example}
  Find all incongruent solutions to $9x\equiv 21\pmod{23}$.
  
  Since$(9,23)=1,$ there is only one  incongruent solution  modulo $23.$ By \nameref{cor:inv-fermat}, $9^{21}$ is the multiplicative inverse of $9$ modulo ${23}$. Thus, $x\equiv 21(9^{21})\pmod{23.}$

  Alternately, $3^{20}$ is the multiplicative inverse of $3^2$ modulo ${23},$ so $x\equiv 21(3^{20})\equiv  (3^{21})7\pmod{23}.$ Since $3^{21}$ is the multiplicative inverse of $3$ modulo $23,$ so $3^{21}\equiv 8\pmod{23}.$ Thus, $x\equiv 7(8)\equiv 10\pmod{23}.$
\end{example}

\begin{example}
  Let $p$ be prime and $a,b\in\Z$ with $p\nmid a$ and $p\nmid b.$ Then $a^p\equiv b^p\pmod{p}$ if and only if $a\equiv b\pmod{p}$.
  
  
  \begin{proof}
    Let $p$ be prime and $a,b\in\Z$ with $p\nmid a$ and $p\nmid b.$ 
      
    ($\Leftarrow$) If $a\equiv  b\pmod{p},$ then $a^p\equiv b^p\pmod{p}$ by repeated applications of Proposition 2.4.
      
    ($\Rightarrow$)  If $a^p\equiv  b^p\pmod{p},$ then by Fermat's Little Theorem, 
      \[a\equiv a^{p-1}a\equiv b^{p-1}b\equiv b\pmod p. \]
  \end{proof}
  \begin{warning}
    This statement is only true for primes. Since
    \[
    1^2\equiv 3^2\equiv 5^2\equiv 7^2\pmod{8},\quad 2^2\equiv 6^2\pmod{8},
    \]
    
    \[
    1^8\equiv 3^8\equiv 5^8\equiv 7^8\pmod{8},\quad  2^8\equiv 6^8\pmod{8}.
    \]
  \end{warning}
\end{example}

%%%%%%%%%%%%%%%%%%%%%%%%%

%%%%%%%%%%%%%%%%%%%%%%%%%%
\subsection*{Multiplicative inverses using Euler's Extension of Fermat's Little Theorem}
%%%%%%%%%%%%%%%%%%%%%%%%%%

\begin{example}\label{ex-euler-solve-cong}
 
  \begin{enumerate}
    \item Find the least nonnegative residue of $29^{202}$ modulo $20$. 
    
    The integers $1,3,7,9,11,13,17,19$ are relatively prime to $20.$ Thus $\phi(20)=8.$ Also note that $29\equiv 9\pmod{20}$ and $202=8(25)+2,$ so 
    \[29^{202}\equiv 9^{202}\equiv(9^8)^{25} 9^2\equiv 1^{25} 9^2\equiv 1\pmod{20}\] 
    
    \item Find the least nonnegative residue of $71^{71}$ modulo $16$.
    
    The integers $1,3,5,7,9,11,13,15$ are relatively prime to $16.$ Thus $\phi(16)=8.$ Also note that $71\equiv 7\pmod{16}$ and $71=8(8)+7,$ so 
      \[71^{71}\equiv 7^{71}\equiv(7^8)^{8} 7^7\equiv 1^{8} 7^7\pmod{16}\] 
      Since $7^8\equiv 7^7 7\equiv 1\pmod{16},$ $7^7$ is the multiplicative inverse of $7$ mulod $16.$

    Using the Euclidean algorithm, 
    \begin{align*}
      16	& = 7(2) + 2,	&2&=16+7(-2)\\
      7	& = 2(3) + 1,	&1&=7-2(3)=7-(16+7(-2))(3)=16(-3)+7(7)
    \end{align*}
    Thus, $7(7)\equiv 1\pmod{16},$ and $7^7\equiv 7\pmod{16}.$
  \end{enumerate}
\end{example}

\begin{corollary}\label{cor:inv-euler}
  Let $a,m\in\Z$ with $m>0.$  If $(a,m)=1,$ then $a^{\phi(m)-1}$ is the multiplicative inverse of $a$ modulo $m$.
\end{corollary}

\begin{example}
  Find all incongruent solutions to $9x\equiv 21\pmod{25}$.
  
  The only positive integers less than $25$ that are \emph{not} relatively prime to $25$ are $5,10,15,20$. Thus, $\phi(25)=24-4=20.$

  Since$(9,25)=1,$ there is only one  incongruent solution  modulo $25.$ By \nameref{cor:inv-euler}, $9^{19}$ is the multiplicative inverse of $9$ modulo ${25}$. Thus, $x\equiv 21(9^{19})\pmod{25.}$

  Alternately, $3^{18}$ is the multiplicative inverse of $3^2$ modulo ${25},$ so $x\equiv 21(3^{18})\equiv  (3^{19})7\pmod{25}.$
\end{example}

The previous example does not ask for the least nonnegative residue, but let's find it anyway.

\begin{example}
  Find the least nonnegative residue of $(9^{19})21$ modulo $25$.

  First, note that $(9^{19})21=(3^2)^19(21)$. From here there are two options:

  \begin{description}
    \item[Factor $21$:]
      \begin{align*}
        (9^{19})21\equiv (3^2)^19(3)(7)
        \equiv (3^{39})(7)
        \equiv (3^{20})(3^{19})(7) \pmod{25}
      \end{align*}
      By \nameref{thm:euler-FlT}, $3^{20}\equiv 1\pmod{25}$ and by \nameref{cor:inv-euler}, $3^{19}$ is the multiplicative inverse of $3$ moddulo $25.$ Since $3(-8)\equiv -24\equiv 1\pmod{25},$ $3^{19}\equiv -8\pmod{25.}$ Thus, 
      \begin{align*}
        (9^{19})21\equiv (-8)(7)\equiv -56\equiv 19 \pmod{25}.
      \end{align*}

    \item[Using $21\equiv -4\pmod{25}$:] 
      \begin{align*}
        (9^{19})21\equiv (3^2)^19(-4)
        \equiv (3^{38})(-4)
        \equiv (3^{20})(3^{18})(-4) \pmod{25}
      \end{align*}
      Since $3^{20}=3^{18}(3^2)\equiv 1\pmod{25}$ by \nameref{thm:euler-FlT}, $3^{18}$ is the multiplicative inverse of $3^2=9$ modulo $25.$ Since $9(-11)\equiv -99\equiv 1\pmod{25},$ we have $3^{18}\equiv -11\pmod{25}.$ Thus, 
      \begin{align*}
        (9^{19})21\equiv (-11)(-4)\equiv 44 \equiv 19 \pmod{25}.
      \end{align*}
  \end{description}
  
\end{example}

\begin{br}\label{br:modpq}
  Let $p,q$ be distinct primes. Prove that $p^{q-1}+q^{p-1}\equiv 1 \pmod{pq}.$
 
 
  \begin{proof} Let $p,q$ be distinct primes. 
   Then $\answer{q^{p-1}\equiv 1}\pmod{p}$ and  $\answer{p^{q-1}\equiv 1}\pmod{q}$ by Fermat's Little Theorem, and $\answer{p^{q-1}\equiv 1}\equiv0\pmod{p}$ and  $\answer{q^{p-1}\equiv 1}\equiv 0\pmod{q}$ by $\answer{\textnormal{definition}}.$
   
   Thus, $p^{q-1}+q^{p-1}\equiv \answer{1} \pmod{p}$ and $p^{q-1}+q^{p-1}\equiv \answer{1} \pmod{p}$ by $\answer{\textnormal{modular addition}}.$

   (Finish proof using definition of congruence modulo $p$ and $q$)
  \end{proof}
\end{br}

%%%%%%%%%%%%%%%%%%%%%%%%%%

\end{document}
