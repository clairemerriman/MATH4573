\documentclass{ximera}
\usepackage{amssymb, latexsym, amsmath, amsthm, graphicx, amsthm,alltt,color, listings,multicol,xr-hyper,hyperref,aliascnt,enumitem}
\usepackage{xfrac}

\usepackage{parskip}
\usepackage[,margin=0.7in]{geometry}
\setlength{\textheight}{8.5in}

\usepackage{epstopdf}

\DeclareGraphicsExtensions{.eps}
\usepackage{tikz}


\usepackage{tkz-euclide}
%\usetkzobj{all}
\tikzstyle geometryDiagrams=[rounded corners=.5pt,ultra thick,color=black]
\colorlet{penColor}{black} % Color of a curve in a plot


\usepackage{subcaption}
\usepackage{float}
\usepackage{fancyhdr}
\usepackage{pdfpages}
\newcounter{includepdfpage}
\usepackage{makecell}


\usepackage{currfile}
\usepackage{xstring}




\graphicspath{  
{./otherDocuments/}
}

\author{Claire Merriman}
\newcommand{\classday}[1]{\def\classday{#1}}

%%%%%%%%%%%%%%%%%%%%%
% Counters and autoref for unnumbered environments
% Not needed??
%%%%%%%%%%%%%%%%%%%%%
\theoremstyle{plain}


\newtheorem*{namedthm}{Theorem}
\newcounter{thm}%makes pointer correct
\providecommand{\thmname}{Theorem}

\makeatletter
\NewDocumentEnvironment{thm*}{o}
 {%
  \IfValueTF{#1}
    {\namedthm[#1]\refstepcounter{thm}\def\@currentlabel{(#1)}}%
    {\namedthm}%
 }
 {%
  \endnamedthm
 }
\makeatother


\newtheorem*{namedprop}{Proposition}
\newcounter{prop}%makes pointer correct
\providecommand{\propname}{Proposition}

\makeatletter
\NewDocumentEnvironment{prop*}{o}
 {%
  \IfValueTF{#1}
    {\namedprop[#1]\refstepcounter{prop}\def\@currentlabel{(#1)}}%
    {\namedprop}%
 }
 {%
  \endnamedprop
 }
\makeatother

\newtheorem*{namedlem}{Lemma}
\newcounter{lem}%makes pointer correct
\providecommand{\lemname}{Lemma}

\makeatletter
\NewDocumentEnvironment{lem*}{o}
 {%
  \IfValueTF{#1}
    {\namedlem[#1]\refstepcounter{lem}\def\@currentlabel{(#1)}}%
    {\namedlem}%
 }
 {%
  \endnamedlem
 }
\makeatother

\newtheorem*{namedcor}{Corollary}
\newcounter{cor}%makes pointer correct
\providecommand{\corname}{Corollary}

\makeatletter
\NewDocumentEnvironment{cor*}{o}
 {%
  \IfValueTF{#1}
    {\namedcor[#1]\refstepcounter{cor}\def\@currentlabel{(#1)}}%
    {\namedcor}%
 }
 {%
  \endnamedcor
 }
\makeatother

\theoremstyle{definition}
\newtheorem*{annotation}{Annotation}
\newtheorem*{rubric}{Rubric}

\newtheorem*{innerrem}{Remark}
\newcounter{rem}%makes pointer correct
\providecommand{\remname}{Remark}

\makeatletter
\NewDocumentEnvironment{rem}{o}
 {%
  \IfValueTF{#1}
    {\innerrem[#1]\refstepcounter{rem}\def\@currentlabel{(#1)}}%
    {\innerrem}%
 }
 {%
  \endinnerrem
 }
\makeatother

\newtheorem*{innerdefn}{Definition}%%placeholder
\newcounter{defn}%makes pointer correct
\providecommand{\defnname}{Definition}

\makeatletter
\NewDocumentEnvironment{defn}{o}
 {%
  \IfValueTF{#1}
    {\innerdefn[#1]\refstepcounter{defn}\def\@currentlabel{(#1)}}%
    {\innerdefn}%
 }
 {%
  \endinnerdefn
 }
\makeatother

\newtheorem*{scratch}{Scratch Work}


\newtheorem*{namedconj}{Conjecture}
\newcounter{conj}%makes pointer correct
\providecommand{\conjname}{Conjecture}
\makeatletter
\NewDocumentEnvironment{conj}{o}
 {%
  \IfValueTF{#1}
    {\innerconj[#1]\refstepcounter{conj}\def\@currentlabel{(#1)}}%
    {\innerconj}%
 }
 {%
  \endinnerconj
 }
\makeatother

\newtheorem*{poll}{Poll question}
\newtheorem{tps}{Think-Pair-Share}[section]


\newenvironment{obj}{
	\textbf{Learning Objectives.} By the end of class, students will be able to:
		\begin{itemize}}
		{\!.\end{itemize}
		}

\newenvironment{pre}{
	\begin{description}
	}{
	\end{description}
}


\newcounter{ex}%makes pointer correct
\providecommand{\exname}{Homework Problem}
\newenvironment{ex}[1][2in]%
{%Env start code
\problemEnvironmentStart{#1}{Homework Problem}
\refstepcounter{ex}
}
{%Env end code
\problemEnvironmentEnd
}

\newcommand{\inlineAnswer}[2][2 cm]{
    \ifhandout{\pdfOnly{\rule{#1}{0.4pt}}}
    \else{\answer{#2}}
    \fi
}


\ifhandout
\newenvironment{shortAnswer}[1][
    \vfill]
        {% Begin then result
        #1
            \begin{freeResponse}
            }
    {% Environment Ending Code
    \end{freeResponse}
    }
\else
\newenvironment{shortAnswer}[1][]
        {\begin{freeResponse}
            }
    {% Environment Ending Code
    \end{freeResponse}
    }
\fi

\let\question\relax
\let\endquestion\relax

\newtheoremstyle{ExerciseStyle}{\topsep}{\topsep}%%% space between body and thm
		{}                      %%% Thm body font
		{}                              %%% Indent amount (empty = no indent)
		{\bfseries}            %%% Thm head font
		{}                              %%% Punctuation after thm head
		{3em}                           %%% Space after thm head
		{{#1}~\thmnumber{#2}\thmnote{ \bfseries(#3)}}%%% Thm head spec
\theoremstyle{ExerciseStyle}
\newtheorem{br}{In-class Problem}

\newenvironment{sketch}
 {\begin{proof}[Sketch of Proof]}
 {\end{proof}}


\newcommand{\gt}{>}
\newcommand{\lt}{<}
\newcommand{\N}{\mathbb N}
\newcommand{\Q}{\mathbb Q}
\newcommand{\Z}{\mathbb Z}
\newcommand{\C}{\mathbb C}
\newcommand{\R}{\mathbb R}
\renewcommand{\H}{\mathbb{H}}
\newcommand{\lcm}{\operatorname{lcm}}
\newcommand{\nequiv}{\not\equiv}
\newcommand{\ord}{\operatorname{ord}}
\newcommand{\ds}{\displaystyle}
\newcommand{\floor}[1]{\left\lfloor #1\right\rfloor}
\newcommand{\legendre}[2]{\left(\frac{#1}{#2}\right)}



%%%%%%%%%%%%



\title{The Euler $\phi$-function}
\begin{document}
\begin{abstract}
\end{abstract}
\maketitle

%%%%%%%%%%%%%%%%%%%%%%%%%%

\begin{obj}
\item Use Euler's Theorem to find the least nonnegative residue modulo a composite
\item Use Euler's Theorem to find the multiplicative inverse of an integer modulo $m$
\item Prove $\phi(4)\phi(5)=\phi(20)$ using an outline that mirrors the proof that $\phi(m)\phi(n)=\phi(mn)$ when $(m,n)=1$
\end{obj}


We will also find a formula for $\phi(n)$ in general. The following exercise will outline the general proof:

\begin{br}
  Let us prove that $\phi(20)=\phi(4)\phi(5)$. First, note that $\phi(4)=\inlineAnswer[1 cm]{2}$ and $\phi(5)=\inlineAnswer[1 cm]{4}$, so we will prove $\phi(20)=\inlineAnswer[1 cm]{8}$.
  \begin{enumerate}
      \item A number $a$ is relatively prime to $20$ if and only if $a$ is relatively prime to $\inlineAnswer[1 cm]{4}$ and $\inlineAnswer[1 cm]{5}.$ 
      \begin{onlineOnly}
          The first blank should be smaller than second blank for the automatic grading to work.
      \end{onlineOnly}
      \begin{hint}
          The number in each blank should be relevant to what we are trying to show.
      \end{hint}
      \item  
      We can partition the positive integers less that or equal to $20$ into 
      \begin{align*}
      & 1\equiv\inlineAnswer[1 cm]{5}   
          \equiv\inlineAnswer[1 cm]{9}
          \equiv\inlineAnswer[1 cm]{13}
          \equiv\inlineAnswer[1 cm]{17}\pmod 4\\
      & 2\equiv\inlineAnswer[1 cm]{6}
          \equiv\inlineAnswer[1 cm]{10}
          \equiv\inlineAnswer[1 cm]{14}
          \equiv\inlineAnswer[1 cm]{18}\pmod 4\\
      & 3\equiv\inlineAnswer[1 cm]{7} 
          \equiv\inlineAnswer[1 cm]{11}
          \equiv\inlineAnswer[1 cm]{15}
          \equiv\inlineAnswer[1 cm]{19}\pmod 4\\
      & 4\equiv\inlineAnswer[1 cm]{8}
          \equiv\inlineAnswer[1 cm]{12} \equiv\inlineAnswer[1 cm]{16} \equiv\inlineAnswer[1 cm]{20}\pmod 4
      \end{align*}

      For any $b$ in the range $1,2,3,4,$ define $s_b$ to be the number of integers $a$ in the range $1,2,\dots, 20$ such that $a\equiv b \pmod 4$ and $\gcd(a,20)=1$. Thus, $s_1=\inlineAnswer[1 cm]{4}, s_2=\inlineAnswer[1 cm]{0},s_3=\inlineAnswer[1 cm]{4},$ and $s_4=\inlineAnswer[1 cm]{0}.$

      We can see that when $(b,4)=1$, $s_b=\phi(\inlineAnswer[1 cm]{4})$ and when $(b,4)>1$, $s_b=\inlineAnswer[1 cm]{0}$.

      \item $\phi(20)=s_1+s_2+s_3+s_4$. Why? 
      \begin{shortAnswer}
          Every positive integers less that or equal to $20$ is counted by exactly one $s_b.$
      \end{shortAnswer}
      
      \item We have seen that $\phi(20)=s_1+s_2+s_3+s_4$, that when $(b,4)=1$, $s_b=\inlineAnswer[1 cm]{\phi(5)},$
      \pdfOnly{\ifhandout
      \footnote{This blank is asking for a function, not a numbers.}\else\fi} 
      \begin{onlineOnly}
          This blank is asking for a function, not a number.
      \end{onlineOnly}
      and that when $(b,4)>1$, $s_b=\inlineAnswer[1 cm]{0}$. To finish the 	``proof" we show that there are $\phi(\inlineAnswer[1 cm]{4})$ integers $b$ where $(b,4)=1$. 
      Thus, we can say that $\phi(20)=\inlineAnswer{\phi(4)\phi(5)}.$ 
  \end{enumerate}

\end{br}

\begin{br}
Repeat the same proof for $m$ and $n$ where $(m,n)=1.$

  \begin{solution}
      Let $m$ and $m$ be relatively prime positive integers. A number $a$ is relatively prime to $mn$ if and only if $a$ is relatively prime to $\inlineAnswer[1 cm]{m}$ and $\inlineAnswer[1 cm]{n}.$ 
      
      
      We can partition the positive integers less that or equal to $mn$ into 
      \begin{align*}
      & 1\equiv\inlineAnswer[1 cm]{m+1}   
          \equiv\inlineAnswer[1 cm]{2m+1}
          \equiv \cdots
          \equiv\inlineAnswer[1 cm]{(n-1)m+1}\pmod m\\
      & 2 \equiv\inlineAnswer[1 cm]{m+2}   
          \equiv\inlineAnswer[1 cm]{2m+2}
          \equiv \cdots
          \equiv\inlineAnswer[1 cm]{(n-1)m+2}\pmod m\\
      & \vdots\\
      & m\equiv\inlineAnswer[1 cm]{2m}   
          \equiv\inlineAnswer[1 cm]{3m}
          \equiv \cdots
          \equiv\inlineAnswer[1 cm]{nm}\pmod m
      \end{align*}

      For any $b$ in the range $1,2,3,\dots,m,$ define $s_b$ to be the number of integers $a$ in the range $1,2,\dots, mn$ such that $a\equiv b \pmod m$ and $\gcd(a,mn)=1$. Thus, when $(b,m)=1$, $s_b=\phi(\inlineAnswer[1 cm]{m})$ and when $(b,m)>1$, $s_b=\inlineAnswer[1 cm]{0}$.

      \begin{shortAnswer}
          Since every positive integers less that or equal to $mn$ is counted by exactly one $s_b,$ $\phi(mn)=s_1+s_2+\cdots+s_m.$
      \end{shortAnswer}

      
      We have seen that $\phi(mn)=s_1+s_2+\dots+s_m$, that when $(b,m)=1$, $s_b=\inlineAnswer[1 cm]{\phi(n)},$
      \pdfOnly{\ifhandout
      \footnote{This blank is asking for a function, not a value.}\else\fi} 
      \begin{onlineOnly}
          This blank is asking for a function, not a value.
      \end{onlineOnly}
      and that when $(b,m)>1$, $s_b=\inlineAnswer[1 cm]{0}$. Since there are $\phi(\inlineAnswer[1 cm]{m})$ integers $b$ where $(b,m)=1$. 
      Thus, we can say that $\phi(mn)=\inlineAnswer{\phi(m)\phi(n)}.$ 
  \end{solution}
\end{br}

\begin{br}\label{br:arithmetic-progression}
    Complete the proof of \nameref{thm:phi-multiplicative} by proving that, if $m, n,$ and $i$ are positive integers with ($m, n) = (m, i) = 1,$ then the integers $i, m + i, 2m +i,..., (n - 1)m +i$ form a complete system of residues modulo $n.$
\end{br}

\end{document}
