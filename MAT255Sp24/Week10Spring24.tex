\documentclass[letterpaper, 11 pt]{ximera}
\usepackage{amssymb, latexsym, amsmath, amsthm, graphicx, amsthm,alltt,color, listings,multicol,hyperref,xr-hyper,aliascnt,enumitem}
\usepackage{xfrac}


\usepackage{parskip}
\usepackage{graphicx}
\usepackage[,margin=0.7in]{geometry}
\setlength{\textheight}{8.5in}
 
\usepackage{tkz-euclide}
%\usetkzobj{all}
\tikzstyle geometryDiagrams=[rounded corners=.5pt,ultra thick,color=black]
\colorlet{penColor}{black} % Color of a curve in a plot


\usepackage{subcaption}
\usepackage{float}
\usepackage{fancyhdr}
\usepackage{pdfpages}
\newcounter{includepdfpage}


\newcommand{\semester}{%
  \ifcase\month
  \or Spring %1
  \or Spring %2
  \or Spring %3
  \or Spring %4
  \or Spring  %5
  \or Fall %8
  \or Fall %9
  \or Fall %10
  \or Fall %11
  \or Fall %12
  \fi
}
\usepackage{currfile}
\usepackage{xstring}


\lhead{\large{Number Theory: MAT-255}}
%Put your Document Title (Camp: Topic) Here
\chead{}
\rhead{\semester 24}
\lfoot{}
\cfoot{}
\rfoot{Page \thepage}
\renewcommand\headrulewidth{0pt}
\renewcommand\footrulewidth{0pt}

\headheight 50pt
\headsep 30pt

\author{Claire Merriman}
\date{Spring 2024}


%%%%%%%%%%%%%%%%%%%%%
% Counters and autoref for unnumbered environments
%%%%%%%%%%%%%%%%%%%%%
\theoremstyle{plain}


\newtheorem*{namedthm}{Theorem}
\newcounter{thm}%makes pointer correct
\providecommand{\thmname}{Proposition}

\makeatletter
\NewDocumentEnvironment{thm*}{o}
 {%
  \IfValueTF{#1}
    {\namedthm[#1]\refstepcounter{thm}\def\@currentlabel{(#1)}}%
    {\namedthm}%
 }
 {%
  \endnamedthm
 }
\makeatother


\newtheorem*{namedprop}{Proposition}
\newcounter{prop}%makes pointer correct
\providecommand{\propname}{Proposition}

\makeatletter
\NewDocumentEnvironment{prop*}{o}
 {%
  \IfValueTF{#1}
    {\namedprop[#1]\refstepcounter{prop}\def\@currentlabel{(#1)}}%
    {\namedprop}%
 }
 {%
  \endnamedprop
 }
\makeatother

\newtheorem*{namedlem}{Lemma}
\newcounter{lem}%makes pointer correct
\providecommand{\lemname}{Lemma}

\makeatletter
\NewDocumentEnvironment{lem*}{o}
 {%
  \IfValueTF{#1}
    {\namedlem[#1]\refstepcounter{lem}\def\@currentlabel{(#1)}}%
    {\namedlem}%
 }
 {%
  \endnamedlem
 }
\makeatother

\newtheorem*{namedcor}{Corollary}
\newcounter{cor}%makes pointer correct
\providecommand{\corname}{Corollary}

\makeatletter
\NewDocumentEnvironment{cor*}{o}
 {%
  \IfValueTF{#1}
    {\namedcor[#1]\refstepcounter{cor}\def\@currentlabel{(#1)}}%
    {\namedcor}%
 }
 {%
  \endnamedcor
 }
\makeatother

\theoremstyle{definition}

\newtheorem*{innerrem}{Remark}
\newcounter{rem}%makes pointer correct
\providecommand{\remname}{Remark}

\makeatletter
\NewDocumentEnvironment{rem}{o}
 {%
  \IfValueTF{#1}
    {\innerrem[#1]\refstepcounter{rem}\def\@currentlabel{(#1)}}%
    {\innerrem}%
 }
 {%
  \endinnerrem
 }
\makeatother

\newtheorem*{innerdefn}{Definition}%%placeholder
\newcounter{defn}%makes pointer correct
\providecommand{\defnname}{Definition}

\makeatletter
\NewDocumentEnvironment{defn}{o}
 {%
  \IfValueTF{#1}
    {\innerdefn[#1]\refstepcounter{defn}\def\@currentlabel{(#1)}}%
    {\innerdefn}%
 }
 {%
  \endinnerdefn
 }
\makeatother

\newtheorem*{scratch}{Scratch Work}


\newtheorem*{namedconj}{Conjecture}
\newcounter{conj}%makes pointer correct
\providecommand{\conjname}{Conjecture}
\makeatletter
\NewDocumentEnvironment{conj}{o}
 {%
  \IfValueTF{#1}
    {\innerconj[#1]\refstepcounter{conj}\def\@currentlabel{(#1)}}%
    {\innerconj}%
 }
 {%
  \endinnerconj
 }
\makeatother

\newtheorem*{poll}{Poll question}
\newtheorem{tps}{Think-Pair-Share}[section]
%\newtheorem{br}{In-class Problem}[section]
\newtheorem*{cs}{Crowd Sourced Proof}

\newlist{checklist}{itemize}{2}
\setlist[checklist]{label=$\square$}

\newenvironment{obj}{
	\textbf{Learning Objectives.} By the end of class, students will be able to:
		\begin{itemize}}
		{\!.\end{itemize}
		}

\newenvironment{pre}{
	\begin{description}
	}{
	\end{description}
}


\newcounter{br}%makes pointer correct
\providecommand{\brname}{In-class Problem}

\newenvironment{br}[1][2in]%
{%Env start code
\problemEnvironmentStart{#1}{In-class Problem}
\refstepcounter{br}
}
{%Env end code
\problemEnvironmentEnd
}

\newcounter{ex}%makes pointer correct
\providecommand{\exname}{Homework Problem}
\newenvironment{ex}[1][2in]%
{%Env start code
\problemEnvironmentStart{#1}{Homework Problem}
\refstepcounter{ex}
}
{%Env end code
\problemEnvironmentEnd
}



\newenvironment{sketch}
 {\begin{proof}[Sketch of Proof]}
 {\end{proof}}
%\newenvironment{hint}
%  {\begin{proof}[Hint]}
%  {\end{proof}}

\newcommand{\gt}{>}
\newcommand{\lt}{<}
\newcommand{\N}{\mathbb N}
\newcommand{\Q}{\mathbb Q}
\newcommand{\Z}{\mathbb Z}
\newcommand{\C}{\mathbb C}
\newcommand{\R}{\mathbb R}
\renewcommand{\H}{\mathbb{H}}
\newcommand{\lcm}{\operatorname{lcm}}
\newcommand{\nequiv}{\not\equiv}
\newcommand{\ord}{\operatorname{ord}}
\newcommand{\ds}{\displaystyle}
\newcommand{\floor}[1]{\left\lfloor #1\right\rfloor}
\newcommand{\legendre}[2]{\left(\frac{#1}{#2}\right)}



%%%%%%%%%%%%



%Imports for cross references
\externaldocument{otherResults}
\externaldocument{Week1Spring24}
\externaldocument{Week2Spring24}
\externaldocument{Week3Spring24}
\externaldocument{Week4Spring24}
\externaldocument{Week5Spring24}
\externaldocument{Week6Spring24}
\externaldocument{Week7Spring24}
\externaldocument{Week8Spring24}
\externaldocument{Week9Spring24}




\StrBetween*[1,1]{\currfilename}{Week}{Sp}[\week]

\title{Week \week--MAT-255 Number Theory}

\begin{document}
\section{Monday, March 25: Quadratic residue of $-1$}
%%%%%%%%%%%%%%%%%%%%%%%%%%

\begin{obj}
    \item Prove \nameref{thm:euler-quads}
	\item Classify when $-1$ is a quadratic residue modulo an odd prime
\end{obj}


\begin{pre}
    \item[Reading] None
\end{pre}

%%%%%%%%%%%%%%%%%%%%%%%%%%
\subsection{Proof of Euler's Criterion}
%%%%%%%%%%%%%%%%%%%%%%%%%%

We will prove \nameref{thm:euler-quads}. 

\begin{thm*}[Euler's Criterion]
    Let $p$ be an odd prime and $a\in\Z$ with $p\nmid a.$ Then \[\legendre{a}{p}\equiv a^{(p-1)/2}\pmod{p}\]
\end{thm*}

\begin{proof}
	Let $p$ be an odd prime and $a\in\Z$ with $p\nmid a.$ If there exists $b\in\Z$ such that $b^2\equiv a\pmod{p},$ then $\legendre{a}{p}=1$ by \hyperref[defn:legendre]{definition}.
	Note that \[a^{(p-1)/2}\equiv (b^2){(p-1)/2}\equiv b^{p-1}\equiv 1\pmod{p}\]
	by \nameref{FlT}. Thus $\legendre{a}{p}\equiv a^{(p-1)/2}\pmod{p}.$
	
	If $a$ is a quadratic nonresidue modulo $p,$ consider the \hyperref[defn:reduced-res-sys]{reduced residue system} $\{1,2,\dots,p-1\}.$ For each element $c$ of the list, there exists a unique element $d$, also on the list, such that $cd\equiv a\pmod{p}$ by \hyperref[thm:lin-cong-solutions]{Theorem 2.6} since $(a,p)=1$. Since $a$ is a quadratic nonresidue by assumption, $c\not\equiv d\pmod{p}.$ Thus, there are $\frac{p-1}{2}$ pairs $c,d$ where $cd\equiv a\pmod{p}.$ Thus, 
		\[
			-1\equiv (p-1)! \equiv a^{(p-1)/2}\pmod{p}
		\]
	by \nameref{Wilson}. Since $a$ is a quadratic nonresidue modulo $p,$ $\legendre{a}{p}=-1\equiv a^{(p-1)/2}\pmod{p}.$
\end{proof}

\begin{remark}
	Some sources define $\legendre{a}{p}=0$ when $p\mid a.$ In this case,  Let $p$ be an odd prime and $a\in\Z$.
	If $p\mid a,$ then $a^{(p-1)/2}\equiv 0^{(p-1)/2}\equiv 0\equiv\legendre{a}{p}\pmod{p}.$
\end{remark}

%%%%%%%%%%%%%%%%%%%%%%%%%%
\subsection{When is $-1$ a quadratic residue?}
%%%%%%%%%%%%%%%%%%%%%%%%%%

\begin{thm*}[Theorem 4.6]\label{thm:residue-neg1}
	Let $p$ be an odd prime number. Then 
	\[
		\legendre{-1}{p}=
			\begin{cases}
 				1, & p\equiv 1\pmod{4}\\
				-1, & p\equiv 3\pmod{4}
			\end{cases}.
	\]
\end{thm*}

\begin{proof}
	Let $p$ be an odd prime number. Then from \nameref{thm:euler-quads}, $\legendre{-1}{p}\equiv (-1)^{(p-1)/2}\pmod{p}.$ Since both values are $\pm1,$ we can say $\legendre{-1}{p}=(-1)^{(p-1)/2}.$

	If $p\equiv 1\pmod{4},$ then there exists $k\in\Z$ such that $p=4k+1.$ Thus, $\frac{p-1}{2}=2k$ and 
		\[
			\legendre{-1}{p}=(-1)^{(p-1)/2}=(-1)^{2k}=1.
		\]
	
	If $p\equiv 3\pmod{4},$ then there exists $k\in\Z$ such that $p=4k+3.$ Thus, $\frac{p-1}{2}=2k+1$ and 
		\[
			\legendre{-1}{p}=(-1)^{(p-1)/2}=(-1)^{2k+1}=-1.
		\]
\end{proof}
%%%%%%%%%%%%%%%%%%%%%%%%%%
\section{Wednesday, March 27: Introduction to quadratic reciprocity}
%%%%%%%%%%%%%%%%%%%%%%%%%%

\begin{obj}
    \item Use quadratic reciprocity to find the Legendre symbol of an integer modulo $p$.
    \item Characterize the primes where $3,-3,5,-5,7,-7$ are quadratic residues.
\end{obj}


\begin{pre}
    \item[Reading:] Scanned notes, Pommersheim-Marks-Flapan Section 11.2 and part of 11.3.
    \item[Turn in:] \emph{Use the results from the reading} to find $\legendre{-3}{71}.$

     \begin{solution}
     	From Theorem 4.5\ref{legendre-mult}, $\legendre{-3}{71}=\legendre{-1}{71}\legendre{3}{71}.$
     	Since $71\equiv3\pmod{4},$ from \nameref{thm:residue-neg1}, $\legendre{-1}{71}=-1,$
	and from \hyperref[quad-rec-useful-form]{quadratic reciprocity}$\legendre{3}{71}=-\legendre{71}{3}.$ Putting this together, we get
	
		\begin{align*}
		 	\legendre{-3}{71}&=\legendre{-1}{71}\legendre{3}{71}=(-1)\left(-\legendre{71}{3}\right)\\
				&=\legendre{71}{3}=\legendre{2}{3}=-1
		\end{align*}
	from Theorem 4.5\ref{legendre-respects-mod} and the fact that $2$ is a quadratic nonresidue modulo $3.$
     \end{solution}
\end{pre}


%%%%%%%%%%%%%%%%%%%%%%%%%%
\subsection{Quadratic reciprocity (20 minutes)}
%%%%%%%%%%%%%%%%%%%%%%%%%%

\begin{thm*}[Quadratic Reciprocity (first statement)]\label{quad-rec-useful-form}
	Let $p$ and $q$ be distinct primes.  
	\begin{enumerate}[label=(\alph*)]
		\item If $p\equiv 1 \pmod{4}$ or $q\equiv 1\pmod{4},$ then $\legendre{p}{q}=\legendre{q}{p}.$
 		\item If $p\equiv q \equiv 3 \pmod{4},$ then $\legendre{p}{q}=-\legendre{q}{p}.$
	\end{enumerate}
\end{thm*}

It will take time to prove the lemmas for quadratic reciprocity, but we can immediately prove some results. 
The first is that this initial statement is equivalent to the more standard statement. This is our first example where one statement of a result (\nameref{quad-rec-useful-form}) is more useful in calculations and proofs, but the proof involves proving an equivalent statement (\nameref{quad-rec-standard-form}).

\begin{thm*}[Law of Quadratic Reciprocity]\label{quad-rec-standard-form} (Theorem 4.9)
	Let $p$ and $q$ be distinct primes.  Then
	\[
		\legendre{p}{q}\legendre{q}{p}=(-1)^{(p-1)/2\ (q-1)/2}=
			\begin{cases}
 				1, & \textnormal{ if } p\equiv 1\pmod{4} \textnormal{ or } q\equiv 1\pmod{4}\\
				-1, & \textnormal{ if } p\equiv q\equiv 3\pmod{4}.
			\end{cases}
	\]
\end{thm*}

\begin{remark}
	We can quickly prove the second equality, 
		\[(-1)^{(p-1)/2\ (q-1)/2}=
			\begin{cases}
 				1, & \textnormal{ if } p\equiv 1\pmod{4} \textnormal{ or } q\equiv 1\pmod{4}\\
				-1, & \textnormal{ if } p\equiv q\equiv 3\pmod{4}.
			\end{cases}\]
	If $p\equiv 1\pmod{4}$ or $q\equiv 1\pmod{4},$ then one of $\frac{p-1}{2}$ and $\frac{q-1}{2}$ is even. Thus $(-1)^{(p-1)/2\ (q-1)/2}=1.$ If $p\equiv q\equiv 3\pmod{4},$ then $\frac{p-1}{2}$ and $\frac{q-1}{2}$ are both odd. Thus $(-1)^{(p-1)/2\ (q-1)/2}=-1.$
\end{remark}

\begin{proposition}
 \nameref{quad-rec-useful-form} is equivalent to \nameref{quad-rec-standard-form}. That is, \nameref{quad-rec-useful-form} implies \nameref{quad-rec-standard-form} and \nameref{quad-rec-standard-form} implies \nameref{quad-rec-useful-form} .
\end{proposition}
\begin{proof}
	We will first show that \nameref{quad-rec-standard-form} implies \nameref{quad-rec-useful-form}.
	
	Let $p$ and $q$ be distinct primes.  Then \nameref{quad-rec-standard-form} says that 
	\[
		\legendre{p}{q}\legendre{q}{p}=
			\begin{cases}
 				1, & \textnormal{ if } p\equiv 1\pmod{4} \textnormal{ or } q\equiv 1\pmod{4}\\
				-1, & \textnormal{ if } p\equiv q\equiv 3\pmod{4}.
			\end{cases}
	\]
	Since $\legendre{p}{q}=\pm 1$ and $\legendre{q}{p}=\pm 1,$ we know that $\legendre{p}{q}\legendre{q}{p}=1$ if and only if $\legendre{p}{q}=\legendre{q}{p}.$ Thus, when $p\equiv 1\pmod{4}$ or $q\equiv 1\pmod{4},$ $\legendre{p}{q}=\legendre{q}{p}.$ Similarly, $\legendre{p}{q}\legendre{q}{p}=-1$ if and only if $\legendre{p}{q}=-\legendre{q}{p}.$ Thus, when $p\equiv q\equiv 3\pmod{4},$ $\legendre{p}{q}=-\legendre{q}{p}.$ 
	
	The other direction is on next homework assignment.
\end{proof}

Notice that we also proved the converse of \nameref{quad-rec-useful-form}:
\begin{corollary}\label{cor:quad-rec}
	Let $p$ and $q$ be distinct primes.  
	\begin{enumerate}[label=(\alph*)]
		\item If $\legendre{p}{q}=\legendre{q}{p},$ then $p\equiv 1 \pmod{4}$ or $q\equiv 1\pmod{4}.$
 		\item If $\legendre{p}{q}=-\legendre{q}{p},$ then $p\equiv q \equiv 3 \pmod{4}.$
	\end{enumerate}
\end{corollary}


%%%%%%%%%%%%%%%%%%%%%%%%%%
\subsection{Quadratic reciprocity practice (30 minutes)}
%%%%%%%%%%%%%%%%%%%%%%%%%%

\begin{br}[Strayer Chapter 4, Exercise 35]
	Let $p$ be an odd prime number. Prove the following statements the following provided outlines, which will help solve the next problem, as well.
	
	\begin{enumerate}[label=(\alph*)]
	\setcounter{enumi}{1}
		\item $\legendre{3}{p}=1$ if and only if $p\equiv \pm1\pmod{12}.$
		\item $\legendre{-3}{p}=1$ if and only if $p\equiv 1\pmod{6}.$	
	\end{enumerate}
\end{br}

\begin{proof}
	\begin{enumerate}[label=(\alph*)]
	\setcounter{enumi}{1}
		\item Since $3\equiv \answer{3}\pmod{4},$\footnote{In this problem, this step is repetitive, but it is needed when $p\neq3$.} we need two cases for \nameref{quad-rec-useful-form}. 
			\begin{enumerate}[label=(\roman*)]
 				\item If $p\equiv 1\pmod{4},$ then $\legendre{3}{p}=\answer{\legendre{p}{3}}$ by \nameref{quad-rec-useful-form}, and $\legendre{p}{3}=1$ if and only if $p\equiv\answer{1\pmod{3}}$. Then $p\equiv \answer{1}\pmod{12},$ and this is the unique equivalence class modulo $12$ by the Chinese Remainder Theorem.
		
				\item If $p\equiv 3\equiv -1\pmod{4},$ then $\legendre{3}{p}=\answer{-\legendre{p}{3}}$ by \nameref{quad-rec-useful-form}, and $\legendre{p}{3}=-1$ if and only if $p\equiv\answer{ 2\equiv-1\pmod{3}}$. Then $p\equiv \answer{-1}\pmod{12},$ and this is the unique equivalence class modulo $12$ by the Chinese Remainder Theorem.
			\end{enumerate}
		Therefore, $\legendre{3}{p}=1$ if and only if $p\equiv \pm1\pmod{12}.$ 
		
		\item From Theorem 4.25\ref{legendre-mult}, $\legendre{-3}{p}= \answer{\legendre{-1}{p}\legendre{3}{p}}.$
		Again, we have two cases.
			\begin{enumerate}[label=(\roman*)]
 				\item If $p\equiv 1\pmod{4},$ then $\legendre{-1}{p}=\answer{1}$ by \nameref{thm:residue-neg1} and
				$\legendre{3}{p}=\answer{\legendre{p}{3}}$ by \nameref{quad-rec-useful-form}. Thus, $\legendre{-3}{p}=\answer{\legendre{p}{3}}=1$ if and only if $p\equiv \answer{1\pmod{3}}.$ Then $p\equiv \answer{1}\pmod{12},$ and this is the unique equivalence class modulo $12$ by the Chinese Remainder Theorem.
		
		\item If $p\equiv 3\equiv -1\pmod{4},$ then $\legendre{-1}{p}=\answer{-1}$ by \nameref{thm:residue-neg1} and
				$\legendre{3}{p}=\answer{-\legendre{p}{3}}$ by \nameref{quad-rec-useful-form}. Thus, $\legendre{-3}{p}=\answer{\legendre{p}{3}}=1$ if and only if $p\equiv \answer{1\pmod{3}}.$ Then $p\equiv \answer{7}\pmod{12},$ and this is the unique equivalence class modulo $12$ by the Chinese Remainder Theorem.
	\end{enumerate}
	Therefore, $\legendre{-3}{p}=1$ if and only if $p\equiv \answer{1,7}\pmod{12},$ which is equivalent to $p\equiv 1\pmod{6}.$ \qedhere
	\end{enumerate}
\end{proof}

\begin{br}[Strayer Chapter 4, Exercise 36]
	Find congruences characterizing all prime numbers $p$ for which the following integers are quadratic residues modulo $p,$ as done in the previous exercise. 
	
	Outline is provided for the first part. Outline for second part added after class.
	\begin{enumerate}[label=(\alph*)]
		\item $5$ 
		\item $-5$	
		\item $7$ 
		\item $-7$	
	\end{enumerate}
\end{br}

\begin{solution}
 	\begin{enumerate}[label=(\alph*)]
		\item Since $5\equiv\answer{1}\pmod{4},$ $\answer{\legendre{5}{p}=\legendre{p}{5}}$ by \nameref{quad-rec-useful-form}. Then $\legendre{5}{p}=\answer{\legendre{p}{5}}=1$ if and only if $p\equiv\answer{ \pm1\pmod{5}}$.

		\item From Theorem 4.25\ref{legendre-mult}, $\legendre{-5}{p}= \answer{\legendre{-1}{p}\legendre{5}{p}}=\answer{\legendre{-1}{p}\legendre{p}{5}}$ by \nameref{quad-rec-useful-form}. Again, we have two cases.
			
			\begin{enumerate}[label=(\roman*)]
 				\item If $p\equiv 1\pmod{4},$ then $\legendre{-1}{p}=\answer{1}$ by \nameref{thm:residue-neg1} and
				$\legendre{5}{p}=\answer{\legendre{p}{5}}$ by \nameref{quad-rec-useful-form}. Thus, $\legendre{-5}{p}=\answer{\legendre{p}{5}}=1$ if and only if $p\equiv \answer{\pm1\pmod{5}}.$ When $p\equiv \answer{1}\pmod{5},$ we have $p\equiv\answer{1}\pmod{\answer{20}}$ and when $p\equiv \answer{-1}\pmod{5},$ we have $p\equiv\answer{9}\pmod{\answer{20}}$. In each case, there is a unique equivalence class modulo $\answer{20}$ by the Chinese Remainder Theorem.
		
		\item If $p\equiv 3\pmod{4},$ then $\legendre{-1}{p}=\answer{-1}$ by \nameref{thm:residue-neg1} and
				$\legendre{5}{p}=\answer{\legendre{p}{5}}$ by \nameref{quad-rec-useful-form}. Thus, $\legendre{-5}{p}=\answer{-\legendre{p}{5}}=1$ if and only if $p\equiv \answer{\pm 2}\pmod{5}.$ When $p\equiv \answer{2}\pmod{5},$ we have $p\equiv\answer{7}\pmod{\answer{20}}$ and when $p\equiv \answer{3}\pmod{5},$ we have $p\equiv\answer{3}\pmod{\answer{20}}$. In each case, there is a unique equivalence class modulo $\answer{20}$ by the Chinese Remainder Theorem.
	\end{enumerate}
	Therefore, $\legendre{-5}{p}=1$ if and only if $p\equiv \answer{1,3,7,9}\pmod{\answer{20}}.$\qedhere
	\end{enumerate}
\end{solution}

\section{Friday March 29: No class for Good Friday}
\end{document}