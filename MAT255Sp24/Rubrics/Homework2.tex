
\documentclass[letterpaper, 11pt]{../ximera}
\usepackage{amssymb, latexsym, amsmath, amsthm, graphicx, amsthm,alltt,color, listings,multicol,hyperref,xr-hyper,aliascnt,enumitem}

\usepackage[,margin=0.7in]{geometry}
\setlength{\textheight}{8.5in}

\usepackage{currfile}
\usepackage{xstring}

\theoremstyle{definition} 

\newtheorem{ex}{Homework Problem}


\author{Claire Merriman}
\date{Spring 2024}

%\linespread{1.5} %double spaces for easier grading/commenting
\newenvironment{writeRubric}{\textbf{Rubric:} \begin{enumerate}[leftmargin=.75in,label=\textbf{\arabic* points}]\setcounter{enumi}{-1}\item Work does not contain enough of the relevant concepts to provide feedback.}{\!\end{enumerate}}

\newenvironment{algRubric}[1]
	{\textbf{Rubric:} \textbf{#1 points} total \begin{itemize}}{\!\end{itemize}}



%\renewcommand\qedsymbol{$\blacksquare$} %uncomment to change the square at the end of the proof to a solid black square
%\renewcommand\qedsymbol{$\spadesuit$} %uncomment to change the square at the end of the proof to a spade. Not formal mathematics, but for this class it's ok to play around with this symbol within reason

 % \newcommand creates a shortcut for a commonly used command
\newcommand{\R}{\mathbb R}
\newcommand{\Z}{\mathbb Z}
\newcommand{\lcm}{\operatorname{lcm}}
\newcommand{\nequiv}{\not\equiv}
\newcommand{\ord}{\operatorname{ord}}
\newcommand{\floor}[1]{\left\lfloor #1\right\rfloor}
\newcommand{\legendre}[2]{\left(\frac{#1}{#2}\right)}

\StrBetween*[1,1]{\currfilename}{Homework}{.tex}[\homework]

\begin{document}

\chapter{MAT-255, Homework \#\homework\ Rubrics}

\section*{Proofs and writing}  %the * means this section will not be numbered
%%%%%%%%%%%%%%%%%%%%%
Strayer Exercise Set 1.1, Exercises 7, 9, 15. Strayer Exercise Set 1.2, Exercises 23, 25. 
\begin{ex}[Strayer Exercise 7]
 Let $a, b \in\Z$ with $a\mid b$. Prove that $a^n\mid b^n$ for every positive integer $n$.
\end{ex}

\begin{writeRubric}
    \item \textbf{Does not demonstrate understanding}
     Contains a reasonable attempt to prove each part, but does not meet the criteria for two points.
    \item \textbf{Needs revisions}
     
    \item \textbf{Demonstrates understanding}
    
    \item \textbf{Exemplary}
        
\end{writeRubric}
                                       \begin{proof}
 
\end{proof}

\begin{ex}[Strayer Exercise 9]
 Let $a, m$ and $n$ be positive integers with $a >1$ Prove that $a^m-1\mid a^n-1$ if and only if $m\mid n.$ (\emph{Hint:} For the ``if" direction, write $n= md$ with $d$ a positive integer and use the factorization $a^{md} d - 1= (a^m - 1) \times  (a^{m(d-1)} + a^{m(d-2)} +\cdot + a^m + 1).$]
\end{ex}

\begin{writeRubric}
    \item \textbf{Does not demonstrate understanding}
     Contains a reasonable attempt to prove each part, but does not meet the criteria for two points.
    \item \textbf{Needs revisions}
     
    \item \textbf{Demonstrates understanding}
    
    \item \textbf{Exemplary}
        
\end{writeRubric}
                                       \begin{proof}
 
\end{proof}

\begin{ex}[Strayer Exercise 15] The following exercises present two alternative versions of the division algorithm. Both versions allow negative divisors; as such, they are more general than Theorem 1.4
	\begin{enumerate}[label=(\alph*)]
		\item\label{abs_value_divide} Let $a$ and $b$ be nonzero integers. Prove that there exists a unique $q,r\in\Z$ such that 
		\[a=bq+r, \quad 0\leq r <|b|.\]
		\item Find the unique $q$ and $r$ guarenteed by the division algorithm of part \ref{abs_value_divide} above with $a=47$ and $b=-6$.
		\item\label{least_remain} Let $a$ and $b$ be nonzero integers. Prove that there exist unique $q,r\in\Z$ such that 
		\[a=bq+r,\quad -\frac{|b|}{2}<r\leq \frac{|b|}{2}.\] 
		This algorithm is called the \emph{absolute least remainders algorithm}.
		\item Find the unique $q$ and $r$ guarenteed by the division algorithm of part \ref{least_remain} above with $a=47$ and $b=-6$.
	\end{enumerate}
\end{ex}

\begin{writeRubric}
    \item \textbf{Does not demonstrate understanding}
     Contains a reasonable attempt to prove each part, but does not meet the criteria for two points.
    \item \textbf{Needs revisions}
     
    \item \textbf{Demonstrates understanding}
    
    \item \textbf{Exemplary}
        
\end{writeRubric}
                                       	
\begin{solution}
 \begin{enumerate}[label=(\alph*)]
		\item %proof for part a
		\item %solution for part b		
		\item %proof for part c
		\item %solution for part d	
\end{enumerate}

\end{solution}

\begin{ex}[Strayer Exercise 23] Prove or disprove the following conjecture, which is similar to Conjecture 1:
		\textbf{Conjecture:} There are infinitely many prime number $p$ for which $p+2$ and $p+4$ are also prime numbers.
\end{ex}

\begin{writeRubric}
    \item \textbf{Does not demonstrate understanding}
     Contains a reasonable attempt to prove each part, but does not meet the criteria for two points.
    \item \textbf{Needs revisions}
     
    \item \textbf{Demonstrates understanding}
    
    \item \textbf{Exemplary}
        
\end{writeRubric}
                                       \begin{solution}
 
\end{solution}

\begin{ex}[Strayer Exercise 25] 
		 \begin{enumerate}[label=(\alph*)]
		 	\item Prove that all odd prime numbers can be expressed as the difference of square of two successive integers.
			\item Prove that no prime numbers can be expressed as the difference of two fourth power integers (\emph{Hint:} Use the factorization tool discussed in the final paragraph of this section.)
		\end{enumerate}
\end{ex}

\begin{writeRubric}
    \item \textbf{Does not demonstrate understanding}
     Contains a reasonable attempt to prove each part, but does not meet the criteria for two points.
    \item \textbf{Needs revisions}
     
    \item \textbf{Demonstrates understanding}
    
    \item \textbf{Exemplary}
        
\end{writeRubric}
                                       

\begin{proof}
 	\begin{enumerate}[label=(\alph*)]
		\item %proof for part a
		\item %proof for part b
	\end{enumerate}
\end{proof}

\begin{ex}[Ernst Problem 2.50]
Consider the following statement: If $x\in\mathbb{Z}$ such that $x^2$ is odd, then $x$ is odd.
The items below can be assembled to form a proof of this statement, but they are currently out of order.  Put them in the proper order.
\begin{enumerate}
	\item Assume that $x$ is an even integer.
	\item We will utilize a proof by contraposition.
	\item Thus, $x^2$ is twice an integer.
	\item Since $x=2k$, we have that $x^2 =(2k)^2 =4k^2$.
	\item Since $k$ is an integer, $2k^2$ is also an integer.
	\item By the definition of even, there is an integer $k$ such that $x=2k$.
	\item We have proved the contrapositive, and hence the desired statement is true.
	\item Assume $x\in \mathbb{Z}$.
	\item By the definition of even integer, $x^2$ is an even integer.
	\item Notice that $x^2 = 2(2k^2)$.
\end{enumerate}
\end{ex}

\begin{writeRubric}
    \item \textbf{Does not demonstrate understanding}
     Contains a reasonable attempt to prove each part, but does not meet the criteria for two points.
    \item \textbf{Needs revisions}
     
    \item \textbf{Demonstrates understanding}
    
    \item \textbf{Exemplary}
        
\end{writeRubric}
                                       \begin{proof}%simply copy/paste the sentences above in the correct order
\end{proof}


	
%%%%%%%%%%%%%%%%%%%%%


\end{document}
