
\documentclass[letterpaper, 11pt]{../ximera}
\usepackage{amssymb, latexsym, amsmath, amsthm, graphicx, amsthm,alltt,color, listings,multicol,hyperref,xr-hyper,aliascnt,enumitem}

\usepackage[,margin=0.7in]{geometry}
\setlength{\textheight}{8.5in}

\usepackage{currfile}
\usepackage{xstring}

\author{Claire Merriman}
\date{Spring 2024}

%\linespread{1.5} %double spaces for easier grading/commenting
\newenvironment{writeRubric}{\textbf{Rubric:} \begin{enumerate}[leftmargin=.75in,label=\textbf{\arabic* points}]\setcounter{enumi}{-1}\item Work does not contain enough of the relevant concepts to provide feedback.}{\!\end{enumerate}}

\newenvironment{algRubric}[1]
	{\textbf{Rubric:} \textbf{#1 points} total \begin{itemize}}{\!\end{itemize}}



%\renewcommand\qedsymbol{$\blacksquare$} %uncomment to change the square at the end of the proof to a solid black square
%\renewcommand\qedsymbol{$\spadesuit$} %uncomment to change the square at the end of the proof to a spade. Not formal mathematics, but for this class it's ok to play around with this symbol within reason

 % \newcommand creates a shortcut for a commonly used command
\newcommand{\R}{\mathbb R}
\newcommand{\Z}{\mathbb Z}
\newcommand{\lcm}{\operatorname{lcm}}


\StrBetween*[1,1]{\currfilename}{Homework}{.tex}[\homework]

\begin{document}

\chapter{Homework \#\homework\ Rubrics}

\section*{Proofs and writing}  %the * means this section will not be numbered
%%%%%%%%%%%%%%%%%%%%%
Strayer Exercise Set 1.3, Exercises 36, 37, 38, 39, 40, 51. Exercises 38, 39, and 40 will be graded as one problem. 
The generalized version of Lemmas from class on Friday are exercises immediately before Exercise 51.
Strayer Exercise Set 1.5, Exercise 83, also on Paper 1.
Strayer Exercise Set 6.1, Exercise 8.
%%%%%%%%%%%%%%%%%%%%%
\begin{ex}[Strayer Chapter 1, Exercise 36]
 	\begin{enumerate}[label=(\alph*)] %enumerate creates a numbered list, label=(\alph*) says to use lower case numbers in parentheses
		\item Do there exist integers $x$ and $y$ such that $x + y = 100$ and $(x, y) = 8$?

		\item Prove that there exist infinitely many pairs of integers $x$ and $y$ such that $x + y = 87$ and $(x, y) = 3$.
	
	\end{enumerate}
\end{ex}

\begin{writeRubric}
    \item \textbf{Does not demonstrate understanding}
     Contains a reasonable attempt to prove each part, but does not meet the criteria for two points.
    \item \textbf{Needs revisions}
     
    \item \textbf{Demonstrates understanding}
    
    \item \textbf{Exemplary}
        
\end{writeRubric}
                                       \begin{proof}
  	\begin{enumerate}[label=(\alph*)] 
		\item %Solution and reasoning or proof

		\item %Proof
	
	\end{enumerate}
\end{proof}

\begin{ex}[Strayer Chapter 1, Exercise 37]
 (Transcribe problem statement)
\end{ex}

\begin{writeRubric}
    \item \textbf{Does not demonstrate understanding}
     Contains a reasonable attempt to prove each part, but does not meet the criteria for two points.
    \item \textbf{Needs revisions}
     
    \item \textbf{Demonstrates understanding}
    
    \item \textbf{Exemplary}
        
\end{writeRubric}
                                       \begin{proof}
 
\end{proof}

\begin{ex}[Strayer Chapter 1, Exercises 38-40]
 
\begin{description}%like itemize, but allows words and phrases instead of bullets without going into the margin
\item[Exercise 38] Let $a$ and $b$ be relatively prime integers. Prove that $(a+b,a-b)$ is either $1$ or $2$.
	\begin{proof}
 		
	\end{proof}

\item[Exercise 39] Let $a$ and $b$ be relatively prime integers. Find all values of $(a+2b,2a+b)$
	\begin{solution}
 		
	\end{solution}
 
\item[Exercise 40] Let $a,b\in\Z$ with $(a,4)=2$ and $(b,4)=2$. Find $(a+b,4)$ and prove that your answer is correct.
	\begin{solution}
 		
	\end{solution} 
\end{description}
\end{ex}

\begin{writeRubric}
    \item \textbf{Does not demonstrate understanding}
     Contains a reasonable attempt to prove each part, but does not meet the criteria for two points.
    \item \textbf{Needs revisions}
     
    \item \textbf{Demonstrates understanding}
    
    \item \textbf{Exemplary}
        
\end{writeRubric}
                                       
\begin{ex}
     Let $a_1,\dots,a_n\in\Z$ with $a_1\neq 0$ and let $d=(a_1,\dots,a_n).$ Show that $c\in\Z$ is a common divisor of $a_1,\dots,a_n$ if and only if $c\mid d.$ 
\end{ex}

\begin{writeRubric}
    \item \textbf{Does not demonstrate understanding}
     Contains a reasonable attempt to prove each part, but does not meet the criteria for two points.
    \item \textbf{Needs revisions}
     
    \item \textbf{Demonstrates understanding}
    
    \item \textbf{Exemplary}
        
\end{writeRubric}
                                       
\begin{ex}[Strayer Chapter 1, Exercise 51]
 Let $a_1,\dots,a_n\in\Z$ with $a_0\neq 0$.  Prove that 
	\[(a_1,\dots,a_n)=((a_1,a_2),a_3,\dots,a_n).\]
Use this method to compute the greatest common divisor of each set of integers in Exercise 34.
\end{ex}

\begin{writeRubric}
    \item \textbf{Does not demonstrate understanding}
     Contains a reasonable attempt to prove each part, but does not meet the criteria for two points.
    \item \textbf{Needs revisions}
     
    \item \textbf{Demonstrates understanding}
    
    \item \textbf{Exemplary}
        
\end{writeRubric}
                                       \begin{proof}
 
\end{proof}
\begin{solution}(Exercise 34)
 
\begin{enumerate}%enumerate creates a numbered list, label=(\alph*) says to use lower case numbers in parentheses
	\item $(18,36,63)$
	
	\item $(30,42,70)$
	
	\item $(0,51,0)$
	
	\item $(35, 55, 77)$
	
	\item $(36, 42, 54, 78)$
	
	\item $(35, 63, 70, 98)$
\end{enumerate}
\end{solution}

\begin{ex}[Strayer Chapter 1, Exercise 83]
Let $a,b\in\Z$ with $a,b>0$ and $(a,b)=1)$. Prove that the arithmetic progression 
	\[a, a+b, a+2b, \dots, a+nb,\dots\]
contains infinitely many composite numbers.
\end{ex}

\begin{writeRubric}
    \item \textbf{Does not demonstrate understanding}
     Contains a reasonable attempt to prove each part, but does not meet the criteria for two points.
    \item \textbf{Needs revisions}
     
    \item \textbf{Demonstrates understanding}
    
    \item \textbf{Exemplary}
        
\end{writeRubric}
                                       \begin{proof}
 
\end{proof}

\begin{ex}[Strayer Chapter 6, Exercise 8]
 
\end{ex}

\begin{writeRubric}
    \item \textbf{Does not demonstrate understanding}
     Contains a reasonable attempt to prove each part, but does not meet the criteria for two points.
    \item \textbf{Needs revisions}
     
    \item \textbf{Demonstrates understanding}
    
    \item \textbf{Exemplary}
        
\end{writeRubric}
                                       \begin{proof}
 
\end{proof}
%%%%%%%%%%%%%%%%%%%%%


\end{document}
