
\documentclass[letterpaper, 11pt]{../ximera}
\usepackage{amssymb, latexsym, amsmath, amsthm, graphicx, amsthm,alltt,color, listings,multicol,hyperref,xr-hyper,aliascnt,enumitem}

\usepackage[,margin=0.7in]{geometry}
\setlength{\textheight}{8.5in}

\usepackage{currfile}
\usepackage{xstring}

\author{Claire Merriman}
\date{Spring 2024}

%\linespread{1.5} %double spaces for easier grading/commenting
\newenvironment{writeRubric}{\textbf{Rubric:} \begin{enumerate}[leftmargin=.75in,label=\textbf{\arabic* points}]\setcounter{enumi}{-1}\item Work does not contain enough of the relevant concepts to provide feedback.}{\!\end{enumerate}}

\newenvironment{algRubric}[1]
	{\textbf{Rubric:} \textbf{#1 points} total \begin{itemize}}{\!\end{itemize}}



%\renewcommand\qedsymbol{$\blacksquare$} %uncomment to change the square at the end of the proof to a solid black square
%\renewcommand\qedsymbol{$\spadesuit$} %uncomment to change the square at the end of the proof to a spade. Not formal mathematics, but for this class it's ok to play around with this symbol within reason

 % \newcommand creates a shortcut for a commonly used command
\newcommand{\R}{\mathbb R}
\newcommand{\Z}{\mathbb Z}
\newcommand{\lcm}{\operatorname{lcm}}


\StrBetween*[1,1]{\currfilename}{Homework}{.tex}[\homework]

\begin{document}

\chapter{MAT-255, Homework \#\homework\ Rubrics}

\section*{Proofs and writing}  %the * means this section will not be numbered
%%%%%%%%%%%%%%%%%%%%%
Strayer Exercise Set 1.1, Exercises 5, 10, 11. Ernst Problem 2.19, Problem 2.37, then either prove or provide a counterexample for the statements. Additional problem provided below.

\begin{ex}[Strayer Exercise 5]
 Prove or disprove the following statements.
	\begin{enumerate}[label=(\alph*)] %enumerate creates a numbered list, label=(\alph*) says to use lower case numbers in parentheses
		\item If $a,b,c,$ and $d$ are integers such that if $a\mid b$ and $c\mid d$, then $a+c\mid b+d$.
		\item If $a,b,c,$ and $d$ are integers such that if $a\mid b$ and $c\mid d$, then $ac\mid bd$.
		\item If $a,b,$ and $c$ are integers such that if $a\nmid b$ and $b\nmid c$, then $a\nmid c$.
	\end{enumerate}
\end{ex}

\begin{writeRubric}
    \item \textbf{Does not demonstrate understanding}
     Contains a reasonable attempt to prove each part, but does not meet the criteria for two points.
    \item \textbf{Needs revisions}
     
    \item \textbf{Demonstrates understanding}
    
    \item \textbf{Exemplary}
        
\end{writeRubric}
                                       
\begin{solution}
 	\begin{enumerate}[label=(\alph*)] %enumerate creates a numbered list, label=(\alph*) says to use lower case numbers in parentheses
		\item %proof or counterexample for part a
		\item %proof or counterexample for part b
		\item %proof or counterexample for part c
	\end{enumerate}
\end{solution}

\begin{ex}[Strayer Exercise 10]
 	\begin{enumerate}[label=(\alph*)]
		\item Let $n\in\Z$. Prove that $3\mid n^3-n$.
		\item Let $n\in\Z$. Prove that $5\mid n^5-n$.
		\item Let $n\in\Z$. Is it true that $4\mid n^4-n$? Provide a proof or counter example.
	\end{enumerate}
\end{ex}

\begin{writeRubric}
    \item \textbf{Does not demonstrate understanding}
     Contains a reasonable attempt to prove each part, but does not meet the criteria for two points.
    \item \textbf{Needs revisions}
     
    \item \textbf{Demonstrates understanding}
    
    \item \textbf{Exemplary}
        
\end{writeRubric}
                                       
\begin{proof}
%  	\begin{enumerate}[label=(\alph*)]
% 		\item %proof for part a
% 		\item %proof for part b
% 		\item %proof or counterexample for part c
% 	\end{enumerate}
\end{proof}

\begin{ex}[Strayer Exercise 11] Use the definition of even and odd from Strayer \textbf{not} Ernst.
 \begin{enumerate}[label=(\alph*)]
 		\item Let $n\in\Z$. Prove that $n$ is an even integer if and only if $n=2m$ with $m\in\mathbb{Z}$.
		\item Let $n\in\Z$. Prove that $n$ is an odd integer if and only if $n=2m+1$ with $m\in\mathbb{Z}$.
		\item Prove that the sum and product of two even integers are even.
		\item Prove that the sum of two odd integers is even and that their product is odd.
		\item Prove that the sum of an even integer and an odd integer is odd and that their product is even.
		\item Prove that the sum of an even integer and an odd integer is odd and their product is even.
	\end{enumerate}

\end{ex}

\begin{writeRubric}
    \item \textbf{Does not demonstrate understanding}
     Contains a reasonable attempt to prove each part, but does not meet the criteria for two points.
    \item \textbf{Needs revisions}
     
    \item \textbf{Demonstrates understanding}
    
    \item \textbf{Exemplary}
        
\end{writeRubric}
                                       
\begin{proof}
%  	\begin{enumerate}[label=(\alph*)]
% 		\item %proof for part a
% 		\item %proof for part b
% 		\item %proof for part c
% 		\item %proof for part d
% 		\item %proof for part e
% 		\item %proof for part f
% 	\end{enumerate}
\end{proof}

\begin{ex}[Ernst Problem 2.19] Let $A$ represent ``6 is an even integer” and $B$ represent ``4 divides 6.” Express each of the following compound propositions in an ordinary English sentence and then determine its truth value.
\begin{enumerate}[label=\alph*.]
	 \item $A\land B$
	 \item $A\lor B$
	 \item $\neg A$
	 \item $\neg B$
	 \item $\neg (A\land B)$
	 \item $\neg(A\lor B)$
	 \item $A\Rightarrow B$
\end{enumerate}
\end{ex}

\begin{writeRubric}
    \item \textbf{Does not demonstrate understanding}
     Contains a reasonable attempt to prove each part, but does not meet the criteria for two points.
    \item \textbf{Needs revisions}
     
    \item \textbf{Demonstrates understanding}
    
    \item \textbf{Exemplary}
        
\end{writeRubric}
                                       
\begin{solution}
%   	\begin{enumerate}[label=(\alph*)]
% 		\item %statement and explanation for part a
% 		\item %statement and explanation for part b
% 		\item %statement and explanation for part c
% 		\item %statement and explanation for part d
% 		\item %statement and explanation for part e
% 		\item %statement and explanation for part f
% 		\item %statement and explanation for part g
% 	\end{enumerate}
\end{solution}


\begin{ex}[Ernst Problem 2.37]
 Let $A$ and $B$ represent the statements from Problem 2.19. Express each of the following in an ordinary English sentence.
 
\begin{enumerate}[label=(\alph*)]
 \item The converse of $A\Rightarrow B$
  \item The contrapositive of $A\Rightarrow B$
\end{enumerate}
\end{ex}

\begin{writeRubric}
    \item \textbf{Does not demonstrate understanding}
     Contains a reasonable attempt to prove each part, but does not meet the criteria for two points.
    \item \textbf{Needs revisions}
     
    \item \textbf{Demonstrates understanding}
    
    \item \textbf{Exemplary}
        
\end{writeRubric}
                                       \begin{solution}
%  Let $A$ represent ``6 is an even integer” and $B$ represent ``4 divides 6.”
%  \begin{enumerate}[label=(\alph*)]
%  \item %the statement for part a. Proof or counterexample for part a.
%   \item %the statement for part b. Proof or counterexample for part b.
% \end{enumerate} 
\end{solution}

\begin{ex}
For each of the following equation, find what real numbers $x$ make the statement true. Prove your statement. 
	\begin{enumerate}
 		\item $\lfloor x \rfloor + \lfloor x \rfloor =\lfloor 2x\rfloor$
		\item $\lfloor x + 3 \rfloor  = 3 +\lfloor x\rfloor$
		\item $\lfloor x +3 \rfloor = 	3 + x$
	\end{enumerate} 
\end{ex}

\begin{writeRubric}
    \item \textbf{Does not demonstrate understanding}
     Contains a reasonable attempt to prove each part, but does not meet the criteria for two points.
    \item \textbf{Needs revisions}
     
    \item \textbf{Demonstrates understanding}
    
    \item \textbf{Exemplary}
        
\end{writeRubric}
                                       
\begin{solution}
%  	\begin{enumerate}
%  		\item If %include condition here, 
% 		then $\lfloor x \rfloor + \lfloor x \rfloor =\lfloor 2x\rfloor$.
		
\begin{proof}
 
\end{proof}
% 		\item If %include condition here, 
% 		then $\lfloor x + 3 \rfloor  = 3 +\lfloor x\rfloor$

\begin{proof}
 
\end{proof}
% 		\item If %include condition here, 
% 		then $\lfloor x +3 \rfloor = 	3 + x$
\begin{proof}
 
\end{proof}
% \end{enumerate} 
\end{solution}
	
%%%%%%%%%%%%%%%%%%%%%


\end{document}
