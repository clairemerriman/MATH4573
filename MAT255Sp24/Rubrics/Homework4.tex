
\documentclass[letterpaper, 11pt]{../ximera}
\usepackage{amssymb, latexsym, amsmath, amsthm, graphicx, amsthm,alltt,color, listings,multicol,hyperref,xr-hyper,aliascnt,enumitem}

\usepackage[,margin=0.7in]{geometry}
\setlength{\textheight}{8.5in}

\usepackage{currfile}
\usepackage{xstring}

\author{Claire Merriman}
\date{Spring 2024}

%\linespread{1.5} %double spaces for easier grading/commenting
\newenvironment{writeRubric}{\textbf{Rubric:} \begin{enumerate}[leftmargin=.75in,label=\textbf{\arabic* points}]\setcounter{enumi}{-1}\item Work does not contain enough of the relevant concepts to provide feedback.}{\!\end{enumerate}}

\newenvironment{algRubric}[1]
	{\textbf{Rubric:} \textbf{#1 points} total \begin{itemize}}{\!\end{itemize}}



%\renewcommand\qedsymbol{$\blacksquare$} %uncomment to change the square at the end of the proof to a solid black square
%\renewcommand\qedsymbol{$\spadesuit$} %uncomment to change the square at the end of the proof to a spade. Not formal mathematics, but for this class it's ok to play around with this symbol within reason

 % \newcommand creates a shortcut for a commonly used command
\newcommand{\R}{\mathbb R}
\newcommand{\Z}{\mathbb Z}
\newcommand{\lcm}{\operatorname{lcm}}


\StrBetween*[1,1]{\currfilename}{Homework}{.tex}[\homework]

\begin{document}

\chapter{MAT-255, Homework \#\homework\ Rubrics}

\section*{Proofs and writing}  %the * means this section will not be numbered
%%%%%%%%%%%%%%%%%%%%%
Strayer Exercise Set 1.5, Exercises 70, 79.
Strayer Exercise Set 2.1, Exercises 11, 13, 14, 17.
Strayer Exercise Set 2.2, Exercise 32.
Strayer Exercise Set 2.3, Exercise 38.
Additional problem below.
%%%%%%%%%%%%%%%%%%%%%
\begin{ex}[Strayer Chapter 1, Exercise 70] 
Prove or disprove the following statements:
  	\begin{enumerate}[label=(\alph*)] 
		\item If $a,b\in\Z,\ a,b>0,$ and $a^2\mid b^3,$ then $a\mid b$. 

		\item If $a,b\in\Z,\ a,b>0,$ and $a^2\mid b^2,$ then $a\mid b$. 
		
		\item If $a\in\Z, a>0,$ $p$ is a prime number and $p^4\mid a^3,$ then $p^2\mid a.$
	
	\end{enumerate}
 \end{ex}

\begin{writeRubric}
    \item \textbf{Does not demonstrate understanding}
     Contains a reasonable attempt to prove each part, but does not meet the criteria for two points.
    \item \textbf{Needs revisions}
     
    \item \textbf{Demonstrates understanding}
    
    \item \textbf{Exemplary}
        
\end{writeRubric}
                                       \begin{proof}
  	\begin{enumerate}[label=(\alph*)] 
		\item %Proof or counterexample

		\item %Proof or counterexample
		
		\item %Proof or counterexample


	\end{enumerate}
\end{proof}

\begin{ex}[Strayer Chapter 1, Exercise 79]
 (Transcribe problem statement)
\end{ex}

\begin{writeRubric}
    \item \textbf{Does not demonstrate understanding}
     Contains a reasonable attempt to prove each part, but does not meet the criteria for two points.
    \item \textbf{Needs revisions}
     
    \item \textbf{Demonstrates understanding}
    
    \item \textbf{Exemplary}
        
\end{writeRubric}
                                       \begin{proof}
 
\end{proof}

\begin{ex}[Strayer Chapter 2, Exercise 11]
 Let $a,b\in\Z$ such that $a\equiv b\pmod{m}.$ Prove that $(a,m)=(b,m).$
\end{ex}

\begin{writeRubric}
    \item \textbf{Does not demonstrate understanding}
     Contains a reasonable attempt to prove each part, but does not meet the criteria for two points.
    \item \textbf{Needs revisions}
     
    \item \textbf{Demonstrates understanding}
    
    \item \textbf{Exemplary}
        
\end{writeRubric}
                                       
\begin{ex}[Strayer Chapter 2, Exercise 13]
  	\begin{enumerate}[label=(\alph*)] 
		\item Let $a$ be an even integer. Prove that $a^2\equiv 0\pmod{4}.$

		\item Let $a$ be an odd integer. Prove that $a^2\equiv 1\pmod{8.}$ Deduce that $a^2\pmod{4}.$ \emph{This means you must use $a^2\equiv 1\pmod{8}$ to prove $a^2\equiv1\pmod{4}$ and not another method.}
		
		\item\label{sum-sq-3} Prove that if $n$ is a positive integer such that $n\equiv 3\pmod 4,$ then $n$ cannot be written as the sum of two squares of integers.
		
		\item Prove or disprove the converse of the statement in part \ref{sum-sq-3} above.
	\end{enumerate}
\end{ex}

\begin{writeRubric}
    \item \textbf{Does not demonstrate understanding}
     Contains a reasonable attempt to prove each part, but does not meet the criteria for two points.
    \item \textbf{Needs revisions}
     
    \item \textbf{Demonstrates understanding}
    
    \item \textbf{Exemplary}
        
\end{writeRubric}
                                       \begin{proof}
   	\begin{enumerate}[label=(\alph*)] 
		\item %Proof 

		\item %Proof 
		
		\item %Proof

		\item %Proof or counterexample
	\end{enumerate}
\end{proof}

\begin{ex}[Strayer Chapter 2, Exercise 14]
Let $n$ be an odd integer not divisible by $3.$ Prove that $n^2\equiv 1\pmod{24}.$
\end{ex}

\begin{writeRubric}
    \item \textbf{Does not demonstrate understanding}
     Contains a reasonable attempt to prove each part, but does not meet the criteria for two points.
    \item \textbf{Needs revisions}
     
    \item \textbf{Demonstrates understanding}
    
    \item \textbf{Exemplary}
        
\end{writeRubric}
                                       \begin{proof}
 
\end{proof}


\begin{ex}[Strayer Chapter 2, Exercise 17]
(transcribe problem statement--also see notation for problem 16)
\end{ex}

\begin{writeRubric}
    \item \textbf{Does not demonstrate understanding}
     Contains a reasonable attempt to prove each part, but does not meet the criteria for two points.
    \item \textbf{Needs revisions}
     
    \item \textbf{Demonstrates understanding}
    
    \item \textbf{Exemplary}
        
\end{writeRubric}
                                       \begin{proof}
 
\end{proof}

\begin{ex}[Strayer Chapter 2, Exercise 32]
 Let $a^\prime$ be the inverse of $a$ modulo $m$ and let $b^\prime$ be the inverse of $b$ modulo $m$. Prove that $a^\prime b^\prime$ is the inverse of $ab$ modulo $m.$
\end{ex}

\begin{writeRubric}
    \item \textbf{Does not demonstrate understanding}
     Contains a reasonable attempt to prove each part, but does not meet the criteria for two points.
    \item \textbf{Needs revisions}
     
    \item \textbf{Demonstrates understanding}
    
    \item \textbf{Exemplary}
        
\end{writeRubric}
                                       \begin{proof}
 
\end{proof}

\begin{ex}[Strayer Chapter 2, Exercise 38]
Prove that the system of linear congruences in one variable given by 
\begin{align*}
 x&\equiv b_1 \pmod{m_1}\\
 x&\equiv b_2 \pmod{m_2}
\end{align*}
is solvable if and only if $(m_1,m_2)\mid (b_1-b_2).$ In this case, prove that the solution is unique modulo $[m_1,m_2].$
\end{ex}

\begin{writeRubric}
    \item \textbf{Does not demonstrate understanding}
     Contains a reasonable attempt to prove each part, but does not meet the criteria for two points.
    \item \textbf{Needs revisions}
     
    \item \textbf{Demonstrates understanding}
    
    \item \textbf{Exemplary}
        
\end{writeRubric}
                                       \begin{proof}
 
\end{proof}


\begin{ex}
   	\begin{enumerate}[label=(\alph*)] 
		\item\label{zero-div} Let $p$ be prime. Prove that $ax\equiv 0\pmod{p}$ implies $a\equiv 0\pmod{p}$ or $x\equiv 0\pmod{p}.$ 

		\item Give an example to show that \ref{zero-div} is false modulo a composite number.
	\end{enumerate}
\end{ex}

\begin{writeRubric}
    \item \textbf{Does not demonstrate understanding}
     Contains a reasonable attempt to prove each part, but does not meet the criteria for two points.
    \item \textbf{Needs revisions}
     
    \item \textbf{Demonstrates understanding}
    
    \item \textbf{Exemplary}
        
\end{writeRubric}
                                       \begin{solution}
    	\begin{enumerate}[label=(\alph*)] 
		\item %Proof 

		\item %Counterexample
	\end{enumerate}
\end{solution}
%%%%%%%%%%%%%%%%%%%%%


\end{document}
