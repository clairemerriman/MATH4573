\usepackage{amssymb, latexsym, amsmath, amsthm, graphicx, amsthm,alltt,color, listings,multicol,hyperref,xr-hyper,aliascnt,enumitem}

\usepackage[,margin=0.7in]{geometry}
\setlength{\textheight}{8.5in}

\usepackage{currfile}
\usepackage{xstring}

\theoremstyle{definition} 

\newtheorem{ex}{Homework Problem}


\author{Claire Merriman}
\date{Spring 2024}

%\linespread{1.5} %double spaces for easier grading/commenting
\newenvironment{writeRubric}{\textbf{Rubric:} \begin{enumerate}[leftmargin=.75in,label=\textbf{\arabic* points}]\setcounter{enumi}{-1}\item Work does not contain enough of the relevant concepts to provide feedback.}{\!\end{enumerate}}

\newenvironment{algRubric}[1]
	{\textbf{Rubric:} \textbf{#1 points} total \begin{itemize}}{\!\end{itemize}}



%\renewcommand\qedsymbol{$\blacksquare$} %uncomment to change the square at the end of the proof to a solid black square
%\renewcommand\qedsymbol{$\spadesuit$} %uncomment to change the square at the end of the proof to a spade. Not formal mathematics, but for this class it's ok to play around with this symbol within reason

 % \newcommand creates a shortcut for a commonly used command
\newcommand{\R}{\mathbb R}
\newcommand{\Z}{\mathbb Z}
\newcommand{\lcm}{\operatorname{lcm}}
\newcommand{\nequiv}{\not\equiv}
\newcommand{\ord}{\operatorname{ord}}
\newcommand{\floor}[1]{\left\lfloor #1\right\rfloor}
\newcommand{\legendre}[2]{\left(\frac{#1}{#2}\right)}