
\documentclass[letterpaper, 11pt]{../ximera}
\usepackage{amssymb, latexsym, amsmath, amsthm, graphicx, amsthm,alltt,color, listings,multicol,hyperref,xr-hyper,aliascnt,enumitem}

\usepackage[,margin=0.7in]{geometry}
\setlength{\textheight}{8.5in}

\usepackage{currfile}
\usepackage{xstring}

\theoremstyle{definition} 

\newtheorem{ex}{Homework Problem}


\author{Claire Merriman}
\date{Spring 2024}

%\linespread{1.5} %double spaces for easier grading/commenting
\newenvironment{writeRubric}{\textbf{Rubric:} \begin{enumerate}[leftmargin=.75in,label=\textbf{\arabic* points}]\setcounter{enumi}{-1}\item Work does not contain enough of the relevant concepts to provide feedback.}{\!\end{enumerate}}

\newenvironment{algRubric}[1]
	{\textbf{Rubric:} \textbf{#1 points} total \begin{itemize}}{\!\end{itemize}}



%\renewcommand\qedsymbol{$\blacksquare$} %uncomment to change the square at the end of the proof to a solid black square
%\renewcommand\qedsymbol{$\spadesuit$} %uncomment to change the square at the end of the proof to a spade. Not formal mathematics, but for this class it's ok to play around with this symbol within reason

 % \newcommand creates a shortcut for a commonly used command
\newcommand{\R}{\mathbb R}
\newcommand{\Z}{\mathbb Z}
\newcommand{\lcm}{\operatorname{lcm}}
\newcommand{\nequiv}{\not\equiv}
\newcommand{\ord}{\operatorname{ord}}
\newcommand{\floor}[1]{\left\lfloor #1\right\rfloor}
\newcommand{\legendre}[2]{\left(\frac{#1}{#2}\right)}

\StrBetween*[1,1]{\currfilename}{Homework}{.tex}[\homework]

\begin{document}

\chapter{MAT-255, Homework \#\homework\ Rubrics}

\section*{Proofs and writing}  %the * means this section will not be numbered
%%%%%%%%%%%%%%%%%%%%%
Exercise Set 2.4, Exercises 43, 44, 45

\noindent Exercise Set 2.5, Exercises 57 (using Fermat's Little Theorem), 60. 

\noindent Exercise Set 2.6,  Exercises 67 (must parallel the proof of Theorem 2.17), 71 using Theorem 2.17, 74 

\noindent Exercise Set 3.2, Exercise 11, Exercise 12 (you may use any result in Section 3.2)

\begin{ex}[Exercise 43] 
	\begin{enumerate}[label=(\alph*)]
 		\item Prove that is $p$ is an odd prime number, then $2(p-3)!\equiv -1 \pmod{p}$.
		\item Find the least nonnegative residue of $2(100!)$ modulo $103$
	\end{enumerate}
\end{ex}

\begin{writeRubric}
    \item \textbf{Does not demonstrate understanding}
     Contains a reasonable attempt to prove each part, but does not meet the criteria for two points.
    \item \textbf{Needs revisions}
     
    \item \textbf{Demonstrates understanding}
    
    \item \textbf{Exemplary}
        
\end{writeRubric}
                                       \begin{solution}
 	\begin{enumerate}[label=(\alph*)]
 		\item 
		\item 
	\end{enumerate}
\end{solution}

\begin{ex}[Exercise 44]
	Let $n\in\Z$ with $n>1.$ Prove that $n$ is a prime number if and only if $n-2)!\equiv 1\pmod{n}.$
\end{ex}

\begin{writeRubric}
    \item \textbf{Does not demonstrate understanding}
     Contains a reasonable attempt to prove each part, but does not meet the criteria for two points.
    \item \textbf{Needs revisions}
     
    \item \textbf{Demonstrates understanding}
    
    \item \textbf{Exemplary}
        
\end{writeRubric}
                                       \begin{proof}
 
\end{proof}
 
\begin{ex}[Exercise 45]
	Let $n$ be a composite integer greater than 4. Prove that $(n-1)!\equiv 0\pmod{n}$.
\end{ex}

\begin{writeRubric}
    \item \textbf{Does not demonstrate understanding}
     Contains a reasonable attempt to prove each part, but does not meet the criteria for two points.
    \item \textbf{Needs revisions}
     
    \item \textbf{Demonstrates understanding}
    
    \item \textbf{Exemplary}
        
\end{writeRubric}
                                       \begin{proof}
 
\end{proof}

\begin{ex}[Exercise 57, using Fermat's Little Theorem] 
Let $n$ be an integer. Prove each congruence below. 
	\begin{enumerate}[label=(\alph*)]
		\item $n^{21} \equiv n \pmod{30}$
		\item $n^7 \equiv n \pmod{42}$
		\item $n^{13} \equiv n \pmod{2730}$. 
	\end{enumerate}
\end{ex}

\begin{writeRubric}
    \item \textbf{Does not demonstrate understanding}
     Contains a reasonable attempt to prove each part, but does not meet the criteria for two points.
    \item \textbf{Needs revisions}
     
    \item \textbf{Demonstrates understanding}
    
    \item \textbf{Exemplary}
        
\end{writeRubric}
                                       \begin{proof}
 	\begin{enumerate}[label=(\alph*)]
		\item 
		\item 
		\item 
	\end{enumerate}
\end{proof}
	
\begin{ex}[Exercise 60] 
	Let $p$ and $q$ be distinct prime numbers with $p-1\mid q-1$.  If $a\in\Z$ with $(a,pq)=1,$ prove that $a^{q-1}\equiv 1 \pmod{pq}$.
\end{ex}

\begin{writeRubric}
    \item \textbf{Does not demonstrate understanding}
     Contains a reasonable attempt to prove each part, but does not meet the criteria for two points.
    \item \textbf{Needs revisions}
     
    \item \textbf{Demonstrates understanding}
    
    \item \textbf{Exemplary}
        
\end{writeRubric}
                                       \begin{proof}
 	%See class February 8
\end{proof}

\begin{ex}[Exercise 67] 
	Prove that $9^8\equiv 1 \pmod{16}$ following the steps in the proof of Euler's Theorem (Theorem 2.17)
\end{ex}

\begin{writeRubric}
    \item \textbf{Does not demonstrate understanding}
     Contains a reasonable attempt to prove each part, but does not meet the criteria for two points.
    \item \textbf{Needs revisions}
     
    \item \textbf{Demonstrates understanding}
    
    \item \textbf{Exemplary}
        
\end{writeRubric}
                                       \begin{proof}
 
\end{proof}

\begin{ex}[Exercise 71, using Theorem 2.17 and the Chinese Remainder Theorem] 
	\begin{enumerate}[label=(\alph*)]
 		\item Let $n$ be an integer not divisible by $3$. Prove that $n^7\equiv n\pmod{63}$.
		\item Let $n$ be an integer divisible by $9$. Prove that $n^7\equiv n \pmod{63}$.
	\end{enumerate}
\end{ex}

\begin{writeRubric}
    \item \textbf{Does not demonstrate understanding}
     Contains a reasonable attempt to prove each part, but does not meet the criteria for two points.
    \item \textbf{Needs revisions}
     
    \item \textbf{Demonstrates understanding}
    
    \item \textbf{Exemplary}
        
\end{writeRubric}
                                       \begin{proof}
	\begin{enumerate}[label=(\alph*)]
 		\item 
		\item 
	\end{enumerate} 
\end{proof}

\begin{ex}[Exercise 74]
	Let $p$ be an odd prime number. Prove that
	\[\left\{\frac{-(p-1)}{2}, \frac{-(p-3)}{2},\dots, -2,-1,1,2,\dots,\frac{p-2}{2},\frac{p-1}{2}\right\}\]
	is a reduced residue system modulo $p$.
\end{ex}

\begin{writeRubric}
    \item \textbf{Does not demonstrate understanding}
     Contains a reasonable attempt to prove each part, but does not meet the criteria for two points.
    \item \textbf{Needs revisions}
     
    \item \textbf{Demonstrates understanding}
    
    \item \textbf{Exemplary}
        
\end{writeRubric}
                                       \begin{proof}
 
\end{proof}


\begin{ex}[Chapter 3, Exercise 11]
 	Complete the proof of Theorem 3.2 by proving that if $m,n$ and $i$ are positive integers with $(m,n)=(m,i)=1,$ then teh integers $i,m+i,2m+i,\dots,(n-1)m+i$ form a complete system of residues modulo $n$.
\end{ex}

\begin{writeRubric}
    \item \textbf{Does not demonstrate understanding}
     Contains a reasonable attempt to prove each part, but does not meet the criteria for two points.
    \item \textbf{Needs revisions}
     
    \item \textbf{Demonstrates understanding}
    
    \item \textbf{Exemplary}
        
\end{writeRubric}
                                       \begin{proof}
 
\end{proof}

\begin{ex}[Chapter 3, Exercise 12]
 	Let $n\in\Z$ with $n>1.$ If $p_1^{a_1}p_2^{a_2}\cdots p_m^{a_m}$ is the prime factorization of $n,$ prove that 
	\[\phi(n)=p_1^{a_1-1}p_2^{a_2-1}\cdots p_m^{a_m-1}\prod_{i=1}^m(p_i-1).\]
\end{ex}

\begin{writeRubric}
    \item \textbf{Does not demonstrate understanding}
     Contains a reasonable attempt to prove each part, but does not meet the criteria for two points.
    \item \textbf{Needs revisions}
     
    \item \textbf{Demonstrates understanding}
    
    \item \textbf{Exemplary}
        
\end{writeRubric}
                                       


\end{document}
