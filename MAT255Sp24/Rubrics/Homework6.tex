
\documentclass[letterpaper, 11pt]{../ximera}
\usepackage{amssymb, latexsym, amsmath, amsthm, graphicx, amsthm,alltt,color, listings,multicol,hyperref,xr-hyper,aliascnt,enumitem}

\usepackage[,margin=0.7in]{geometry}
\setlength{\textheight}{8.5in}

\usepackage{currfile}
\usepackage{xstring}

\author{Claire Merriman}
\date{Spring 2024}

%\linespread{1.5} %double spaces for easier grading/commenting
\newenvironment{writeRubric}{\textbf{Rubric:} \begin{enumerate}[leftmargin=.75in,label=\textbf{\arabic* points}]\setcounter{enumi}{-1}\item Work does not contain enough of the relevant concepts to provide feedback.}{\!\end{enumerate}}

\newenvironment{algRubric}[1]
	{\textbf{Rubric:} \textbf{#1 points} total \begin{itemize}}{\!\end{itemize}}



%\renewcommand\qedsymbol{$\blacksquare$} %uncomment to change the square at the end of the proof to a solid black square
%\renewcommand\qedsymbol{$\spadesuit$} %uncomment to change the square at the end of the proof to a spade. Not formal mathematics, but for this class it's ok to play around with this symbol within reason

 % \newcommand creates a shortcut for a commonly used command
\newcommand{\R}{\mathbb R}
\newcommand{\Z}{\mathbb Z}
\newcommand{\lcm}{\operatorname{lcm}}


\StrBetween*[1,1]{\currfilename}{Homework}{.tex}[\homework]

\begin{document}

\chapter{Homework \#\homework\ Rubrics}

\section*{Proofs and writing}  %the * means this section will not be numbered
%%%%%%%%%%%%%%%%%%%%%
Exercise Set 5.1, Exercises 5, 6, 7, 8 (check transcribed problems for mistakes)

\noindent Additional problems below

\begin{ex}[Chapter 5, Exercise 5] 
	\begin{enumerate}[label=(\alph*)]
 		\item\label{order_mult} Let $m$ be a positive integer and let $a$ and $b$ be integers relatively prime to $m$ with $(\ord_m a, \ord_m b)=1$. Prove that $\ord_m(ab)=(\ord_m a)(\ord_m b)$.
		\item Show that the hypothesis $(\ord_m a, \ord_m b)=1$ cannot be eliminated from part (a). What can be said about $\ord_m(ab)$ if $(\ord_m a, \ord_m b)\neq1$?
	\end{enumerate}
 
\end{ex}	

\begin{writeRubric}
    \item \textbf{Does not demonstrate understanding}
     Contains a reasonable attempt to prove each part, but does not meet the criteria for two points.
    \item \textbf{Needs revisions}
     
    \item \textbf{Demonstrates understanding}
    
    \item \textbf{Exemplary}
        
\end{writeRubric}
                                       	
\begin{solution}
 	\begin{enumerate}[label=(\alph*)]
 		\item
		\item 
	\end{enumerate}

\end{solution}
	
\begin{ex}[Chapter 5, Exercise 6]
	Let $m$ be a positive integer and let $d\mid \phi(m)$ with $d>0$. Prove or disprove that there exists $a\in\Z$ with $\ord_m a=d$.
\end{ex}

\begin{writeRubric}
    \item \textbf{Does not demonstrate understanding}
     Contains a reasonable attempt to prove each part, but does not meet the criteria for two points.
    \item \textbf{Needs revisions}
     
    \item \textbf{Demonstrates understanding}
    
    \item \textbf{Exemplary}
        
\end{writeRubric}
                                       	
\begin{solution}
 
\end{solution}

\begin{ex}[Chapter 5, Exercise 7] Let $m$ be a positive integer and let $a\in\Z$ with $(a,m)=1$.
	\begin{enumerate}[label=(\alph*)]
 		\item Prove that if $\ord_m a= xy$ (with $x$ and $y$ positive integers), then $\ord_m(a^x)=y$.
		\item Prove that if $\ord_m a =m-1$, then $m$ is a prime number.
	\end{enumerate}
\end{ex}

\begin{writeRubric}
    \item \textbf{Does not demonstrate understanding}
     Contains a reasonable attempt to prove each part, but does not meet the criteria for two points.
    \item \textbf{Needs revisions}
     
    \item \textbf{Demonstrates understanding}
    
    \item \textbf{Exemplary}
        
\end{writeRubric}
                                       	
\begin{proof}
 	\begin{enumerate}[label=(\alph*)]
 		\item 
		\item 
	\end{enumerate}
\end{proof}

\begin{ex}[Chapter 5, Exercise 8] 
	Let $a$ and $n$ be positive integers with $a>1$. Prove that $n\mid \phi(a^n-1)$. (\emph{Hint:} Consider $\ord_{(a^n-1)} a$). 
\end{ex}

\begin{writeRubric}
    \item \textbf{Does not demonstrate understanding}
     Contains a reasonable attempt to prove each part, but does not meet the criteria for two points.
    \item \textbf{Needs revisions}
     
    \item \textbf{Demonstrates understanding}
    
    \item \textbf{Exemplary}
        
\end{writeRubric}
                                       
	
\begin{solution}
 
\end{solution}

\begin{ex}
    Prove the following statement, which is the converse of Proposition 5.3 (restricted to primes): 
	\begin{prop*}
 	    Let $p$ be prime, and let $a\in\Z.$ If every $b\in\Z$ such that $p\nmid b$ is congruent to a power of $a$ modulo $p,$ then ${a}$ is a primitive root modulo $p$.

	\end{prop*}
\end{ex}

\begin{writeRubric}
    \item \textbf{Does not demonstrate understanding}
     Contains a reasonable attempt to prove each part, but does not meet the criteria for two points.
    \item \textbf{Needs revisions}
     
    \item \textbf{Demonstrates understanding}
    
    \item \textbf{Exemplary}
        
\end{writeRubric}
                                       
\begin{proof}
 
\end{proof}

\begin{ex}
    Prove the following generalization of \nameref{order_mult}\ref{order_mult}:
    
    
    \begin{lem*}
        Let $n\in\Z$ and let $x_1,x_2,\dots,x_m$ be reduced residues modulo $n$.  Suppose that for all $i\neq j,$ $\ord_n(x_i)$ and $\ord_n(x_j)$ are relatively prime. Then \[\ord_n(x_1 x_2\cdots x_m)=(\ord_n x_1)(\ord_n x_2)\cdots (\ord_n x_m).\]
    \end{lem*}
\end{ex}

\begin{writeRubric}
    \item \textbf{Does not demonstrate understanding}
     Contains a reasonable attempt to prove each part, but does not meet the criteria for two points.
    \item \textbf{Needs revisions}
     
    \item \textbf{Demonstrates understanding}
    
    \item \textbf{Exemplary}
        
\end{writeRubric}
                                       
\begin{proof}
 
\end{proof}

\begin{ex}
	If $p$ is a prime and $r$ is a primitive root modulo $p,$ is $-r$ also a primitive root modulo $p$? Prove or provide a counter example.
\end{ex}

\begin{writeRubric}
    \item \textbf{Does not demonstrate understanding}
     Contains a reasonable attempt to prove each part, but does not meet the criteria for two points.
    \item \textbf{Needs revisions}
     
    \item \textbf{Demonstrates understanding}
    
    \item \textbf{Exemplary}
        
\end{writeRubric}
                                       
\begin{solution}
 
\end{solution}

\begin{ex}
	If $p$ is a prime with $p\equiv 1\pmod{4}$ and $r$ is a primitive root modulo $p,$ is $-r$ also a primitive root modulo $p$? Prove or provide a counter example.
\end{ex}

\begin{writeRubric}
    \item \textbf{Does not demonstrate understanding}
     Contains a reasonable attempt to prove each part, but does not meet the criteria for two points.
    \item \textbf{Needs revisions}
     
    \item \textbf{Demonstrates understanding}
    
    \item \textbf{Exemplary}
        
\end{writeRubric}
                                       
\begin{solution}
 
\end{solution}

\begin{ex}
	Let $p$ be prime and $a\in\Z$ such that $p\nmid a.$ Suppose that $\ord_p a> \frac{p-1}{2}.$ Prove that $a$ is a primitive root modulo $p.$
\end{ex}

\begin{writeRubric}
    \item \textbf{Does not demonstrate understanding}
     Contains a reasonable attempt to prove each part, but does not meet the criteria for two points.
    \item \textbf{Needs revisions}
     
    \item \textbf{Demonstrates understanding}
    
    \item \textbf{Exemplary}
        
\end{writeRubric}
                                       
\begin{proof}
 
\end{proof}

%%%%%%%%%%%%%%%%%%%%%


\end{document}
