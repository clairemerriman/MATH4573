
\documentclass[letterpaper, 11pt]{../ximera}
\usepackage{amssymb, latexsym, amsmath, amsthm, graphicx, amsthm,alltt,color, listings,multicol,hyperref,xr-hyper,aliascnt,enumitem}

\usepackage[,margin=0.7in]{geometry}
\setlength{\textheight}{8.5in}

\usepackage{currfile}
\usepackage{xstring}

\author{Claire Merriman}
\date{Spring 2024}

%\linespread{1.5} %double spaces for easier grading/commenting
\newenvironment{writeRubric}{\textbf{Rubric:} \begin{enumerate}[leftmargin=.75in,label=\textbf{\arabic* points}]\setcounter{enumi}{-1}\item Work does not contain enough of the relevant concepts to provide feedback.}{\!\end{enumerate}}

\newenvironment{algRubric}[1]
	{\textbf{Rubric:} \textbf{#1 points} total \begin{itemize}}{\!\end{itemize}}



%\renewcommand\qedsymbol{$\blacksquare$} %uncomment to change the square at the end of the proof to a solid black square
%\renewcommand\qedsymbol{$\spadesuit$} %uncomment to change the square at the end of the proof to a spade. Not formal mathematics, but for this class it's ok to play around with this symbol within reason

 % \newcommand creates a shortcut for a commonly used command
\newcommand{\R}{\mathbb R}
\newcommand{\Z}{\mathbb Z}
\newcommand{\lcm}{\operatorname{lcm}}


\StrBetween*[1,1]{\currfilename}{Homework}{.tex}[\homework]

\begin{document}

\chapter{Homework \#\homework\ Rubrics}

\section*{Proofs and writing}  %the * means this section will not be numbered
%%%%%%%%%%%%%%%%%%%%%
Exercise Set 5.1, Exercises 5, 6, 7, 8 (check transcribed problems for mistakes)

\noindent Additional problems below

\begin{ex}[Chapter 5, Exercise 5] 
	\begin{enumerate}[label=(\alph*)]
 		\item\label{order_mult} Let $m$ be a positive integer and let $a$ and $b$ be integers relatively prime to $m$ with $(\ord_m a, \ord_m b)=1$. Prove that $\ord_m(ab)=(\ord_m a)(\ord_m b)$.
		\item Show that the hypothesis $(\ord_m a, \ord_m b)=1$ cannot be eliminated from part (a). What can be said about $\ord_m(ab)$ if $(\ord_m a, \ord_m b)\neq1$?
	\end{enumerate}
 
\end{ex}	

\begin{writeRubric}
    \item \textbf{Does not demonstrate understanding}
     Proof contains enough of the relevant concepts to provide feedback, but does not meet the criteria for 2 points. Or contains circular reasoning.
    \item \textbf{Needs revisions}
    Work shows understanding of the definition of $\operatorname{ord}_n(a),$ but it has significant gaps or errors. Proves that $\ord_m(ab)\mid(\ord_m(a))(\ord_m(b)),$ but not that $\ord_m(a)\mid(\ord_m(ab))(\ord_m(b))$ (or other case), with mathematically correct answer and example for part (b). Or some missing details in proof for part (a) with no example in part (b). Writing may be difficult to follow. It needs further review and significant revisions.
     
    \item \textbf{Demonstrates understanding}
    Mathematically correct proof, with some gaps or errors in one direction, possibly with minor arithmetic, spelling, or grammatical errors. Or uses informal mathematical writing.
    
    \item \textbf{Exemplary} Mathematically correct solutions for both parts, using the definition of $\operatorname{ord}_n(a)$ and exponent rules. Gives and example for part (b) with an answer related to the results in this section. Work is easy to follow with formal mathematical writing.
        
\end{writeRubric}
                                       	
\begin{solution}
 	\begin{enumerate}[label=(\alph*)]
 		\item
		\item 
	\end{enumerate}

\end{solution}
	
\begin{ex}[Chapter 5, Exercise 6]
	Let $m$ be a positive integer and let $d\mid \phi(m)$ with $d>0$. Prove or disprove that there exists $a\in\Z$ with $\ord_m a=d$.
\end{ex}

\begin{writeRubric}
    \item \textbf{Does not demonstrate understanding}
     Contains a reasonable attempt to prove each part, but does not meet the criteria for two points.
     
    \item \textbf{Needs revisions}
    Proof contains enough of the relevant concepts to provide feedback, but does not meet the criteria for 2 points. Or contains circular reasoning.
 \item Work shows understanding of the relationship between $\ord_m (a)$ and $\phi(m)$, with significant missing details. Example is incorrect.
     
    \item \textbf{Demonstrates understanding}
    Example of $m$ and $d\mid\phi(m)$ where there are no elements of order $d$, but no justification for the fact there are no elements of order $d$.
 Or uses informal mathematical writing.

    \item \textbf{Exemplary}
       Mathematically correct solutions for both parts, using the definition of $\operatorname{ord}_n(a)$ and exponent rules. Work is easy to follow with formal mathematical writing.
\end{writeRubric}
                                       	
\begin{solution}
 
\end{solution}

\begin{ex}[Chapter 5, Exercise 7] Let $m$ be a positive integer and let $a\in\Z$ with $(a,m)=1$.
	\begin{enumerate}[label=(\alph*)]
 		\item Prove that if $\ord_m a= xy$ (with $x$ and $y$ positive integers), then $\ord_m(a^x)=y$.
		\item Prove that if $\ord_m a =m-1$, then $m$ is a prime number.
	\end{enumerate}
\end{ex}

\begin{writeRubric}
    \item \textbf{Does not demonstrate understanding}
          Proof contains enough of the relevant concepts to provide feedback, but does not meet the criteria for 2 points. Proofs that use Fermat's Little Theorem to conclude $m$ is prime cannot receive more than 1 point--pseudoprimes are a counterexample to this statement.
          
    \item \textbf{Needs revisions}
	Work shows understanding of the relationship between $\ord_m (a)$ and $\phi(m)$, with significant missing details. Proves that $\ord_m(a^x)\mid y,$ but not that $\ord_m(a^x)=y$, with mathematically correct proof for part (b).
 
    \item \textbf{Demonstrates understanding}
    Mathematically correct proof, with some gaps or errors in one part, possibly with minor arithmetic, spelling, or grammatical errors. Or uses informal mathematical writing.
    
    \item \textbf{Exemplary}
    Mathematically correct solutions for both parts, using the definition of $\operatorname{ord}_n(a)$ and exponent rules. Work is easy to follow with formal mathematical writing.
\end{writeRubric}
                                       	
\begin{proof}
 	\begin{enumerate}[label=(\alph*)]
 		\item 
		\item 
	\end{enumerate}
\end{proof}

\begin{ex}[Chapter 5, Exercise 8] 
	Let $a$ and $n$ be positive integers with $a>1$. Prove that $n\mid \phi(a^n-1)$. (\emph{Hint:} Consider $\ord_{(a^n-1)} a$). 
\end{ex}

\begin{writeRubric}
    \item \textbf{Does not demonstrate understanding}
     Contains a reasonable attempt to prove each part, but does not meet the criteria for two points.
    \item \textbf{Needs revisions}
    States that $n=\ord_{a^n-1}a$, but only proves that $n\mid \ord_{a^n-1}a$. Does not explain
 the relationship between $\ord_{a^n-1}a$ and $\phi(a^n-1)$. 

     
    \item \textbf{Demonstrates understanding}
    
    \item \textbf{Exemplary}
        
\end{writeRubric}
                                       
	
\begin{solution}
 
\end{solution}

\begin{ex}
    Prove the following statement, which is the converse of Proposition 5.3 (restricted to primes): 
	\begin{proposition}
 	    Let $p$ be prime, and let $a\in\Z.$ If every $b\in\Z$ such that $p\nmid b$ is congruent to a power of $a$ modulo $p,$ then ${a}$ is a primitive root modulo $p$.

	\end{proposition}
\end{ex}

\begin{writeRubric}
    \item \textbf{Does not demonstrate understanding}
     Contains a reasonable attempt to prove each part, but does not meet the criteria for two points.
    \item \textbf{Needs revisions}
     
    \item \textbf{Demonstrates understanding}
    
    \item \textbf{Exemplary}
        
\end{writeRubric}
                                       
\begin{proof}
 
\end{proof}

\begin{ex}
    Prove the following generalization of \nameref{order_mult}\ref{order_mult}:
    
    
    \begin{lemma}
        Let $n\in\Z$ and let $x_1,x_2,\dots,x_m$ be reduced residues modulo $n$.  Suppose that for all $i\neq j,$ $\ord_n(x_i)$ and $\ord_n(x_j)$ are relatively prime. Then \[\ord_n(x_1 x_2\cdots x_m)=(\ord_n x_1)(\ord_n x_2)\cdots (\ord_n x_m).\]
    \end{lemma}
\end{ex}

\begin{writeRubric}
    \item \textbf{Does not demonstrate understanding}
     Contains a reasonable attempt to prove each part, but does not meet the criteria for two points.
    \item \textbf{Needs revisions}
     
    \item \textbf{Demonstrates understanding}
    
    \item \textbf{Exemplary}
        
\end{writeRubric}
                                       
\begin{proof}
 
\end{proof}

\begin{ex}
	If $p$ is a prime and $r$ is a primitive root modulo $p,$ is $-r$ also a primitive root modulo $p$? Prove or provide a counter example.
\end{ex}

\begin{writeRubric}
    \item \textbf{Does not demonstrate understanding}
     Contains a reasonable attempt to prove each part, but does not meet the criteria for two points.
    \item \textbf{Needs revisions}
     
    \item \textbf{Demonstrates understanding}
    
    \item \textbf{Exemplary}
        
\end{writeRubric}
                                       
\begin{solution}
 
\end{solution}

\begin{ex}
	If $p$ is a prime with $p\equiv 1\pmod{4}$ and $r$ is a primitive root modulo $p,$ is $-r$ also a primitive root modulo $p$? Prove or provide a counter example.
\end{ex}

\begin{writeRubric}
    \item \textbf{Does not demonstrate understanding}
     Contains a reasonable attempt to prove each part, but does not meet the criteria for two points.
    \item \textbf{Needs revisions}
     
    \item \textbf{Demonstrates understanding}
    
    \item \textbf{Exemplary}
        
\end{writeRubric}
                                       
\begin{solution}
 
\end{solution}

\begin{ex}
	Let $p$ be prime and $a\in\Z$ such that $p\nmid a.$ Suppose that $\ord_p a> \frac{p-1}{2}.$ Prove that $a$ is a primitive root modulo $p.$
\end{ex}

\begin{writeRubric}
    \item \textbf{Does not demonstrate understanding}
     Contains a reasonable attempt to prove each part, but does not meet the criteria for two points.
    \item \textbf{Needs revisions}
     
    \item \textbf{Demonstrates understanding}
    
    \item \textbf{Exemplary}
        
\end{writeRubric}
                                       
\begin{proof}
 
\end{proof}

%%%%%%%%%%%%%%%%%%%%%


\end{document}
