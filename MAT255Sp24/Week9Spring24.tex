\documentclass[letterpaper, 11 pt]{ximera}
\usepackage{amssymb, latexsym, amsmath, amsthm, graphicx, amsthm,alltt,color, listings,multicol,hyperref,xr-hyper,aliascnt,enumitem}
\usepackage{xfrac}


\usepackage{parskip}
\usepackage{graphicx}
\usepackage[,margin=0.7in]{geometry}
\setlength{\textheight}{8.5in}
 
\usepackage{tkz-euclide}
%\usetkzobj{all}
\tikzstyle geometryDiagrams=[rounded corners=.5pt,ultra thick,color=black]
\colorlet{penColor}{black} % Color of a curve in a plot


\usepackage{subcaption}
\usepackage{float}
\usepackage{fancyhdr}
\usepackage{pdfpages}
\newcounter{includepdfpage}


\newcommand{\semester}{%
  \ifcase\month
  \or Spring %1
  \or Spring %2
  \or Spring %3
  \or Spring %4
  \or Spring  %5
  \or Fall %8
  \or Fall %9
  \or Fall %10
  \or Fall %11
  \or Fall %12
  \fi
}
\usepackage{currfile}
\usepackage{xstring}


\lhead{\large{Number Theory: MAT-255}}
%Put your Document Title (Camp: Topic) Here
\chead{}
\rhead{\semester 24}
\lfoot{}
\cfoot{}
\rfoot{Page \thepage}
\renewcommand\headrulewidth{0pt}
\renewcommand\footrulewidth{0pt}

\headheight 50pt
\headsep 30pt

\author{Claire Merriman}
\date{Spring 2024}


%%%%%%%%%%%%%%%%%%%%%
% Counters and autoref for unnumbered environments
%%%%%%%%%%%%%%%%%%%%%
\theoremstyle{plain}


\newtheorem*{namedthm}{Theorem}
\newcounter{thm}%makes pointer correct
\providecommand{\thmname}{Proposition}

\makeatletter
\NewDocumentEnvironment{thm*}{o}
 {%
  \IfValueTF{#1}
    {\namedthm[#1]\refstepcounter{thm}\def\@currentlabel{(#1)}}%
    {\namedthm}%
 }
 {%
  \endnamedthm
 }
\makeatother


\newtheorem*{namedprop}{Proposition}
\newcounter{prop}%makes pointer correct
\providecommand{\propname}{Proposition}

\makeatletter
\NewDocumentEnvironment{prop*}{o}
 {%
  \IfValueTF{#1}
    {\namedprop[#1]\refstepcounter{prop}\def\@currentlabel{(#1)}}%
    {\namedprop}%
 }
 {%
  \endnamedprop
 }
\makeatother

\newtheorem*{namedlem}{Lemma}
\newcounter{lem}%makes pointer correct
\providecommand{\lemname}{Lemma}

\makeatletter
\NewDocumentEnvironment{lem*}{o}
 {%
  \IfValueTF{#1}
    {\namedlem[#1]\refstepcounter{lem}\def\@currentlabel{(#1)}}%
    {\namedlem}%
 }
 {%
  \endnamedlem
 }
\makeatother

\newtheorem*{namedcor}{Corollary}
\newcounter{cor}%makes pointer correct
\providecommand{\corname}{Corollary}

\makeatletter
\NewDocumentEnvironment{cor*}{o}
 {%
  \IfValueTF{#1}
    {\namedcor[#1]\refstepcounter{cor}\def\@currentlabel{(#1)}}%
    {\namedcor}%
 }
 {%
  \endnamedcor
 }
\makeatother

\theoremstyle{definition}

\newtheorem*{innerrem}{Remark}
\newcounter{rem}%makes pointer correct
\providecommand{\remname}{Remark}

\makeatletter
\NewDocumentEnvironment{rem}{o}
 {%
  \IfValueTF{#1}
    {\innerrem[#1]\refstepcounter{rem}\def\@currentlabel{(#1)}}%
    {\innerrem}%
 }
 {%
  \endinnerrem
 }
\makeatother

\newtheorem*{innerdefn}{Definition}%%placeholder
\newcounter{defn}%makes pointer correct
\providecommand{\defnname}{Definition}

\makeatletter
\NewDocumentEnvironment{defn}{o}
 {%
  \IfValueTF{#1}
    {\innerdefn[#1]\refstepcounter{defn}\def\@currentlabel{(#1)}}%
    {\innerdefn}%
 }
 {%
  \endinnerdefn
 }
\makeatother

\newtheorem*{scratch}{Scratch Work}


\newtheorem*{namedconj}{Conjecture}
\newcounter{conj}%makes pointer correct
\providecommand{\conjname}{Conjecture}
\makeatletter
\NewDocumentEnvironment{conj}{o}
 {%
  \IfValueTF{#1}
    {\innerconj[#1]\refstepcounter{conj}\def\@currentlabel{(#1)}}%
    {\innerconj}%
 }
 {%
  \endinnerconj
 }
\makeatother

\newtheorem*{poll}{Poll question}
\newtheorem{tps}{Think-Pair-Share}[section]
%\newtheorem{br}{In-class Problem}[section]
\newtheorem*{cs}{Crowd Sourced Proof}

\newlist{checklist}{itemize}{2}
\setlist[checklist]{label=$\square$}

\newenvironment{obj}{
	\textbf{Learning Objectives.} By the end of class, students will be able to:
		\begin{itemize}}
		{\!.\end{itemize}
		}

\newenvironment{pre}{
	\begin{description}
	}{
	\end{description}
}


\newcounter{br}%makes pointer correct
\providecommand{\brname}{In-class Problem}

\newenvironment{br}[1][2in]%
{%Env start code
\problemEnvironmentStart{#1}{In-class Problem}
\refstepcounter{br}
}
{%Env end code
\problemEnvironmentEnd
}

\newcounter{ex}%makes pointer correct
\providecommand{\exname}{Homework Problem}
\newenvironment{ex}[1][2in]%
{%Env start code
\problemEnvironmentStart{#1}{Homework Problem}
\refstepcounter{ex}
}
{%Env end code
\problemEnvironmentEnd
}



\newenvironment{sketch}
 {\begin{proof}[Sketch of Proof]}
 {\end{proof}}
%\newenvironment{hint}
%  {\begin{proof}[Hint]}
%  {\end{proof}}

\newcommand{\gt}{>}
\newcommand{\lt}{<}
\newcommand{\N}{\mathbb N}
\newcommand{\Q}{\mathbb Q}
\newcommand{\Z}{\mathbb Z}
\newcommand{\C}{\mathbb C}
\newcommand{\R}{\mathbb R}
\renewcommand{\H}{\mathbb{H}}
\newcommand{\lcm}{\operatorname{lcm}}
\newcommand{\nequiv}{\not\equiv}
\newcommand{\ord}{\operatorname{ord}}
\newcommand{\ds}{\displaystyle}
\newcommand{\floor}[1]{\left\lfloor #1\right\rfloor}
\newcommand{\legendre}[2]{\left(\frac{#1}{#2}\right)}



%%%%%%%%%%%%



%Imports for cross references
\externaldocument{otherResults}
\externaldocument{Week1Spring24}
\externaldocument{Week2Spring24}
\externaldocument{Week3Spring24}
\externaldocument{Week4Spring24}
\externaldocument{Week5Spring24}
\externaldocument{Week6Spring24}
\externaldocument{Week7Spring24}
\externaldocument{Week8Spring24}

%
%\theoremstyle{definition}
%\newtheorem*{annotation}{Annotation}



\StrBetween*[1,1]{\currfilename}{Week}{Sp}[\week]

\title{Week \week--MAT-255 Number Theory}

\begin{document}
\section{Monday, March 18: Proof of \nameref{thm:prime-roots}}
%%%%%%%%%%%%%%%%%%%%%%%%%%

\begin{obj}
    \item Find the number of roots of unity modulo $m$
    \item Prove primitive roots exist modulo a prime
\end{obj}


\begin{pre}
    \item[Reading] None
\end{pre}

%%%%%%%%%%%%%%%%%%%%%%%%%%
\subsection{Roots of unity (35 minutes)}
%%%%%%%%%%%%%%%%%%%%%%%%%%

Finish proof of \nameref{prop:roots-unity}


\begin{br}\label{br:condition-root-unity}
    Let $p$ be prime, $m$ a positive integer, and $d=(m,p-1).$ Prove that $a^m\equiv 1\pmod{p}$ if and only if $a^d\equiv 1\pmod{p}.$


    \begin{solution}
        Let $p$ be prime, $m$ a positive integer, and $d=(m,p-1).$ Let $a\in\Z$. If $p\mid a,$ then $a^i\answer{\equiv 0\pmod{p}}$ for all positive integers $i$. 
        Thus, we are only considering $a\in\Z$ such that $p\nmid a.$
        Otherwise, $a^{p-1}\equiv 1\pmod{p}$ by $\answer{\textnormal{\nameref{FlT}}}$.
        
        By \nameref{prop:order_divides_phi}, $a^m\equiv 1\pmod{p}$ if and only if $\answer{\ord_p a \mid m}$. Similarly, $\answer{a^{p-1}\equiv 1\pmod{p}}$ if and only if $\answer{\ord_p a \mid p-1}$. Thus, $\answer{\ord_p a}$ is a common divisor of $\answer{m}$ and $\answer{p-1}$. Combining \nameref{lem:gcd_mult} and \nameref{lem:gcd_trans} gives $\ord_p a$ is a common divisor of   $\answer{m}$ and $\answer{p-1}$ if and only if $\ord_p a\mid d$. One final application of \nameref{prop:order_divides_phi} gives $\answer{\ord_p a\mid d}$ if and only if $\answer{a^d\equiv 1\pmod{p}}$.
    \end{solution}
\end{br}

\begin{br}\label{br:more-roots-unity}
        Let $p$ be prime and $m$ a positive integer. Prove that 
        \[
            x^m\equiv 1\pmod{p}
        \]
        has exactly $(m,p-1)$ incongruent solutions modulo $p.$


        \begin{proof}
            Let $p$ be prime, $m$ a positive integer, and $d=(m,p-1).$ 
            By \autoref{br:condition-root-unity}, $x^m\equiv 1\pmod{p}$ if and only if $x^d\equiv 1\pmod{p}$. By \nameref{prop:roots-unity} there are exactly $d$ solutions to $x^d\equiv 1\pmod{p}.$ Thus, there are exactly $d$ solutions to $x^m\equiv 1\pmod{p}.$
        \end{proof}
\end{br}

%%%%%%%%%%%%%%%%%%%%%%%%%%
\subsection{Primitive roots modulo a prime (15 minutes)}
%%%%%%%%%%%%%%%%%%%%%%%%%%

We will now prove the existence of primivitive roots modulo a prime combining the two methods from the reading: we will show that when $d\mid p-1,$ there are $\phi(d)$ incongruent integers of order $d$ modulo $p,$ like Strayer. However, we will prove this using the method from \nameref{read-lem:prime-power-order} instead of results from Chapter 3.

\begin{thm*}[Theorem 5.9]\label{thm:number-element-order-d}
    Let $p$ be a prime and let $d\in\Z$ with $d>0$ and $d\mid p-1.$ Then there are exactly $\phi(d)$ incongruent integers of order $d$ modulo $p.$
\end{thm*}


\begin{proof}
    Let $p$ be a prime and let $d\in\Z$ with $d>0$ and $d\mid p-1.$ First we will prove the theorem for $d=q^s$ modulo $p$ where $q$ is prime and $s$ is a nonnegative integer.

    By \nameref{prop:roots-unity}, there are exactly $q^s$ incongruent solutions to 
    \begin{equation}\label{eq:unity-2a}
        x^{q^s}\equiv 1\pmod{p}
    \end{equation} and exactly $q^{s-1}$ incongruent solutions to 
    \begin{equation}\label{eq:unity-2b}
        x^{q^{s-1}}\equiv 1\pmod{p}.
    \end{equation}
    Since $(x^{q^{s-1}})^q=x^{q^s},$ all solutions to \eqref{eq:unity-2b} are solutions to \eqref{eq:unity-2a}. 
    Thus, there are exactly $q^s-q^{s-1}=q^{s-1}(q-1)$ integers $a$ where $a^{q^s}\equiv 1\pmod{p}$ and $a^{q^{s-1}}\not\equiv 1\pmod{p}.$ Thus, by \nameref{prop:order_divides_phi}, $\ord_p a \mid q^s$ and $\ord_p a\nmid q^{s-1}.$ Since $q$ is prime, $\ord_p a =q^s.$ By \nameref{thm:phi-prime-power}, $\phi(q^s)=q^s-q^{s-1}=q^{s-1}(q-1),$ so we have shown there are $\phi(q^s)$ incongruent integers with order $q^s$ modulo $p$.

    Now we will prove the general case. Let 
    \[d=q_1^{s_1}q_2^{s_2}\cdots q_k^{s_k}\]
    for distinct primes $q_1,q_2,\dots,q_k$ and positive integers $s_1,s_2,\dots,s_k.$ Let $a_1,a_2,\dots,a_k$ be elements of order $q_1^{s_1},q_2^{s_2},\dots, q_k^{s_k}$ respectively.
    %Since there are $\phi(q_1^{s_1})$ options for $a_1,$ $\phi(q_2^{s_2})$ options for $a_2,$ etc there are $\phi(q_1^{s_1})\phi(q_2^{s_2})\cdots\phi(q_k^{s_k})=\phi(d)$ ways to form $a=a_1a_2\cdots a_k.$ 
    Consider $a=a_1a_2\cdots a_k$ and $a^2, a^3,\dots,a^d$. By Homework 6, Problem 6, $a$ has order $q_1^{s_1}q_2^{s_2}\cdots q_k^{s_k}=d.$ 
    By \nameref{prop:roots-unity}, there are exactly $d$ solutions to $x^d\equiv 1\pmod{p}$. \begin{annotation}
        This is where we ended class on Monday.
    \end{annotation}
    Thus, $a, a^2,\dots,a^d$ are all incongruent solutions to $x^d\equiv 1\pmod{p}$ by \nameref{prop:order_divides_phi}. 
    By \nameref{prop:compare-order},
    $\ord_p a^i=\frac{d}{(d,i)}=d$ if and only if $(d,i)=1.$ Since there are $\phi(d)$ such integers $i,$ there are in fact $\phi(d)$ incongruent integers with order $d$ modulo $p.$
\end{proof}


\begin{cor*}[Corollary 5.10]\label{cor:number-prime-roots}
    Let $p$ be prime. There are exactly $\phi(p-1)$ primitive roots modulo $p.$
\end{cor*}


%%%%%%%%%%%%%%%%%%%%%%%%%%
\section{Wednesday, March 20: Introduction to quadratic residues}
%%%%%%%%%%%%%%%%%%%%%%%%%%

\begin{obj}
    \item Define a quadratic residue modulo $m$
    \item Prove that the quadratic congruence $x^2\equiv a\pmod{p}$ has zero or one solution modulo a prime when $p\nmid a$
    \item Use the solution to a quadratic congruence modulo a prime to find the other solution
\end{obj}


\begin{pre}
    \item[Reading:] Strayer Section 4.1
    \item[Turn in:] Exercise 3
     Find all incongruent solutions of the quadratic congruence $x^2\equiv 1\pmod{8}.$ Is it not true that quadratic congruences have either no solutions or exactly two incongruent solutions? Explain.

     \begin{solution}
        As we have seen on many previous questions, $x^2\equiv 1\pmod{8}$ for all odd numbers. So there are $4$ incongruent solutions modulo $8$, which is not a contradiction because $8$ is not an odd prime number.
     \end{solution}
\end{pre}

%%%%%%%%%%%%%%%%%%%%%%%%%%
\subsection{Finish proof of the existence of primitive roots modulo a prime (10 minutes)}
%%%%%%%%%%%%%%%%%%%%%%%%%%

%%%%%%%%%%%%%%%%%%%%%%%%%%
\subsection{Quadratic residues (40 minutes)}
%%%%%%%%%%%%%%%%%%%%%%%%%%


\begin{definition}[quadratic residue]\label{defn:quad-residue}
    Let $a,m\in\Z$ with $m>0$ and $(a,m)=1.$ The $a$ is said to be a \emph{quadratic residue modulo $m$} if the quadratic congruence $x^2\equiv a\pmod{m}$ is solvable in $\Z.$ Otherwise, $a$ is said to be a \emph{quadratic nonresidue modulo $m$}.
\end{definition}

\begin{remark}
    When finding squares modulo $m,$ we only need to check up to $\frac{m}{2},$ since $(-a)^2=a^2$ and $m-a\equiv -a\pmod{m}$
\end{remark}

\begin{br}
    Find all incongruent quadratic residues and nonresidues modulo $2,3,4,5,6,7,8,$ and $9$.
    % \begin{enumerate}
    %     \item $3$
    %     \item $4$
    %     \item $5$
    %     \item $7$
    %     \item $6$
    %     \item $8$
    %     \item $9$
    %     \item $10$
    % \end{enumerate}


    \begin{solution}
        I also included solutions modulo $10,11,12$

        \begin{tabular}{p{1.5cm}|p{4cm}p{3cm}p{3cm}}
            Modulus & least nonnegative reduced residues & quadratic residues & quadratic nonresidues \\\hline
            $2$ & $1$   
                & $1$ 
                & N/A \\
            $3$ & $\answer{1,2}$ 
                & $\answer{1}$
                & $\answer{2}$\\
            $4$ & $\answer{1,3}$ 
                & $\answer{1}$
                & $\answer{3}$\\
            $5$ & $\answer{1,2,3,4}$
                & $\answer{1,4}$
                & $\answer{2,3}$\\
            $6$ & $\answer{1,5}$
                & $\answer{1}$
                & $\answer{5}$\\
            $7$ & $\answer{1,2,3,4,5}$
                & $\answer{1,2,4}$
                & $\answer{3,5,6}$\\
            $8$ & $\answer{1,3,5,7}$
                & $\answer{1}$
                & $\answer{3,5,7}$\\
            $9$ & $\answer{1,2,4,5,7,8}$
                & $\answer{1,4,7}$
                & $\answer{2,4,8}$\\
            $10$ & $\answer{1,3,7,9}$
                & $\answer{1,9}$
                & $\answer{3,7}$\\
            $11$ & $\answer{1,2,3,4,5,6,7,8,9,10}$
                & $\answer{1,3,4,5,9}$
                & $\answer{2,6,7,8,10}$\\
            $12$ & $\answer{1,5,7,11}$
                & $\answer{1}$
                & $\answer{5,7,11}$\\
        \end{tabular}
    \end{solution}
\end{br}

\begin{lem*}[Generalized Porism 4.2]\label{lem:roots-pairs}
    Let $a,m\in\Z$ with $m>0$ and $(a,m)=1.$ If the quadratic congruence $x^2\equiv a\pmod{m}$ is solvable, say with $x=x_0,$ then  $m-x_0$ is also a solution. If $m\gt 2,$ then $x_0\not\equiv m-x_0\pmod{m},$ and solutions occur in pairs.
\end{lem*}

\begin{proof}
    Let $a,m\in\Z$ with $m>0$ and $(a,m)=1.$ If the quadratic congruence $x^2\equiv a\pmod{m}$ is solvable, say with $x=x_0.$ Then 
    \[(m-x_0)^2\equiv (-x_0)^2\equiv x_0^2\equiv a\pmod{m}.\]
    
    If $x_0\equiv m-x_0\pmod{m},$ then $2x_0\equiv m\equiv 0 \pmod{m}$ and $m\mid 2x_0$ by definition. Since $(a,m)=1,$ it must be that $(x_0,m)=1$ since $(x_0,m)\mid(a,m).$ Thus, $m\mid 2,$ so $m=2.$ Therefore, when  $m\gt 2,$ then $x_0\not\equiv m-x_0\pmod{m},$ and solutions occur in pairs.
\end{proof}


\begin{remark}
    Since $x_0\equiv m-x_0\pmod{m}$ implies $x_0\equiv \frac{m}{2},$ we can say that if $x^2\equiv a\pmod{m}$ is solvable and $\frac{m}{2}$ is \emph{not} a solution, then solutions occur in pairs.
\end{remark}

\begin{prop*}[Proposition 4.1]\label{prop:number-sqrts}
    Let $p$ be an odd prime number and let $a\in\Z$ with $p\mid a.$ Then the quadratic congruence $x^2\equiv a\pmod{p}$ has either no solutions or exactly two incongruent solutions modulo $p$.
\end{prop*}

\begin{proof}
	Let $p$ be an odd prime number and let $a\in\Z$ with $p\mid a.$ Consider the quadratic congruence $x^2\equiv a\pmod{p}.$ If no solutions exist, we are done.
	
	If solutions to the quadratic congruence exist, then \nameref{lem:roots-pairs} says that there are at least two solutions, since $p>2.$ \nameref{thm:lagrange} says that there are at most two solutions to $x^2-a\equiv 0\pmod{p}$ and therefore $x^2\equiv a\pmod{p}.$ Thus, there are exactly two incongruent solutions modulo $p.$
\end{proof}

\begin{prop*}[Proposition 4.3]\label{prop:number-quad-residues}
	Let $p$ be an odd prime number. Then there are exactly $\frac{p-1}{2}$ incongruent quadratic residues modulo $p$ and exactly $\frac{p-1}{2}$ incongruent quadratic nonresidues modulo $p.$
\end{prop*}
\begin{proof}
	Consider the $p-1$ quadratic congruences
	\begin{align*}
 		x^2&\equiv 1\pmod{p}\\
		x^2&\equiv 2\pmod{p}\\
		&\vdots\\
		x^2&\equiv p-1\pmod{p}.
	\end{align*}
	Since each congruence has either zero or two incongruent solutions modulo $p$ by \nameref{prop:number-sqrts}, and no integer is a solution to more than one of the congruences, exactly half are solvable. Therefore,  there are exactly $\frac{p-1}{2}$ incongruent quadratic residues modulo $p$ and exactly $\frac{p-1}{2}$ incongruent quadratic nonresidues modulo $p.$
\end{proof}

%%%%%%%%%%%%%%%%%%%%%%%%%%
\section{Friday, March 22: Legendre symbol}
%%%%%%%%%%%%%%%%%%%%%%%%%%

\begin{obj}
    \item Define the Legendre symbol
    \item Prove basic facts about the Legendre symbol
    \item Use the definition and basic facts to find the Legendre symbol for specific examples
\end{obj}


\begin{pre}
    \item[Reading:] Strayer Section 4.2 through Example 4
    \item[Turn in:] Exercise 12
     Use Euler's Criterion to evaluate the following Legendre symbols 
	\begin{enumerate}
 		\item $\left(\frac{11}{23}\right)$
		
		\begin{solution}
 			$\left(\frac{11}{23}\right)\equiv 11^{(23-1)/2}\equiv 11^{11}\pmod{23}$ By Euler's Criterion. Then
			\[11^{11}\equiv (11^{2})^{5}(11)\equiv 6^5(11)\equiv (6^2)(6^3)(11)\equiv (13)(9)(11)\equiv(-90)(11)\equiv -1\pmod{23}\]
		\end{solution}
		
		\item $\left(\frac{-6}{11}\right)$
		
		\begin{solution}
 			$\left(\frac{-6}{11}\right)\equiv (-6)^{(11-1)/2}\equiv (-6)^{5}\pmod{11}$ By Euler's Criterion. Then
			\[(-6)^{5}\equiv ((6)^{2})^{2}(-6)\equiv 3^2(-6)\equiv -54 \equiv 1\pmod{11}\]
		\end{solution}
	\end{enumerate}
 

    \end{pre}
\end{document}