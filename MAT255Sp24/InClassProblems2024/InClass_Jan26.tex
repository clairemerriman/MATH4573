\documentclass[handout]{ximera}
\usepackage{amsmath,multicol,amsthm,alltt,color, listings,xr-hyper,hyperref}
\usepackage{xparse}

\usepackage{parskip}
\usepackage[,margin=0.7in]{geometry}
\setlength{\textheight}{8.5in}

%%%fonts
%\usepackage{euler}
\usepackage{pbsi} %% Answer font

\usepackage{epstopdf}

\DeclareGraphicsExtensions{.eps}

%
%\usepackage{tkz-euclide}
%%\usetkzobj{all}
%\tikzstyle geometryDiagrams=[rounded corners=.5pt,ultra thick,color=black]
%\colorlet{penColor}{black} % Color of a curve in a plot


\usepackage{subcaption}
\usepackage{float}
\usepackage{fancyhdr}
%\usepackage{pdfpages}
%\newcounter{includepdfpage}
\usepackage{makecell}

%
%\usepackage{currfile}
%\usepackage{xstring}


\lhead{\large{Number Theory: MAT-255}}
\chead{}
\rhead{Spring 2024}
\lfoot{}
\cfoot{}
\rfoot{Page \thepage}
\renewcommand\headrulewidth{0pt}
\renewcommand\footrulewidth{0pt}

\headheight 50pt
\headsep 30pt

\author{Claire Merriman}

%%%%%%%%%%%%%%%%%%%%%
% Create handoutstyle for in class assignments
%%%%%%%%%%%%%%%%%%%%%
\makeatletter
 \newcommand\handoutstyle{%
  \def\activitystyle{activity-handout}
  \def\maketitle{\addtocounter{titlenumber}{1}%
  \addcontentsline{toc}{section}{\@date}%
        \setcounter{br}{0}}
 }

\newcommand{\handoutAbstract}{\begin{abstract}
\end{abstract}}
\makeatother

%%%%%%%%%%%%%%%%%%%%%
% Counters and autoref for unnumbered environments
%%%%%%%%%%%%%%%%%%%%%
\theoremstyle{plain}


\newtheorem*{namedthm}{Theorem}
\newcounter{thm}%makes pointer correct
\providecommand{\thmname}{Proposition}

\makeatletter
\NewDocumentEnvironment{thm*}{o}
 {%
  \IfValueTF{#1}
    {\namedthm[#1]\refstepcounter{thm}\def\@currentlabel{(#1)}}%
    {\namedthm}%
 }
 {%
  \endnamedthm
 }
\makeatother


\newtheorem*{namedprop}{Proposition}
\newcounter{prop}%makes pointer correct
\providecommand{\propname}{Proposition}

\makeatletter
\NewDocumentEnvironment{prop*}{o}
 {%
  \IfValueTF{#1}
    {\namedprop[#1]\refstepcounter{prop}\def\@currentlabel{(#1)}}%
    {\namedprop}%
 }
 {%
  \endnamedprop
 }
\makeatother

\newtheorem*{namedlem}{Lemma}
\newcounter{lem}%makes pointer correct
\providecommand{\lemname}{Lemma}

\makeatletter
\NewDocumentEnvironment{lem*}{o}
 {%
  \IfValueTF{#1}
    {\namedlem[#1]\refstepcounter{lem}\def\@currentlabel{(#1)}}%
    {\namedlem}%
 }
 {%
  \endnamedlem
 }
\makeatother

\newtheorem*{namedcor}{Corollary}
\newcounter{cor}%makes pointer correct
\providecommand{\corname}{Corollary}

\makeatletter
\NewDocumentEnvironment{cor*}{o}
 {%
  \IfValueTF{#1}
    {\namedcor[#1]\refstepcounter{cor}\def\@currentlabel{(#1)}}%
    {\namedcor}%
 }
 {%
  \endnamedcor
 }
\makeatother

\theoremstyle{definition}
\newtheorem*{annotation}{Annotation}
\newtheorem*{rubric}{Rubric}

\newtheorem*{innerrem}{Remark}
\newcounter{rem}%makes pointer correct
\providecommand{\remname}{Remark}

\makeatletter
\NewDocumentEnvironment{rem}{o}
 {%
  \IfValueTF{#1}
    {\innerrem[#1]\refstepcounter{rem}\def\@currentlabel{(#1)}}%
    {\innerrem}%
 }
 {%
  \endinnerrem
 }
\makeatother

\newtheorem*{innerdefn}{Definition}%%placeholder
\newcounter{defn}%makes pointer correct
\providecommand{\defnname}{Definition}

\makeatletter
\NewDocumentEnvironment{defn}{o}
 {%
  \IfValueTF{#1}
    {\innerdefn[#1]\refstepcounter{defn}\def\@currentlabel{(#1)}}%
    {\innerdefn}%
 }
 {%
  \endinnerdefn
 }
\makeatother

\newtheorem*{scratch}{Scratch Work}


\newtheorem*{namedconj}{Conjecture}
\newcounter{conj}%makes pointer correct
\providecommand{\conjname}{Conjecture}
\makeatletter
\NewDocumentEnvironment{conj}{o}
 {%
  \IfValueTF{#1}
    {\innerconj[#1]\refstepcounter{conj}\def\@currentlabel{(#1)}}%
    {\innerconj}%
 }
 {%
  \endinnerconj
 }
\makeatother

%\let\br\relax
%\let\endbr\relax

%\newcounter{br}%makes pointer correct
%\counterwithin{br}{section}
%
%\newenvironment{br}[1][2in]%
%{%Env start code
%\problemEnvironmentStart{#1}{In-class Problem}
%\refstepcounter{br}
%\stepcounter{problem}
%}
%{%Env end code
%\problemEnvironmentEnd
%}

\let\question\relax
\let\endquestion\relax

\newtheoremstyle{ExerciseStyle}{\topsep}{\topsep}%%% space between body and thm
		{}                      %%% Thm body font
		{}                              %%% Indent amount (empty = no indent)
		{\bfseries}            %%% Thm head font
		{}                              %%% Punctuation after thm head
		{3em}                           %%% Space after thm head
		{{#1}~\thmnumber{#2}\thmnote{ \bfseries(#3)}}%%% Thm head spec
\theoremstyle{ExerciseStyle}
\newtheorem{br}{In-class Problem}


\newcounter{ex}%makes pointer correct
\providecommand{\exname}{Homework Problem}
\newenvironment{ex}[1][2in]%
{%Env start code
\problemEnvironmentStart{#1}{Homework Problem}
\refstepcounter{ex}
}
{%Env end code
\problemEnvironmentEnd
}

\newcommand{\inlineAnswer}[2][2 cm]{
    \ifhandout{\pdfOnly{\rule{#1}{0.4pt}}}
    \else{\answer{#2}}
    \fi
}

\ifhandout
\newenvironment{shortAnswer}[1][
    \vfill]
        {% Begin then result
        #1
            \begin{freeResponse}
            }
    {% Environment Ending Code
    \end{freeResponse}
    }
\else
\newenvironment{shortAnswer}[1][]
        {\begin{freeResponse}
            }
    {% Environment Ending Code
    \end{freeResponse}
    }
\fi

\newenvironment{sketch}
 {\begin{proof}[Sketch of Proof]}
 {\end{proof}}


\newcommand{\gt}{>}
\newcommand{\lt}{<}
\newcommand{\N}{\mathbb N}
\newcommand{\Q}{\mathbb Q}
\newcommand{\Z}{\mathbb Z}
\newcommand{\C}{\mathbb C}
\newcommand{\R}{\mathbb R}
\renewcommand{\H}{\mathbb{H}}
\newcommand{\lcm}{\operatorname{lcm}}
\newcommand{\nequiv}{\not\equiv}
\newcommand{\ord}{\operatorname{ord}}
\newcommand{\ds}{\displaystyle}
\newcommand{\floor}[1]{\left\lfloor #1\right\rfloor}
\newcommand{\legendre}[2]{\left(\frac{#1}{#2}\right)}



%%%%%%%%%%%%




\title{January 26, 2024}

\begin{document}
\handoutAbstract
\maketitle
    \begin{center}%
        {\large \scshape MAT-255-- Number Theory \hfill Spring 2024 \hfill In Class Work January 26}%
    
        {\large
            Your Name: \hrulefill \quad Group Members:\hrulefill \quad \hrulefill
	    \par}%
    \end{center}%
  
Use the first principle of mathematical induction to prove each statement.


\begin{br}[Ernst Theorem 4.5]
    For all $n\in\mathbb{N}$, 3 divides $4^{n}-1$.

    \begin{proof}
        We proceed by induction. The base case is $n=1$. Since $\inlineAnswer{3\mid 4^1-1},$
        we are done.


        The induction hypothesis is that if $k \geq 1$ and $n = k,$ then  $\inlineAnswer{3\mid 4^k-1}.$ We want to show that $\inlineAnswer{3\mid 4^{k+1}-1}.$

        \begin{shortAnswer}[Complete the proof:\vfill]
            Then by the definition of divides, there exists $m$ such that $3m=4^k-1$. Rewriting this equation, we get $3m+1=4^k$. Multiplying both sides by $4$ gives $4(3m)+4=4^{k+1}$, or $3(4m+1)=4^{k+1}-1$. Therefore, $3\mid 4^{k+1}-1$.
        \end{shortAnswer}
    \end{proof}
\end{br}


\begin{br}[Ernst Theorem 4.7]
    Let $p_{1}, p_{2}, \ldots, p_{n}$ be $n$ distinct points arranged on a circle.  Then the number of line segments joining all pairs of points is $\frac{n^{2}-n}{2}$.

    \begin{proof}
        We proceed by induction. The base case is $n=1$. Since 
        \begin{shortAnswer}[\hrulefill]
            There are $6\frac{1^{2}-1}{2}=0$ line segments connecting the only point
        \end{shortAnswer}
        we are done.

        The induction hypothesis is that if $k \geq 1$ and $n = k,$ then 
        \begin{shortAnswer}
            there are $\frac{k^{2}-k}{2}$ line segments joining all pairs of distinct points $p_{1}, p_{2}, \ldots, p_{k}$ arranged on a circle.
        \end{shortAnswer}

        We want to show that
        \begin{shortAnswer}
            there are $\frac{(k+1)^{2}-(k+1)}{2}$ line segments joining all pairs of distinct points $p_{1}, p_{2}, \ldots, p_{k}, p_{k+1}$ arranged on a circle.
        \end{shortAnswer}

        \begin{shortAnswer}[Complete the proof:\vfill]
            Adding a $k+1^{st}$ point adds an additional $k$ pairs of points. Then there are $\frac{k^{2}-k}{2}+k=\frac{k^{2}+k}{2}=\frac{(k+1)^2-(k+1)}{2}$ line segments joining all pairs of distinct points $p_{1}, p_{2}, \ldots, p_{k}, p_{k+1}$ arranged on a circle.
        \end{shortAnswer}
    \end{proof}
\end{br}

\pdfOnly{
    \ifhandout{
        \pagebreak}
        \else
        \fi}

\begin{br}
    If $n$ is a positive integer, then 
        \[1^3+2^3+3^3+\cdots+n^3
            =\frac{n^2(n+1)^2}{4}.\]
	\begin{proof}
        We proceed by induction. The base case is $n=1$. Since  $\inlineAnswer{1^3=\frac{1^2(1+1)^2}{4}},$ we are done.

        The induction hypothesis is that if $k \geq 1$ and $n = k,$ then 
        \[\inlineAnswer[0cm]{1^3+2^3+3^3+\cdots+n^3
            =\frac{n^2(n+1)^2}{4}}\]
        We want to show that 
        \[\inlineAnswer[0cm]{1^3+2^3+3^3+\cdots+n^3+(n+1)^3
            =\frac{(n+1)^2(n+2)^2}{4}}\]
        
        \begin{shortAnswer}[Complete the proof:\vfill]
            \begin{align*}
                1^3+2^3+3^3+\cdots+k^3
                    &=\frac{k^2(k+1)^2}{4} \\
                1^3+2^3+3^3+\cdots+k^3+(k+1)^3
                    &=\frac{k^2(k+1)^2}{4}+(k+1)^3=\frac{k^2(k+1)^2+4(k+1)^3}{4}\\
                &=\frac{(k+1)^2(k^2+4k+4)}{4}
                    =\frac{(k+1)^2(k+2)^2}{4}
            \end{align*}
        \end{shortAnswer}
    \end{proof}
\end{br}

\begin{br}
    If $n$ is an integer with $n\geq 5,$ then 
    \[2^n>n^2.\]
	
    \begin{proof}
        We proceed by induction. The base case is $n=5$. Since $\inlineAnswer{2^5>5^2},$
        we are done.

        The induction hypothesis is that if $k \geq 5$ and $n = k,$ then $\inlineAnswer{2^k>k^2}.$ 
        We want to show that $\inlineAnswer{2^{k+1}>(k+1)^2}.$

        \begin{shortAnswer}[Complete the proof:\vfill]
            Multiplying both sides by $k$ gives $k2^{k}>2^{k+1}>k^2.$
        \end{shortAnswer}
        
    \end{proof}
\end{br}

\pdfOnly{\ifhandout{
    \pagebreak}
    \else
    \fi}

Recall the notation $\gcd(a,b)=(a,b).$
\begin{br}
    Let $a_1,a_2,\dots,a_n\in\Z$ with $a_1\neq 0$. Prove that \[(a_1,\dots,a_n)=((a_1,a_2,a_3,\dots,a_{n-1}),a_n).\]

    \begin{hint}
        Try solving the $k=3$ case as part of your scratch work.
    \end{hint}
 
	\begin{proof}
        We proceed by induction. The base case is $n=2,$ since the statement we are trying to prove requires at least two inputs. Since 
            \[\inlineAnswer[0 cm]{(a_1,a_2)
                =((a_1,a_2))}\] 
        we are done.
        
        The induction hypothesis is that if $k \geq 2$ and $n = k,$ then 
            \[\inlineAnswer[0 cm]{(a_1,a_2,\dots,a_k)
                =((a_1,a_2,a_{k-1}),a_k)}\] 
        We want to prove that 
            \[\inlineAnswer[0 cm]{(a_1,a_2,\dots,a_{k+1})
                =((a_1,a_2,a_{k}),a_{k+1})}\] 
        \begin{shortAnswer}[Complete the proof:\vfill]
            Let $d_k=(a_1,a_2,a_3,\dots,a_{k}),$
            $e=((a_1,a_2,a_3,\dots,a_{k}),a_{k+1})=(d_k,a_{k+1}),$ and $f= (a_1,a_2,a_3,\dots,a_{k},a_{k+1}).$ We will show that $e\mid f$ and $f\mid e$. Since both $e$ and $f$ are positive, this will prove that $e=f$.

            Note that $e\mid (a_1,a_2,a_3,\dots,a_{k})$ and $e\mid a_{k+1}$ by definition. 
            Since $(a_1,\dots,a_k)=((a_1,a_2,a_3,\dots,a_{k-1}),a_k)=(d_{k-1},a_k)$ by the induction hypothesis, $e\mid d_{k-1}$ and $e\mid a_k$ by defintion of $(d_{k-1},a_k).$ Again, by the induction hypothesis, $d_{k-1}=(a_1,a_2,a_3,\dots,a_{k-1})=((a_1,a_2,a_3,\dots,a_{k-2}),a_{k-1})=(d_{k-2},a_{k-1}),$ so $e\mid a_{k-1}$ and $e\mid d_{k-2}$ by defintion of $(d_{k-2},a_{k-1}).$ Repeat this process until we get $(a_1,a_2,a_3)=((a_1,a_2),a_3)$, so $e\mid a_3$ and $e\mid (a_1,a_2)$ by definition of $((a_1,a_2),a_3)$. Thus $e\mid a_1,a_2,\dots,a_{k+1}$ by repeated applications of the induction hypothesis and the definition of greatest common divisor. By Problem 4 on Homework 3, $e\mid f.$

            To show that $f\mid e,$ we note that $f\mid a_1,a_2,\dots, a_k,a_{k+1}$ by definition. Then $f\mid d_k$ by Problem 4 on Homework 3. Since $e=(d_k,a_k),$ we have that $f\mid e$ by Problem 4 on Homework 3.
        \end{shortAnswer}
    \end{proof}
\end{br}

\pdfOnly{\ifhandout{
    \pagebreak}
    \else
    \fi}

\begin{br}
    Redo the following proofs using induction:
 	\begin{br}
		Let $n\in\Z$. Prove that $3\mid n^3-n$.
		
        \begin{proof}
            We proceed by induction. The base case is $n=1$. Since $\inlineAnswer{3\mid 1^3-1=3},$
            we are done.

            The induction hypothesis is that if $k \geq 1$ and $n = k,$ then $\inlineAnswer{3\mid k^3-k}.$
            We want to show that $\inlineAnswer{3\mid (k+1)^3-(k+1)}.$
            \begin{shortAnswer}
                Since $3\mid 3(k^2 + k)$ by the definition of divides, and $3\mid k^3-k$ by the induction hypothesis, $k^3 - k +3(k^2 + k)$ by linear combination. Note that 
                \begin{align*}
                    k^3 - k +3(k^2 + k)
                    &= k^3+3k^2 + 3k +1 - k- 1\\
                    &= (k+1)^3-(k+1).
                \end{align*}
                Thus, $3\mid (k+1)^3-(k+1).$
            \end{shortAnswer}
        \end{proof}
    \end{br}

		
    \begin{br}
        Let $n\in\Z$. Prove that $5\mid n^5-n$.
		\begin{proof}
            We proceed by induction. The base case is $n=1$. Since $\inlineAnswer{5\mid 1^5-1=0},$
            we are done.

            The induction hypothesis is that if $k \geq 1$ and $n = k,$ then $\inlineAnswer{5\mid k^5-k}.$
            We want to show that $\inlineAnswer{5\mid (k+1)^5-(k+1)}.$
            \begin{shortAnswer}
                Since $5\mid 5(k^4 + 2k^3 +2k^2 + k)$ by the definition of divides, and $5\mid k^5-k$ by the induction hypothesis, $k^5 - k +5(k^4 + 2k^3 +2k^2 + k)$ by linear combination. Note that 
                \begin{align*}
                    k^5 - k +5(k^4 + 2k^3 +2k^2 + k)
                    &= k^5 + 5k^4 + 10k^3 + 10k^2+ 5k +1 - k- 1\\
                    &= (k+1)^5-(k+1).
                \end{align*}
                Thus, $5\mid (k+1)^5-(k+1).$
            \end{shortAnswer}
        \end{proof}
	\end{br}
\end{br}
  \end{document}