\documentclass[handout]{ximera}
\usepackage{amsmath,multicol,amsthm,alltt,color, listings,xr-hyper,hyperref}
\usepackage{xparse}

\usepackage{parskip}
\usepackage[,margin=0.7in]{geometry}
\setlength{\textheight}{8.5in}

%%%fonts
%\usepackage{euler}
\usepackage{pbsi} %% Answer font

\usepackage{epstopdf}

\DeclareGraphicsExtensions{.eps}

%
%\usepackage{tkz-euclide}
%%\usetkzobj{all}
%\tikzstyle geometryDiagrams=[rounded corners=.5pt,ultra thick,color=black]
%\colorlet{penColor}{black} % Color of a curve in a plot


\usepackage{subcaption}
\usepackage{float}
\usepackage{fancyhdr}
%\usepackage{pdfpages}
%\newcounter{includepdfpage}
\usepackage{makecell}

%
%\usepackage{currfile}
%\usepackage{xstring}


\lhead{\large{Number Theory: MAT-255}}
\chead{}
\rhead{Spring 2024}
\lfoot{}
\cfoot{}
\rfoot{Page \thepage}
\renewcommand\headrulewidth{0pt}
\renewcommand\footrulewidth{0pt}

\headheight 50pt
\headsep 30pt

\author{Claire Merriman}

%%%%%%%%%%%%%%%%%%%%%
% Create handoutstyle for in class assignments
%%%%%%%%%%%%%%%%%%%%%
\makeatletter
 \newcommand\handoutstyle{%
  \def\activitystyle{activity-handout}
  \def\maketitle{\addtocounter{titlenumber}{1}%
  \addcontentsline{toc}{section}{\@date}%
        \setcounter{br}{0}}
 }

\newcommand{\handoutAbstract}{\begin{abstract}
\end{abstract}}
\makeatother

%%%%%%%%%%%%%%%%%%%%%
% Counters and autoref for unnumbered environments
%%%%%%%%%%%%%%%%%%%%%
\theoremstyle{plain}


\newtheorem*{namedthm}{Theorem}
\newcounter{thm}%makes pointer correct
\providecommand{\thmname}{Proposition}

\makeatletter
\NewDocumentEnvironment{thm*}{o}
 {%
  \IfValueTF{#1}
    {\namedthm[#1]\refstepcounter{thm}\def\@currentlabel{(#1)}}%
    {\namedthm}%
 }
 {%
  \endnamedthm
 }
\makeatother


\newtheorem*{namedprop}{Proposition}
\newcounter{prop}%makes pointer correct
\providecommand{\propname}{Proposition}

\makeatletter
\NewDocumentEnvironment{prop*}{o}
 {%
  \IfValueTF{#1}
    {\namedprop[#1]\refstepcounter{prop}\def\@currentlabel{(#1)}}%
    {\namedprop}%
 }
 {%
  \endnamedprop
 }
\makeatother

\newtheorem*{namedlem}{Lemma}
\newcounter{lem}%makes pointer correct
\providecommand{\lemname}{Lemma}

\makeatletter
\NewDocumentEnvironment{lem*}{o}
 {%
  \IfValueTF{#1}
    {\namedlem[#1]\refstepcounter{lem}\def\@currentlabel{(#1)}}%
    {\namedlem}%
 }
 {%
  \endnamedlem
 }
\makeatother

\newtheorem*{namedcor}{Corollary}
\newcounter{cor}%makes pointer correct
\providecommand{\corname}{Corollary}

\makeatletter
\NewDocumentEnvironment{cor*}{o}
 {%
  \IfValueTF{#1}
    {\namedcor[#1]\refstepcounter{cor}\def\@currentlabel{(#1)}}%
    {\namedcor}%
 }
 {%
  \endnamedcor
 }
\makeatother

\theoremstyle{definition}
\newtheorem*{annotation}{Annotation}
\newtheorem*{rubric}{Rubric}

\newtheorem*{innerrem}{Remark}
\newcounter{rem}%makes pointer correct
\providecommand{\remname}{Remark}

\makeatletter
\NewDocumentEnvironment{rem}{o}
 {%
  \IfValueTF{#1}
    {\innerrem[#1]\refstepcounter{rem}\def\@currentlabel{(#1)}}%
    {\innerrem}%
 }
 {%
  \endinnerrem
 }
\makeatother

\newtheorem*{innerdefn}{Definition}%%placeholder
\newcounter{defn}%makes pointer correct
\providecommand{\defnname}{Definition}

\makeatletter
\NewDocumentEnvironment{defn}{o}
 {%
  \IfValueTF{#1}
    {\innerdefn[#1]\refstepcounter{defn}\def\@currentlabel{(#1)}}%
    {\innerdefn}%
 }
 {%
  \endinnerdefn
 }
\makeatother

\newtheorem*{scratch}{Scratch Work}


\newtheorem*{namedconj}{Conjecture}
\newcounter{conj}%makes pointer correct
\providecommand{\conjname}{Conjecture}
\makeatletter
\NewDocumentEnvironment{conj}{o}
 {%
  \IfValueTF{#1}
    {\innerconj[#1]\refstepcounter{conj}\def\@currentlabel{(#1)}}%
    {\innerconj}%
 }
 {%
  \endinnerconj
 }
\makeatother

%\let\br\relax
%\let\endbr\relax

%\newcounter{br}%makes pointer correct
%\counterwithin{br}{section}
%
%\newenvironment{br}[1][2in]%
%{%Env start code
%\problemEnvironmentStart{#1}{In-class Problem}
%\refstepcounter{br}
%\stepcounter{problem}
%}
%{%Env end code
%\problemEnvironmentEnd
%}

\let\question\relax
\let\endquestion\relax

\newtheoremstyle{ExerciseStyle}{\topsep}{\topsep}%%% space between body and thm
		{}                      %%% Thm body font
		{}                              %%% Indent amount (empty = no indent)
		{\bfseries}            %%% Thm head font
		{}                              %%% Punctuation after thm head
		{3em}                           %%% Space after thm head
		{{#1}~\thmnumber{#2}\thmnote{ \bfseries(#3)}}%%% Thm head spec
\theoremstyle{ExerciseStyle}
\newtheorem{br}{In-class Problem}


\newcounter{ex}%makes pointer correct
\providecommand{\exname}{Homework Problem}
\newenvironment{ex}[1][2in]%
{%Env start code
\problemEnvironmentStart{#1}{Homework Problem}
\refstepcounter{ex}
}
{%Env end code
\problemEnvironmentEnd
}

\newcommand{\inlineAnswer}[2][2 cm]{
    \ifhandout{\pdfOnly{\rule{#1}{0.4pt}}}
    \else{\answer{#2}}
    \fi
}

\ifhandout
\newenvironment{shortAnswer}[1][
    \vfill]
        {% Begin then result
        #1
            \begin{freeResponse}
            }
    {% Environment Ending Code
    \end{freeResponse}
    }
\else
\newenvironment{shortAnswer}[1][]
        {\begin{freeResponse}
            }
    {% Environment Ending Code
    \end{freeResponse}
    }
\fi

\newenvironment{sketch}
 {\begin{proof}[Sketch of Proof]}
 {\end{proof}}


\newcommand{\gt}{>}
\newcommand{\lt}{<}
\newcommand{\N}{\mathbb N}
\newcommand{\Q}{\mathbb Q}
\newcommand{\Z}{\mathbb Z}
\newcommand{\C}{\mathbb C}
\newcommand{\R}{\mathbb R}
\renewcommand{\H}{\mathbb{H}}
\newcommand{\lcm}{\operatorname{lcm}}
\newcommand{\nequiv}{\not\equiv}
\newcommand{\ord}{\operatorname{ord}}
\newcommand{\ds}{\displaystyle}
\newcommand{\floor}[1]{\left\lfloor #1\right\rfloor}
\newcommand{\legendre}[2]{\left(\frac{#1}{#2}\right)}



%%%%%%%%%%%%



\date{March 1, 2024}

\begin{document}
\handoutAbstract
\maketitle
  \begin{center}%
    {\large \scshape MAT-255-- Number Theory \hfill Spring 2024 \hfill In Class Work March 1}%
    
    {\large
        Your Name: \hrulefill \quad Group Members:\hrulefill \quad \hrulefill
	\par}%
  \end{center}%
  


\begin{br}
    Repeat the proof from last class to prove \begin{theorem}[Theorem 3.2]\label{thm:phi-multiplicative}
        Let $m$ and $n$ be positive integers where $(m,n)=1$. Then $\phi(mn)=\phi(m)\phi(n).$
    \end{theorem}

    \begin{proof}
        Let $m$ and $m$ be relatively prime positive integers. A number $a$ is relatively prime to $mn$ if and only if $a$ is relatively prime to $\inlineAnswer[1 cm]{m}$ and $\inlineAnswer[1 cm]{n}.$ 
        
        
        We can partition the positive integers less that or equal to $mn$ into 
        \begin{align*}
        & 1\equiv\inlineAnswer[1 cm]{m+1}   
            \equiv\inlineAnswer[1 cm]{2m+1}
            \equiv \cdots
            \equiv\inlineAnswer[1 cm]{(n-1)m+1}\pmod m\\
        & 2 \equiv\inlineAnswer[1 cm]{m+2}   
            \equiv\inlineAnswer[1 cm]{2m+2}
            \equiv \cdots
            \equiv\inlineAnswer[1 cm]{(n-1)m+2}\pmod m\\
        & \vdots\\
        & m\equiv\inlineAnswer[1 cm]{2m}   
            \equiv\inlineAnswer[1 cm]{3m}
            \equiv \cdots
            \equiv\inlineAnswer[1 cm]{nm}\pmod m
        \end{align*}

        For any $b$ in the range $1,2,3,\dots,m,$ define $s_b$ to be the number of integers $a$ in the range $1,2,\dots, mn$ such that $a\equiv b \pmod m$ and $\gcd(a,mn)=1$. Thus, when $(b,m)=1$, $s_b=\phi(\inlineAnswer[1 cm]{m})$ and when $(b,m)>1$, $s_b=\inlineAnswer[1 cm]{0}$.

        \begin{shortAnswer}[\vspace{1in}]
            Since every positive integers less that or equal to $mn$ is counted by exactly one $s_b,$ $\phi(mn)=s_1+s_2+\cdots+s_m.$
        \end{shortAnswer}

        
        We have seen that $\phi(mn)=s_1+s_2+\dots+s_m$, that when $(b,m)=1$, $s_b=\inlineAnswer[1 cm]{\phi(n)},$
        \pdfOnly{\ifhandout
        \footnote{This blank is asking for a function, not a value.}\else\fi} 
        \begin{onlineOnly}
            This blank is asking for a function, not a value.
        \end{onlineOnly}
        and that when $(b,m)>1$, $s_b=\inlineAnswer[1 cm]{0}$. Since there are $\phi(\inlineAnswer[1 cm]{m})$ integers $b$ where $(b,m)=1$. 
        Thus, we can say that $\phi(mn)=\inlineAnswer{\phi(m)\phi(n)}.$ 
    \end{proof}
\end{br}


\begin{br}
    Complete the proof of \nameref{thm:phi-multiplicative} by proving 
    \begin{proposition}
        If $m, n,$ and $i$ are positive integers with ($m, n) = (m, i) = 1,$ then the integers \[i, m + i, 2m +i,\dots, (n - 1)m +i\] form a complete system of residues modulo $n.$
    \end{proposition}
\end{br}

\end{document}