\documentclass[handout]{ximera}
\usepackage{amsmath,multicol,amsthm,alltt,color, listings,xr-hyper,hyperref}
\usepackage{xparse}

\usepackage{parskip}
\usepackage[,margin=0.7in]{geometry}
\setlength{\textheight}{8.5in}

%%%fonts
%\usepackage{euler}
\usepackage{pbsi} %% Answer font

\usepackage{epstopdf}

\DeclareGraphicsExtensions{.eps}

%
%\usepackage{tkz-euclide}
%%\usetkzobj{all}
%\tikzstyle geometryDiagrams=[rounded corners=.5pt,ultra thick,color=black]
%\colorlet{penColor}{black} % Color of a curve in a plot


\usepackage{subcaption}
\usepackage{float}
\usepackage{fancyhdr}
%\usepackage{pdfpages}
%\newcounter{includepdfpage}
\usepackage{makecell}

%
%\usepackage{currfile}
%\usepackage{xstring}


\lhead{\large{Number Theory: MAT-255}}
\chead{}
\rhead{Spring 2024}
\lfoot{}
\cfoot{}
\rfoot{Page \thepage}
\renewcommand\headrulewidth{0pt}
\renewcommand\footrulewidth{0pt}

\headheight 50pt
\headsep 30pt

\author{Claire Merriman}

%%%%%%%%%%%%%%%%%%%%%
% Create handoutstyle for in class assignments
%%%%%%%%%%%%%%%%%%%%%
\makeatletter
 \newcommand\handoutstyle{%
  \def\activitystyle{activity-handout}
  \def\maketitle{\addtocounter{titlenumber}{1}%
  \addcontentsline{toc}{section}{\@date}%
        \setcounter{br}{0}}
 }

\newcommand{\handoutAbstract}{\begin{abstract}
\end{abstract}}
\makeatother

%%%%%%%%%%%%%%%%%%%%%
% Counters and autoref for unnumbered environments
%%%%%%%%%%%%%%%%%%%%%
\theoremstyle{plain}


\newtheorem*{namedthm}{Theorem}
\newcounter{thm}%makes pointer correct
\providecommand{\thmname}{Proposition}

\makeatletter
\NewDocumentEnvironment{thm*}{o}
 {%
  \IfValueTF{#1}
    {\namedthm[#1]\refstepcounter{thm}\def\@currentlabel{(#1)}}%
    {\namedthm}%
 }
 {%
  \endnamedthm
 }
\makeatother


\newtheorem*{namedprop}{Proposition}
\newcounter{prop}%makes pointer correct
\providecommand{\propname}{Proposition}

\makeatletter
\NewDocumentEnvironment{prop*}{o}
 {%
  \IfValueTF{#1}
    {\namedprop[#1]\refstepcounter{prop}\def\@currentlabel{(#1)}}%
    {\namedprop}%
 }
 {%
  \endnamedprop
 }
\makeatother

\newtheorem*{namedlem}{Lemma}
\newcounter{lem}%makes pointer correct
\providecommand{\lemname}{Lemma}

\makeatletter
\NewDocumentEnvironment{lem*}{o}
 {%
  \IfValueTF{#1}
    {\namedlem[#1]\refstepcounter{lem}\def\@currentlabel{(#1)}}%
    {\namedlem}%
 }
 {%
  \endnamedlem
 }
\makeatother

\newtheorem*{namedcor}{Corollary}
\newcounter{cor}%makes pointer correct
\providecommand{\corname}{Corollary}

\makeatletter
\NewDocumentEnvironment{cor*}{o}
 {%
  \IfValueTF{#1}
    {\namedcor[#1]\refstepcounter{cor}\def\@currentlabel{(#1)}}%
    {\namedcor}%
 }
 {%
  \endnamedcor
 }
\makeatother

\theoremstyle{definition}
\newtheorem*{annotation}{Annotation}
\newtheorem*{rubric}{Rubric}

\newtheorem*{innerrem}{Remark}
\newcounter{rem}%makes pointer correct
\providecommand{\remname}{Remark}

\makeatletter
\NewDocumentEnvironment{rem}{o}
 {%
  \IfValueTF{#1}
    {\innerrem[#1]\refstepcounter{rem}\def\@currentlabel{(#1)}}%
    {\innerrem}%
 }
 {%
  \endinnerrem
 }
\makeatother

\newtheorem*{innerdefn}{Definition}%%placeholder
\newcounter{defn}%makes pointer correct
\providecommand{\defnname}{Definition}

\makeatletter
\NewDocumentEnvironment{defn}{o}
 {%
  \IfValueTF{#1}
    {\innerdefn[#1]\refstepcounter{defn}\def\@currentlabel{(#1)}}%
    {\innerdefn}%
 }
 {%
  \endinnerdefn
 }
\makeatother

\newtheorem*{scratch}{Scratch Work}


\newtheorem*{namedconj}{Conjecture}
\newcounter{conj}%makes pointer correct
\providecommand{\conjname}{Conjecture}
\makeatletter
\NewDocumentEnvironment{conj}{o}
 {%
  \IfValueTF{#1}
    {\innerconj[#1]\refstepcounter{conj}\def\@currentlabel{(#1)}}%
    {\innerconj}%
 }
 {%
  \endinnerconj
 }
\makeatother

%\let\br\relax
%\let\endbr\relax

%\newcounter{br}%makes pointer correct
%\counterwithin{br}{section}
%
%\newenvironment{br}[1][2in]%
%{%Env start code
%\problemEnvironmentStart{#1}{In-class Problem}
%\refstepcounter{br}
%\stepcounter{problem}
%}
%{%Env end code
%\problemEnvironmentEnd
%}

\let\question\relax
\let\endquestion\relax

\newtheoremstyle{ExerciseStyle}{\topsep}{\topsep}%%% space between body and thm
		{}                      %%% Thm body font
		{}                              %%% Indent amount (empty = no indent)
		{\bfseries}            %%% Thm head font
		{}                              %%% Punctuation after thm head
		{3em}                           %%% Space after thm head
		{{#1}~\thmnumber{#2}\thmnote{ \bfseries(#3)}}%%% Thm head spec
\theoremstyle{ExerciseStyle}
\newtheorem{br}{In-class Problem}


\newcounter{ex}%makes pointer correct
\providecommand{\exname}{Homework Problem}
\newenvironment{ex}[1][2in]%
{%Env start code
\problemEnvironmentStart{#1}{Homework Problem}
\refstepcounter{ex}
}
{%Env end code
\problemEnvironmentEnd
}

\newcommand{\inlineAnswer}[2][2 cm]{
    \ifhandout{\pdfOnly{\rule{#1}{0.4pt}}}
    \else{\answer{#2}}
    \fi
}

\ifhandout
\newenvironment{shortAnswer}[1][
    \vfill]
        {% Begin then result
        #1
            \begin{freeResponse}
            }
    {% Environment Ending Code
    \end{freeResponse}
    }
\else
\newenvironment{shortAnswer}[1][]
        {\begin{freeResponse}
            }
    {% Environment Ending Code
    \end{freeResponse}
    }
\fi

\newenvironment{sketch}
 {\begin{proof}[Sketch of Proof]}
 {\end{proof}}


\newcommand{\gt}{>}
\newcommand{\lt}{<}
\newcommand{\N}{\mathbb N}
\newcommand{\Q}{\mathbb Q}
\newcommand{\Z}{\mathbb Z}
\newcommand{\C}{\mathbb C}
\newcommand{\R}{\mathbb R}
\renewcommand{\H}{\mathbb{H}}
\newcommand{\lcm}{\operatorname{lcm}}
\newcommand{\nequiv}{\not\equiv}
\newcommand{\ord}{\operatorname{ord}}
\newcommand{\ds}{\displaystyle}
\newcommand{\floor}[1]{\left\lfloor #1\right\rfloor}
\newcommand{\legendre}[2]{\left(\frac{#1}{#2}\right)}



%%%%%%%%%%%%




\date{March 27, 2024}

\begin{document}
\handoutAbstract
\maketitle
 	\begin{center}%
    	{\large \scshape MAT-255-- Number Theory 
			\hfill Spring 2024 
			\hfill In Class Work March 27}%
    
		{\large Your Name: \hrulefill \quad 
			Group Members:\hrulefill \quad 
			\hrulefill
			\par}%
 	\end{center}%
	 
From class March 20:

\begin{tabular}{cll}
	Modulus & Quadratic residues & Quadratic nonresidues\\\hline
	$2$	& $1$ 	& None\\
	$3$	& $1$	& $2$\\
	$5$	& $1,4$	& $2,3$\\
	$7$	& $1,2,4$	& $3,5,6$
\end{tabular}

\begin{proposition}[Proposition 4.5]
	Let $p$ be an odd prime number and $a,b\in\Z$ with $p\nmid a$ and $p\nmid b.$ Then 
	\begin{enumerate}[label=(\alph*)]
		\item $\legendre{a^2}{p}=1$ \label{squares-are-square}
		\item If $a\equiv b\pmod{p}$ then $\legendre{a}{p}=\legendre{b}{p}$ \label{legendre-respects-mod}
		\item $\legendre{ab}{p}=\legendre{a}{p}\legendre{b}{p}$ \label{legendre-mult}
	\end{enumerate}
\end{proposition}

\begin{theorem}[Theorem 4.6]\label{thm:residue-neg1}
	Let $p$ be an odd prime number. Then 
	\[
		\legendre{-1}{p}=
			\begin{cases}
 				1, & p\equiv 1\pmod{4}\\
				-1, & p\equiv 3\pmod{4}
			\end{cases}.
	\]
\end{theorem}

\begin{theorem}[Quadratic reciprocity]\label{quad-rec}
	Let $p$ and $q$ be distinct primes.  
	\begin{enumerate}[label=(\alph*)]
		\item If $p\equiv 1 \pmod{4}$ or $q\equiv 1\pmod{4},$ then $\legendre{p}{q}=\legendre{q}{p}$
 		\item If $p\equiv q \equiv 3 \pmod{4},$ then $\legendre{p}{q}=-\legendre{q}{p}$
	\end{enumerate}
\end{theorem}




\begin{br}[Strayer Chapter 4, Exercise 35]
	Let $p$ be an odd prime number. Prove the following statements the following provided outlines, which will help solve the next problem, as well.
	
	\begin{enumerate}
		\item $\legendre{3}{p}=1$ if and only if $p\equiv \pm1\pmod{12}.$
		\item $\legendre{-3}{p}=1$ if and only if $p\equiv 1\pmod{6}.$	
	\end{enumerate}
\end{br}

\begin{proof}
	\begin{enumerate}
        \item Since $3\equiv \inlineAnswer{3}\pmod{4},$\footnote{In this problem, this step is repetitive, but it is needed when $p\neq3$.} we need two cases for \nameref{quad-rec}. 
            \begin{enumerate}
                \item If $p\equiv 1\pmod{4},$ then $\legendre{3}{p}=\inlineAnswer{\legendre{p}{3}}$ by \nameref{quad-rec}, and $\legendre{p}{3}=1$ if and only if $p\equiv\inlineAnswer{1\pmod{3}}$. Then $p\equiv \inlineAnswer{1}\pmod{12},$ and this is the unique equivalence class modulo $12$ by the Chinese Remainder Theorem.
            
                \item If $p\equiv 3\equiv -1\pmod{4},$ then $\legendre{3}{p}=\inlineAnswer{-\legendre{p}{3}}$ by \nameref{quad-rec}, and $\legendre{p}{3}=-1$ if and only if $p\equiv\inlineAnswer{ 2\equiv-1\pmod{3}}$. Then $p\equiv \inlineAnswer{-1}\pmod{12},$ and this is the unique equivalence class modulo $12$ by the Chinese Remainder Theorem.
            \end{enumerate}

        Therefore, $\legendre{3}{p}=1$ if and only if $p\equiv \pm1\pmod{12}.$ 
            
        \item From Theorem 4.25\ref{legendre-mult}, $\legendre{-3}{p}= \inlineAnswer{\legendre{-1}{p}\legendre{3}{p}}.$
        Again, we have two cases.
            \begin{enumerate}
                \item If $p\equiv 1\pmod{4},$ then $\legendre{-1}{p}=\inlineAnswer{1}$ by \nameref{thm:residue-neg1} and
                $\legendre{3}{p}=\inlineAnswer{\legendre{p}{3}}$ by \nameref{quad-rec}. Thus, $\legendre{-3}{p}=\inlineAnswer{\legendre{p}{3}}=1$ if and only if $p\equiv \inlineAnswer{1\pmod{3}}.$ Then $p\equiv \inlineAnswer{1}\pmod{12},$ and this is the unique equivalence class modulo $12$ by the Chinese Remainder Theorem.
            
                \item If $p\equiv 3\equiv -1\pmod{4},$ then $\legendre{-1}{p}=\inlineAnswer{-1}$ by \nameref{thm:residue-neg1} and
                $\legendre{3}{p}=\inlineAnswer{-\legendre{p}{3}}$ by \nameref{quad-rec}. Thus, $\legendre{-3}{p}=\inlineAnswer{\legendre{p}{3}}=1$ if and only if $p\equiv \inlineAnswer{1\pmod{3}}.$ Then $p\equiv \inlineAnswer{7}\pmod{12},$ and this is the unique equivalence class modulo $12$ by the Chinese Remainder Theorem.
            \end{enumerate}
        Therefore, $\legendre{-3}{p}=1$ if and only if $p\equiv \inlineAnswer{1,7}\pmod{12},$ which is equivalent to $p\equiv 1\pmod{6}.$ 
    \end{enumerate}
\end{proof}

\begin{br}[Strayer Chapter 4, Exercise 36]
	Find congruences characterizing all prime numbers $p$ for which the following integers are quadratic residues modulo $p,$ as done in the previous exercise. 
	
	Outline is provided for the first part.
	\begin{enumerate}
		\item $5$ 
		\item $-5$	
		\item $7$ 
		\item $-7$	
	\end{enumerate}



    \begin{proof}
        \begin{enumerate}
            \item Since $5\equiv\inlineAnswer{1}\pmod{4},$ $\inlineAnswer[1.25 in]{\legendre{5}{p}=\legendre{p}{5}}$ by \nameref{quad-rec}. Then $\legendre{5}{p}=\inlineAnswer[1cm]{\legendre{p}{5}}=1$ if and only if $\inlineAnswer[1.25 in]{p\equiv 1, 4 \pmod{5}}.$

            \pdfOnly{\ifhandout
                \vfill\else\fi}
        \end{enumerate}
    \end{proof}
\end{br}
\end{document}