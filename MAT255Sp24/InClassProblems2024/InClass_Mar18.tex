\documentclass[handout]{ximera}
\usepackage{amsmath,multicol,amsthm,alltt,color, listings,xr-hyper,hyperref}
\usepackage{xparse}

\usepackage{parskip}
\usepackage[,margin=0.7in]{geometry}
\setlength{\textheight}{8.5in}

%%%fonts
%\usepackage{euler}
\usepackage{pbsi} %% Answer font

\usepackage{epstopdf}

\DeclareGraphicsExtensions{.eps}

%
%\usepackage{tkz-euclide}
%%\usetkzobj{all}
%\tikzstyle geometryDiagrams=[rounded corners=.5pt,ultra thick,color=black]
%\colorlet{penColor}{black} % Color of a curve in a plot


\usepackage{subcaption}
\usepackage{float}
\usepackage{fancyhdr}
%\usepackage{pdfpages}
%\newcounter{includepdfpage}
\usepackage{makecell}

%
%\usepackage{currfile}
%\usepackage{xstring}


\lhead{\large{Number Theory: MAT-255}}
\chead{}
\rhead{Spring 2024}
\lfoot{}
\cfoot{}
\rfoot{Page \thepage}
\renewcommand\headrulewidth{0pt}
\renewcommand\footrulewidth{0pt}

\headheight 50pt
\headsep 30pt

\author{Claire Merriman}

%%%%%%%%%%%%%%%%%%%%%
% Create handoutstyle for in class assignments
%%%%%%%%%%%%%%%%%%%%%
\makeatletter
 \newcommand\handoutstyle{%
  \def\activitystyle{activity-handout}
  \def\maketitle{\addtocounter{titlenumber}{1}%
  \addcontentsline{toc}{section}{\@date}%
        \setcounter{br}{0}}
 }

\newcommand{\handoutAbstract}{\begin{abstract}
\end{abstract}}
\makeatother

%%%%%%%%%%%%%%%%%%%%%
% Counters and autoref for unnumbered environments
%%%%%%%%%%%%%%%%%%%%%
\theoremstyle{plain}


\newtheorem*{namedthm}{Theorem}
\newcounter{thm}%makes pointer correct
\providecommand{\thmname}{Proposition}

\makeatletter
\NewDocumentEnvironment{thm*}{o}
 {%
  \IfValueTF{#1}
    {\namedthm[#1]\refstepcounter{thm}\def\@currentlabel{(#1)}}%
    {\namedthm}%
 }
 {%
  \endnamedthm
 }
\makeatother


\newtheorem*{namedprop}{Proposition}
\newcounter{prop}%makes pointer correct
\providecommand{\propname}{Proposition}

\makeatletter
\NewDocumentEnvironment{prop*}{o}
 {%
  \IfValueTF{#1}
    {\namedprop[#1]\refstepcounter{prop}\def\@currentlabel{(#1)}}%
    {\namedprop}%
 }
 {%
  \endnamedprop
 }
\makeatother

\newtheorem*{namedlem}{Lemma}
\newcounter{lem}%makes pointer correct
\providecommand{\lemname}{Lemma}

\makeatletter
\NewDocumentEnvironment{lem*}{o}
 {%
  \IfValueTF{#1}
    {\namedlem[#1]\refstepcounter{lem}\def\@currentlabel{(#1)}}%
    {\namedlem}%
 }
 {%
  \endnamedlem
 }
\makeatother

\newtheorem*{namedcor}{Corollary}
\newcounter{cor}%makes pointer correct
\providecommand{\corname}{Corollary}

\makeatletter
\NewDocumentEnvironment{cor*}{o}
 {%
  \IfValueTF{#1}
    {\namedcor[#1]\refstepcounter{cor}\def\@currentlabel{(#1)}}%
    {\namedcor}%
 }
 {%
  \endnamedcor
 }
\makeatother

\theoremstyle{definition}
\newtheorem*{annotation}{Annotation}
\newtheorem*{rubric}{Rubric}

\newtheorem*{innerrem}{Remark}
\newcounter{rem}%makes pointer correct
\providecommand{\remname}{Remark}

\makeatletter
\NewDocumentEnvironment{rem}{o}
 {%
  \IfValueTF{#1}
    {\innerrem[#1]\refstepcounter{rem}\def\@currentlabel{(#1)}}%
    {\innerrem}%
 }
 {%
  \endinnerrem
 }
\makeatother

\newtheorem*{innerdefn}{Definition}%%placeholder
\newcounter{defn}%makes pointer correct
\providecommand{\defnname}{Definition}

\makeatletter
\NewDocumentEnvironment{defn}{o}
 {%
  \IfValueTF{#1}
    {\innerdefn[#1]\refstepcounter{defn}\def\@currentlabel{(#1)}}%
    {\innerdefn}%
 }
 {%
  \endinnerdefn
 }
\makeatother

\newtheorem*{scratch}{Scratch Work}


\newtheorem*{namedconj}{Conjecture}
\newcounter{conj}%makes pointer correct
\providecommand{\conjname}{Conjecture}
\makeatletter
\NewDocumentEnvironment{conj}{o}
 {%
  \IfValueTF{#1}
    {\innerconj[#1]\refstepcounter{conj}\def\@currentlabel{(#1)}}%
    {\innerconj}%
 }
 {%
  \endinnerconj
 }
\makeatother

%\let\br\relax
%\let\endbr\relax

%\newcounter{br}%makes pointer correct
%\counterwithin{br}{section}
%
%\newenvironment{br}[1][2in]%
%{%Env start code
%\problemEnvironmentStart{#1}{In-class Problem}
%\refstepcounter{br}
%\stepcounter{problem}
%}
%{%Env end code
%\problemEnvironmentEnd
%}

\let\question\relax
\let\endquestion\relax

\newtheoremstyle{ExerciseStyle}{\topsep}{\topsep}%%% space between body and thm
		{}                      %%% Thm body font
		{}                              %%% Indent amount (empty = no indent)
		{\bfseries}            %%% Thm head font
		{}                              %%% Punctuation after thm head
		{3em}                           %%% Space after thm head
		{{#1}~\thmnumber{#2}\thmnote{ \bfseries(#3)}}%%% Thm head spec
\theoremstyle{ExerciseStyle}
\newtheorem{br}{In-class Problem}


\newcounter{ex}%makes pointer correct
\providecommand{\exname}{Homework Problem}
\newenvironment{ex}[1][2in]%
{%Env start code
\problemEnvironmentStart{#1}{Homework Problem}
\refstepcounter{ex}
}
{%Env end code
\problemEnvironmentEnd
}

\newcommand{\inlineAnswer}[2][2 cm]{
    \ifhandout{\pdfOnly{\rule{#1}{0.4pt}}}
    \else{\answer{#2}}
    \fi
}

\ifhandout
\newenvironment{shortAnswer}[1][
    \vfill]
        {% Begin then result
        #1
            \begin{freeResponse}
            }
    {% Environment Ending Code
    \end{freeResponse}
    }
\else
\newenvironment{shortAnswer}[1][]
        {\begin{freeResponse}
            }
    {% Environment Ending Code
    \end{freeResponse}
    }
\fi

\newenvironment{sketch}
 {\begin{proof}[Sketch of Proof]}
 {\end{proof}}


\newcommand{\gt}{>}
\newcommand{\lt}{<}
\newcommand{\N}{\mathbb N}
\newcommand{\Q}{\mathbb Q}
\newcommand{\Z}{\mathbb Z}
\newcommand{\C}{\mathbb C}
\newcommand{\R}{\mathbb R}
\renewcommand{\H}{\mathbb{H}}
\newcommand{\lcm}{\operatorname{lcm}}
\newcommand{\nequiv}{\not\equiv}
\newcommand{\ord}{\operatorname{ord}}
\newcommand{\ds}{\displaystyle}
\newcommand{\floor}[1]{\left\lfloor #1\right\rfloor}
\newcommand{\legendre}[2]{\left(\frac{#1}{#2}\right)}



%%%%%%%%%%%%



\date{March 18, 2024}

\begin{document}
\handoutAbstract
\maketitle
 	\begin{center}%
    		{\large \scshape MAT-255-- Number Theory 
			\hfill Spring 2024 
			\hfill In Class Work March 18}%
    
		{\large Your Name: \hrulefill \quad 
			Group Members:\hrulefill \quad 
			\hrulefill
			\par}%
 	\end{center}%
	
 \subsection*{Previous Results}
 \begin{lemma}\label{lem:gcd_mult}
	Let $a,b\in\Z,$ not both zero. Then any  common divisor of $a$ and $b$ divides the greatest common divisor.
\end{lemma}


\begin{lemma}\label{lem:gcd_trans}
 	Let $a,b\in\Z,$ not both zero. Then any divisor of $(a,b)$ is a common divisor of $a$ and $b$.
\end{lemma}
 
 \begin{proposition}[Proposition 5.1]\label{prop:order_divides_phi}
    Let $a,m\in\Z$ with $m>0$ and $(a,m)=1.$ Then $a^n\equiv 1\pmod{m}$ for some positive integer $n$ if and only if $\ord_m a\mid n.$ In particular, $\ord_m a\mid \phi(m).$
\end{proposition}

\subsection*{Problems}

\begin{br}
    Let $p$ be prime, $m$ a positive integer, and $d=(m,p-1).$ Prove that $a^m\equiv 1\pmod{p}$ if and only if $a^d\equiv 1\pmod{p}.$

	\begin{proof}
        Let $p$ be prime, $m$ a positive integer, and $d=(m,p-1).$ Let $a\in\Z$. If $p\mid a,$ then $a^i\equiv \inlineAnswer{0\pmod{p}}$ for all positive integers. 
        Otherwise, $a^{p-1}\equiv 1\pmod{p}$ by $\inlineAnswer[1 in]{Fermat's\ Little\ Theorem}.$
        
        By \nameref{prop:order_divides_phi}, $a^m\equiv 1\pmod{p}$ if and only if $\inlineAnswer{\ord_p a \mid m}$. Similarly, $\inlineAnswer{a^{p-1}\equiv 1\pmod{p}}$ if and only if $\inlineAnswer{\ord_p a \mid p-1}$. Thus, $\inlineAnswer{\ord_p a}$ is a common divisor of $\inlineAnswer{m}$ and $\inlineAnswer{p-1}$. Combining Lemmas \autoref{lem:gcd_mult}
        and \autoref{lem:gcd_trans} gives $\ord_p a$ is a common divisor of   $\inlineAnswer{m}$ and $\inlineAnswer{p-1}$ if and only if $\ord_p a\mid d$. One final application of \nameref{prop:order_divides_phi} gives $\inlineAnswer{\ord_p a\mid d}$ if and only if $\inlineAnswer{a^d\equiv 1\pmod{p}}$.
    \end{proof}
\end{br}
\pdfOnly{\ifhandout{
    \vfill
    Problem 2 on back page 
    \pagebreak}
    \else\fi
}

\begin{br}
    Let $p$ be prime and $m$ a positive integer. Prove that 
        \[x^m\equiv 1\pmod{p}\]
    has exactly $(m,p-1)$ incongruent solutions modulo $p.$


    \begin{proof}
        Let $p$ be prime, $m$ a positive integer, and $d=(m,p-1).$ 
        From Problem 1,
        \begin{shortAnswer}[\vspace{1in}]
            $x^m\equiv 1\pmod{p}$ if and only if $x^d\equiv 1\pmod{p}$.
        \end{shortAnswer}

            
        \emph{Now find a result that allows you to finish the proof in 1-2 sentences.}

        \begin{shortAnswer}[\vspace{1in}]
            By Proposition 5.8 there are exactly $d$ solutions to $x^d\equiv 1\pmod{p}.$ Thus, there are exactly $d$ solutions to $x^m\equiv 1\pmod{p}.$
        \end{shortAnswer}
    \end{proof}
\end{br}

\pdfOnly{\ifhandout{If you have time, start working on this problem from the homework.}
    \else\fi}

\begin{br}
    Prove the following statement, which is the converse of Proposition 5.4 (for a prime):

    Let $p$ be prime, and let $a\in\Z.$ If every $b\in\Z$ such that $p\nmid b$ is congruent to a power of $a$ modulo $p,$ then ${a}$ is a primitive root modulo $p$.
    
    \begin{solution}
        Let $p$ be prime, and let $a\in\Z$ such that every integer $b\in\Z$ where $p\nmid b$ is congruent to $a^i$ modulo $p$ for some positive integer $i$. Thus, $(a,p)=1,$ otherwise $1$ would not be congruent to a power of $a$. By Proposition 5.2, $a^i\equiv a^j\pmod{p}$ if and only if $i\equiv j\pmod{p-1}.$ Thus, $a^1,a^2,\dots,a^{p-1}$ are distinct congruence classes and only one of $a^1,a^2,\dots,a^{p-1}$ is congruent to $1$ modulo $p.$ By Fermat's Little Theorem, $a^{p-1}\equiv1\pmod{p},$ so $\ord_p a=p-1.$
    \end{solution}
\end{br}
\end{document}