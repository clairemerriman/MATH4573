\documentclass[handout]{ximera}
\usepackage{amsmath,multicol,amsthm,alltt,color, listings,xr-hyper,hyperref}
\usepackage{xparse}

\usepackage{parskip}
\usepackage[,margin=0.7in]{geometry}
\setlength{\textheight}{8.5in}

%%%fonts
%\usepackage{euler}
\usepackage{pbsi} %% Answer font

\usepackage{epstopdf}

\DeclareGraphicsExtensions{.eps}

%
%\usepackage{tkz-euclide}
%%\usetkzobj{all}
%\tikzstyle geometryDiagrams=[rounded corners=.5pt,ultra thick,color=black]
%\colorlet{penColor}{black} % Color of a curve in a plot


\usepackage{subcaption}
\usepackage{float}
\usepackage{fancyhdr}
%\usepackage{pdfpages}
%\newcounter{includepdfpage}
\usepackage{makecell}

%
%\usepackage{currfile}
%\usepackage{xstring}


\lhead{\large{Number Theory: MAT-255}}
\chead{}
\rhead{Spring 2024}
\lfoot{}
\cfoot{}
\rfoot{Page \thepage}
\renewcommand\headrulewidth{0pt}
\renewcommand\footrulewidth{0pt}

\headheight 50pt
\headsep 30pt

\author{Claire Merriman}

%%%%%%%%%%%%%%%%%%%%%
% Create handoutstyle for in class assignments
%%%%%%%%%%%%%%%%%%%%%
\makeatletter
 \newcommand\handoutstyle{%
  \def\activitystyle{activity-handout}
  \def\maketitle{\addtocounter{titlenumber}{1}%
  \addcontentsline{toc}{section}{\@date}%
        \setcounter{br}{0}}
 }

\newcommand{\handoutAbstract}{\begin{abstract}
\end{abstract}}
\makeatother

%%%%%%%%%%%%%%%%%%%%%
% Counters and autoref for unnumbered environments
%%%%%%%%%%%%%%%%%%%%%
\theoremstyle{plain}


\newtheorem*{namedthm}{Theorem}
\newcounter{thm}%makes pointer correct
\providecommand{\thmname}{Proposition}

\makeatletter
\NewDocumentEnvironment{thm*}{o}
 {%
  \IfValueTF{#1}
    {\namedthm[#1]\refstepcounter{thm}\def\@currentlabel{(#1)}}%
    {\namedthm}%
 }
 {%
  \endnamedthm
 }
\makeatother


\newtheorem*{namedprop}{Proposition}
\newcounter{prop}%makes pointer correct
\providecommand{\propname}{Proposition}

\makeatletter
\NewDocumentEnvironment{prop*}{o}
 {%
  \IfValueTF{#1}
    {\namedprop[#1]\refstepcounter{prop}\def\@currentlabel{(#1)}}%
    {\namedprop}%
 }
 {%
  \endnamedprop
 }
\makeatother

\newtheorem*{namedlem}{Lemma}
\newcounter{lem}%makes pointer correct
\providecommand{\lemname}{Lemma}

\makeatletter
\NewDocumentEnvironment{lem*}{o}
 {%
  \IfValueTF{#1}
    {\namedlem[#1]\refstepcounter{lem}\def\@currentlabel{(#1)}}%
    {\namedlem}%
 }
 {%
  \endnamedlem
 }
\makeatother

\newtheorem*{namedcor}{Corollary}
\newcounter{cor}%makes pointer correct
\providecommand{\corname}{Corollary}

\makeatletter
\NewDocumentEnvironment{cor*}{o}
 {%
  \IfValueTF{#1}
    {\namedcor[#1]\refstepcounter{cor}\def\@currentlabel{(#1)}}%
    {\namedcor}%
 }
 {%
  \endnamedcor
 }
\makeatother

\theoremstyle{definition}
\newtheorem*{annotation}{Annotation}
\newtheorem*{rubric}{Rubric}

\newtheorem*{innerrem}{Remark}
\newcounter{rem}%makes pointer correct
\providecommand{\remname}{Remark}

\makeatletter
\NewDocumentEnvironment{rem}{o}
 {%
  \IfValueTF{#1}
    {\innerrem[#1]\refstepcounter{rem}\def\@currentlabel{(#1)}}%
    {\innerrem}%
 }
 {%
  \endinnerrem
 }
\makeatother

\newtheorem*{innerdefn}{Definition}%%placeholder
\newcounter{defn}%makes pointer correct
\providecommand{\defnname}{Definition}

\makeatletter
\NewDocumentEnvironment{defn}{o}
 {%
  \IfValueTF{#1}
    {\innerdefn[#1]\refstepcounter{defn}\def\@currentlabel{(#1)}}%
    {\innerdefn}%
 }
 {%
  \endinnerdefn
 }
\makeatother

\newtheorem*{scratch}{Scratch Work}


\newtheorem*{namedconj}{Conjecture}
\newcounter{conj}%makes pointer correct
\providecommand{\conjname}{Conjecture}
\makeatletter
\NewDocumentEnvironment{conj}{o}
 {%
  \IfValueTF{#1}
    {\innerconj[#1]\refstepcounter{conj}\def\@currentlabel{(#1)}}%
    {\innerconj}%
 }
 {%
  \endinnerconj
 }
\makeatother

%\let\br\relax
%\let\endbr\relax

%\newcounter{br}%makes pointer correct
%\counterwithin{br}{section}
%
%\newenvironment{br}[1][2in]%
%{%Env start code
%\problemEnvironmentStart{#1}{In-class Problem}
%\refstepcounter{br}
%\stepcounter{problem}
%}
%{%Env end code
%\problemEnvironmentEnd
%}

\let\question\relax
\let\endquestion\relax

\newtheoremstyle{ExerciseStyle}{\topsep}{\topsep}%%% space between body and thm
		{}                      %%% Thm body font
		{}                              %%% Indent amount (empty = no indent)
		{\bfseries}            %%% Thm head font
		{}                              %%% Punctuation after thm head
		{3em}                           %%% Space after thm head
		{{#1}~\thmnumber{#2}\thmnote{ \bfseries(#3)}}%%% Thm head spec
\theoremstyle{ExerciseStyle}
\newtheorem{br}{In-class Problem}


\newcounter{ex}%makes pointer correct
\providecommand{\exname}{Homework Problem}
\newenvironment{ex}[1][2in]%
{%Env start code
\problemEnvironmentStart{#1}{Homework Problem}
\refstepcounter{ex}
}
{%Env end code
\problemEnvironmentEnd
}

\newcommand{\inlineAnswer}[2][2 cm]{
    \ifhandout{\pdfOnly{\rule{#1}{0.4pt}}}
    \else{\answer{#2}}
    \fi
}

\ifhandout
\newenvironment{shortAnswer}[1][
    \vfill]
        {% Begin then result
        #1
            \begin{freeResponse}
            }
    {% Environment Ending Code
    \end{freeResponse}
    }
\else
\newenvironment{shortAnswer}[1][]
        {\begin{freeResponse}
            }
    {% Environment Ending Code
    \end{freeResponse}
    }
\fi

\newenvironment{sketch}
 {\begin{proof}[Sketch of Proof]}
 {\end{proof}}


\newcommand{\gt}{>}
\newcommand{\lt}{<}
\newcommand{\N}{\mathbb N}
\newcommand{\Q}{\mathbb Q}
\newcommand{\Z}{\mathbb Z}
\newcommand{\C}{\mathbb C}
\newcommand{\R}{\mathbb R}
\renewcommand{\H}{\mathbb{H}}
\newcommand{\lcm}{\operatorname{lcm}}
\newcommand{\nequiv}{\not\equiv}
\newcommand{\ord}{\operatorname{ord}}
\newcommand{\ds}{\displaystyle}
\newcommand{\floor}[1]{\left\lfloor #1\right\rfloor}
\newcommand{\legendre}[2]{\left(\frac{#1}{#2}\right)}



%%%%%%%%%%%%




\date{January 19, 2024}

\begin{document}
\handoutAbstract
\maketitle
  \begin{center}%
    {\large \scshape MAT-255-- Number Theory \hfill Spring 2024 \hfill In Class Work January 19}%
    
    {\large
        Your Name: \hrulefill \quad 
        Group Members:\hrulefill \quad \hrulefill
        \par}%
  \end{center}%

Use the proofs of the following propositions as a guide.

\begin{proposition}%[Strayer, Proposition 1.1]
    Let $a,b\in\Z$. If $a\mid b$ and $b \mid c$, then $a\mid c$.
    \begin{proof}
        Since $a\mid b$ and $b \mid c$, there exist $d,e\in\Z$ such that $b=ae$ and $c=bf$. Combining these, we see \[c=bf=(ae)f=a(ef),\] so $a\mid c$.
    \end{proof}
\end{proposition}



\begin{proposition}%[Strayer, Proposition 1.2] 
    Let $a,b,c,m,n\in\Z$.
    If $c\mid a$ and $c\mid b$ then $c\mid ma+nb$.

    \begin{proof}
        Let $a,b,c,m,n\in\Z$ such that $c\mid a$ and $c\mid b$. Then by definition of divisibility, there exists $j,k\in\Z$ such that $cj=a$ and $ck=b$. Thus, \[ma+nb=m(cj)+n(ck)=c(mj+nk).\] Therefore, $c\mid ma+nb$ by definition.
    \end{proof}
\end{proposition}

\begin{br}%[Exercise Set 1.1, Exercise 5]\label{divisfacts}
    Prove or disprove the following statements.
    \begin{enumerate}
        \item If $a,b,c,$ and $d$ are integers such that if $a\mid b$ and $c\mid d$, then $a+c\mid b+d$.
        \item If $a,b,c,$ and $d$ are integers such that if $a\mid b$ and $c\mid d$, then $ac\mid bd$.
        \item If $a,b,$ and $c$ are integers such that if $a\nmid b$ and $b\nmid c$, then $a\nmid c$.
    \end{enumerate}
\end{br}

\pdfOnly{\ifhandout{
    \vfill
    \vfill}
\else
\fi}



\begin{br}
    Construct a truth table for $A\rightarrow B, \neg (A\rightarrow B)$ and $A\land \neg B$
    \begin{solution}
 
        \begin{tabular}{c|c|c|c|c}
            $A$ 	& $B$	& $A\Rightarrow B$ 	& $\neg (A\Rightarrow B)$ & $A\land \neg B$\\\hline
            T 	& T		& T 				& F					& F	\\
            T 	& F 		& F 				& T					& T\\
            F 	& T 		& T 				& F					& F\\
            F 	& F 		& T 				& F					& F\\
        \end{tabular}
    
    \end{solution}
    \pdfOnly{\ifhandout{
        \vfill
        \vfill}
    \else
    \fi}
\end{br}

 
\pdfOnly{\ifhandout{
    Pause for more lecture. If there is time, complete the following problem.

    \pagebreak}
\else
\fi}
 
\begin{br}
Prove that our two definitions of even are equivalent using the following outline:
    \begin{proposition}
        Let $n\in\Z$. Then there is some $k\in\Z$ such that $n=2k$ if and only if $2\mid n$.

        \begin{proof}
            $(\Rightarrow)$  Let $n\in\Z$. Assume that there is some $k\in\Z$ such that $n=2k$.  
            Thus, $2\mid n$ 
            \begin{shortAnswer}[\hrulefill]
                by definition of divides.
            \end{shortAnswer}

 
            $(\Leftarrow)$  Let $n\in\Z$. Assume that $2\mid n$. Then, there is some $k\in\Z$ such that $n=2k$
            \begin{shortAnswer}[\hrulefill.]
                by definition of divides.
            \end{shortAnswer}
        \end{proof}
    \end{proposition}
\end{br}


\begin{br}
    Prove that our two definitions of odd are equivalent using the following outline:
        \begin{proposition}
        Let $n\in\Z$. Then there is some $k\in\Z$ such that $n=2k+1$ if and only if $2\nmid k$.

        \begin{proof}
            $(\Rightarrow)$  Let $n\in\Z$. Assume that there is some $k\in\Z$ such that $n=2k+1$. Then 
            \begin{shortAnswer}[\vspace{3cm}]
                by the division algorithm, there exists unique $q,r\in\Z$ such that $n=2q+r$ and $0\leq r <2.$
            \end{shortAnswer}
            Thus, $2\nmid k$.

            $(\Leftarrow)$  Let $n\in\Z$. Assume that $2\nmid k$. Then
            \begin{shortAnswer}[\vspace{3cm}]
                by the division algorithm, there exists unique $q,r\in\Z$ such that $n=2q+r$ and $0< r <2.$ Thus, $r=1.1$ 
            \end{shortAnswer}
            Thus, there is some $k\in\Z$ such that $n=2k+1$.
        \end{proof}
    \end{proposition}
\end{br}
\end{document}








