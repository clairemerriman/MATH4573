<<<<<<< Updated upstream
\documentclass[handout]{ximera}
\usepackage{amsmath,multicol,amsthm,alltt,color, listings,xr-hyper,hyperref}
\usepackage{xparse}

\usepackage{parskip}
\usepackage[,margin=0.7in]{geometry}
\setlength{\textheight}{8.5in}

%%%fonts
%\usepackage{euler}
\usepackage{pbsi} %% Answer font

\usepackage{epstopdf}

\DeclareGraphicsExtensions{.eps}

%
%\usepackage{tkz-euclide}
%%\usetkzobj{all}
%\tikzstyle geometryDiagrams=[rounded corners=.5pt,ultra thick,color=black]
%\colorlet{penColor}{black} % Color of a curve in a plot


\usepackage{subcaption}
\usepackage{float}
\usepackage{fancyhdr}
%\usepackage{pdfpages}
%\newcounter{includepdfpage}
\usepackage{makecell}

%
%\usepackage{currfile}
%\usepackage{xstring}


\lhead{\large{Number Theory: MAT-255}}
\chead{}
\rhead{Spring 2024}
\lfoot{}
\cfoot{}
\rfoot{Page \thepage}
\renewcommand\headrulewidth{0pt}
\renewcommand\footrulewidth{0pt}

\headheight 50pt
\headsep 30pt

\author{Claire Merriman}

%%%%%%%%%%%%%%%%%%%%%
% Create handoutstyle for in class assignments
%%%%%%%%%%%%%%%%%%%%%
\makeatletter
 \newcommand\handoutstyle{%
  \def\activitystyle{activity-handout}
  \def\maketitle{\addtocounter{titlenumber}{1}%
  \addcontentsline{toc}{section}{\@date}%
        \setcounter{br}{0}}
 }

\newcommand{\handoutAbstract}{\begin{abstract}
\end{abstract}}
\makeatother

%%%%%%%%%%%%%%%%%%%%%
% Counters and autoref for unnumbered environments
%%%%%%%%%%%%%%%%%%%%%
\theoremstyle{plain}


\newtheorem*{namedthm}{Theorem}
\newcounter{thm}%makes pointer correct
\providecommand{\thmname}{Proposition}

\makeatletter
\NewDocumentEnvironment{thm*}{o}
 {%
  \IfValueTF{#1}
    {\namedthm[#1]\refstepcounter{thm}\def\@currentlabel{(#1)}}%
    {\namedthm}%
 }
 {%
  \endnamedthm
 }
\makeatother


\newtheorem*{namedprop}{Proposition}
\newcounter{prop}%makes pointer correct
\providecommand{\propname}{Proposition}

\makeatletter
\NewDocumentEnvironment{prop*}{o}
 {%
  \IfValueTF{#1}
    {\namedprop[#1]\refstepcounter{prop}\def\@currentlabel{(#1)}}%
    {\namedprop}%
 }
 {%
  \endnamedprop
 }
\makeatother

\newtheorem*{namedlem}{Lemma}
\newcounter{lem}%makes pointer correct
\providecommand{\lemname}{Lemma}

\makeatletter
\NewDocumentEnvironment{lem*}{o}
 {%
  \IfValueTF{#1}
    {\namedlem[#1]\refstepcounter{lem}\def\@currentlabel{(#1)}}%
    {\namedlem}%
 }
 {%
  \endnamedlem
 }
\makeatother

\newtheorem*{namedcor}{Corollary}
\newcounter{cor}%makes pointer correct
\providecommand{\corname}{Corollary}

\makeatletter
\NewDocumentEnvironment{cor*}{o}
 {%
  \IfValueTF{#1}
    {\namedcor[#1]\refstepcounter{cor}\def\@currentlabel{(#1)}}%
    {\namedcor}%
 }
 {%
  \endnamedcor
 }
\makeatother

\theoremstyle{definition}
\newtheorem*{annotation}{Annotation}
\newtheorem*{rubric}{Rubric}

\newtheorem*{innerrem}{Remark}
\newcounter{rem}%makes pointer correct
\providecommand{\remname}{Remark}

\makeatletter
\NewDocumentEnvironment{rem}{o}
 {%
  \IfValueTF{#1}
    {\innerrem[#1]\refstepcounter{rem}\def\@currentlabel{(#1)}}%
    {\innerrem}%
 }
 {%
  \endinnerrem
 }
\makeatother

\newtheorem*{innerdefn}{Definition}%%placeholder
\newcounter{defn}%makes pointer correct
\providecommand{\defnname}{Definition}

\makeatletter
\NewDocumentEnvironment{defn}{o}
 {%
  \IfValueTF{#1}
    {\innerdefn[#1]\refstepcounter{defn}\def\@currentlabel{(#1)}}%
    {\innerdefn}%
 }
 {%
  \endinnerdefn
 }
\makeatother

\newtheorem*{scratch}{Scratch Work}


\newtheorem*{namedconj}{Conjecture}
\newcounter{conj}%makes pointer correct
\providecommand{\conjname}{Conjecture}
\makeatletter
\NewDocumentEnvironment{conj}{o}
 {%
  \IfValueTF{#1}
    {\innerconj[#1]\refstepcounter{conj}\def\@currentlabel{(#1)}}%
    {\innerconj}%
 }
 {%
  \endinnerconj
 }
\makeatother

%\let\br\relax
%\let\endbr\relax

%\newcounter{br}%makes pointer correct
%\counterwithin{br}{section}
%
%\newenvironment{br}[1][2in]%
%{%Env start code
%\problemEnvironmentStart{#1}{In-class Problem}
%\refstepcounter{br}
%\stepcounter{problem}
%}
%{%Env end code
%\problemEnvironmentEnd
%}

\let\question\relax
\let\endquestion\relax

\newtheoremstyle{ExerciseStyle}{\topsep}{\topsep}%%% space between body and thm
		{}                      %%% Thm body font
		{}                              %%% Indent amount (empty = no indent)
		{\bfseries}            %%% Thm head font
		{}                              %%% Punctuation after thm head
		{3em}                           %%% Space after thm head
		{{#1}~\thmnumber{#2}\thmnote{ \bfseries(#3)}}%%% Thm head spec
\theoremstyle{ExerciseStyle}
\newtheorem{br}{In-class Problem}


\newcounter{ex}%makes pointer correct
\providecommand{\exname}{Homework Problem}
\newenvironment{ex}[1][2in]%
{%Env start code
\problemEnvironmentStart{#1}{Homework Problem}
\refstepcounter{ex}
}
{%Env end code
\problemEnvironmentEnd
}

\newcommand{\inlineAnswer}[2][2 cm]{
    \ifhandout{\pdfOnly{\rule{#1}{0.4pt}}}
    \else{\answer{#2}}
    \fi
}

\ifhandout
\newenvironment{shortAnswer}[1][
    \vfill]
        {% Begin then result
        #1
            \begin{freeResponse}
            }
    {% Environment Ending Code
    \end{freeResponse}
    }
\else
\newenvironment{shortAnswer}[1][]
        {\begin{freeResponse}
            }
    {% Environment Ending Code
    \end{freeResponse}
    }
\fi

\newenvironment{sketch}
 {\begin{proof}[Sketch of Proof]}
 {\end{proof}}


\newcommand{\gt}{>}
\newcommand{\lt}{<}
\newcommand{\N}{\mathbb N}
\newcommand{\Q}{\mathbb Q}
\newcommand{\Z}{\mathbb Z}
\newcommand{\C}{\mathbb C}
\newcommand{\R}{\mathbb R}
\renewcommand{\H}{\mathbb{H}}
\newcommand{\lcm}{\operatorname{lcm}}
\newcommand{\nequiv}{\not\equiv}
\newcommand{\ord}{\operatorname{ord}}
\newcommand{\ds}{\displaystyle}
\newcommand{\floor}[1]{\left\lfloor #1\right\rfloor}
\newcommand{\legendre}[2]{\left(\frac{#1}{#2}\right)}



%%%%%%%%%%%%




\date{April 1, 2024}
=======
\documentclass[handout,nooutcomes]{ximera}

\usepackage{amssymb, latexsym, amsmath, amsthm, graphicx, amsthm,alltt,color, listings,multicol,xr-hyper,hyperref,aliascnt,enumitem}
\usepackage{xfrac}

\usepackage{parskip}
\usepackage[,margin=0.7in]{geometry}
\setlength{\textheight}{8.5in}

\usepackage{epstopdf}

\DeclareGraphicsExtensions{.eps}
\usepackage{tikz}


\usepackage{tkz-euclide}
%\usetkzobj{all}
\tikzstyle geometryDiagrams=[rounded corners=.5pt,ultra thick,color=black]
\colorlet{penColor}{black} % Color of a curve in a plot


\usepackage{subcaption}
\usepackage{float}
\usepackage{fancyhdr}
\usepackage{pdfpages}
\newcounter{includepdfpage}
\usepackage{makecell}


\usepackage{currfile}
\usepackage{xstring}




\graphicspath{  
{./otherDocuments/}
}

\author{Claire Merriman}
\newcommand{\classday}[1]{\def\classday{#1}}

%%%%%%%%%%%%%%%%%%%%%
% Counters and autoref for unnumbered environments
% Not needed??
%%%%%%%%%%%%%%%%%%%%%
\theoremstyle{plain}


\newtheorem*{namedthm}{Theorem}
\newcounter{thm}%makes pointer correct
\providecommand{\thmname}{Theorem}

\makeatletter
\NewDocumentEnvironment{thm*}{o}
 {%
  \IfValueTF{#1}
    {\namedthm[#1]\refstepcounter{thm}\def\@currentlabel{(#1)}}%
    {\namedthm}%
 }
 {%
  \endnamedthm
 }
\makeatother


\newtheorem*{namedprop}{Proposition}
\newcounter{prop}%makes pointer correct
\providecommand{\propname}{Proposition}

\makeatletter
\NewDocumentEnvironment{prop*}{o}
 {%
  \IfValueTF{#1}
    {\namedprop[#1]\refstepcounter{prop}\def\@currentlabel{(#1)}}%
    {\namedprop}%
 }
 {%
  \endnamedprop
 }
\makeatother

\newtheorem*{namedlem}{Lemma}
\newcounter{lem}%makes pointer correct
\providecommand{\lemname}{Lemma}

\makeatletter
\NewDocumentEnvironment{lem*}{o}
 {%
  \IfValueTF{#1}
    {\namedlem[#1]\refstepcounter{lem}\def\@currentlabel{(#1)}}%
    {\namedlem}%
 }
 {%
  \endnamedlem
 }
\makeatother

\newtheorem*{namedcor}{Corollary}
\newcounter{cor}%makes pointer correct
\providecommand{\corname}{Corollary}

\makeatletter
\NewDocumentEnvironment{cor*}{o}
 {%
  \IfValueTF{#1}
    {\namedcor[#1]\refstepcounter{cor}\def\@currentlabel{(#1)}}%
    {\namedcor}%
 }
 {%
  \endnamedcor
 }
\makeatother

\theoremstyle{definition}
\newtheorem*{annotation}{Annotation}
\newtheorem*{rubric}{Rubric}

\newtheorem*{innerrem}{Remark}
\newcounter{rem}%makes pointer correct
\providecommand{\remname}{Remark}

\makeatletter
\NewDocumentEnvironment{rem}{o}
 {%
  \IfValueTF{#1}
    {\innerrem[#1]\refstepcounter{rem}\def\@currentlabel{(#1)}}%
    {\innerrem}%
 }
 {%
  \endinnerrem
 }
\makeatother

\newtheorem*{innerdefn}{Definition}%%placeholder
\newcounter{defn}%makes pointer correct
\providecommand{\defnname}{Definition}

\makeatletter
\NewDocumentEnvironment{defn}{o}
 {%
  \IfValueTF{#1}
    {\innerdefn[#1]\refstepcounter{defn}\def\@currentlabel{(#1)}}%
    {\innerdefn}%
 }
 {%
  \endinnerdefn
 }
\makeatother

\newtheorem*{scratch}{Scratch Work}


\newtheorem*{namedconj}{Conjecture}
\newcounter{conj}%makes pointer correct
\providecommand{\conjname}{Conjecture}
\makeatletter
\NewDocumentEnvironment{conj}{o}
 {%
  \IfValueTF{#1}
    {\innerconj[#1]\refstepcounter{conj}\def\@currentlabel{(#1)}}%
    {\innerconj}%
 }
 {%
  \endinnerconj
 }
\makeatother

\newtheorem*{poll}{Poll question}
\newtheorem{tps}{Think-Pair-Share}[section]


\newenvironment{obj}{
	\textbf{Learning Objectives.} By the end of class, students will be able to:
		\begin{itemize}}
		{\!.\end{itemize}
		}

\newenvironment{pre}{
	\begin{description}
	}{
	\end{description}
}


\newcounter{ex}%makes pointer correct
\providecommand{\exname}{Homework Problem}
\newenvironment{ex}[1][2in]%
{%Env start code
\problemEnvironmentStart{#1}{Homework Problem}
\refstepcounter{ex}
}
{%Env end code
\problemEnvironmentEnd
}

\newcommand{\inlineAnswer}[2][2 cm]{
    \ifhandout{\pdfOnly{\rule{#1}{0.4pt}}}
    \else{\answer{#2}}
    \fi
}


\ifhandout
\newenvironment{shortAnswer}[1][
    \vfill]
        {% Begin then result
        #1
            \begin{freeResponse}
            }
    {% Environment Ending Code
    \end{freeResponse}
    }
\else
\newenvironment{shortAnswer}[1][]
        {\begin{freeResponse}
            }
    {% Environment Ending Code
    \end{freeResponse}
    }
\fi

\let\question\relax
\let\endquestion\relax

\newtheoremstyle{ExerciseStyle}{\topsep}{\topsep}%%% space between body and thm
		{}                      %%% Thm body font
		{}                              %%% Indent amount (empty = no indent)
		{\bfseries}            %%% Thm head font
		{}                              %%% Punctuation after thm head
		{3em}                           %%% Space after thm head
		{{#1}~\thmnumber{#2}\thmnote{ \bfseries(#3)}}%%% Thm head spec
\theoremstyle{ExerciseStyle}
\newtheorem{br}{In-class Problem}

\newenvironment{sketch}
 {\begin{proof}[Sketch of Proof]}
 {\end{proof}}


\newcommand{\gt}{>}
\newcommand{\lt}{<}
\newcommand{\N}{\mathbb N}
\newcommand{\Q}{\mathbb Q}
\newcommand{\Z}{\mathbb Z}
\newcommand{\C}{\mathbb C}
\newcommand{\R}{\mathbb R}
\renewcommand{\H}{\mathbb{H}}
\newcommand{\lcm}{\operatorname{lcm}}
\newcommand{\nequiv}{\not\equiv}
\newcommand{\ord}{\operatorname{ord}}
\newcommand{\ds}{\displaystyle}
\newcommand{\floor}[1]{\left\lfloor #1\right\rfloor}
\newcommand{\legendre}[2]{\left(\frac{#1}{#2}\right)}



%%%%%%%%%%%%


\lhead{\large{Number Theory: MAT-255}}
%Put your Document Title (Camp: Topic) Here
\chead{}
\rhead{Spring 2024}
\lfoot{}
\cfoot{}
\rfoot{Page \thepage}
\renewcommand\headrulewidth{0pt}
\renewcommand\footrulewidth{0pt}

\headheight 50pt
\headsep 30pt




%%%%%%%%%%%%%%%%%%%%%
% Create a chapter divider where there is not an intro file
% Should be a place holder until there is a file
%%%%%%%%%%%%%%%%%%%%%
\newcommand{\chapter}[1]{\addtocounter{titlenumber}{1}%
{\flushleft\LARGE\sffamily\bfseries\thetitlenumber\hspace{1em}#1 \par }%
{\vskip .6em\noindent\textit\theabstract\setcounter{problem}{0}\setcounter{sectiontitlenumber}{0}}%
\par\vspace{2em}
\phantomsection\addcontentsline{toc}{section}{\textbf{\thetitlenumber\hspace{1em}#1}}%
}

<<<<<<< Updated upstream

=======
\makeatletter
\renewcommand\chapterstyle{%
  \def\activitystyle{activity-chapter}
  \def\maketitle{%
    \addtocounter{titlenumber}{1}%
        {\flushleft\small\sffamily\bfseries\@pretitle\par\vspace{-1.5em}}%
        {\flushleft\LARGE\sffamily\bfseries\thetitlenumber\hspace{1em}\@title \par }%
        {\vskip .6em\noindent\textit\theabstract\setcounter{problem}{0}\setcounter{sectiontitlenumber}{0}}%
        \par\vspace{2em}
        \phantomsection\addcontentsline{toc}{section}{\textbf{\thetitlenumber\hspace{1em}\@title}}%
        \let\section\subsection
        \let\subsection\subsubsection
    }}

\renewcommand\sectionstyle{%
    \def\activitystyle{activity-section}
    \def\maketitle{%
        \addtocounter{sectiontitlenumber}{1}
        {\flushleft\small\sffamily\bfseries\@pretitle\par\vspace{-1.5em}}%
        {\flushleft\Large\sffamily\bfseries\thetitlenumber.\thesectiontitlenumber\hspace{1em}\@title \par}%
        {\vskip .6em\noindent\textit\theabstract}%
        \par\vspace{2em}
        \phantomsection\addcontentsline{toc}{subsection}{\thetitlenumber.\thesectiontitlenumber\hspace{1em}\@title}%
        \let\section\subsubsection
        \let\subsection\subsubsubsection
    }}

\makeatother
>>>>>>> Stashed changes

%%%%%%%%%%%%%%%%%%%%%
% Create handoutstyle for in class assignments
%%%%%%%%%%%%%%%%%%%%%
<<<<<<< Updated upstream
\makeatletter
 \newcommand\handoutstyle{%
    \addtocounter{titlenumber}{1}%
    \phantomsection\addcontentsline{toc}{section}{\@date}%
        \setcounter{br}{0}}
%
%
%\newcommand{\handoutTitle}{\title[%
%\textnormal{\large \scshape MAT-255-- Number Theory \hfill Spring 2024 \hfill In Class Work \classday}%
%
%\textnormal{\large
%Your Name: \hrulefill \quad Group Members:\hrulefill \quad \hrulefill}%
%\vspace{-5em}]{}}
=======
\newcommand{\handoutTitle}{
    \title[%
        \textnormal{\large \scshape MAT-255-- Number Theory \hfill Spring 2024 \hfill In Class Work \classday}%
        
        \textnormal{\large
        Your Name: \hrulefill \quad Group Members:\hrulefill \quad \hrulefill}%
        \vspace{-5em}]{}
    }


\makeatletter
\newcommand\handoutstyle{%
\def\activitystyle{activity-handout}
\def\maketitle{
    \renewcommand{\handoutTitle}{\title{In Class \classday}}
    \phantomsection\addcontentsline{toc}{subsection}{\@date}%
      \setcounter{br}{0}
      \setcounter{theorem}{0}
       \setcounter{proposition}{0}
       \setcounter{lemma}{0}}
   }
>>>>>>> Stashed changes

\newcommand{\handoutAbstract}{\begin{abstract}
\end{abstract}}

\makeatother

\classday{April 1}


\handoutTitle
        
\date{\classday, 2024}
>>>>>>> Stashed changes

\begin{document}
\handoutAbstract
\maketitle
<<<<<<< Updated upstream
 	\begin{center}%
    	{\large \scshape MAT-255-- Number Theory 
			\hfill Spring 2024 
			\hfill In Class Work April 1}%
    
		{\large Your Name: \hrulefill \quad 
			Group Members:\hrulefill \quad 
			\hrulefill
			\par}%
 	\end{center}%
=======
>>>>>>> Stashed changes
	 
\section*{Results}
\begin{theorem}[Euler's Criterion]\label{thm:euler-quads}
	Let $p$ be an odd prime and $a\in\Z$ with $p\nmid a.$ Then \[\legendre{a}{p}\equiv a^{(p-1)/2}\pmod{p}\]
\end{theorem}

%\begin{proposition}
%	Let $p$ be an odd prime number and $a,b\in\Z$ with $p\nmid a$ and $p\nmid b.$ Then 
%	\begin{enumerate}[label=(\alph*)]
%		\item $\legendre{a^2}{p}=1$ \label{squares-are-square}
%		\item If $a\equiv b\pmod{p}$ then $\legendre{a}{p}=\legendre{b}{p}$ \label{legendre-respects-mod}
%		\item $\legendre{ab}{p}=\legendre{a}{p}\legendre{b}{p}$ \label{legendre-mult}
%	\end{enumerate}
%\end{proposition}

\begin{theorem}[Theorem 4.6]\label{thm:residue-neg1}
	Let $p$ be an odd prime number. Then 
	\[
		\legendre{-1}{p}=
			\begin{cases}
 				1, & p\equiv 1\pmod{4}\\
				-1, & p\equiv 3\pmod{4}
			\end{cases}.
	\]
\end{theorem}

\begin{theorem}[Quadratic reciprocity]\label{quad-rec}
	Let $p$ and $q$ be distinct primes.  
	\begin{enumerate}[label=(\alph*)]
		\item If $p\equiv 1 \pmod{4}$ or $q\equiv 1\pmod{4},$ then $\legendre{p}{q}=\legendre{q}{p}$
 		\item If $p\equiv q \equiv 3 \pmod{4},$ then $\legendre{p}{q}=-\legendre{q}{p}$
	\end{enumerate}
\end{theorem}

\begin{lemma}[Gauss's Lemma]\label{lem:gauss}
	Let $p$ be an odd prime number and like $a\in\Z$ with $p\nmid a$. Let $n$ be the number of least positive residues of the integers $a,2a,3a,\dots,\frac{p-1}{a}$ modulo $p$ that are greater than $\frac{p}{2}.$ Then \[\legendre{a}{p}=(-1)^n.\]
\end{lemma}


\section*{Problems}

We can combine these results to find the Legendre symbol many different ways.

\begin{br}
	Use the following methods to find $\legendre{-6}{11}$:
 
	\begin{enumerate}
		\item Euler's Criterion, from March 22: 
		
        $\left(\frac{-6}{11}\right)\equiv (-6)^{(11-1)/2}\equiv (-6)^{5}\pmod{11}$ By Euler's Criterion. Then
				\[
					(-6)^{5}\equiv ((6)^{2})^{2}(-6)\equiv 3^2(-6)\equiv -54 \equiv 1\pmod{11}\qedhere
				\]
		
		
		\item Factor into $\legendre{-6}{11}=\legendre{-1}{11}\legendre{2}{11}\legendre{3}{11}=(\inlineAnswer[1cm]{-1})\legendre{2}{11}\legendre{3}{11}$. From here, we will explore the various was to find $\legendre{2}{11}$ and $\legendre{3}{11}$.
		\pdfOnly{\ifhandout
        \pagebreak
        \else\fi}
		
		\begin{enumerate}
 			\item Find $\legendre{2}{11}$ using the specified method:
			\begin{itemize}
			    \item Using \nameref{thm:euler-quads}.
				\begin{solution} 
					From  \nameref{thm:euler-quads}, 
						\[
							\legendre{2}{11}\equiv 2^{(11-1)/2}\equiv 32\equiv -1\pmod{11}.
						\]
				\end{solution}
        \pdfOnly{\ifhandout
            \vfill
            \else\fi}
					
			
				\item Using \nameref{lem:gauss}.
				\begin{solution} 
					First, find the least nonnegative residues of $2, 2(2), 3(2),4(2), 5(2)$ modulo $11.$ 
					These are \[2,4,6,8,10,\] 
					and $n=\inlineAnswer{3}$ are greater than $\frac{11}{2}.$ Thus, by \nameref{lem:gauss}, \[\legendre{2}{11}=(-1)^{\inlineAnswer{3}}=\inlineAnswer{3}.\qedhere\]
				\end{solution}
                \pdfOnly{\ifhandout
                    \vfill
                    \else\fi}
			\end{itemize}
				
			\item Find $\legendre{3}{11}$ using the specified method:
			\begin{itemize}
				\item Using \nameref{thm:euler-quads}.
				\begin{solution} 
					From  \nameref{thm:euler-quads}, 
						\[
							\legendre{3}{11}\equiv 3^{(11-1)/2}
							\equiv (-2)^2(3)\equiv 
							1\pmod{11}.\qedhere
						\]
				\end{solution}
                \pdfOnly{\ifhandout
                    \vfill
                    \else\fi}
                                
			
				\item Using \nameref{quad-rec}
				\begin{solution}
					Since $11\equiv 3\pmod{4},$ $\legendre{3}{11}=-\legendre{11}{3}=-\legendre{2}{3}=1.$
			 	\end{solution}
                \pdfOnly{\ifhandout
                    \vfill
                    \else\fi}
							

				\item Using \nameref{lem:gauss}.
				\begin{solution} 
					First, find the least nonnegative residues of $3, 2(3), 3(3),4(3), 5(3)$ modulo $11.$ 
					These are \[\inlineAnswer{3,6,9,1,4}\] and $n=\inlineAnswer{2}$ are greater than $\frac{11}{2}.$ Thus, by \nameref{lem:gauss}, \[\legendre{3}{11}=(-1)^{\inlineAnswer{2}}=\inlineAnswer{1}.\qedhere\]
				\end{solution}
        \pdfOnly{\ifhandout
            \vfill
            \else\fi}
			\end{itemize}


		\end{enumerate}
		Thus, $\legendre{-6}{11}=\inlineAnswer[1cm]{1}$

		\item Use that $-6\equiv 5\pmod{11},$ so $\legendre{-6}{11}=\legendre{5}{11}.$ Then find $\legendre{5}{11}$ the specified method:
			
		\begin{enumerate}
			\item Using \nameref{thm:euler-quads}.
			\begin{solution} 
				From  \nameref{thm:euler-quads}, 
				\[
                    \legendre{5}{11}\equiv 5^{(11-1)/2}
                    \equiv (3)^2(5)\equiv
                    1\pmod{11}.
                \]
			\end{solution}
        \pdfOnly{\ifhandout
            \vfill
            \else\fi}
				
	
			\item Using \nameref{quad-rec}
			\begin{solution}
                Since $5\equiv 1\pmod{4},$ $\legendre{5}{11}=\legendre{11}{5}=\legendre{1}{5}=1.$
			\end{solution}
        \pdfOnly{\ifhandout
            \vfill
            \else\fi}
					

			\item Using \nameref{lem:gauss}.
				\begin{solution} 
					First, find the least nonnegative residues of $5, 2(5), 3(5),4(5), 5(5)$ modulo $11.$ 
					These are \[\inlineAnswer{5,10,4,9,2},\]
                    and $n=\inlineAnswer{2}$ are greater than $\frac{11}{2}.$ Thus, by \nameref{lem:gauss}, \[\legendre{5}{11}=(-1)^{\inlineAnswer{2}}=\inlineAnswer{1}.\]
				\end{solution}
        \pdfOnly{\ifhandout
            \vfill
            \else\fi}
		\end{enumerate}
    \end{enumerate}
\end{br}


\begin{br}
	Now we will examine the Legendre symbol of $2$ using Gauss's Lemma. First, note that $2,2(2),3(2),\dots,2(\frac{p-1}{2})$ are already least nonnegative residues modulo $p.$ It will be slightly easier to count how many are \emph{less than} $\frac{p}{2},$ then subtract from the total number, $\frac{p-1}{2}.$

	Let $k\in\Z$ with $1\leq k\leq \frac{p-1}{2}.$ Then $2k< \frac{p}{2}$ if and only if $k <\inlineAnswer[1cm]{\floor{\frac{p}{4}}}.$ Thus, $\frac{p-1}{2}-\floor{\inlineAnswer[1cm]{\frac{p}{4}}}$ of $2,2(2),3(2),\dots,2(\frac{p-1}{2})$ are greater than $\frac{p}{2}.$ 
    \begin{hint}
        The two blanks should be the same, and also go in the blanks below
    \end{hint}

	Now complete this table

	\pdfOnly{\renewcommand{\arraystretch}{2}}
	\begin{tabular}{c|p{1.5cm}|p{2.5cm}|p{7cm}|p{3cm}}
        $p$ & $\floor{\inlineAnswer[1 cm]{\frac{p}{4}}}$ & $\frac{p-1}{2}-\floor{\inlineAnswer[1 cm]{\frac{p}{4}}}$ & $2,2(2),3(2),\dots,2(\frac{p-1}{2})$ & $\legendre{2}{p}$\\\hline
        $3$ & \begin{prompt}
        $\answer{0}$\end{prompt}	& \begin{prompt}
        $\answer{1}$\end{prompt}
            & \makecell[l]{Less than $\tfrac{3}{2}:$ \begin{prompt}
        $\answer{N/A}$\end{prompt}
            \\Greater than $\tfrac{3}{2}:$ \begin{prompt} 
            $\answer{2}$\end{prompt}} 
            & \begin{prompt}
            $(-1)^{\answer{1}}=\answer{-1}$
            \end{prompt} \\\hline
        $5$ & \begin{prompt}
        $\answer{1}$\end{prompt}	& \begin{prompt}
            $\answer{1}$\end{prompt}
            & \makecell[l]{Less than $\tfrac{5}{2}:$ \begin{prompt}
            $\answer{2}$\end{prompt}
                \\Greater than $\tfrac{5}{2}:$ \begin{prompt}
            $\answer{4}$\end{prompt}}
                & \begin{prompt}
            $(-1)^{\answer{1}}=\answer{-1}$\end{prompt}\\\hline
        $7$ & \begin{prompt}
            $\answer{1}$\end{prompt}	& \begin{prompt}
            $\answer{2}$\end{prompt} 
                & \makecell[l]{Less than $\tfrac{7}{2}:$ \begin{prompt}
            $\answer{2}$\end{prompt}
                \\Greater than $\tfrac{7}{2}:$ \begin{prompt}
            $\answer{4,6}$\end{prompt}}
                & \begin{prompt}
            $(-1)^{\answer{2}}=\answer{1}$\end{prompt}\\\hline
        $11$ & \begin{prompt}
            $\answer{2}$\end{prompt}	& \begin{prompt}
            $\answer{3}$\end{prompt}
                & \makecell[l]{Less than $\tfrac{11}{2}:$ \begin{prompt}
            $\answer{2,4}$\end{prompt}
                \\Greater than $\tfrac{11}{2}:$ \begin{prompt}
            $\answer{6,8,10}$\end{prompt}}
                & \begin{prompt}
            $(-1)^{\answer{3}}=\answer{-1}$\end{prompt}\\\hline
        $13$ & \begin{prompt}
            $\answer{3}$\end{prompt}	& \begin{prompt}
            $\answer{3}$\end{prompt}
                & \makecell[l]{Less than $\tfrac{13}{2}:$ \begin{prompt}
            $\answer{2,4,6}$\end{prompt}
                \\Greater than $\tfrac{13}{2}:$ \begin{prompt}
            $\answer{8,10,12}$\end{prompt}}
                & \begin{prompt}
            $(-1)^{\answer{3}}=\answer{-1}$\end{prompt}\\\hline
        $17$ & \begin{prompt}
            $\answer{4}$\end{prompt}	& \begin{prompt}
            $\answer{4}$\end{prompt}
                & \makecell[l]{Less than $\tfrac{17}{2}:$ \begin{prompt}
            $\answer{2,4,6,8}$\end{prompt}
                \\Greater than $\tfrac{17}{2}:$ \begin{prompt}
            $\answer{10,12,14,16}$\end{prompt}}
                & \begin{prompt}
            $(-1)^{\answer{4}}=\answer{1}$\end{prompt}\\\hline
        $19$ & \begin{prompt}
            $\answer{4}$\end{prompt}	& \begin{prompt}
            $\answer{5}$\end{prompt}
                & \makecell[l]{Less than $\tfrac{19}{2}:$ \begin{prompt}
            $\answer{2,4,6,8}$\end{prompt}
            \\Greater than $\tfrac{17}{2}:$ \begin{prompt}
            $\answer{10,12,14,16,18}$\end{prompt}}
            & \begin{prompt}
            $(-1)^{\answer{5}}=\answer{-1}$\end{prompt}\\\hline
            $p$ & \begin{prompt}
            $\answer{5}$\end{prompt}	& \begin{prompt}
            $\answer{6}$\end{prompt}
                & \makecell[l]{Less than $\tfrac{17}{2}:$ \begin{prompt}
            $\answer{2,4,6,8,10}$\end{prompt}
                \\Greater than $\tfrac{17}{2}:$ \begin{prompt}
            $\answer{12,14,16,18,20,22}$\end{prompt}}
                & \begin{prompt}
            $(-1)^{\answer{6}}=\answer{1}$\end{prompt}\\\hline
    \end{tabular}
\end{br}
\end{document}