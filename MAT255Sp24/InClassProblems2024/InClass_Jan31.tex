\documentclass[handout]{ximera}
\usepackage{amsmath,multicol,amsthm,alltt,color, listings,xr-hyper,hyperref}
\usepackage{xparse}

\usepackage{parskip}
\usepackage[,margin=0.7in]{geometry}
\setlength{\textheight}{8.5in}

%%%fonts
%\usepackage{euler}
\usepackage{pbsi} %% Answer font

\usepackage{epstopdf}

\DeclareGraphicsExtensions{.eps}

%
%\usepackage{tkz-euclide}
%%\usetkzobj{all}
%\tikzstyle geometryDiagrams=[rounded corners=.5pt,ultra thick,color=black]
%\colorlet{penColor}{black} % Color of a curve in a plot


\usepackage{subcaption}
\usepackage{float}
\usepackage{fancyhdr}
%\usepackage{pdfpages}
%\newcounter{includepdfpage}
\usepackage{makecell}

%
%\usepackage{currfile}
%\usepackage{xstring}


\lhead{\large{Number Theory: MAT-255}}
\chead{}
\rhead{Spring 2024}
\lfoot{}
\cfoot{}
\rfoot{Page \thepage}
\renewcommand\headrulewidth{0pt}
\renewcommand\footrulewidth{0pt}

\headheight 50pt
\headsep 30pt

\author{Claire Merriman}

%%%%%%%%%%%%%%%%%%%%%
% Create handoutstyle for in class assignments
%%%%%%%%%%%%%%%%%%%%%
\makeatletter
\newcommand\handoutstyle{%
  \def\activitystyle{activity-section}
  \def\maketitle{}{}{%
    	\addtocounter{titlenumber}{1}%
                {\setcounter{problem}{0}
                 \setcounter{br}{0}
                \setcounter{sectiontitlenumber}{0}}%
                    \par\vspace{2em}
                    \phantomsection\addcontentsline{toc}{section}
                    {\thetitlenumber\hspace{1em}\textbf{\@title}}%
        }}


\newcommand{\handoutAbstract}{\begin{abstract}
\end{abstract}}
\makeatother

%%%%%%%%%%%%%%%%%%%%%
% Counters and autoref for unnumbered environments
%%%%%%%%%%%%%%%%%%%%%
\theoremstyle{plain}


\newtheorem*{namedthm}{Theorem}
\newcounter{thm}%makes pointer correct
\providecommand{\thmname}{Proposition}

\makeatletter
\NewDocumentEnvironment{thm*}{o}
 {%
  \IfValueTF{#1}
    {\namedthm[#1]\refstepcounter{thm}\def\@currentlabel{(#1)}}%
    {\namedthm}%
 }
 {%
  \endnamedthm
 }
\makeatother


\newtheorem*{namedprop}{Proposition}
\newcounter{prop}%makes pointer correct
\providecommand{\propname}{Proposition}

\makeatletter
\NewDocumentEnvironment{prop*}{o}
 {%
  \IfValueTF{#1}
    {\namedprop[#1]\refstepcounter{prop}\def\@currentlabel{(#1)}}%
    {\namedprop}%
 }
 {%
  \endnamedprop
 }
\makeatother

\newtheorem*{namedlem}{Lemma}
\newcounter{lem}%makes pointer correct
\providecommand{\lemname}{Lemma}

\makeatletter
\NewDocumentEnvironment{lem*}{o}
 {%
  \IfValueTF{#1}
    {\namedlem[#1]\refstepcounter{lem}\def\@currentlabel{(#1)}}%
    {\namedlem}%
 }
 {%
  \endnamedlem
 }
\makeatother

\newtheorem*{namedcor}{Corollary}
\newcounter{cor}%makes pointer correct
\providecommand{\corname}{Corollary}

\makeatletter
\NewDocumentEnvironment{cor*}{o}
 {%
  \IfValueTF{#1}
    {\namedcor[#1]\refstepcounter{cor}\def\@currentlabel{(#1)}}%
    {\namedcor}%
 }
 {%
  \endnamedcor
 }
\makeatother

\theoremstyle{definition}
\newtheorem*{annotation}{Annotation}
\newtheorem*{rubric}{Rubric}

\newtheorem*{innerrem}{Remark}
\newcounter{rem}%makes pointer correct
\providecommand{\remname}{Remark}

\makeatletter
\NewDocumentEnvironment{rem}{o}
 {%
  \IfValueTF{#1}
    {\innerrem[#1]\refstepcounter{rem}\def\@currentlabel{(#1)}}%
    {\innerrem}%
 }
 {%
  \endinnerrem
 }
\makeatother

\newtheorem*{innerdefn}{Definition}%%placeholder
\newcounter{defn}%makes pointer correct
\providecommand{\defnname}{Definition}

\makeatletter
\NewDocumentEnvironment{defn}{o}
 {%
  \IfValueTF{#1}
    {\innerdefn[#1]\refstepcounter{defn}\def\@currentlabel{(#1)}}%
    {\innerdefn}%
 }
 {%
  \endinnerdefn
 }
\makeatother

\newtheorem*{scratch}{Scratch Work}


\newtheorem*{namedconj}{Conjecture}
\newcounter{conj}%makes pointer correct
\providecommand{\conjname}{Conjecture}
\makeatletter
\NewDocumentEnvironment{conj}{o}
 {%
  \IfValueTF{#1}
    {\innerconj[#1]\refstepcounter{conj}\def\@currentlabel{(#1)}}%
    {\innerconj}%
 }
 {%
  \endinnerconj
 }
\makeatother



\newcounter{br}%makes pointer correct
\counterwithin{br}{section}

\newenvironment{br}[1][2in]%
{%Env start code
\problemEnvironmentStart{#1}{In-class Problem}
\refstepcounter{br}
\stepcounter{problem}
}
{%Env end code
\problemEnvironmentEnd
}

\newcounter{ex}%makes pointer correct
\providecommand{\exname}{Homework Problem}
\newenvironment{ex}[1][2in]%
{%Env start code
\problemEnvironmentStart{#1}{Homework Problem}
\refstepcounter{ex}
}
{%Env end code
\problemEnvironmentEnd
}

\newcommand{\inlineAnswer}[2][2 cm]{
    \ifhandout{\pdfOnly{\rule{#1}{0.4pt}}}
    \else{\answer{#2}}
    \fi
}

\ifhandout
\newenvironment{shortAnswer}[1][
    \vfill]
        {% Begin then result
        #1
            \begin{freeResponse}
            }
    {% Environment Ending Code
    \end{freeResponse}
    }
\else
\newenvironment{shortAnswer}[1][]
        {\begin{freeResponse}
            }
    {% Environment Ending Code
    \end{freeResponse}
    }
\fi

\newenvironment{sketch}
 {\begin{proof}[Sketch of Proof]}
 {\end{proof}}


\newcommand{\gt}{>}
\newcommand{\lt}{<}
\newcommand{\N}{\mathbb N}
\newcommand{\Q}{\mathbb Q}
\newcommand{\Z}{\mathbb Z}
\newcommand{\C}{\mathbb C}
\newcommand{\R}{\mathbb R}
\renewcommand{\H}{\mathbb{H}}
\newcommand{\lcm}{\operatorname{lcm}}
\newcommand{\nequiv}{\not\equiv}
\newcommand{\ord}{\operatorname{ord}}
\newcommand{\ds}{\displaystyle}
\newcommand{\floor}[1]{\left\lfloor #1\right\rfloor}
\newcommand{\legendre}[2]{\left(\frac{#1}{#2}\right)}



%%%%%%%%%%%%



\date{January 31, 2024}

\begin{document}
\handoutAbstract
\maketitle
    \begin{center}%
        {\large \scshape MAT-255-- Number Theory \hfill Spring 2024 \hfill In Class Work January 31}%
    
        {\large
            Your Name: \hrulefill \quad Group Members:\hrulefill \quad \hrulefill
	    \par}%
    \end{center}%

 
\begin{br}%[Strayer Exercises 32 and 54]
    Find the greatest common divisors of the pairs of integers below and write the greatest common divisor as a linear combination of the integers.
    \begin{enumerate}
        \item $(21,28)$
        
        \begin{solution}
            By inspection: $28-21=7$.

            Using the Euclidean Algorithm:
            $a=28,b=21$
            \begin{align*}
                28 & = 21(1)+7 &q_1=1,r_1=7 &&7=21(1)+28(-1)\\
                21 & = 7(3) +0 & q_2=3, r_2=0
            \end{align*}
            so $28+(-1)21=7=(28,21)$
        \end{solution}

        \item $(32,56)$
        \begin{solution}
            Using the Euclidean Algorithm:
            $a=56,b=32$
            \begin{align*}
                56 & = 32(1)+24 &q_1=1,r_1=24 &&24=56(1)+32(-1)\\
                32 & = 24(1) +8 & q_2=1, r_2=8 &&8=32(1)+24(-1)=32(1)+(56(1)+32(-1))(-1)=32(2)+56(-1)\\
                32&=8(4)+0 & q_3=4, r_3=0.
            \end{align*}
            so $56(-1)+32(2)=8=(56,32)$
        \end{solution}

        \item $(0,113)$
        \begin{solution}
            Since $0=113(0)$, $(0,113)=113=0(0)=113(1)$.
        \end{solution}
        
        \item $(78,708)$
        \begin{solution}
            Using the Euclidean Algorithm:
            $a=708,b=78$
            \begin{align*}
                708 & = 78(9)+6 &q_1=9,r_1=6 &&6=708(1)+78(-9)\\
                78 & = 6(13) +0 & q_2=13, r_2=0.
            \end{align*}
            so $708(1)+78(-6)=6=(78,708)$
        \end{solution}
    \end{enumerate}
    \pdfOnly{\ifhandout{
        \vfill}
        \else
        \fi}
\end{br}

\pdfOnly{\ifhandout{
    Pause for more lecture.}
    \else
    \fi}


\begin{br}
	Let $p$ be prime.
	\begin{enumerate}
		\item If $(a,b)=p$, what are the possible values of $(a^2,b)$? Of $(a^3,b)$? Of $(a^2,b^3)$?
		
		\begin{solution}
			If $(a,b)=p$, then there exist $j,k\in\Z$ such that $a=pj, b=pk$, and $p\nmid j$ or $p\nmid k$ (otherwise $(a,b)=p^2$). 
			\[a^2=p^2j^2,\quad
			a^3=p^3j^3,\quad
			b^3=p^3k^3\]
			Then $(a^2,b)$ is $p$ if $p\nmid k$ or $p^2$ if $p\mid k$; and
			$(a^3,b)$ is $p$ if $p\nmid k,$ $p^2$ if $p\mid k$ and $p^2\nmid k,$ or $p^3$ if $p^2\mid k$. 
			
			If $p\mid j,$ then $p\nmid k$ and   
			$(a^2,b^3)=p^3$.
			If $p\nmid j,$ then   
			$(a^2,b^3)=p^2.$
		\end{solution}
		\item If $(a,b)=p$ and $(b,p^3)=p^2$, find $(ab,p^4)$ and $(a+b,p^4)$.
		
		\begin{solution}
			There exists $j,k\in\Z$ such that $a=pj, b=p^2k,$ and $p\nmid k, p\nmid k$. 
			Then $ab=p^3jk$ and $a+b=pj+p^2k=p(j+pk)$. Thus, $(ab,p^4)=p^3$ and $(a+b,p^4)=p.$
		\end{solution}
	\end{enumerate}
    \pdfOnly{\ifhandout{
        \vfill}
        \else
        \fi}
\end{br}
  
  \end{document}