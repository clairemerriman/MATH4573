\documentclass[handout]{ximera}
\usepackage{amsmath,multicol,amsthm,alltt,color, listings,xr-hyper,hyperref}
\usepackage{xparse}

\usepackage{parskip}
\usepackage[,margin=0.7in]{geometry}
\setlength{\textheight}{8.5in}

%%%fonts
%\usepackage{euler}
\usepackage{pbsi} %% Answer font

\usepackage{epstopdf}

\DeclareGraphicsExtensions{.eps}

%
%\usepackage{tkz-euclide}
%%\usetkzobj{all}
%\tikzstyle geometryDiagrams=[rounded corners=.5pt,ultra thick,color=black]
%\colorlet{penColor}{black} % Color of a curve in a plot


\usepackage{subcaption}
\usepackage{float}
\usepackage{fancyhdr}
%\usepackage{pdfpages}
%\newcounter{includepdfpage}
\usepackage{makecell}

%
%\usepackage{currfile}
%\usepackage{xstring}


\lhead{\large{Number Theory: MAT-255}}
\chead{}
\rhead{Spring 2024}
\lfoot{}
\cfoot{}
\rfoot{Page \thepage}
\renewcommand\headrulewidth{0pt}
\renewcommand\footrulewidth{0pt}

\headheight 50pt
\headsep 30pt

\author{Claire Merriman}

%%%%%%%%%%%%%%%%%%%%%
% Create handoutstyle for in class assignments
%%%%%%%%%%%%%%%%%%%%%
\makeatletter
 \newcommand\handoutstyle{%
  \def\activitystyle{activity-handout}
  \def\maketitle{\addtocounter{titlenumber}{1}%
  \addcontentsline{toc}{section}{\@date}%
        \setcounter{br}{0}}
 }

\newcommand{\handoutAbstract}{\begin{abstract}
\end{abstract}}
\makeatother

%%%%%%%%%%%%%%%%%%%%%
% Counters and autoref for unnumbered environments
%%%%%%%%%%%%%%%%%%%%%
\theoremstyle{plain}


\newtheorem*{namedthm}{Theorem}
\newcounter{thm}%makes pointer correct
\providecommand{\thmname}{Proposition}

\makeatletter
\NewDocumentEnvironment{thm*}{o}
 {%
  \IfValueTF{#1}
    {\namedthm[#1]\refstepcounter{thm}\def\@currentlabel{(#1)}}%
    {\namedthm}%
 }
 {%
  \endnamedthm
 }
\makeatother


\newtheorem*{namedprop}{Proposition}
\newcounter{prop}%makes pointer correct
\providecommand{\propname}{Proposition}

\makeatletter
\NewDocumentEnvironment{prop*}{o}
 {%
  \IfValueTF{#1}
    {\namedprop[#1]\refstepcounter{prop}\def\@currentlabel{(#1)}}%
    {\namedprop}%
 }
 {%
  \endnamedprop
 }
\makeatother

\newtheorem*{namedlem}{Lemma}
\newcounter{lem}%makes pointer correct
\providecommand{\lemname}{Lemma}

\makeatletter
\NewDocumentEnvironment{lem*}{o}
 {%
  \IfValueTF{#1}
    {\namedlem[#1]\refstepcounter{lem}\def\@currentlabel{(#1)}}%
    {\namedlem}%
 }
 {%
  \endnamedlem
 }
\makeatother

\newtheorem*{namedcor}{Corollary}
\newcounter{cor}%makes pointer correct
\providecommand{\corname}{Corollary}

\makeatletter
\NewDocumentEnvironment{cor*}{o}
 {%
  \IfValueTF{#1}
    {\namedcor[#1]\refstepcounter{cor}\def\@currentlabel{(#1)}}%
    {\namedcor}%
 }
 {%
  \endnamedcor
 }
\makeatother

\theoremstyle{definition}
\newtheorem*{annotation}{Annotation}
\newtheorem*{rubric}{Rubric}

\newtheorem*{innerrem}{Remark}
\newcounter{rem}%makes pointer correct
\providecommand{\remname}{Remark}

\makeatletter
\NewDocumentEnvironment{rem}{o}
 {%
  \IfValueTF{#1}
    {\innerrem[#1]\refstepcounter{rem}\def\@currentlabel{(#1)}}%
    {\innerrem}%
 }
 {%
  \endinnerrem
 }
\makeatother

\newtheorem*{innerdefn}{Definition}%%placeholder
\newcounter{defn}%makes pointer correct
\providecommand{\defnname}{Definition}

\makeatletter
\NewDocumentEnvironment{defn}{o}
 {%
  \IfValueTF{#1}
    {\innerdefn[#1]\refstepcounter{defn}\def\@currentlabel{(#1)}}%
    {\innerdefn}%
 }
 {%
  \endinnerdefn
 }
\makeatother

\newtheorem*{scratch}{Scratch Work}


\newtheorem*{namedconj}{Conjecture}
\newcounter{conj}%makes pointer correct
\providecommand{\conjname}{Conjecture}
\makeatletter
\NewDocumentEnvironment{conj}{o}
 {%
  \IfValueTF{#1}
    {\innerconj[#1]\refstepcounter{conj}\def\@currentlabel{(#1)}}%
    {\innerconj}%
 }
 {%
  \endinnerconj
 }
\makeatother

%\let\br\relax
%\let\endbr\relax

%\newcounter{br}%makes pointer correct
%\counterwithin{br}{section}
%
%\newenvironment{br}[1][2in]%
%{%Env start code
%\problemEnvironmentStart{#1}{In-class Problem}
%\refstepcounter{br}
%\stepcounter{problem}
%}
%{%Env end code
%\problemEnvironmentEnd
%}

\let\question\relax
\let\endquestion\relax

\newtheoremstyle{ExerciseStyle}{\topsep}{\topsep}%%% space between body and thm
		{}                      %%% Thm body font
		{}                              %%% Indent amount (empty = no indent)
		{\bfseries}            %%% Thm head font
		{}                              %%% Punctuation after thm head
		{3em}                           %%% Space after thm head
		{{#1}~\thmnumber{#2}\thmnote{ \bfseries(#3)}}%%% Thm head spec
\theoremstyle{ExerciseStyle}
\newtheorem{br}{In-class Problem}


\newcounter{ex}%makes pointer correct
\providecommand{\exname}{Homework Problem}
\newenvironment{ex}[1][2in]%
{%Env start code
\problemEnvironmentStart{#1}{Homework Problem}
\refstepcounter{ex}
}
{%Env end code
\problemEnvironmentEnd
}

\newcommand{\inlineAnswer}[2][2 cm]{
    \ifhandout{\pdfOnly{\rule{#1}{0.4pt}}}
    \else{\answer{#2}}
    \fi
}

\ifhandout
\newenvironment{shortAnswer}[1][
    \vfill]
        {% Begin then result
        #1
            \begin{freeResponse}
            }
    {% Environment Ending Code
    \end{freeResponse}
    }
\else
\newenvironment{shortAnswer}[1][]
        {\begin{freeResponse}
            }
    {% Environment Ending Code
    \end{freeResponse}
    }
\fi

\newenvironment{sketch}
 {\begin{proof}[Sketch of Proof]}
 {\end{proof}}


\newcommand{\gt}{>}
\newcommand{\lt}{<}
\newcommand{\N}{\mathbb N}
\newcommand{\Q}{\mathbb Q}
\newcommand{\Z}{\mathbb Z}
\newcommand{\C}{\mathbb C}
\newcommand{\R}{\mathbb R}
\renewcommand{\H}{\mathbb{H}}
\newcommand{\lcm}{\operatorname{lcm}}
\newcommand{\nequiv}{\not\equiv}
\newcommand{\ord}{\operatorname{ord}}
\newcommand{\ds}{\displaystyle}
\newcommand{\floor}[1]{\left\lfloor #1\right\rfloor}
\newcommand{\legendre}[2]{\left(\frac{#1}{#2}\right)}



%%%%%%%%%%%%




\date{January 22, 2024}

\begin{document}
\handoutAbstract
\maketitle
    \begin{center}%
            {\large \scshape MAT-255-- Number Theory \hfill Spring 2024 \hfill In Class Work January 22}%
        
        {\large
            Your Name: \hrulefill \quad Group Members:\hrulefill \quad \hrulefill
	    \par}%
    \end{center}%

\begin{br} 
    Use the division algorithm on $a=47, b=6$ and $a=281, b=13$.
    \begin{solution}
        When $a=47, b=6,$ we have $q=7,$ and $r=5$ since \[47=6(7)+5\quad and \quad 0\leq 5 <6.\]
        When $a=281, b=13,$ we have $q=21,$ and $r=8$ since \[47=13(21)+8\quad and \quad 0\leq 8 <13.\]
    \end{solution}
    \pdfOnly{\ifhandout{
        \vfill}
        \else
        \fi}
\end{br}

\begin{br} Let $a$ and $b$ be nonzero integers. Prove that there exists a unique $q,r\in\Z$ such that 
    \[a=bq+r, \quad 0\leq r <|b|.\]
    \begin{enumerate}
        \item Use the division algorithm to prove this statement as a corollary. That is, use the \emph{conclusion} of the division algorithm as part of the proof.  Use the following outline:
        \begin{enumerate}
            \item  Let $a$ and $b$ be nonzero integers. Since $|b|>0$, the division algorithm says that there exist unique $p,s\in\Z$ such that
            $\inlineAnswer{a=p|b|+s}$
            and
            $\inlineAnswer{0\leq s<|b|}.$
                
            \item There are two cases:
                \begin{enumerate}
                    \item When $\inlineAnswer{b>0},$
                    the conditions are already met, and $r=\inlineAnswer{s}$ and $q=\inlineAnswer{b}.$
                    
                    \item Otherwise,
                    $\inlineAnswer{b<0},$ 
                    $r=\inlineAnswer{s}$ and $q=\inlineAnswer{-b}.$
                \end{enumerate}
            
            \item Since both cases used that the $p,s$ are unique, then $q,r$ are also unique
        \end{enumerate}
        
        \item Use the \emph{proof} of the division algorithm as a template to prove this statement. That is, repeat the steps, adjusting as necessary, but do not use the conclusion.
            \begin{enumerate}
                \item In the proof of the division algorithm, we let $q=\floor{\frac{a}{b}}$. Here we have two cases:
                \begin{enumerate}
                    \item When $\inlineAnswer{b>0},$
                    $q=\inlineAnswer{\floor{\frac{a}{b}}}$ and $r=\inlineAnswer{a-bq}.$
                        
                    \begin{onlineOnly}
                        \begin{hint}
                            The \TeX code for the floor function is \verb|\lfloor ... \rfloor|
                        \end{hint}
                    \end{onlineOnly}
                    as in the proof of the division algorithm. 
  
                    \item When $\inlineAnswer{b<0},$
                    $q=\inlineAnswer{-\floor{\frac{a}{b}}}$ and $r=\inlineAnswer{a-bq}.$
                \end{enumerate}
                
                \item Summarizing these statements, rewrite $q,r$ in terms of $a$ and $b$, as in the original proof of the division algorithm.
            
        
            \item Now use your scratch work and follow the outline of the proof of the division algorithm to provide a new proof \emph{without referencing the division algorithm.}
        
            \pdfOnly{\ifhandout{
                \vfill
                \vfill}
                \else
                \fi}
        \end{enumerate}
    \end{enumerate}
\end{br}

\end{document}








