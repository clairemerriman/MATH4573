\documentclass[letterpaper, 11 pt, instructornotes]{ximera}
<<<<<<< Updated upstream
\usepackage{amssymb, latexsym, amsmath, amsthm, graphicx, amsthm,alltt,color, listings,multicol,xr-hyper,hyperref,aliascnt,enumitem}
=======
\usepackage{amssymb, latexsym, amsmath, amsthm, graphicx, amsthm,alltt,color, listings,multicol,hyperref}
\usepackage[capitalise,nameinlink]{cleveref}
>>>>>>> Stashed changes
\usepackage{xfrac}

\usepackage{parskip}
\usepackage[,margin=0.7in]{geometry}
\setlength{\textheight}{8.5in}

\usepackage{epstopdf}

\DeclareGraphicsExtensions{.eps}
\usepackage{tikz}


\usepackage{tkz-euclide}
%\usetkzobj{all}
\tikzstyle geometryDiagrams=[rounded corners=.5pt,ultra thick,color=black]
\colorlet{penColor}{black} % Color of a curve in a plot


\usepackage{subcaption}
\usepackage{float}
\usepackage{fancyhdr}
\usepackage{pdfpages}
\newcounter{includepdfpage}
\usepackage{makecell}


\usepackage{currfile}
\usepackage{xstring}




\graphicspath{  
{./otherDocuments/}
}

\author{Claire Merriman}
\newcommand{\classday}[1]{\def\classday{#1}}

%%%%%%%%%%%%%%%%%%%%%
% Counters and autoref for unnumbered environments
% Not needed??
%%%%%%%%%%%%%%%%%%%%%
<<<<<<< Updated upstream
\theoremstyle{plain}


\newtheorem*{namedthm}{Theorem}
\newcounter{thm}%makes pointer correct
\providecommand{\thmname}{Theorem}
=======

\crefname{problem}{problem}{problems}


% \theoremstyle{plain}


% \newtheorem*{namedthm}{Theorem}
% \newcounter{thm}%makes pointer correct
% \providecommand{\thmname}{Theorem}
>>>>>>> Stashed changes

\makeatletter
\NewDocumentEnvironment{thm*}{o}
 {%
  \IfValueTF{#1}
    {\namedthm[#1]\refstepcounter{thm}\def\@currentlabel{(#1)}}%
    {\namedthm}%
 }
 {%
  \endnamedthm
 }
\makeatother


\newtheorem*{namedprop}{Proposition}
\newcounter{prop}%makes pointer correct
\providecommand{\propname}{Proposition}

\makeatletter
\NewDocumentEnvironment{prop*}{o}
 {%
  \IfValueTF{#1}
    {\namedprop[#1]\refstepcounter{prop}\def\@currentlabel{(#1)}}%
    {\namedprop}%
 }
 {%
  \endnamedprop
 }
\makeatother

\newtheorem*{namedlem}{Lemma}
\newcounter{lem}%makes pointer correct
\providecommand{\lemname}{Lemma}

\makeatletter
\NewDocumentEnvironment{lem*}{o}
 {%
  \IfValueTF{#1}
    {\namedlem[#1]\refstepcounter{lem}\def\@currentlabel{(#1)}}%
    {\namedlem}%
 }
 {%
  \endnamedlem
 }
\makeatother

\newtheorem*{namedcor}{Corollary}
\newcounter{cor}%makes pointer correct
\providecommand{\corname}{Corollary}

\makeatletter
\NewDocumentEnvironment{cor*}{o}
 {%
  \IfValueTF{#1}
    {\namedcor[#1]\refstepcounter{cor}\def\@currentlabel{(#1)}}%
    {\namedcor}%
 }
 {%
  \endnamedcor
 }
\makeatother

\theoremstyle{definition}
\newtheorem*{annotation}{Annotation}
\newtheorem*{rubric}{Rubric}

\newtheorem*{innerrem}{Remark}
\newcounter{rem}%makes pointer correct
\providecommand{\remname}{Remark}

\makeatletter
\NewDocumentEnvironment{rem}{o}
 {%
  \IfValueTF{#1}
    {\innerrem[#1]\refstepcounter{rem}\def\@currentlabel{(#1)}}%
    {\innerrem}%
 }
 {%
  \endinnerrem
 }
\makeatother

\newtheorem*{innerdefn}{Definition}%%placeholder
\newcounter{defn}%makes pointer correct
\providecommand{\defnname}{Definition}

\makeatletter
\NewDocumentEnvironment{defn}{o}
 {%
  \IfValueTF{#1}
    {\innerdefn[#1]\refstepcounter{defn}\def\@currentlabel{(#1)}}%
    {\innerdefn}%
 }
 {%
  \endinnerdefn
 }
\makeatother

\newtheorem*{scratch}{Scratch Work}


\newtheorem*{namedconj}{Conjecture}
\newcounter{conj}%makes pointer correct
\providecommand{\conjname}{Conjecture}
\makeatletter
\NewDocumentEnvironment{conj}{o}
 {%
  \IfValueTF{#1}
    {\innerconj[#1]\refstepcounter{conj}\def\@currentlabel{(#1)}}%
    {\innerconj}%
 }
 {%
  \endinnerconj
 }
\makeatother

\newtheorem*{poll}{Poll question}
\newtheorem{tps}{Think-Pair-Share}[section]


\newenvironment{obj}{
	\textbf{Learning Objectives.} By the end of class, students will be able to:
		\begin{itemize}}
		{\!.\end{itemize}
		}

<<<<<<< Updated upstream
\newenvironment{pre}{
	\begin{description}
	}{
	\end{description}
}
=======

\ifinstructornotes
\newenvironment{pre}
  {{\textbf Reading assignment:}
  \begin{description}
    }{
	\end{description}
  }
\else
\newenvironment{pre}{ 
  \begin{trivlist}
  \item[]}
  {\end{trivlist}}
\fi
>>>>>>> Stashed changes


\newcounter{ex}%makes pointer correct
\providecommand{\exname}{Homework Problem}
\newenvironment{ex}[1][2in]%
{%Env start code
\problemEnvironmentStart{#1}{Homework Problem}
\refstepcounter{ex}
}
{%Env end code
\problemEnvironmentEnd
}

\newcommand{\inlineAnswer}[2][2 cm]{
    \ifhandout{\pdfOnly{\rule{#1}{0.4pt}}}
    \else{\answer{#2}}
    \fi
}


\ifhandout
\newenvironment{shortAnswer}[1][
    \vfill]
        {% Begin then result
        #1
            \begin{freeResponse}
            }
    {% Environment Ending Code
    \end{freeResponse}
    }
\else
\newenvironment{shortAnswer}[1][]
        {\begin{freeResponse}
            }
    {% Environment Ending Code
    \end{freeResponse}
    }
\fi

\let\question\relax
\let\endquestion\relax

\newtheoremstyle{ExerciseStyle}{\topsep}{\topsep}%%% space between body and thm
		{}                      %%% Thm body font
		{}                              %%% Indent amount (empty = no indent)
		{\bfseries}            %%% Thm head font
		{}                              %%% Punctuation after thm head
		{3em}                           %%% Space after thm head
		{{#1}~\thmnumber{#2}\thmnote{ \bfseries(#3)}}%%% Thm head spec
\theoremstyle{ExerciseStyle}
\newtheorem{br}{In-class Problem}

\newenvironment{sketch}
 {\begin{proof}[Sketch of Proof]}
 {\end{proof}}


\newcommand{\gt}{>}
\newcommand{\lt}{<}
\newcommand{\N}{\mathbb N}
\newcommand{\Q}{\mathbb Q}
\newcommand{\Z}{\mathbb Z}
\newcommand{\C}{\mathbb C}
\newcommand{\R}{\mathbb R}
\renewcommand{\H}{\mathbb{H}}
\newcommand{\lcm}{\operatorname{lcm}}
\newcommand{\nequiv}{\not\equiv}
\newcommand{\ord}{\operatorname{ord}}
\newcommand{\ds}{\displaystyle}
\newcommand{\floor}[1]{\left\lfloor #1\right\rfloor}
\newcommand{\legendre}[2]{\left(\frac{#1}{#2}\right)}



%%%%%%%%%%%%



\title{Geometry Pythagorean Triples}


\begin{document}
\begin{abstract}
 Project on geometry and Pythagorean Triples.
\end{abstract}
\maketitle


\begin{exploration} 
\begin{instructorNotes}
	Based on the proof in \cite{jones-jones} Section 11.5 The classification of Pythagorean triples.
\end{instructorNotes}
 \begin{definition}
 	A \emph{rational point} is a point $(x,y)$ whose coordinates $x$ and $y$ are both rational numbers.
\end{definition}

Use rational points on the unit circle and trigonometry to derive the formula for generating Pythagorean triples. 

\begin{problem}\label{prob:rational-form}
 	Let $(x,y)$ be a rational point on the unit circle. That is, there exists $a,b,c\in\Z$ such that $x=\frac{a}{c}$ and $y=\frac{b}{c}$.
	
	Explain why we can write $x$ and $y$ with the same denominator.
\begin{rubric}
 2 points.
\end{rubric}
\end{problem}

\begin{problem}
Let $(x,y)$ be any point on the unit circle, ie $x^2+y^2=1.$
	Consider the line segment between $(0,0)$ and $(x,y).$ As long as $(x,y)$ is not $(-1,0),$ the slope of this line is \[t=\frac{y}{1+x}=\tan\theta\] where $\theta$ is the angle between the line segment and the $x$-axis. 
	\begin{enumerate}
 		\item Show that \[x=\frac{1-t^2}{1+t^2}, y=\frac{2t}{1+t^2}\] either using algebra or trig identities.
		\item Show that $(x,y)$ is a rational point not equal to $(-1,0)$\footnote{$x=-1$ and $y=0$ is the limit as $t\to\infty$} if and only if $t$ is rational.
		\item Let $t=\frac{m}{n}.$ Prove that if $x,y>0,$ then $m>n, (m,n)=1,$ exactly one of $m,n$ is even, and all nontrivial Pythagorean triples have the form $a=k(m^2-n^2), b=k(2mn), c=k(m^+n^2),$ for some $k\in\Z.$ 
\begin{hint}
Use that $k=\frac{c}{m^2+n^2}$.
\end{hint}
 
	\end{enumerate}
\begin{rubric}
 4 points if individual, 3 if presenting as a pair.
\end{rubric}
\end{problem}

\end{exploration}

\begin{exploration}
The following problems are from \emph{Number Theory: A Lively Introduction with Proofs, Applications, and Stories} by Erica Flapan, Tim Marks, and James Pommersheim Chapter 15, Some Nonlinear Diophantine Equations,15.5 A Geometric look at the Equation $x^4 + y^4 = z^2$ \cite{theo}.

\begin{theorem}[Fermat's Right Triangle Theorem]
The does not exist a right triangle with rational side lengths and area a perfect square.
\end{theorem}

\begin{problem}
 	Let $x,y,z$ be positive integers be the side lengths of a triangle with hypotenuse $z$ and the area of the triangle is a perfect square.  Prove that if $z$ is the smallest such integer, then $(x,y,z)$ is a primitive Pythagorean triple.
	
\begin{rubric}
 4 points if individual, 3 if presenting as a pair.
\end{rubric}
\end{problem}

\begin{problem}
	Let $x,y,z$ as in the previous problem. Then there exists $m, n\in\Z$ with $m>n>0, (m,n)=1$ and exactly one of $m$ and $n$ even such that $x=m^2-n^2,\ y=2mn,\ $ and $z=m^2+n^2$. Furthermore, $m,\ n,\ m+n$ and $m-n$ are perfect squares. 
	
	\begin{enumerate}
		\item\label{setup}  Prove that if $c,d\in\Z$ such that $c^2=m+n$, and $d^2=m-n$, then exactly one of $c+d$ and $c-d$ is divisible by $4$.
		
		\item\label{case1-part1} If $4\mid c+d,$ prove that $\frac{b^2}{4}=\frac{c+d}{4}\frac{c-d}{2},$ where the two factors on the right side of the equation are relatively prime. 
		
		\item Show that there exists $s,t\in\Z$ such that $\frac{c+d}{4}=s^2$ and $\frac{c-d}{2}=t^2$ 
		
		\item\label{case1-part3} Show that $(2s^2,t^2,a)$ is a Pythagorean triple and finish the proof of the right triangle theorem for this case. 
		
		\item Repeat parts \ref{case1-part1}-\ref{case1-part3} for $4\mid c-d.$ 
	\end{enumerate}
\begin{rubric}
 Part \ref{setup} 4 points if individual, 3 if presenting as a pair.  Parts \ref{case1-part1}-\ref{case1-part3} 3 points per case if individual, 2 points per case if presenting as a pair.
 
 Problem total: 10 points if individual, 7 points if pair.
\end{rubric}
\end{problem}
\end{exploration}

\begin{exploration}(If presenting as a pair)
 \begin{definition}
 A positive integer $n$ is a \emph{congruent number} if there exists a right triangle whose sides are all rational numbers and whose area is $n$. 
\end{definition}

\begin{problem}
Show that the following are congruent numbers:
\begin{itemize}
 \item $6$.
 \item $5$.
\end{itemize} 
\begin{rubric}
 2 points.
\end{rubric}
\end{problem}

\begin{problem}
Prove that there is no right triangle with \emph{integer} sides whose area is $5$.
 There is no right triangle with \emph{integer} sides whose area is $1$.
 \begin{rubric}
 3 points.
\end{rubric}
\end{problem}

\end{exploration}
\end{document}