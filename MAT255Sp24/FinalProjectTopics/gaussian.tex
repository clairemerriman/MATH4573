\documentclass[letterpaper, 11 pt,handout]{ximera}
\usepackage{amssymb, latexsym, amsmath, amsthm, graphicx, amsthm,alltt,color, listings,multicol,xr-hyper,hyperref,aliascnt,enumitem}
\usepackage{xfrac}

\usepackage{parskip}
\usepackage[,margin=0.7in]{geometry}
\setlength{\textheight}{8.5in}

\usepackage{epstopdf}

\DeclareGraphicsExtensions{.eps}
\usepackage{tikz}


\usepackage{tkz-euclide}
%\usetkzobj{all}
\tikzstyle geometryDiagrams=[rounded corners=.5pt,ultra thick,color=black]
\colorlet{penColor}{black} % Color of a curve in a plot


\usepackage{subcaption}
\usepackage{float}
\usepackage{fancyhdr}
\usepackage{pdfpages}
\newcounter{includepdfpage}
\usepackage{makecell}


\usepackage{currfile}
\usepackage{xstring}




\graphicspath{  
{./otherDocuments/}
}

\author{Claire Merriman}
\newcommand{\classday}[1]{\def\classday{#1}}

%%%%%%%%%%%%%%%%%%%%%
% Counters and autoref for unnumbered environments
% Not needed??
%%%%%%%%%%%%%%%%%%%%%
\theoremstyle{plain}


\newtheorem*{namedthm}{Theorem}
\newcounter{thm}%makes pointer correct
\providecommand{\thmname}{Theorem}

\makeatletter
\NewDocumentEnvironment{thm*}{o}
 {%
  \IfValueTF{#1}
    {\namedthm[#1]\refstepcounter{thm}\def\@currentlabel{(#1)}}%
    {\namedthm}%
 }
 {%
  \endnamedthm
 }
\makeatother


\newtheorem*{namedprop}{Proposition}
\newcounter{prop}%makes pointer correct
\providecommand{\propname}{Proposition}

\makeatletter
\NewDocumentEnvironment{prop*}{o}
 {%
  \IfValueTF{#1}
    {\namedprop[#1]\refstepcounter{prop}\def\@currentlabel{(#1)}}%
    {\namedprop}%
 }
 {%
  \endnamedprop
 }
\makeatother

\newtheorem*{namedlem}{Lemma}
\newcounter{lem}%makes pointer correct
\providecommand{\lemname}{Lemma}

\makeatletter
\NewDocumentEnvironment{lem*}{o}
 {%
  \IfValueTF{#1}
    {\namedlem[#1]\refstepcounter{lem}\def\@currentlabel{(#1)}}%
    {\namedlem}%
 }
 {%
  \endnamedlem
 }
\makeatother

\newtheorem*{namedcor}{Corollary}
\newcounter{cor}%makes pointer correct
\providecommand{\corname}{Corollary}

\makeatletter
\NewDocumentEnvironment{cor*}{o}
 {%
  \IfValueTF{#1}
    {\namedcor[#1]\refstepcounter{cor}\def\@currentlabel{(#1)}}%
    {\namedcor}%
 }
 {%
  \endnamedcor
 }
\makeatother

\theoremstyle{definition}
\newtheorem*{annotation}{Annotation}
\newtheorem*{rubric}{Rubric}

\newtheorem*{innerrem}{Remark}
\newcounter{rem}%makes pointer correct
\providecommand{\remname}{Remark}

\makeatletter
\NewDocumentEnvironment{rem}{o}
 {%
  \IfValueTF{#1}
    {\innerrem[#1]\refstepcounter{rem}\def\@currentlabel{(#1)}}%
    {\innerrem}%
 }
 {%
  \endinnerrem
 }
\makeatother

\newtheorem*{innerdefn}{Definition}%%placeholder
\newcounter{defn}%makes pointer correct
\providecommand{\defnname}{Definition}

\makeatletter
\NewDocumentEnvironment{defn}{o}
 {%
  \IfValueTF{#1}
    {\innerdefn[#1]\refstepcounter{defn}\def\@currentlabel{(#1)}}%
    {\innerdefn}%
 }
 {%
  \endinnerdefn
 }
\makeatother

\newtheorem*{scratch}{Scratch Work}


\newtheorem*{namedconj}{Conjecture}
\newcounter{conj}%makes pointer correct
\providecommand{\conjname}{Conjecture}
\makeatletter
\NewDocumentEnvironment{conj}{o}
 {%
  \IfValueTF{#1}
    {\innerconj[#1]\refstepcounter{conj}\def\@currentlabel{(#1)}}%
    {\innerconj}%
 }
 {%
  \endinnerconj
 }
\makeatother

\newtheorem*{poll}{Poll question}
\newtheorem{tps}{Think-Pair-Share}[section]


\newenvironment{obj}{
	\textbf{Learning Objectives.} By the end of class, students will be able to:
		\begin{itemize}}
		{\!.\end{itemize}
		}

\newenvironment{pre}{
	\begin{description}
	}{
	\end{description}
}


\newcounter{ex}%makes pointer correct
\providecommand{\exname}{Homework Problem}
\newenvironment{ex}[1][2in]%
{%Env start code
\problemEnvironmentStart{#1}{Homework Problem}
\refstepcounter{ex}
}
{%Env end code
\problemEnvironmentEnd
}

\newcommand{\inlineAnswer}[2][2 cm]{
    \ifhandout{\pdfOnly{\rule{#1}{0.4pt}}}
    \else{\answer{#2}}
    \fi
}


\ifhandout
\newenvironment{shortAnswer}[1][
    \vfill]
        {% Begin then result
        #1
            \begin{freeResponse}
            }
    {% Environment Ending Code
    \end{freeResponse}
    }
\else
\newenvironment{shortAnswer}[1][]
        {\begin{freeResponse}
            }
    {% Environment Ending Code
    \end{freeResponse}
    }
\fi

\let\question\relax
\let\endquestion\relax

\newtheoremstyle{ExerciseStyle}{\topsep}{\topsep}%%% space between body and thm
		{}                      %%% Thm body font
		{}                              %%% Indent amount (empty = no indent)
		{\bfseries}            %%% Thm head font
		{}                              %%% Punctuation after thm head
		{3em}                           %%% Space after thm head
		{{#1}~\thmnumber{#2}\thmnote{ \bfseries(#3)}}%%% Thm head spec
\theoremstyle{ExerciseStyle}
\newtheorem{br}{In-class Problem}

\newenvironment{sketch}
 {\begin{proof}[Sketch of Proof]}
 {\end{proof}}


\newcommand{\gt}{>}
\newcommand{\lt}{<}
\newcommand{\N}{\mathbb N}
\newcommand{\Q}{\mathbb Q}
\newcommand{\Z}{\mathbb Z}
\newcommand{\C}{\mathbb C}
\newcommand{\R}{\mathbb R}
\renewcommand{\H}{\mathbb{H}}
\newcommand{\lcm}{\operatorname{lcm}}
\newcommand{\nequiv}{\not\equiv}
\newcommand{\ord}{\operatorname{ord}}
\newcommand{\ds}{\displaystyle}
\newcommand{\floor}[1]{\left\lfloor #1\right\rfloor}
\newcommand{\legendre}[2]{\left(\frac{#1}{#2}\right)}



%%%%%%%%%%%%



\title{Gaussian Integers}


\begin{document}
\begin{abstract}
 Project on Gaussian Integers.
\end{abstract}
\maketitle

The following problems are from \emph{Number Theory: A Lively Introduction with Proofs, Applications, and Stories} by Erica Flapan, Tim Marks, and James Pommersheim.

	Read the scanned notes on Moodle. Present as much as necessarily for classmates to follow.
	
\begin{rubric}
 	Present as much as necessarily for classmates to follow: 4 points if individual, 3 points if pair.
\end{rubric}

\begin{exploration}
Here I will use slightly more standard notation, which will match iwth the $\Z[\sqrt{d}]$ topic and complex analysis.
 	\begin{definition}
 		The set of \emph{Gaussian integers}, denoted $\Z[i]$ (read ``$\Z$ adjoin $i$") is defined by 
 \[\Z[i]=\{a+bi : a,b\in\Z\},\]
 where $i^2=-1$.
 
 	Let $z=a+bi$ be a Gaussian integer. The \emph{complex  conjugate of $z$} is $\overline{z}=a-bi$. Define the \emph{norm of $z$}, as
 \[N(z)=|z|=z\overline{z}=(a+bi)(a-bi)=a^2+b^2.\]
	\end{definition}

\begin{lem*}[Lemma 13.1.4]\label{norm-mult}
 	Let $r$ and $s$ be Gaussian integers. Then 
 		\[N(rs)=N(r)N(s).\]
 \end{lem*}

\begin{problem}
Prove \nameref{norm-mult}
	\begin{rubric}
 		4 points if individual, 3 points if pair.
	\end{rubric}
\end{problem}

\begin{problem}
Let $d$ and $z$ be Gaussian integers. 
\begin{enumerate}
	\item Prove that if $d\mid z$, then $|d|\mid |z|$.
		\item Prove or provide a counterexample for if $|d|\mid |z|,$ then $d\mid z.$ 
\end{enumerate}
	\begin{rubric}
 		4 points if individual, 3 points if pair.
	\end{rubric}
\end{problem}

\begin{definition}
 Let $p\in\Z[i]$ such that $p$ is not a unit. We say $p$ is \emph{prime} if for every $a,b\in\Z[i],$ $p=ab$ implies that $a$ is a unit or $b$ is a unit.
\end{definition}

\begin{lem*}[Lemma 13.2.6]\label{prime_norm}
 Let $z$ be a Gaussian integer. If $N(z)$ is a prime in the (regular) integers, then $z$ is a prime as a Gaussian integer.
 %
\end{lem*}

\begin{problem}
Prove \nameref{prime_norm}.
	\begin{rubric}
 		4 points if individual, 3 points if pair.
	\end{rubric}
\end{problem}



\begin{definition}
 Let $a,d\in\Z[i]$. We say $a$ and $b$ are \emph{relatively prime} if for all $d\in\Z[i],$ $d\mid a$ and $d\mid b$ implies that $d$ is a unit.
\end{definition}

\begin{theorem}[Division Theorem for Gaussian Integers]\label{div-gauss-int}
 Let $a,d\in\Z[i]$ with $b\neq 0$. Then there exist Gaussian integers $q$ and $r$ such that $a=qb+r$ with $N(r)<N(b)$
 %Exercises 15-17
\end{theorem}

\begin{problem}(If presenting as a pair)
Prove \nameref{div-gauss-int} using Exercises 15-17 in the scanned notes.
	\begin{rubric}
 		5 points.
	\end{rubric}
\end{problem}

\begin{theorem}[Theorem 13.5.4]
 Suppose $p\in\Z$ is prime. Then $p$ is a prime in $\Z[i]$ if and only if $p\equiv 3\pmod{4}$.
 %
\end{theorem}



\begin{theorem}[Theorem 13.5.5]\label{thm:guass-primes}
 Let $z\in\Z[i]$. Then $z$ is a prime Gaussian integer if and only if one of the following conditions holds:
 
\begin{itemize}
 \item $N(z)=2$,
 \item $N(z)$ is a prime integer congruent to $1$ modulo $4$,
 \item $z$ is a unit times a prime integer congruent to $3$ modulo $4$.
\end{itemize}
%
\end{theorem}

\begin{problem}
Prove \nameref{thm:guass-primes}
	\begin{rubric}
 		4 points if individual, 3 points if pair.
	\end{rubric}
\end{problem}
\end{exploration}

\end{document}