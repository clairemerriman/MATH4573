\documentclass[letterpaper, 11 pt]{ximera}
\usepackage{amssymb, latexsym, amsmath, amsthm, graphicx, amsthm,alltt,color, listings,multicol,xr-hyper,hyperref,aliascnt,enumitem}
\usepackage{xfrac}

\usepackage{parskip}
\usepackage[,margin=0.7in]{geometry}
\setlength{\textheight}{8.5in}

\usepackage{epstopdf}

\DeclareGraphicsExtensions{.eps}
\usepackage{tikz}


\usepackage{tkz-euclide}
%\usetkzobj{all}
\tikzstyle geometryDiagrams=[rounded corners=.5pt,ultra thick,color=black]
\colorlet{penColor}{black} % Color of a curve in a plot


\usepackage{subcaption}
\usepackage{float}
\usepackage{fancyhdr}
\usepackage{pdfpages}
\newcounter{includepdfpage}
\usepackage{makecell}


\usepackage{currfile}
\usepackage{xstring}




\graphicspath{  
{./otherDocuments/}
}

\author{Claire Merriman}
\newcommand{\classday}[1]{\def\classday{#1}}

%%%%%%%%%%%%%%%%%%%%%
% Counters and autoref for unnumbered environments
% Not needed??
%%%%%%%%%%%%%%%%%%%%%
\theoremstyle{plain}


\newtheorem*{namedthm}{Theorem}
\newcounter{thm}%makes pointer correct
\providecommand{\thmname}{Theorem}

\makeatletter
\NewDocumentEnvironment{thm*}{o}
 {%
  \IfValueTF{#1}
    {\namedthm[#1]\refstepcounter{thm}\def\@currentlabel{(#1)}}%
    {\namedthm}%
 }
 {%
  \endnamedthm
 }
\makeatother


\newtheorem*{namedprop}{Proposition}
\newcounter{prop}%makes pointer correct
\providecommand{\propname}{Proposition}

\makeatletter
\NewDocumentEnvironment{prop*}{o}
 {%
  \IfValueTF{#1}
    {\namedprop[#1]\refstepcounter{prop}\def\@currentlabel{(#1)}}%
    {\namedprop}%
 }
 {%
  \endnamedprop
 }
\makeatother

\newtheorem*{namedlem}{Lemma}
\newcounter{lem}%makes pointer correct
\providecommand{\lemname}{Lemma}

\makeatletter
\NewDocumentEnvironment{lem*}{o}
 {%
  \IfValueTF{#1}
    {\namedlem[#1]\refstepcounter{lem}\def\@currentlabel{(#1)}}%
    {\namedlem}%
 }
 {%
  \endnamedlem
 }
\makeatother

\newtheorem*{namedcor}{Corollary}
\newcounter{cor}%makes pointer correct
\providecommand{\corname}{Corollary}

\makeatletter
\NewDocumentEnvironment{cor*}{o}
 {%
  \IfValueTF{#1}
    {\namedcor[#1]\refstepcounter{cor}\def\@currentlabel{(#1)}}%
    {\namedcor}%
 }
 {%
  \endnamedcor
 }
\makeatother

\theoremstyle{definition}
\newtheorem*{annotation}{Annotation}
\newtheorem*{rubric}{Rubric}

\newtheorem*{innerrem}{Remark}
\newcounter{rem}%makes pointer correct
\providecommand{\remname}{Remark}

\makeatletter
\NewDocumentEnvironment{rem}{o}
 {%
  \IfValueTF{#1}
    {\innerrem[#1]\refstepcounter{rem}\def\@currentlabel{(#1)}}%
    {\innerrem}%
 }
 {%
  \endinnerrem
 }
\makeatother

\newtheorem*{innerdefn}{Definition}%%placeholder
\newcounter{defn}%makes pointer correct
\providecommand{\defnname}{Definition}

\makeatletter
\NewDocumentEnvironment{defn}{o}
 {%
  \IfValueTF{#1}
    {\innerdefn[#1]\refstepcounter{defn}\def\@currentlabel{(#1)}}%
    {\innerdefn}%
 }
 {%
  \endinnerdefn
 }
\makeatother

\newtheorem*{scratch}{Scratch Work}


\newtheorem*{namedconj}{Conjecture}
\newcounter{conj}%makes pointer correct
\providecommand{\conjname}{Conjecture}
\makeatletter
\NewDocumentEnvironment{conj}{o}
 {%
  \IfValueTF{#1}
    {\innerconj[#1]\refstepcounter{conj}\def\@currentlabel{(#1)}}%
    {\innerconj}%
 }
 {%
  \endinnerconj
 }
\makeatother

\newtheorem*{poll}{Poll question}
\newtheorem{tps}{Think-Pair-Share}[section]


\newenvironment{obj}{
	\textbf{Learning Objectives.} By the end of class, students will be able to:
		\begin{itemize}}
		{\!.\end{itemize}
		}

\newenvironment{pre}{
	\begin{description}
	}{
	\end{description}
}


\newcounter{ex}%makes pointer correct
\providecommand{\exname}{Homework Problem}
\newenvironment{ex}[1][2in]%
{%Env start code
\problemEnvironmentStart{#1}{Homework Problem}
\refstepcounter{ex}
}
{%Env end code
\problemEnvironmentEnd
}

\newcommand{\inlineAnswer}[2][2 cm]{
    \ifhandout{\pdfOnly{\rule{#1}{0.4pt}}}
    \else{\answer{#2}}
    \fi
}


\ifhandout
\newenvironment{shortAnswer}[1][
    \vfill]
        {% Begin then result
        #1
            \begin{freeResponse}
            }
    {% Environment Ending Code
    \end{freeResponse}
    }
\else
\newenvironment{shortAnswer}[1][]
        {\begin{freeResponse}
            }
    {% Environment Ending Code
    \end{freeResponse}
    }
\fi

\let\question\relax
\let\endquestion\relax

\newtheoremstyle{ExerciseStyle}{\topsep}{\topsep}%%% space between body and thm
		{}                      %%% Thm body font
		{}                              %%% Indent amount (empty = no indent)
		{\bfseries}            %%% Thm head font
		{}                              %%% Punctuation after thm head
		{3em}                           %%% Space after thm head
		{{#1}~\thmnumber{#2}\thmnote{ \bfseries(#3)}}%%% Thm head spec
\theoremstyle{ExerciseStyle}
\newtheorem{br}{In-class Problem}

\newenvironment{sketch}
 {\begin{proof}[Sketch of Proof]}
 {\end{proof}}


\newcommand{\gt}{>}
\newcommand{\lt}{<}
\newcommand{\N}{\mathbb N}
\newcommand{\Q}{\mathbb Q}
\newcommand{\Z}{\mathbb Z}
\newcommand{\C}{\mathbb C}
\newcommand{\R}{\mathbb R}
\renewcommand{\H}{\mathbb{H}}
\newcommand{\lcm}{\operatorname{lcm}}
\newcommand{\nequiv}{\not\equiv}
\newcommand{\ord}{\operatorname{ord}}
\newcommand{\ds}{\displaystyle}
\newcommand{\floor}[1]{\left\lfloor #1\right\rfloor}
\newcommand{\legendre}[2]{\left(\frac{#1}{#2}\right)}



%%%%%%%%%%%%



\title{Farey Fractions}


\begin{document}
\begin{abstract}
 Project on Farey Fractions.
\end{abstract}
\maketitle

The following problems are based on \emph{The Theory of Number} by Ivan Niven, Herbert S. Zuckerman, and Hugh L. Montgomery \cite{niven-zuckerman-montgomery}.

The subject of \emph{Diophantine Approximation} concerns approximating real numbers with rational numbers.  One way to do this is with \emph{Farey fractions}--which we can generate by adding ``wrong." However, there is a much simpler way to generate Farey fractions. One goal is to show these two methods give the same sequences.

\begin{exploration}
 \begin{definition}
 The \emph{$N^{th}$ Farey sequence} is the list of all fractions, written from smallest to largest, between $0$ and $1$ where the denominator is less than or equal to $N$ when written as a reduced fraction. We write this sequence as $\mathcal{F}_N$.
\end{definition}

The first three Farey sequences are:
\begin{align*}
    \mathcal{F}_1 &= \left\{\frac 01, \frac 11 \right\}, \\
    \mathcal{F}_2 &= \left\{ \frac 01, \frac 12, \frac 11 \right\}, \\
    \mathcal{F}_3 &= \left\{ \frac 01, \frac 13, \frac 12, \frac 23, \frac 11 \right\}
    \end{align*}


\begin{problem}
	Prove that for a positive integer $N\geq 2$, the number of elements of $\mathcal{F}_N$ with denominator $N$ is equal to $\phi(N)$.
\begin{rubric}
5 points if individual, 4 points if pair.
\end{rubric}
\end{problem}
\end{exploration}


\begin{exploration}
 Another way to generate the Farey sequence is using a table. In the first row, write $\frac 01$ and $\frac 11$.

To form the second row, copy the first row. Then insert $\frac{0+1}{1+1}$ between $\frac 01$ and $\frac 11$.

To form the $n^{th}$ row, copy the $(n-1)^{st}$ row. Then for each $\frac ab, \frac cd$ in the $(n-1)$ row, if $b+d\leq n$, insert $\frac{a+c}{b+d}$ between $\frac ab$ and $\frac cd$.

\begin{table}
 \begin{tabular}{ c c c c c c c c c c c c c }
 $\frac{0}{1}$ &&&&&&&&&&&& $\frac{1}{1}$\\
 $\frac{0}{1}$ & & & & &$\frac{1}{2}$&&&&&&& $\frac{1}{1}$\\
 $\frac{0}{1}$ & & & $\frac{1}{3}$& &$\frac{1}{2}$& & $\frac{2}{3}$&&&&& $\frac{1}{1}$\\
  $\frac 01$ &  & & $ \frac 14$ & $ \frac 13$ & & $ \frac 12$ & $ \frac 35$ & $\frac 23$ & $ \frac 34$ &  &  & $ \frac 11 $\\
  $\frac 01$ &  & $ \frac 15$ & $ \frac 14$ & $ \frac 13$ & $ \frac 25$ & $ \frac 12$ & $ \frac 35$ & $\frac 23$ & $ \frac 34$ & $ \frac 45$ &  & $ \frac 11 $\\
 $\frac 01$ & $ \frac 16$ & $ \frac 15$ & $ \frac 14$ & $ \frac 13$ & $ \frac 25$ & $ \frac 12$ & $ \frac 35$ & $\frac 23$ & $ \frac 34$ & $ \frac 45$ & $\frac 56$ & $ \frac 11 $
\end{tabular}
\end{table}

\begin{lemma}[Niven-Zukerman-Montgomery, Theorem 6.1]
 If $\frac{a}{b}$ and $\frac{c}{d}$ are consecutive fractions in the $n^{th}$ row of the table, with $\frac ab$ to the left of $\frac cd$, then $bc-ad=1$.
\end{lemma}
This lemma matches with the definition of a \emph{Farey pair} from Strayer Chapter 7, Student Project 2.

\begin{problem}
	Prove this lemma.
	
	
\begin{rubric}
 5 points if individual, 4 points if pair.
\end{rubric}
\end{problem}

\begin{problem}(If presenting as a pair)
	Strayer Chapter 7, Student Project 2.
\begin{rubric}
 Parts (a)-(c): 4 points.
 Part (d): 4 points.
\end{rubric}
\end{problem}

\begin{problem}(If presenting as an individual)
  Prove \begin{corollary}[Niven-Zukerman-Montgomery, Corollary 6.2]
 Every $\frac{a}{b}$ in the table is in reduced form.
\end{corollary}

\begin{corollary}[Niven-Zukerman-Montgomery, Corollary 6.3]
 The fractions in each row are listed in order from smallest to largest.
\end{corollary}

\begin{rubric}
 5 points.
\end{rubric}
\end{problem}



\begin{problem}
	Prove that if $0\leq m \leq n$ and $(m,n)=1$, then fraction $\frac{m}{n}$ is in the $n^{th}$ row of the table. Therefore, algorithmic definition of the $n^{th}$ Farey sequence equivalent to the definition: The \emph{$N^{th}$ Farey sequence} is the list of all fractions, written from smallest to largest, between $0$ and $1$ where the denominator is less than or equal to $N$ when written as a reduced fraction.
	
\begin{rubric}
 5 points if individual, 4 points if pair.
\end{rubric}
\end{problem}
\end{exploration}
\end{document}