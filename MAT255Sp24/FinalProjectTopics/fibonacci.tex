\documentclass[letterpaper, 11 pt]{ximera}
\usepackage{amssymb, latexsym, amsmath, amsthm, graphicx, amsthm,alltt,color, listings,multicol,xr-hyper,hyperref,aliascnt,enumitem}
\usepackage{xfrac}

\usepackage{parskip}
\usepackage[,margin=0.7in]{geometry}
\setlength{\textheight}{8.5in}

\usepackage{epstopdf}

\DeclareGraphicsExtensions{.eps}
\usepackage{tikz}


\usepackage{tkz-euclide}
%\usetkzobj{all}
\tikzstyle geometryDiagrams=[rounded corners=.5pt,ultra thick,color=black]
\colorlet{penColor}{black} % Color of a curve in a plot


\usepackage{subcaption}
\usepackage{float}
\usepackage{fancyhdr}
\usepackage{pdfpages}
\newcounter{includepdfpage}
\usepackage{makecell}


\usepackage{currfile}
\usepackage{xstring}




\graphicspath{  
{./otherDocuments/}
}

\author{Claire Merriman}
\newcommand{\classday}[1]{\def\classday{#1}}

%%%%%%%%%%%%%%%%%%%%%
% Counters and autoref for unnumbered environments
% Not needed??
%%%%%%%%%%%%%%%%%%%%%
\theoremstyle{plain}


\newtheorem*{namedthm}{Theorem}
\newcounter{thm}%makes pointer correct
\providecommand{\thmname}{Theorem}

\makeatletter
\NewDocumentEnvironment{thm*}{o}
 {%
  \IfValueTF{#1}
    {\namedthm[#1]\refstepcounter{thm}\def\@currentlabel{(#1)}}%
    {\namedthm}%
 }
 {%
  \endnamedthm
 }
\makeatother


\newtheorem*{namedprop}{Proposition}
\newcounter{prop}%makes pointer correct
\providecommand{\propname}{Proposition}

\makeatletter
\NewDocumentEnvironment{prop*}{o}
 {%
  \IfValueTF{#1}
    {\namedprop[#1]\refstepcounter{prop}\def\@currentlabel{(#1)}}%
    {\namedprop}%
 }
 {%
  \endnamedprop
 }
\makeatother

\newtheorem*{namedlem}{Lemma}
\newcounter{lem}%makes pointer correct
\providecommand{\lemname}{Lemma}

\makeatletter
\NewDocumentEnvironment{lem*}{o}
 {%
  \IfValueTF{#1}
    {\namedlem[#1]\refstepcounter{lem}\def\@currentlabel{(#1)}}%
    {\namedlem}%
 }
 {%
  \endnamedlem
 }
\makeatother

\newtheorem*{namedcor}{Corollary}
\newcounter{cor}%makes pointer correct
\providecommand{\corname}{Corollary}

\makeatletter
\NewDocumentEnvironment{cor*}{o}
 {%
  \IfValueTF{#1}
    {\namedcor[#1]\refstepcounter{cor}\def\@currentlabel{(#1)}}%
    {\namedcor}%
 }
 {%
  \endnamedcor
 }
\makeatother

\theoremstyle{definition}
\newtheorem*{annotation}{Annotation}
\newtheorem*{rubric}{Rubric}

\newtheorem*{innerrem}{Remark}
\newcounter{rem}%makes pointer correct
\providecommand{\remname}{Remark}

\makeatletter
\NewDocumentEnvironment{rem}{o}
 {%
  \IfValueTF{#1}
    {\innerrem[#1]\refstepcounter{rem}\def\@currentlabel{(#1)}}%
    {\innerrem}%
 }
 {%
  \endinnerrem
 }
\makeatother

\newtheorem*{innerdefn}{Definition}%%placeholder
\newcounter{defn}%makes pointer correct
\providecommand{\defnname}{Definition}

\makeatletter
\NewDocumentEnvironment{defn}{o}
 {%
  \IfValueTF{#1}
    {\innerdefn[#1]\refstepcounter{defn}\def\@currentlabel{(#1)}}%
    {\innerdefn}%
 }
 {%
  \endinnerdefn
 }
\makeatother

\newtheorem*{scratch}{Scratch Work}


\newtheorem*{namedconj}{Conjecture}
\newcounter{conj}%makes pointer correct
\providecommand{\conjname}{Conjecture}
\makeatletter
\NewDocumentEnvironment{conj}{o}
 {%
  \IfValueTF{#1}
    {\innerconj[#1]\refstepcounter{conj}\def\@currentlabel{(#1)}}%
    {\innerconj}%
 }
 {%
  \endinnerconj
 }
\makeatother

\newtheorem*{poll}{Poll question}
\newtheorem{tps}{Think-Pair-Share}[section]


\newenvironment{obj}{
	\textbf{Learning Objectives.} By the end of class, students will be able to:
		\begin{itemize}}
		{\!.\end{itemize}
		}

\newenvironment{pre}{
	\begin{description}
	}{
	\end{description}
}


\newcounter{ex}%makes pointer correct
\providecommand{\exname}{Homework Problem}
\newenvironment{ex}[1][2in]%
{%Env start code
\problemEnvironmentStart{#1}{Homework Problem}
\refstepcounter{ex}
}
{%Env end code
\problemEnvironmentEnd
}

\newcommand{\inlineAnswer}[2][2 cm]{
    \ifhandout{\pdfOnly{\rule{#1}{0.4pt}}}
    \else{\answer{#2}}
    \fi
}


\ifhandout
\newenvironment{shortAnswer}[1][
    \vfill]
        {% Begin then result
        #1
            \begin{freeResponse}
            }
    {% Environment Ending Code
    \end{freeResponse}
    }
\else
\newenvironment{shortAnswer}[1][]
        {\begin{freeResponse}
            }
    {% Environment Ending Code
    \end{freeResponse}
    }
\fi

\let\question\relax
\let\endquestion\relax

\newtheoremstyle{ExerciseStyle}{\topsep}{\topsep}%%% space between body and thm
		{}                      %%% Thm body font
		{}                              %%% Indent amount (empty = no indent)
		{\bfseries}            %%% Thm head font
		{}                              %%% Punctuation after thm head
		{3em}                           %%% Space after thm head
		{{#1}~\thmnumber{#2}\thmnote{ \bfseries(#3)}}%%% Thm head spec
\theoremstyle{ExerciseStyle}
\newtheorem{br}{In-class Problem}

\newenvironment{sketch}
 {\begin{proof}[Sketch of Proof]}
 {\end{proof}}


\newcommand{\gt}{>}
\newcommand{\lt}{<}
\newcommand{\N}{\mathbb N}
\newcommand{\Q}{\mathbb Q}
\newcommand{\Z}{\mathbb Z}
\newcommand{\C}{\mathbb C}
\newcommand{\R}{\mathbb R}
\renewcommand{\H}{\mathbb{H}}
\newcommand{\lcm}{\operatorname{lcm}}
\newcommand{\nequiv}{\not\equiv}
\newcommand{\ord}{\operatorname{ord}}
\newcommand{\ds}{\displaystyle}
\newcommand{\floor}[1]{\left\lfloor #1\right\rfloor}
\newcommand{\legendre}[2]{\left(\frac{#1}{#2}\right)}



%%%%%%%%%%%%



\title{Fibonacci Sequence}


\begin{document}
\begin{abstract}
 Project on Fibonacci and Lucas sequence.
\end{abstract}
\maketitle

\begin{exploration}
 Start by reviewing the Fibonacci Sequence from Strayer Appendix A.2: 
 $F_1=1, F_2=1, F_{k+1}=F_k+F_{k-1}$ for $k\geq3.$ 
\begin{problem}
	Prove that $\ds\lim_{k\to\infty}\frac{F_{k+1}}{F_k}=\phi$ where $\phi=\frac{1+\sqrt{5}}{2}.$
	
\begin{rubric} 5 points if individual project, 3 points if presenting as a pair.
\end{rubric}
\end{problem}

\begin{problem}
 	Prove that for every positive integer $k,$ 
	\[F_1+F_2+\cdots+F_k=F_{k+2}-1.\]
	
\begin{rubric} 5 points if individual project, 3 points if presenting as a pair.
\end{rubric}
\end{problem}

\begin{problem}(If presenting as a pair)
 	Strayer Exercise Set A, Exercise 2.
	\begin{rubric} 6 points.
\end{rubric}
\end{problem}

	
%\begin{problem}
% 	Suppose $P_n$ is the sequence defined by $P_1=2, P_2=2, P_{n}=P_{n-1}P_{n-2}$ for $n\geq 3.$ Prove that $P_n=2^{F_n}$ for all positive integers $n.$
%\end{problem}
\end{exploration}

The following problems are from \emph{Number Theory: A Lively Introduction with Proofs, Applications, and Stories} by Erica Flapan, Tim Marks, and James Pommersheim.


\begin{exploration}
 	The following problems are from \emph{Number Theory: A Lively Introduction with Proofs, Applications, and Stories} by Erica Flapan, Tim Marks, and James Pommersheim.

	The Lucas numbers are similar to the Fibonacci numbers, where $L_1=1, L_2=3, L_{k+1}=L_k+L_{k-1}$ for $k\geq3.$ 

	\begin{problem}
 		\begin{enumerate}
 			\item Make a table of the first $12$ Lucas numbers. \emph{You do not need to present this part}
			\item Use your results from part (a) to calculate the ratios of pairs of consecutive Lucas numbers. \emph{You do not need to present this part}
			\item Make a conjecture about the value of  $\ds\lim_{k\to\infty}\frac{L_{k+1}}{L_k}.$ \emph{You do not need to present this part}
			\item Prove your conjecture is correct. \emph{You \textbf{do} need to present this part}
		\end{enumerate}
		\begin{rubric} 5 points if individual project, 3 points if presenting as a pair.
\end{rubric}
	\end{problem}
	
\begin{problem}(If presenting as a pair)
 		\begin{enumerate}
 			\item Calculate $L_1, L_1+L_2, L_1+L_2+L_3, L_1+L_2+L_3+L_4.$ \emph{You do not need to present this part}
			\item Make a conjecture about the relationship between  $L_1+L_2+L_3 +\cdots+L_n$ and the number $L_{n+2}.$ \emph{You do not need to present this part}
			\item Prove your conjecture is correct. \emph{You \textbf{do} need to present this part}
		\end{enumerate}
		\begin{rubric} 5 points if individual project, 3 points if presenting as a pair.
\end{rubric}
\end{problem}
\end{exploration}
\end{document}