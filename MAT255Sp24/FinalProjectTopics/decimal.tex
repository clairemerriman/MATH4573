\documentclass[letterpaper, 11 pt,handout,hints]{ximera}
<<<<<<< Updated upstream
\usepackage{amssymb, latexsym, amsmath, amsthm, graphicx, amsthm,alltt,color, listings,multicol,xr-hyper,hyperref,aliascnt,enumitem}
=======
\usepackage{amssymb, latexsym, amsmath, amsthm, graphicx, amsthm,alltt,color, listings,multicol,hyperref}
\usepackage[capitalise,nameinlink]{cleveref}
>>>>>>> Stashed changes
\usepackage{xfrac}

\usepackage{parskip}
\usepackage[,margin=0.7in]{geometry}
\setlength{\textheight}{8.5in}

\usepackage{epstopdf}

\DeclareGraphicsExtensions{.eps}
\usepackage{tikz}


\usepackage{tkz-euclide}
%\usetkzobj{all}
\tikzstyle geometryDiagrams=[rounded corners=.5pt,ultra thick,color=black]
\colorlet{penColor}{black} % Color of a curve in a plot


\usepackage{subcaption}
\usepackage{float}
\usepackage{fancyhdr}
\usepackage{pdfpages}
\newcounter{includepdfpage}
\usepackage{makecell}


\usepackage{currfile}
\usepackage{xstring}




\graphicspath{  
{./otherDocuments/}
}

\author{Claire Merriman}
\newcommand{\classday}[1]{\def\classday{#1}}

%%%%%%%%%%%%%%%%%%%%%
% Counters and autoref for unnumbered environments
% Not needed??
%%%%%%%%%%%%%%%%%%%%%
<<<<<<< Updated upstream
\theoremstyle{plain}


\newtheorem*{namedthm}{Theorem}
\newcounter{thm}%makes pointer correct
\providecommand{\thmname}{Theorem}
=======

\crefname{problem}{problem}{problems}


% \theoremstyle{plain}


% \newtheorem*{namedthm}{Theorem}
% \newcounter{thm}%makes pointer correct
% \providecommand{\thmname}{Theorem}
>>>>>>> Stashed changes

\makeatletter
\NewDocumentEnvironment{thm*}{o}
 {%
  \IfValueTF{#1}
    {\namedthm[#1]\refstepcounter{thm}\def\@currentlabel{(#1)}}%
    {\namedthm}%
 }
 {%
  \endnamedthm
 }
\makeatother


\newtheorem*{namedprop}{Proposition}
\newcounter{prop}%makes pointer correct
\providecommand{\propname}{Proposition}

\makeatletter
\NewDocumentEnvironment{prop*}{o}
 {%
  \IfValueTF{#1}
    {\namedprop[#1]\refstepcounter{prop}\def\@currentlabel{(#1)}}%
    {\namedprop}%
 }
 {%
  \endnamedprop
 }
\makeatother

\newtheorem*{namedlem}{Lemma}
\newcounter{lem}%makes pointer correct
\providecommand{\lemname}{Lemma}

\makeatletter
\NewDocumentEnvironment{lem*}{o}
 {%
  \IfValueTF{#1}
    {\namedlem[#1]\refstepcounter{lem}\def\@currentlabel{(#1)}}%
    {\namedlem}%
 }
 {%
  \endnamedlem
 }
\makeatother

\newtheorem*{namedcor}{Corollary}
\newcounter{cor}%makes pointer correct
\providecommand{\corname}{Corollary}

\makeatletter
\NewDocumentEnvironment{cor*}{o}
 {%
  \IfValueTF{#1}
    {\namedcor[#1]\refstepcounter{cor}\def\@currentlabel{(#1)}}%
    {\namedcor}%
 }
 {%
  \endnamedcor
 }
\makeatother

\theoremstyle{definition}
\newtheorem*{annotation}{Annotation}
\newtheorem*{rubric}{Rubric}

\newtheorem*{innerrem}{Remark}
\newcounter{rem}%makes pointer correct
\providecommand{\remname}{Remark}

\makeatletter
\NewDocumentEnvironment{rem}{o}
 {%
  \IfValueTF{#1}
    {\innerrem[#1]\refstepcounter{rem}\def\@currentlabel{(#1)}}%
    {\innerrem}%
 }
 {%
  \endinnerrem
 }
\makeatother

\newtheorem*{innerdefn}{Definition}%%placeholder
\newcounter{defn}%makes pointer correct
\providecommand{\defnname}{Definition}

\makeatletter
\NewDocumentEnvironment{defn}{o}
 {%
  \IfValueTF{#1}
    {\innerdefn[#1]\refstepcounter{defn}\def\@currentlabel{(#1)}}%
    {\innerdefn}%
 }
 {%
  \endinnerdefn
 }
\makeatother

\newtheorem*{scratch}{Scratch Work}


\newtheorem*{namedconj}{Conjecture}
\newcounter{conj}%makes pointer correct
\providecommand{\conjname}{Conjecture}
\makeatletter
\NewDocumentEnvironment{conj}{o}
 {%
  \IfValueTF{#1}
    {\innerconj[#1]\refstepcounter{conj}\def\@currentlabel{(#1)}}%
    {\innerconj}%
 }
 {%
  \endinnerconj
 }
\makeatother

\newtheorem*{poll}{Poll question}
\newtheorem{tps}{Think-Pair-Share}[section]


\newenvironment{obj}{
	\textbf{Learning Objectives.} By the end of class, students will be able to:
		\begin{itemize}}
		{\!.\end{itemize}
		}

<<<<<<< Updated upstream
\newenvironment{pre}{
	\begin{description}
	}{
	\end{description}
}
=======

\ifinstructornotes
\newenvironment{pre}
  {{\textbf Reading assignment:}
  \begin{description}
    }{
	\end{description}
  }
\else
\newenvironment{pre}{ 
  \begin{trivlist}
  \item[]}
  {\end{trivlist}}
\fi
>>>>>>> Stashed changes


\newcounter{ex}%makes pointer correct
\providecommand{\exname}{Homework Problem}
\newenvironment{ex}[1][2in]%
{%Env start code
\problemEnvironmentStart{#1}{Homework Problem}
\refstepcounter{ex}
}
{%Env end code
\problemEnvironmentEnd
}

\newcommand{\inlineAnswer}[2][2 cm]{
    \ifhandout{\pdfOnly{\rule{#1}{0.4pt}}}
    \else{\answer{#2}}
    \fi
}


\ifhandout
\newenvironment{shortAnswer}[1][
    \vfill]
        {% Begin then result
        #1
            \begin{freeResponse}
            }
    {% Environment Ending Code
    \end{freeResponse}
    }
\else
\newenvironment{shortAnswer}[1][]
        {\begin{freeResponse}
            }
    {% Environment Ending Code
    \end{freeResponse}
    }
\fi

\let\question\relax
\let\endquestion\relax

\newtheoremstyle{ExerciseStyle}{\topsep}{\topsep}%%% space between body and thm
		{}                      %%% Thm body font
		{}                              %%% Indent amount (empty = no indent)
		{\bfseries}            %%% Thm head font
		{}                              %%% Punctuation after thm head
		{3em}                           %%% Space after thm head
		{{#1}~\thmnumber{#2}\thmnote{ \bfseries(#3)}}%%% Thm head spec
\theoremstyle{ExerciseStyle}
\newtheorem{br}{In-class Problem}

\newenvironment{sketch}
 {\begin{proof}[Sketch of Proof]}
 {\end{proof}}


\newcommand{\gt}{>}
\newcommand{\lt}{<}
\newcommand{\N}{\mathbb N}
\newcommand{\Q}{\mathbb Q}
\newcommand{\Z}{\mathbb Z}
\newcommand{\C}{\mathbb C}
\newcommand{\R}{\mathbb R}
\renewcommand{\H}{\mathbb{H}}
\newcommand{\lcm}{\operatorname{lcm}}
\newcommand{\nequiv}{\not\equiv}
\newcommand{\ord}{\operatorname{ord}}
\newcommand{\ds}{\displaystyle}
\newcommand{\floor}[1]{\left\lfloor #1\right\rfloor}
\newcommand{\legendre}[2]{\left(\frac{#1}{#2}\right)}



%%%%%%%%%%%%



\title{Decimal expansions}


\begin{document}
\begin{abstract}
 Project on decimal expansions.
\end{abstract}
\maketitle

In this project, you will classify decimal (and duadecimal) expansions are finite, periodic, or aperiodic.

Read Strayer Section 7.1.
\begin{rubric}
 Introducing topic, definitions, and any necessary results: 4 points 
\end{rubric}

\begin{exploration}
	We say that the decimal expansion of a real number is \emph{finite} if it terminates after a finite number of digits. If the pattern of digits eventually repeats, such as $\frac{1}{3}=0.33333\dots=0.\overline{3}$ or $\frac{1}{28}=0.03571428571428\dots=0.03\overline{571428},$ we say the decimal is \emph{periodic}. We could also say that finite decimal expansions are periodic, where the periodic part is $\overline{0}.$ Otherwise, we say the decimal is \emph{aperiodic}.


\begin{problem}\label{prob:dec-condition}
 
	\begin{enumerate}
		\item\label{decimal-ex} Find the decimal expansions of $\frac{3}{2}, \frac{2}{3},\frac{7}{25},\frac{2}{7},\frac{3}{20},\frac{2}{15}$ using WolframAlpha or another resource or method that will tell you if the decimal is periodic. These particular examples should be fine with a regular calculator. \emph{You do not need to present these answers, they are to help with the next part.}
		\item Complete and prove the statement:		
	\begin{conjecture}
 		Let $a,b\in\Z$ with $b\neq 0$ and $(a,b)=1.$ Then $\frac{a}{b}$ has a finite decimal expansion if and only if $b$ has no prime factors other than $\answer{2,5}.$
	\end{conjecture}
	\end{enumerate}
\begin{rubric}
 4 points if individual, 3 points if pair.
\end{rubric}
\end{problem}

\begin{problem}
	Prove that if a real number has terminating decimal expansion, then it must be rational. Prove that if a real number has a periodic decimal expansion, then it must be rational. 
\begin{hint}
 Generalize the idea in Example 4 and the proof of Proposition 7.4.
\end{hint}
\begin{rubric}
 6 points if individual, 4 points if pair.
\end{rubric}
\end{problem}
\end{exploration}

\begin{exploration}
 
\begin{definition}
	Let $x\in\R$ with $0<x<1.$ Then the \emph{base 12} expansion of $x$ is \[\sum_{n=1}^\infty \frac{d_n}{12^n}=0.d_1 d_2\dots, \quad d_i\in\{1,2,\dots,11,12\}.\]
	If there exist a positive integer $r$ and $N$ such that $d_n=d_{n+r}$ for all $n\geq N$, then $x$ is \emph{periodic} and $d_Nd_{N+1}\dots d_{N+r}$ is the periodic part of the base 12 expansion.
	The base 12 expansion of $x$ is finite if the periodic part is $0.$
	
	If $m$ is a positive integer, then the base 12 expansion of $m$ is  
	\[\sum_{k=1}^n {a_k}{12^k}, \quad a_i\in\{1,2,\dots,11,12\}.\]
	Combining these definitions gives the base 12 expansion of any positive real number (the only reason to separate the definitions is dealing with the indices of the summation)
\end{definition}

\begin{example}\label{ex:duodec}
	\begin{enumerate}
 		\item To find the base 12 expansion of $\frac{3}{2},$ first write $\frac{3}{2}=1+\frac{1}{2}.$ Since $1\in\{1,2,\dots,12\}$ the base 12 expansion of $1$ is still $1.$ Then we find the base $12$ expansion of $\frac{1}{2}:$
			\begin{align*}
 				\frac{1}{2}&=\frac{a_1}{12}+\frac{a_2}{12^2}+\cdots\\
				6(1)&={a_1}+\frac{a_2}{12}+\cdots\\
			\end{align*}
		Then we can make $a_1=6$ and the rest of the $a_i=0.$ So the base 12 expansion of $\frac{3}{2}$ is $1.6$

		\item To find the base 12 expansion of $\frac{1}{10},$ 
			\begin{align*}
				\frac{1}{11}&=\frac{a_1}{12}+\frac{a_2}{12^2}+\cdots\\
				12\left(\frac{1}{11}\right)&={a_1}+\frac{a_2}{12}+\cdots\\
				1+\frac{1}{11}&={a_1}+\frac{a_2}{12}+\cdots\\
			\end{align*}
			So $a_1=1,$ and we are back to where we started, finding the base 12 expansion of $\frac{1}{11}.$ Thus $\frac{1}{11}=0.\overline{1}.$
		\item Often $a$ is used to represent $10$ and $b$ is used to represent $11.$ The base 12 expansion of \[\frac{10}{11}=b\frac{1}{11}=b\left(\frac{1}{12}+\frac{1}{12^2}+\dots+\right)=0.\overline{b}.\]
	\end{enumerate}
	
\end{example}

\begin{problem}
 	Adjust the technique from Example \ref{ex:duodec} or check with \href{https://www.wolframalpha.com/input?i=1\%2F11+to+base+12}{WolframAlpha} to find the base 12 expansions of the fractions from Problem 1.1\ref{decimal-ex}. Present some of these in class: you do not need to present the work, only the expansion.
\begin{rubric}
 2 points.
\end{rubric}
 \end{problem}
 
 
\begin{problem}
  	Complete and prove the statement:		
		\begin{conjecture}
 			Let $a,b\in\Z$ with $b\neq 0$ and $(a,b)=1.$ Then $\frac{a}{b}$ has a finite base 12 expansion if and only if $b$ has no prime factors other than $\answer{2,3}.$
		\end{conjecture}
\begin{rubric}
 4 points if individual, 3 points if pair.
\end{rubric}
\end{problem}

\begin{problem}(If presenting as a pair)
	Prove that if a real number has terminating base 12 expansion, then it must be rational.
\begin{hint}
 Generalize the idea in Example 4 and the proof of Proposition 7.4.
\end{hint}
\begin{rubric}
 4 points.
\end{rubric}
\end{problem}
\end{exploration}
\end{document}