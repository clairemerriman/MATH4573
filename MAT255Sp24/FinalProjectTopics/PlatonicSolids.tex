\documentclass[letterpaper, 11 pt]{ximera}
<<<<<<< Updated upstream
\usepackage{amssymb, latexsym, amsmath, amsthm, graphicx, amsthm,alltt,color, listings,multicol,xr-hyper,hyperref,aliascnt,enumitem}
=======
\usepackage{amssymb, latexsym, amsmath, amsthm, graphicx, amsthm,alltt,color, listings,multicol,hyperref}
\usepackage[capitalise,nameinlink]{cleveref}
>>>>>>> Stashed changes
\usepackage{xfrac}

\usepackage{parskip}
\usepackage[,margin=0.7in]{geometry}
\setlength{\textheight}{8.5in}

\usepackage{epstopdf}

\DeclareGraphicsExtensions{.eps}
\usepackage{tikz}


\usepackage{tkz-euclide}
%\usetkzobj{all}
\tikzstyle geometryDiagrams=[rounded corners=.5pt,ultra thick,color=black]
\colorlet{penColor}{black} % Color of a curve in a plot


\usepackage{subcaption}
\usepackage{float}
\usepackage{fancyhdr}
\usepackage{pdfpages}
\newcounter{includepdfpage}
\usepackage{makecell}


\usepackage{currfile}
\usepackage{xstring}




\graphicspath{  
{./otherDocuments/}
}

\author{Claire Merriman}
\newcommand{\classday}[1]{\def\classday{#1}}

%%%%%%%%%%%%%%%%%%%%%
% Counters and autoref for unnumbered environments
% Not needed??
%%%%%%%%%%%%%%%%%%%%%
<<<<<<< Updated upstream
\theoremstyle{plain}


\newtheorem*{namedthm}{Theorem}
\newcounter{thm}%makes pointer correct
\providecommand{\thmname}{Theorem}
=======

\crefname{problem}{problem}{problems}


% \theoremstyle{plain}


% \newtheorem*{namedthm}{Theorem}
% \newcounter{thm}%makes pointer correct
% \providecommand{\thmname}{Theorem}
>>>>>>> Stashed changes

\makeatletter
\NewDocumentEnvironment{thm*}{o}
 {%
  \IfValueTF{#1}
    {\namedthm[#1]\refstepcounter{thm}\def\@currentlabel{(#1)}}%
    {\namedthm}%
 }
 {%
  \endnamedthm
 }
\makeatother


\newtheorem*{namedprop}{Proposition}
\newcounter{prop}%makes pointer correct
\providecommand{\propname}{Proposition}

\makeatletter
\NewDocumentEnvironment{prop*}{o}
 {%
  \IfValueTF{#1}
    {\namedprop[#1]\refstepcounter{prop}\def\@currentlabel{(#1)}}%
    {\namedprop}%
 }
 {%
  \endnamedprop
 }
\makeatother

\newtheorem*{namedlem}{Lemma}
\newcounter{lem}%makes pointer correct
\providecommand{\lemname}{Lemma}

\makeatletter
\NewDocumentEnvironment{lem*}{o}
 {%
  \IfValueTF{#1}
    {\namedlem[#1]\refstepcounter{lem}\def\@currentlabel{(#1)}}%
    {\namedlem}%
 }
 {%
  \endnamedlem
 }
\makeatother

\newtheorem*{namedcor}{Corollary}
\newcounter{cor}%makes pointer correct
\providecommand{\corname}{Corollary}

\makeatletter
\NewDocumentEnvironment{cor*}{o}
 {%
  \IfValueTF{#1}
    {\namedcor[#1]\refstepcounter{cor}\def\@currentlabel{(#1)}}%
    {\namedcor}%
 }
 {%
  \endnamedcor
 }
\makeatother

\theoremstyle{definition}
\newtheorem*{annotation}{Annotation}
\newtheorem*{rubric}{Rubric}

\newtheorem*{innerrem}{Remark}
\newcounter{rem}%makes pointer correct
\providecommand{\remname}{Remark}

\makeatletter
\NewDocumentEnvironment{rem}{o}
 {%
  \IfValueTF{#1}
    {\innerrem[#1]\refstepcounter{rem}\def\@currentlabel{(#1)}}%
    {\innerrem}%
 }
 {%
  \endinnerrem
 }
\makeatother

\newtheorem*{innerdefn}{Definition}%%placeholder
\newcounter{defn}%makes pointer correct
\providecommand{\defnname}{Definition}

\makeatletter
\NewDocumentEnvironment{defn}{o}
 {%
  \IfValueTF{#1}
    {\innerdefn[#1]\refstepcounter{defn}\def\@currentlabel{(#1)}}%
    {\innerdefn}%
 }
 {%
  \endinnerdefn
 }
\makeatother

\newtheorem*{scratch}{Scratch Work}


\newtheorem*{namedconj}{Conjecture}
\newcounter{conj}%makes pointer correct
\providecommand{\conjname}{Conjecture}
\makeatletter
\NewDocumentEnvironment{conj}{o}
 {%
  \IfValueTF{#1}
    {\innerconj[#1]\refstepcounter{conj}\def\@currentlabel{(#1)}}%
    {\innerconj}%
 }
 {%
  \endinnerconj
 }
\makeatother

\newtheorem*{poll}{Poll question}
\newtheorem{tps}{Think-Pair-Share}[section]


\newenvironment{obj}{
	\textbf{Learning Objectives.} By the end of class, students will be able to:
		\begin{itemize}}
		{\!.\end{itemize}
		}

<<<<<<< Updated upstream
\newenvironment{pre}{
	\begin{description}
	}{
	\end{description}
}
=======

\ifinstructornotes
\newenvironment{pre}
  {{\textbf Reading assignment:}
  \begin{description}
    }{
	\end{description}
  }
\else
\newenvironment{pre}{ 
  \begin{trivlist}
  \item[]}
  {\end{trivlist}}
\fi
>>>>>>> Stashed changes


\newcounter{ex}%makes pointer correct
\providecommand{\exname}{Homework Problem}
\newenvironment{ex}[1][2in]%
{%Env start code
\problemEnvironmentStart{#1}{Homework Problem}
\refstepcounter{ex}
}
{%Env end code
\problemEnvironmentEnd
}

\newcommand{\inlineAnswer}[2][2 cm]{
    \ifhandout{\pdfOnly{\rule{#1}{0.4pt}}}
    \else{\answer{#2}}
    \fi
}


\ifhandout
\newenvironment{shortAnswer}[1][
    \vfill]
        {% Begin then result
        #1
            \begin{freeResponse}
            }
    {% Environment Ending Code
    \end{freeResponse}
    }
\else
\newenvironment{shortAnswer}[1][]
        {\begin{freeResponse}
            }
    {% Environment Ending Code
    \end{freeResponse}
    }
\fi

\let\question\relax
\let\endquestion\relax

\newtheoremstyle{ExerciseStyle}{\topsep}{\topsep}%%% space between body and thm
		{}                      %%% Thm body font
		{}                              %%% Indent amount (empty = no indent)
		{\bfseries}            %%% Thm head font
		{}                              %%% Punctuation after thm head
		{3em}                           %%% Space after thm head
		{{#1}~\thmnumber{#2}\thmnote{ \bfseries(#3)}}%%% Thm head spec
\theoremstyle{ExerciseStyle}
\newtheorem{br}{In-class Problem}

\newenvironment{sketch}
 {\begin{proof}[Sketch of Proof]}
 {\end{proof}}


\newcommand{\gt}{>}
\newcommand{\lt}{<}
\newcommand{\N}{\mathbb N}
\newcommand{\Q}{\mathbb Q}
\newcommand{\Z}{\mathbb Z}
\newcommand{\C}{\mathbb C}
\newcommand{\R}{\mathbb R}
\renewcommand{\H}{\mathbb{H}}
\newcommand{\lcm}{\operatorname{lcm}}
\newcommand{\nequiv}{\not\equiv}
\newcommand{\ord}{\operatorname{ord}}
\newcommand{\ds}{\displaystyle}
\newcommand{\floor}[1]{\left\lfloor #1\right\rfloor}
\newcommand{\legendre}[2]{\left(\frac{#1}{#2}\right)}



%%%%%%%%%%%%


\usepackage{tikz}

\title{Platonic Solids}


\begin{document}
\begin{exploration}
A mathematical \emph{graph} is a set of vertices connected by edges. Each edge connects two (not necessarily distinct) vertices. A graph is \emph{planar} if it can be drawn without any of the edges crossing. If it is possible to get from any vertex to any other vertex by traversing edges, the graph is \emph{connected.} Some examples of connected planar graphs are in \autoref{fig:graphs}.
    
    \begin{figure}
    	\begin{subfigure}{0.3\textwidth}
		\begin{center}
			\begin{tikzpicture}[scale=1.5]
    				% Vertices
    				\foreach \x/\y/\name in {0/0/A, 1/0/B, 1/1/C, 0/1/D, 0/1/E, 2/1/F} {
        				\coordinate (\name) at (\x,\y);
        				\draw[fill=black] (\name) circle (1.5pt);
    				}
    
    				% Edges
    				\draw (A) -- (B) -- (C) -- (D) -- (A);
    				\draw (D) -- (E) -- (F);
				\end{tikzpicture}
			\end{center}
		\caption{A graph with $5$ vertices, $5$ edges, and $2$ faces.}
        \end{subfigure}
        \begin{subfigure}{0.3\textwidth}
		\begin{center}
			\begin{tikzpicture}[scale=2]
    				% Vertices
   			 	\foreach \x/\y/\name in {0/0/A, 1/0/B, 2/0/C, 1/.5/D, 0/1/E, 2/1/F} {
        				\coordinate (\name) at (\x,\y);
        				\draw[fill=black] (\name) circle (1.5pt);
    				}
    
    				% Edges
    				\draw (A) -- (B) -- (C) -- (F) -- (E) -- (A) -- (D) -- (B);
    				\draw (D) -- (E);
			\end{tikzpicture}
             \end{center}
            \caption{A graph with $6$ vertices, $8$ edges, and $4$ faces.}
        \end{subfigure}
        \begin{subfigure}{0.3\textwidth}
		\begin{center}
			\begin{tikzpicture}[scale=2]
    				% Vertices
   			 	\foreach \x/\y/\name in {0/0/A, 1/0/B, 2/0/C, 1/.5/D, 0/1/E} {
        				\coordinate (\name) at (\x,\y);
        				\draw[fill=black] (\name) circle (1.5pt);
    				}
    
    				% Edges
    				\draw (A) -- (B) -- (C) -- (E) -- (A) -- (D) -- (B);
    				\draw (D) -- (E);
			\end{tikzpicture}
             \end{center}
            \caption{A graph with $5$ vertices, $7$ edges, and $4$ faces.}
        \end{subfigure}
        \caption{Examples of connected planar graphs}\label{fig:graphs}
    \end{figure}

A connected planar graph divides the plane into disjoint enclosed areas called \emph{faces}. Note that the entire region outside of the graph is a face.


\begin{theorem}[Euler's Formula]\label{thm:euler-form}
	Call the number of vertices in a graph $V,$ the number of edges $E,$ and the number of faces $F.$ For a connected planar graph, \[V-E+F=2.\]
\end{theorem}

The following problems are from \emph{Number Theory: A Lively Introduction with Proofs, Applications, and Stories} by Erica Flapan, Tim Marks, and James Pommersheim.

\begin{problem}

	\begin{enumerate}
 		\item The simplest graph consists of a single vertex with no edges. Verify that \nameref{thm:euler-form} is true for this graph.
		\item We can construct any planar graph by starting with a single vertex and repeatedly doing one of two moves:
		\begin{description}
 			\item[Move 1:] Starting at any vertex, draw a new edge that does not cross any existing edge, and then place a vertex at the end of your new edge.
			\item[Move 2:] Connect any two existing vertices with a new edge that does not cross any existing edge.
		\end{description}
		Show that in doing either move, the value of $V-E+F$ does not change.
	
		\item Why does these facts lead to a proof that \nameref{thm:euler-form} holds true for all connected planar graphs?
	\end{enumerate}
 
\end{problem}	
\end{exploration}


\begin{exploration}
A polyhedron is a closed, three-dimensional shape with flat polygonal faces and straight edges. 
\begin{problem}
 Explain why we can draw the vertices, edges, and faces of polyhedra as a planar graph, so \nameref{thm:euler-form} is also true for polyhedra. See \autoref{fig:cube} for an example.
\end{problem}

\begin{figure}
\begin{center}
	\begin{tikzpicture}[scale=1.5]
    		% Vertices
    		\foreach \x/\y/\name in {0/0/A, 1/0/B, 1/1/C, 0/1/D, -.5/-.5/E, 1.5/-.5/F, 1.5/1.5/G, -.5/1.5/H} {
        		\coordinate (\name) at (\x,\y);
        		\draw[fill=black] (\name) circle (1.5pt);
    		}
    
    		% Edges
    		\draw (A) -- (B) -- (C) -- (D) -- (A);
    		\draw (E) -- (F) -- (G) -- (H) -- (E);
    		\draw (A) -- (E);
    		\draw (B) -- (F);
    		\draw (C) -- (G);
    		\draw (D) -- (H);
	\end{tikzpicture}
 	\caption{A cube drawn as a planar graph.}\label{fig:cube}
 \end{center}
\end{figure}



A \emph{regular polygon} is a convex polygon where every side is the same length and every angle is the same size. 
A \emph{platonic solid} is a convex polyhedron where every is the same regular polygon, and the same number of faces meet at each vertex.



\begin{theorem}\label{thm:num-plat-solids}
	 There are only five platonic solids. 
\end{theorem}

\begin{problem}
Let $n$ be the number of edges for each face and $m$ be the number of faces that meet at a vertex.
	\begin{enumerate}
		\item Explain why $E= \frac{1}{2}(Fn)$ and $V=\frac{Fn}{m}.$
		\item Use \nameref{thm:euler-form} to finish the proof of \autoref{thm:num-plat-solids}.
	\end{enumerate}
\end{problem}


\begin{problem}(If presenting as a pair)
	Prove that if every face of a polyhedron is a pentagon or a hexagon and if three faces meet at every vertex, then the polyhedron has exactly twelve pentagonal faces.
\end{problem}

\begin{problem}(If presenting as a pair)
	Prove that if every face of a polyhedron is a triangle and five or six faces meet at every vertex, then the polyhedron has exactly twelve vertices where five faces meet.
\end{problem}
 
\end{exploration}
\end{document}