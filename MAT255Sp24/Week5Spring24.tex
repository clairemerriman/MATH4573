\documentclass[letterpaper, 11 pt]{ximera}
\usepackage{amssymb, latexsym, amsmath, amsthm, graphicx, amsthm,alltt,color, listings,multicol,hyperref,enumitem,tikz}
\usepackage{xfrac}

\usepackage{parskip}
\usepackage{graphicx}
\usepackage[,margin=0.7in]{geometry}
\setlength{\textheight}{8.5in}
 
\usepackage{tkz-euclide}
%\usetkzobj{all}
\tikzstyle geometryDiagrams=[rounded corners=.5pt,ultra thick,color=black]
\colorlet{penColor}{black} % Color of a curve in a plot


\usepackage{subcaption}
\usepackage{float}
\usepackage{fancyhdr}
\usepackage{pdfpages}
\newcounter{includepdfpage}


\newcommand{\semester}{%
  \ifcase\month
  \or Spring %1
  \or Spring %2
  \or Spring %3
  \or Spring %4
  \or Spring  %5
  \or Fall %8
  \or Fall %9
  \or Fall %10
  \or Fall %11
  \or Fall %12
  \fi
}
\usepackage{currfile}
\usepackage{xstring}



\lhead{\large{Number Theory: MAT-255}}
%Put your Document Title (Camp: Topic) Here
\chead{}
\rhead{\semester 24}
\lfoot{}
\cfoot{}
\rfoot{Page \thepage}
\renewcommand\headrulewidth{0pt}
\renewcommand\footrulewidth{0pt}

\headheight 50pt
\headsep 30pt

\author{Spring 2024}
\date{}
 \theoremstyle{plain}
 
\newtheorem{thm}{Theorem}%[section] % reset theorem numbering for each section
\newtheorem*{thm*}{Theorem}
\newtheorem{prop}[thm]{Proposition}
\newtheorem*{prop*}{Proposition}
\newtheorem{lem}[thm]{Lemma}
\newtheorem*{lem*}{Lemma}
\newtheorem{cor}[thm]{Corollary}
\newtheorem*{cor*}{Corollary}

\theoremstyle{definition}
\newtheorem*{rem}{Remark}
\newtheorem*{defn}{Definition}
\newtheorem{defn_num}{Definition}
\newtheorem*{ex}{Exercise}
\newtheorem*{scratch}{Scratch Work}

%\newtheorem*{focus}{Central Focus}
%\newtheorem*{obj}{Objectives}

\newtheorem*{conj}{Conjecture}

\newtheorem*{poll}{Poll question}
\newtheorem{tps}{Think-Pair-Share}[section]
%\newtheorem{br}{In-class Problem}[section]
\newtheorem*{cs}{Crowd Sourced Proof}

\newlist{checklist}{itemize}{2}
\setlist[checklist]{label=$\square$}

\newenvironment{obj}{
	\textbf{Learning Objectives.} By the end of class, students will be able to:
		\begin{itemize}}
		{\!.\end{itemize}
		}

\newenvironment{pre}{
	\begin{description}
	}{
	\end{description}
}


\newenvironment{br}[1][2in]%
{%Env start code
\problemEnvironmentStart{#1}{In-class Problem}
}
{%Env end code
\problemEnvironmentEnd
}


%\newenvironment{solution}
%  {\begin{proof}[Solution]}
%  {\end{proof}}
%\newenvironment{hint}
%  {\begin{proof}[Hint]}
%  {\end{proof}}

\newcommand{\gt}{>}
\newcommand{\lt}{<}
\newcommand{\N}{\mathbb N}
\newcommand{\Q}{\mathbb Q}
\newcommand{\Z}{\mathbb Z}
\newcommand{\C}{\mathbb C}
\newcommand{\R}{\mathbb R}
\renewcommand{\H}{\mathbb{H}}
\newcommand{\lcm}{\operatorname{lcm}}
\newcommand{\nequiv}{\not\equiv}
\newcommand{\ord}{\operatorname{ord}}
\newcommand{\ds}{\displaystyle}


%Imports for cross references
\externaldocument{otherResults}
\externaldocument{Week1Spring24}
\externaldocument{Week2Spring24}
\externaldocument{Week3Spring24}
\externaldocument{Week4Spring24}


\StrBetween*[1,1]{\currfilename}{Week}{Sp}[\week]

\title{Week \week--MAT-255 Number Theory}

\begin{document}


%%%%%%%%%%%%%%%%%%%%%%%%%%
\section{Monday, February 12: Peer review and introduction to congruences}
%%%%%%%%%%%%%%%%%%%%%%%%%%

\begin{obj}
\item Prove basic facts about modular arithmetic.
 \item Understand gaps in argument and writing of proof of Exercise 83. Give classmates useful feedback on their proofs.
\end{obj}
%%%%%%%%%%%%%%%%%%%%%%%%%
\subsection{Proposition and examples of $\pmod m$ (30 minutes)}
%%%%%%%%%%%%%%%%%%%%%%%%%%

\begin{defn}[$a\equiv b\pmod{m}$]\label{defn:mod-equiv-all}
Let $a,b,m\in\Z$ with $m>0.$ From Friday, we have the following equivalent definitions of congruence modulo $m:$
\begin{enumerate}
 \item $a\equiv b \pmod m$ if and only if\footnote{all definitions are if and only if} $m\mid b-a$ (standard definition, generalizing even/odd based on divisibility)
 \item $a\equiv b \pmod m$ if and only if $a$ and $b$ have the same remainder with divided by $m.$ That is, That is, there exists unique $q_1,q_2,r\in\Z$ such that  $a=mq_1+r,\   b=mq_2+r,\  0\leq r<m.$ (definition generalizing even/odd based on remainder)
\item $a\equiv b\pmod m$ if and only if $a$ and $b$ differ by a multiple of $m.$ That is, $b=a+mk$ for some $k\in\Z.$ (arithmetic progression definition)
\end{enumerate}
\end{defn}

Different statements of the definition will be useful in different situations

\begin{proposition}[Restatement of Propositions 2.1, 2.4, and 2.5]\label{prop:equiv-arith}
 Let $a,b,c,d,m\in\Z$ with $m>0,$ then:
\begin{enumerate}
\item $a\equiv b \pmod{m}$ and $b\equiv c \pmod{m}$ implies $a\equiv c \pmod{m}$
\item\label{equiv-add} $a\equiv b \pmod{m}$ and $c\equiv d \pmod{m}$ implies $a+c \equiv b+d \pmod{m}$ 
\item\label{equiv-multiply} $a\equiv b\pmod{m}$ and $c\equiv d \pmod{m}$ implies $ac\equiv bd \pmod{m}$.
\item $a\equiv b \pmod{m}$ and $d\mid m$, $d>0$ implies $a\equiv b \pmod{d}$
\item\label{equiv-upmod} $a\equiv b \pmod{m}$ implies $ac\equiv bc \pmod{mc}$ for $c>0$.
\end{enumerate}
\end{proposition}
\begin{proof}
  Let $a,b,c,d,m\in\Z$ with $m>0.$
  
\begin{enumerate}
 \item Assume $a\equiv b \pmod{m}$ and $b\equiv c \pmod{m}.$ Then using the second definition of equivalence, there exists $q_1,q_2,q_3,r\in\Z$ such that 
\begin{align*}
 a&=mq_1+r, \qquad 0\leq r<m,\\
 b&=mq_2+r, \qquad 0\leq r<m,\\
 c&=mq_3+r, \qquad 0\leq r<m. 
\end{align*}
Thus, $a$ and $c$ have the same remainder when divided by $m,$ so $a\equiv c\pmod m.$

\item[2/3.] Assume $a\equiv b \pmod{m}$ and $c\equiv d \pmod{m}.$ Then by the third definition of equivalence, there exists $j,k\in\Z$ such that $b=a+mj$ and $d=c+mk.$ Thus, 
\begin{align*}
 b+d&=a+c+m(j+k), \qquad\qquad\textnormal{and}\\
 bd &= ac+m(ak+cj+mjk).
\end{align*}
Thus, $a+c\equiv b+d\pmod m$ and $ac=bd\pmod m.$

\setcounter{enumi}{3}
\item Assume $a\equiv b \pmod{m},$ and $d>0$ with $d\mid m.$ From the first definition of equivalence modulo $m,$ $m\mid b-a$. Since division is transitive, $d\mid b-a,$ so $a\equiv b\pmod d.$

\item Assume $a\equiv b \pmod{m},$ and $c>0.$ From the third definition of equivalence modulo $m,$ there exists $k\in\Z$ such that $b=a+mk.$ Thus, $bc=ac+mck,$ so $ac\equiv bc \pmod{mc}.$ \qedhere
\end{enumerate}
\end{proof}

\begin{example}
Note that $2\equiv 5 \pmod 3$. Then $4\equiv 10 \pmod 3$ by Proposition \autoref{prop:equiv-arith} \ref{equiv-multiply}, since $2\equiv 2\pmod 3.$ From part \ref{equiv-upmod}, $4\equiv 10 \pmod 6,$ but $2\not\equiv 5\pmod 6.$ 
\end{example}


%%%%%%%%%%%%%%%%%%%%%%%%%
\subsection{Peer review Chapter 1 Exercise 83 (20 minutes)}
%%%%%%%%%%%%%%%%%%%%%%%%%%

%%%%%%%%%%%%%%%%%%%%%%%%%%
\section{Wednesday, February 14: More congruence facts}
%%%%%%%%%%%%%%%%%%%%%%%%%%
%%%%%%%%%%%%%%%%%%%%%%%%%%

\begin{obj}
 \item Prove that $\{0,1,\dots,m-1\}$ is a complete residue system modulo $m$.
\end{obj}
%%%%%%%%%%%%%%%%%%%%%%%%%%

%%%%%%%%%%%%%%%%%%%%%%%%%
\subsection{Basic facts of working modulo $m$ (35 minutes)}
%%%%%%%%%%%%%%%%%%%%%%%%%%

\begin{defn}[complete residue system]\label{defn:complete-residue}
 Let $a,m\in\Z$ with $m>0$. We call the set of all $b\in\Z$ such that $a\equiv b \pmod{m}$ the \emph{equivalence class of $a$.} A set of integers such that every integer is congruent modulo $m$ is called a \emph{complete residue system modulo $m$.}
\end{defn}

\begin{prop*}[Consequence 2.2, rephrased]\label{cor:mod-partition}
 Let $m$ be a positive integer. Then equivalence modulo $m$ partition the integers. That is, every integer is in exactly one equivalence class modulo $m$.
\end{prop*}
\begin{proof}
This is an immediate consequence of the fact that equivalence modulo $m$ is an equivalence relation.
\end{proof}
 
 Notice that this arguments also simplifies the proof the $\{0,1,\dots,m-1\}$ is a complete residue system modulo $m$.
\begin{prop*}[Proposition 2.3]\label{prop:complete-residue}
The set $\{0,1,\dots,m-1\}$ is a complete residue system modulo $m$.
\end{prop*}
\begin{proof}
Let $a,m\in\Z$ with $m>0$. By the \nameref{div-alg}, there exist unique $q,r\in\Z$ such that $a=qm+r$ with $0\leq r <m$. In fact, since $0\leq r<m,$ we know $r=0,1,\dots, m-2,$ or $m-1$. Therefore, every integer is in the equivalence class of $0,1,\dots, m-2$ or $m-1$ modulo $m$.
Since every integer is in exactly one equivalence class modulo $m$, and the remainder from the division algorithm is unique, it is not possible for $a$ to be equivalent to any other element of $\{0,1,\dots,m-1\}$. 
\end{proof}



\begin{br}
 Practice: addition and multiplication tables modulo $3,4,5,6,7$. I am adding $9$ to include an odd composite.
 
\begin{description}
 \item[Modulo $3$] 
 \begin{align*}
& \begin{array}{c||c|c|c}
 + & [0] & [1] & [2]\\ \hline\hline
 [0] & [0] & [1] & [2] \\ \hline
 [1] & [1] & [2] & [0] \\ \hline
 [2]  & [2] & [0]& [1]\\ \hline
\end{array}
& \begin{array}{c||c|c|c}
 \ast & [0] & [1] & [2]\\ \hline\hline
 [0] & [0] & [0] & [0] \\ \hline
 [1] & [0] & [1] & [2] \\ \hline
 [2] & [0] & [2]& [1]\\ \hline
\end{array}
\end{align*}

 \item[Modulo $4$] 
 \begin{align*}
& \begin{array}{c||c|c|c|c}
 + & [0] & [1] & [2] & [3]\\ \hline\hline
 [0] & [0] & [1] & [2] & [3]\\ \hline
 [1] & [1] & [2] & [3] & [0] \\ \hline
 [2] & [2] & [3] & [0] & [1]\\ \hline
 [3] & [3] & [0] & [1] & [2]\\ \hline
\end{array}
& \begin{array}{c||c|c|c|c}
 \ast & [0] & [1] & [2] & [3]\\ \hline\hline
 [0] & [0] & [0] & [0] & [0]\\ \hline
 [1] & [0] & [1] & [2] & [3]\\ \hline
 [2] & [0] & [2] & [0] & [2]\\ \hline
 [3] & [0] & [3] & [2] & [1]\\ \hline
\end{array}
\end{align*}


 \item[Modulo $5$] 
 \begin{align*}
& \begin{array}{c||c|c|c|c|c}
 + & [0] & [1] & [2] & [3] & [4]\\ \hline\hline
 [0] & [0] & [1] & [2] & [3] & [4]\\ \hline
 [1] & [1] & [2] & [3] & [4] & [0] \\ \hline
 [2] & [2] & [3] & [4] & [0] & [1]\\ \hline
 [3] & [3] & [4] & [0] & [1] & [0]\\ \hline
 [4] & [4] & [0] & [1] & [2] & [3]\\ \hline
\end{array}
& \begin{array}{c||c|c|c|c|c}
 \ast & [0] & [1] & [2] & [3] & [4]\\ \hline\hline
 [0] & [0] & [0] & [0] & [0] & [0]\\ \hline
 [1] & [0] & [1] & [2] & [3] & [4]\\ \hline
 [2] & [0] & [2] & [4] & [1] & [3]\\ \hline
 [3] & [0] & [3] & [1] & [4] & [2]\\ \hline
 [4] & [0] & [4] & [3] & [2] & [1]\\ \hline
\end{array}
\end{align*}

 \item[Modulo $6$] 
 \begin{align*}
& \begin{array}{c||c|c|c|c|c|c}
 + & [0] & [1] & [2] & [3] & [4] & [5]\\ \hline\hline
 [0] & [0] & [1] & [2] & [3] & [4] & [5]\\ \hline
 [1] & [1] & [2] & [3] & [4] & [5] & [0] \\ \hline
 [2] & [2] & [3] & [4] & [5] & [0] & [1]\\ \hline
 [3] & [3] & [4] & [5] & [0] & [1] & [2]\\ \hline
 [4] & [4] & [5] & [0] & [1] & [2] & [3]\\ \hline
 [5] & [5] & [0] & [1] & [2] & [3] & [4]\\ \hline
\end{array}
& \begin{array}{c||c|c|c|c|c|c}
 \ast & [0] & [1] & [2] & [3] & [4] & [5]\\ \hline\hline
 [0] & [0] & [0] & [0] & [0] & [0] & [0]\\ \hline
 [1] & [0] & [1] & [2] & [3] & [4] & [5]\\ \hline
 [2] & [0] & [2] & [4] & [0] & [2] & [4]\\ \hline
 [3] & [0] & [3] & [0] & [3] & [0] & [3]\\ \hline
 [4] & [0] & [4] & [2] & [0] & [4] & [2]\\ \hline
 [5] & [0] & [5] & [4] & [3] & [2] & [1]\\ \hline
\end{array}
\end{align*}

 \item[Modulo $7$] 
 \begin{align*}
& \begin{array}{c||c|c|c|c|c|c|c}
 + & [0] & [1] & [2] & [3] & [4] & [5] & [6] \\ \hline\hline
 [0] & [0] & [1] & [2] & [3] & [4] & [5] & [6]\\ \hline
 [1] & [1] & [2] & [3] & [4] & [5] & [6]  & [0] \\ \hline
 [2] & [2] & [3] & [4] & [5] & [6]  & [0] & [1]\\ \hline
 [3] & [3] & [4] & [5] & [6]  & [0] & [1] & [2]\\ \hline
 [4] & [4] & [5] & [6]  & [0] & [1] & [2] & [3]\\ \hline
 [5] & [5] & [6] & [0] & [1] & [2] & [3] & [4]\\ \hline
 [6] & [6] & [0] & [1] & [2] & [3] & [4] & [5]\\ \hline
\end{array}
& \begin{array}{c||c|c|c|c|c|c|c}
 \ast & [0] & [1] & [2] & [3] & [4] & [5] & [6] \\ \hline\hline
 [0] & [0] & [0] & [0] & [0] & [0] & [0] & [0]\\ \hline
 [1] & [0] & [1] & [2] & [3] & [4] & [5] & [6]\\ \hline
 [2] & [0] & [2] & [4] & [6] & [1] & [3] & [5]\\ \hline
 [3] & [0] & [3] & [6] & [2] & [5] & [1] & [4]\\ \hline
 [4] & [0] & [4] & [1] & [5] & [2] & [6] & [3]\\ \hline
 [5] & [0] & [5] & [3] & [1] & [6] & [4] & [2]\\ \hline
 [6] & [0] & [6] & [5] & [4] & [3] & [2] & [1]\\ \hline
\end{array}
\end{align*}


 \item[Modulo $8$] 
 \begin{align*}
& \begin{array}{c||c|c|c|c|c|c|c|c}
 + & [0] & [1] & [2] & [3] & [4] & [5] & [6] & [7]\\ \hline\hline
 [0] & [0] & [1] & [2] & [3] & [4] & [5] & [6] & [7]\\ \hline
 [1] & [1] & [2] & [3] & [4] & [5] & [6] & [7] & [0] \\ \hline
 [2] & [2] & [3] & [4] & [5] & [6] & [7] & [0] & [1]\\ \hline
 [3] & [3] & [4] & [5] & [6] & [7] & [0] & [1] & [2]\\ \hline
 [4] & [4] & [5] & [6] & [7] & [0] & [1] & [2] & [3]\\ \hline
 [5] & [5] & [6] & [7] & [0] & [1] & [2] & [3] & [4]\\ \hline
 [6] & [6] & [7] & [0] & [1] & [2] & [3] & [4] & [5]\\ \hline
 [7] & [7] & [0] & [1] & [2] & [3] & [4] & [5] & [6]\\ \hline
\end{array}
& \begin{array}{c||c|c|c|c|c|c|c|c}
 \ast & [0] & [1] & [2] & [3] & [4] & [5] & [6] & [7]\\ \hline\hline
 [0] & [0] & [0] & [0] & [0] & [0] & [0] & [0] & [0]\\ \hline
 [1] & [0] & [1] & [2] & [3] & [4] & [5] & [6] & [7] \\ \hline
 [2] & [0] & [2] & [4] & [6] & [0] & [2] & [4] & [6]\\ \hline
 [3] & [0] & [3] & [6] & [1] & [4] & [7] & [2] & [5]\\ \hline
 [4] & [0] & [4] & [0] & [4] & [0] & [4] & [0] & [4]\\ \hline
 [5] & [0] & [5] & [2] & [7] & [4] & [1] & [6] & [3]\\ \hline
 [6] & [0] & [6] & [4] & [2] & [0] & [6] & [4] & [2]\\ \hline
 [7] & [0] & [7] & [6] & [5] & [4] & [3] & [2]& [1]\\ \hline
\end{array}
\end{align*}


 \item[Modulo $9$] 
 \begin{align*}
& \begin{array}{c||c|c|c|c|c|c|c|c|c}
 + & [0] & [1] & [2] & [3] & [4] & [5] & [6] & [7] & [8]\\ \hline\hline
 [0] & [0] & [1] & [2] & [3] & [4] & [5] & [6] & [7] & [8]\\ \hline
 [1] & [1] & [2] & [3] & [4] & [5] & [6] & [7] & [8] & [0] \\ \hline
 [2] & [2] & [3] & [4] & [5] & [6] & [7] & [8] & [0] & [1]\\ \hline
 [3] & [3] & [4] & [5] & [6] & [7] & [8] & [0] & [1] & [2]\\ \hline
 [4] & [4] & [5] & [6] & [7] & [8] & [0] & [1] & [2] & [3]\\ \hline
 [5] & [5] & [6] & [7] & [8] & [0] & [1] & [2] & [3] & [4]\\ \hline
 [6] & [6] & [7] & [8] & [0] & [1] & [2] & [3] & [4] & [5]\\ \hline
 [7] & [7] & [8] & [0] & [1] & [2] & [3] & [4] & [5] & [6]\\ \hline
 [8] & [8] & [0] & [1] & [2] & [3] & [4] & [5] & [6] & [7]\\ \hline
\end{array}
& \begin{array}{c||c|c|c|c|c|c|c|c|c}
 \ast & [0] & [1] & [2] & [3] & [4] & [5] & [6] & [7] & [8]\\ \hline\hline
 [0] & [0] & [0] & [0] & [0] & [0] & [0] & [0] & [0] & [0]\\ \hline
 [1] & [0] & [1] & [2] & [3] & [4] & [5] & [6] & [7] & [8]\\ \hline
 [2] & [0] & [2] & [4] & [6] & [0] & [1] & [3] & [5] & [7] \\ \hline
 [3] & [0] & [3] & [6] & [0] & [3] & [6] & [0] & [3] & [6]\\ \hline
 [4] & [0] & [4] & [8] & [3] & [7] & [2] & [6] & [1] & [5]\\ \hline
 [5] & [0] & [5] & [1] & [6] & [2] & [7] & [3] & [8] & [4]\\ \hline
 [6] & [0] & [6] & [3] & [0] & [6] & [3] & [0] & [6] & [3]\\ \hline
 [7] & [0] & [7] & [5] & [3] & [1] & [8] & [6] & [4] & [2]\\ \hline
 [8] & [0] & [8] & [7] & [6] & [5] & [4] & [3] & [2]& [1]\\ \hline
\end{array}
\end{align*}
\end{description}
\end{br}


%%%%%%%%%%%%%%%%%%%%%%%%%%
\section{Friday, February 16: Linear congruences in one variable}
%%%%%%%%%%%%%%%%%%%%%%%%%%
%%%%%%%%%%%%%%%%%%%%%%%%%%
\begin{obj}
	\item Prove when a linear congruence in one variable has a solution
	\item Find all solutions to a linear congruence given a particular solution
	\item Find the number of incongruent solutions to a linear congruence
\end{obj}

\begin{pre} \item Paper 1 due
\end{pre}
%%%%%%%%%%%%%%%%%%%%%%%%%%
\subsection{Quiz (10 min)}
%%%%%%%%%%%%%%%%%%%%%%%%%%

%%%%%%%%%%%%%%%%%%%%%%%%%%
\subsection{Linear congruences in one variable (40 min)}
%%%%%%%%%%%%%%%%%%%%%%%%%%
\begin{rem}\label{remark-add-inverse}
Let $a,b,m\in\Z$ with $m>0$. Every row/column of addition modulo $m$ contains $\{0,1,\dots,m-1\}$. 

We can also say that $a+x\equiv b\pmod m$ always has a solution, since $x\equiv b-a\pmod m$.
\end{rem}

\begin{thm*}[Strayer Theorem 2.6 and Porism 2.7]\label{thm:lin-cong-solutions}
Let $a,b,m\in\Z$ with $m>0,$ and $d=(a,m)$. The linear congruence in one variable $ax\equiv b\pmod m$ has a solution if and only if $d\mid b$. When $d\mid b,$ there are exactly $d$ incongruent solutions modulo $m$ corresponding to the congruence classes \[x_0, x_0+\frac{m}{d},\dots,x_0+\frac{(d-1)m}{d} \pmod m.\]
\end{thm*}
\begin{proof}
 Let $a,b,m\in\Z$ with $m>0$, and $d=(a,m)$. From the definition of congruence modulo $m$, $ax\equiv b \pmod m$ if and only if $m\mid (ax-b)$. That is, $ax\equiv b \pmod m$ if and only if $my=ax-b$ for some $y\in\Z$ from the definition of divisibility. Sine $ax-my=b$ is a linear Diophantine equation, \nameref{thm:linear-dioph} says solutions exist if and only if $(a,-m)=d\mid b.$
 
In the case that solutions exist, let $x_0, y_0$ be a particular solution to the linear Diophantine equation. Then $x_0$ is also a solution to the linear congruence in one variable, since $a x_0-my_0=b,$ implies $ax_0\equiv b\pmod m$. From \nameref{thm:linear-dioph}, all solutions have the from $x=x_0+\frac{mn}{d}$ for all $n\in\Z.$  We need to show that these solutions are in exactly $d$ distinct congruence classes modulo $m.$
 
Consider the solutions $x_0+\frac{mi}{d}$ and $x_0+\frac{mj}{d}$ for some integers $i$ and $j.$ Then $x_0+\frac{mj}{d}\equiv x_0+\frac{mk}{d} \pmod m$ if and only if $m\mid \left(\frac{mi}{d}-\frac{mj}{d}\right).$ That is, if and only if there exists $k\in\Z$ such that $mk=\frac{mi}{d}-\frac{mj}{d}$. Rearranging this equation, we get that $x_0+\frac{mj}{d}\equiv x_0+\frac{mk}{d} \pmod m$ if and only if $dk=i-j.$ Thus, $i\equiv j\pmod$ by definition of equivalence modulo $d.$ Thus, the incongruent solutions to $ax\equiv b\pmod m$ are the congruence classes \[x_0, x_0+\frac{m}{d},\dots,x_0+\frac{(d-1)m}{d} \pmod m.\qedhere\]
\end{proof}

\begin{example} Let's consider several linear congruences modulo $12.$
\begin{itemize}
\item The linear congruence $2x\equiv 1\pmod{12}$ has no solutions, since $2\nmid1.$
\item The linear congruence $8x\equiv b \pmod{12}$ has a solution if and only if $4\mid b$. Considering the least nonnegative residues, the options for $b$ are:
	\begin{itemize}
 	\item $8x\equiv 0\pmod{12}.$ The incongruent solutions are $0,3,6,9\pmod{12}.$
	\item $8x\equiv 4\pmod{12}.$ The incongruent solutions are $2,5,8,11\pmod{12}.$ Notice we cannot divide across the equivalence, since $2x\equiv 1\pmod{12}$ has no solutions. 
	\item $8x\equiv 8\pmod{12}.$ The incongruent solutions are $1,4,7,10\pmod{12}.$ 
	\end{itemize}
 \item The linear congruence $5x\equiv 1 \pmod{12}$ has solution $x\equiv 5\pmod{12}.$ Since $(5,12)=1,$ the solution is unique.
 \item The linear congruence $5x\equiv 7 \pmod{12}$ has solution $x\equiv -1\equiv 11\pmod{12}.$ Since $(5,12)=1,$ the solution is unique. Note that instead of $12+5(-1)=7,$ we could have done \[5(5x)\equiv 5(7)\equiv 11\pmod{12}.\]
\end{itemize}
\end{example}


\end{document}