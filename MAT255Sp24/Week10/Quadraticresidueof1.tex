\documentclass{ximera}
\usepackage{amssymb, latexsym, amsmath, amsthm, graphicx, amsthm,alltt,color, listings,multicol,xr-hyper,hyperref,aliascnt,enumitem}
\usepackage{xfrac}

\usepackage{parskip}
\usepackage[,margin=0.7in]{geometry}
\setlength{\textheight}{8.5in}

\usepackage{epstopdf}

\DeclareGraphicsExtensions{.eps}
\usepackage{tikz}


\usepackage{tkz-euclide}
%\usetkzobj{all}
\tikzstyle geometryDiagrams=[rounded corners=.5pt,ultra thick,color=black]
\colorlet{penColor}{black} % Color of a curve in a plot


\usepackage{subcaption}
\usepackage{float}
\usepackage{fancyhdr}
\usepackage{pdfpages}
\newcounter{includepdfpage}
\usepackage{makecell}


\usepackage{currfile}
\usepackage{xstring}




\graphicspath{  
{./otherDocuments/}
}

\author{Claire Merriman}
\newcommand{\classday}[1]{\def\classday{#1}}

%%%%%%%%%%%%%%%%%%%%%
% Counters and autoref for unnumbered environments
% Not needed??
%%%%%%%%%%%%%%%%%%%%%
\theoremstyle{plain}


\newtheorem*{namedthm}{Theorem}
\newcounter{thm}%makes pointer correct
\providecommand{\thmname}{Theorem}

\makeatletter
\NewDocumentEnvironment{thm*}{o}
 {%
  \IfValueTF{#1}
    {\namedthm[#1]\refstepcounter{thm}\def\@currentlabel{(#1)}}%
    {\namedthm}%
 }
 {%
  \endnamedthm
 }
\makeatother


\newtheorem*{namedprop}{Proposition}
\newcounter{prop}%makes pointer correct
\providecommand{\propname}{Proposition}

\makeatletter
\NewDocumentEnvironment{prop*}{o}
 {%
  \IfValueTF{#1}
    {\namedprop[#1]\refstepcounter{prop}\def\@currentlabel{(#1)}}%
    {\namedprop}%
 }
 {%
  \endnamedprop
 }
\makeatother

\newtheorem*{namedlem}{Lemma}
\newcounter{lem}%makes pointer correct
\providecommand{\lemname}{Lemma}

\makeatletter
\NewDocumentEnvironment{lem*}{o}
 {%
  \IfValueTF{#1}
    {\namedlem[#1]\refstepcounter{lem}\def\@currentlabel{(#1)}}%
    {\namedlem}%
 }
 {%
  \endnamedlem
 }
\makeatother

\newtheorem*{namedcor}{Corollary}
\newcounter{cor}%makes pointer correct
\providecommand{\corname}{Corollary}

\makeatletter
\NewDocumentEnvironment{cor*}{o}
 {%
  \IfValueTF{#1}
    {\namedcor[#1]\refstepcounter{cor}\def\@currentlabel{(#1)}}%
    {\namedcor}%
 }
 {%
  \endnamedcor
 }
\makeatother

\theoremstyle{definition}
\newtheorem*{annotation}{Annotation}
\newtheorem*{rubric}{Rubric}

\newtheorem*{innerrem}{Remark}
\newcounter{rem}%makes pointer correct
\providecommand{\remname}{Remark}

\makeatletter
\NewDocumentEnvironment{rem}{o}
 {%
  \IfValueTF{#1}
    {\innerrem[#1]\refstepcounter{rem}\def\@currentlabel{(#1)}}%
    {\innerrem}%
 }
 {%
  \endinnerrem
 }
\makeatother

\newtheorem*{innerdefn}{Definition}%%placeholder
\newcounter{defn}%makes pointer correct
\providecommand{\defnname}{Definition}

\makeatletter
\NewDocumentEnvironment{defn}{o}
 {%
  \IfValueTF{#1}
    {\innerdefn[#1]\refstepcounter{defn}\def\@currentlabel{(#1)}}%
    {\innerdefn}%
 }
 {%
  \endinnerdefn
 }
\makeatother

\newtheorem*{scratch}{Scratch Work}


\newtheorem*{namedconj}{Conjecture}
\newcounter{conj}%makes pointer correct
\providecommand{\conjname}{Conjecture}
\makeatletter
\NewDocumentEnvironment{conj}{o}
 {%
  \IfValueTF{#1}
    {\innerconj[#1]\refstepcounter{conj}\def\@currentlabel{(#1)}}%
    {\innerconj}%
 }
 {%
  \endinnerconj
 }
\makeatother

\newtheorem*{poll}{Poll question}
\newtheorem{tps}{Think-Pair-Share}[section]


\newenvironment{obj}{
	\textbf{Learning Objectives.} By the end of class, students will be able to:
		\begin{itemize}}
		{\!.\end{itemize}
		}

\newenvironment{pre}{
	\begin{description}
	}{
	\end{description}
}


\newcounter{ex}%makes pointer correct
\providecommand{\exname}{Homework Problem}
\newenvironment{ex}[1][2in]%
{%Env start code
\problemEnvironmentStart{#1}{Homework Problem}
\refstepcounter{ex}
}
{%Env end code
\problemEnvironmentEnd
}

\newcommand{\inlineAnswer}[2][2 cm]{
    \ifhandout{\pdfOnly{\rule{#1}{0.4pt}}}
    \else{\answer{#2}}
    \fi
}


\ifhandout
\newenvironment{shortAnswer}[1][
    \vfill]
        {% Begin then result
        #1
            \begin{freeResponse}
            }
    {% Environment Ending Code
    \end{freeResponse}
    }
\else
\newenvironment{shortAnswer}[1][]
        {\begin{freeResponse}
            }
    {% Environment Ending Code
    \end{freeResponse}
    }
\fi

\let\question\relax
\let\endquestion\relax

\newtheoremstyle{ExerciseStyle}{\topsep}{\topsep}%%% space between body and thm
		{}                      %%% Thm body font
		{}                              %%% Indent amount (empty = no indent)
		{\bfseries}            %%% Thm head font
		{}                              %%% Punctuation after thm head
		{3em}                           %%% Space after thm head
		{{#1}~\thmnumber{#2}\thmnote{ \bfseries(#3)}}%%% Thm head spec
\theoremstyle{ExerciseStyle}
\newtheorem{br}{In-class Problem}

\newenvironment{sketch}
 {\begin{proof}[Sketch of Proof]}
 {\end{proof}}


\newcommand{\gt}{>}
\newcommand{\lt}{<}
\newcommand{\N}{\mathbb N}
\newcommand{\Q}{\mathbb Q}
\newcommand{\Z}{\mathbb Z}
\newcommand{\C}{\mathbb C}
\newcommand{\R}{\mathbb R}
\renewcommand{\H}{\mathbb{H}}
\newcommand{\lcm}{\operatorname{lcm}}
\newcommand{\nequiv}{\not\equiv}
\newcommand{\ord}{\operatorname{ord}}
\newcommand{\ds}{\displaystyle}
\newcommand{\floor}[1]{\left\lfloor #1\right\rfloor}
\newcommand{\legendre}[2]{\left(\frac{#1}{#2}\right)}



%%%%%%%%%%%%



\title{Quadratic residue of $-1$}
\begin{document}
\begin{abstract}
\end{abstract}
\maketitle

%%%%%%%%%%%%%%%%%%%%%%%%%%

\begin{obj}
    \item Prove \nameref{thm:euler-quads}
	\item Classify when $-1$ is a quadratic residue modulo an odd prime
\end{obj}


\begin{pre}
    \item[Reading] None
\end{pre}

%%%%%%%%%%%%%%%%%%%%%%%%%%
\subsection{Proof of Euler's Criterion}
%%%%%%%%%%%%%%%%%%%%%%%%%%

We will prove \nameref{thm:euler-quads}. 

\begin{theorem}[Euler's Criterion]
    Let $p$ be an odd prime and $a\in\Z$ with $p\nmid a.$ Then \[\legendre{a}{p}\equiv a^{(p-1)/2}\pmod{p}\]
\end{theorem}

\begin{proof}
	Let $p$ be an odd prime and $a\in\Z$ with $p\nmid a.$ If there exists $b\in\Z$ such that $b^2\equiv a\pmod{p},$ then $\legendre{a}{p}=1$ by \hyperref[defn:legendre]{definition}.
	Note that \[a^{(p-1)/2}\equiv (b^2){(p-1)/2}\equiv b^{p-1}\equiv 1\pmod{p}\]
	by \nameref{FlT}. Thus $\legendre{a}{p}\equiv a^{(p-1)/2}\pmod{p}.$
	
	If $a$ is a quadratic nonresidue modulo $p,$ consider the \hyperref[defn:reduced-res-sys]{reduced residue system} $\{1,2,\dots,p-1\}.$ For each element $c$ of the list, there exists a unique element $d$, also on the list, such that $cd\equiv a\pmod{p}$ by \hyperref[thm:lin-cong-solutions]{Theorem 2.6} since $(a,p)=1$. Since $a$ is a quadratic nonresidue by assumption, $c\not\equiv d\pmod{p}.$ Thus, there are $\frac{p-1}{2}$ pairs $c,d$ where $cd\equiv a\pmod{p}.$ Thus, 
		\[
			-1\equiv (p-1)! \equiv a^{(p-1)/2}\pmod{p}
		\]
	by \nameref{Wilson}. Since $a$ is a quadratic nonresidue modulo $p,$ $\legendre{a}{p}=-1\equiv a^{(p-1)/2}\pmod{p}.$
\end{proof}

\begin{remark}
	Some sources define $\legendre{a}{p}=0$ when $p\mid a.$ In this case,  Let $p$ be an odd prime and $a\in\Z$.
	If $p\mid a,$ then $a^{(p-1)/2}\equiv 0^{(p-1)/2}\equiv 0\equiv\legendre{a}{p}\pmod{p}.$
\end{remark}

%%%%%%%%%%%%%%%%%%%%%%%%%%
\subsection{When is $-1$ a quadratic residue?}
%%%%%%%%%%%%%%%%%%%%%%%%%%

\begin{theorem}[Theorem 4.6]\label{thm:residue-neg1}
	Let $p$ be an odd prime number. Then 
	\[
		\legendre{-1}{p}=
			\begin{cases}
 				1, & p\equiv 1\pmod{4}\\
				-1, & p\equiv 3\pmod{4}
			\end{cases}.
	\]
\end{theorem}

\begin{proof}
	Let $p$ be an odd prime number. Then from \nameref{thm:euler-quads}, $\legendre{-1}{p}\equiv (-1)^{(p-1)/2}\pmod{p}.$ Since both values are $\pm1,$ we can say $\legendre{-1}{p}=(-1)^{(p-1)/2}.$

	If $p\equiv 1\pmod{4},$ then there exists $k\in\Z$ such that $p=4k+1.$ Thus, $\frac{p-1}{2}=2k$ and 
		\[
			\legendre{-1}{p}=(-1)^{(p-1)/2}=(-1)^{2k}=1.
		\]
	
	If $p\equiv 3\pmod{4},$ then there exists $k\in\Z$ such that $p=4k+3.$ Thus, $\frac{p-1}{2}=2k+1$ and 
		\[
			\legendre{-1}{p}=(-1)^{(p-1)/2}=(-1)^{2k+1}=-1.
		\]
\end{proof}
%%%%%%%%%%%%%%%%%%%%%%%%%%


\end{document}
