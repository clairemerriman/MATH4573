\documentclass{ximera}
\usepackage{amssymb, latexsym, amsmath, amsthm, graphicx, amsthm,alltt,color, listings,multicol,xr-hyper,hyperref,aliascnt,enumitem}
\usepackage{xfrac}

\usepackage{parskip}
\usepackage[,margin=0.7in]{geometry}
\setlength{\textheight}{8.5in}

\usepackage{epstopdf}

\DeclareGraphicsExtensions{.eps}
\usepackage{tikz}


\usepackage{tkz-euclide}
%\usetkzobj{all}
\tikzstyle geometryDiagrams=[rounded corners=.5pt,ultra thick,color=black]
\colorlet{penColor}{black} % Color of a curve in a plot


\usepackage{subcaption}
\usepackage{float}
\usepackage{fancyhdr}
\usepackage{pdfpages}
\newcounter{includepdfpage}
\usepackage{makecell}


\usepackage{currfile}
\usepackage{xstring}




\graphicspath{  
{./otherDocuments/}
}

\author{Claire Merriman}
\newcommand{\classday}[1]{\def\classday{#1}}

%%%%%%%%%%%%%%%%%%%%%
% Counters and autoref for unnumbered environments
% Not needed??
%%%%%%%%%%%%%%%%%%%%%
\theoremstyle{plain}


\newtheorem*{namedthm}{Theorem}
\newcounter{thm}%makes pointer correct
\providecommand{\thmname}{Theorem}

\makeatletter
\NewDocumentEnvironment{thm*}{o}
 {%
  \IfValueTF{#1}
    {\namedthm[#1]\refstepcounter{thm}\def\@currentlabel{(#1)}}%
    {\namedthm}%
 }
 {%
  \endnamedthm
 }
\makeatother


\newtheorem*{namedprop}{Proposition}
\newcounter{prop}%makes pointer correct
\providecommand{\propname}{Proposition}

\makeatletter
\NewDocumentEnvironment{prop*}{o}
 {%
  \IfValueTF{#1}
    {\namedprop[#1]\refstepcounter{prop}\def\@currentlabel{(#1)}}%
    {\namedprop}%
 }
 {%
  \endnamedprop
 }
\makeatother

\newtheorem*{namedlem}{Lemma}
\newcounter{lem}%makes pointer correct
\providecommand{\lemname}{Lemma}

\makeatletter
\NewDocumentEnvironment{lem*}{o}
 {%
  \IfValueTF{#1}
    {\namedlem[#1]\refstepcounter{lem}\def\@currentlabel{(#1)}}%
    {\namedlem}%
 }
 {%
  \endnamedlem
 }
\makeatother

\newtheorem*{namedcor}{Corollary}
\newcounter{cor}%makes pointer correct
\providecommand{\corname}{Corollary}

\makeatletter
\NewDocumentEnvironment{cor*}{o}
 {%
  \IfValueTF{#1}
    {\namedcor[#1]\refstepcounter{cor}\def\@currentlabel{(#1)}}%
    {\namedcor}%
 }
 {%
  \endnamedcor
 }
\makeatother

\theoremstyle{definition}
\newtheorem*{annotation}{Annotation}
\newtheorem*{rubric}{Rubric}

\newtheorem*{innerrem}{Remark}
\newcounter{rem}%makes pointer correct
\providecommand{\remname}{Remark}

\makeatletter
\NewDocumentEnvironment{rem}{o}
 {%
  \IfValueTF{#1}
    {\innerrem[#1]\refstepcounter{rem}\def\@currentlabel{(#1)}}%
    {\innerrem}%
 }
 {%
  \endinnerrem
 }
\makeatother

\newtheorem*{innerdefn}{Definition}%%placeholder
\newcounter{defn}%makes pointer correct
\providecommand{\defnname}{Definition}

\makeatletter
\NewDocumentEnvironment{defn}{o}
 {%
  \IfValueTF{#1}
    {\innerdefn[#1]\refstepcounter{defn}\def\@currentlabel{(#1)}}%
    {\innerdefn}%
 }
 {%
  \endinnerdefn
 }
\makeatother

\newtheorem*{scratch}{Scratch Work}


\newtheorem*{namedconj}{Conjecture}
\newcounter{conj}%makes pointer correct
\providecommand{\conjname}{Conjecture}
\makeatletter
\NewDocumentEnvironment{conj}{o}
 {%
  \IfValueTF{#1}
    {\innerconj[#1]\refstepcounter{conj}\def\@currentlabel{(#1)}}%
    {\innerconj}%
 }
 {%
  \endinnerconj
 }
\makeatother

\newtheorem*{poll}{Poll question}
\newtheorem{tps}{Think-Pair-Share}[section]


\newenvironment{obj}{
	\textbf{Learning Objectives.} By the end of class, students will be able to:
		\begin{itemize}}
		{\!.\end{itemize}
		}

\newenvironment{pre}{
	\begin{description}
	}{
	\end{description}
}


\newcounter{ex}%makes pointer correct
\providecommand{\exname}{Homework Problem}
\newenvironment{ex}[1][2in]%
{%Env start code
\problemEnvironmentStart{#1}{Homework Problem}
\refstepcounter{ex}
}
{%Env end code
\problemEnvironmentEnd
}

\newcommand{\inlineAnswer}[2][2 cm]{
    \ifhandout{\pdfOnly{\rule{#1}{0.4pt}}}
    \else{\answer{#2}}
    \fi
}


\ifhandout
\newenvironment{shortAnswer}[1][
    \vfill]
        {% Begin then result
        #1
            \begin{freeResponse}
            }
    {% Environment Ending Code
    \end{freeResponse}
    }
\else
\newenvironment{shortAnswer}[1][]
        {\begin{freeResponse}
            }
    {% Environment Ending Code
    \end{freeResponse}
    }
\fi

\let\question\relax
\let\endquestion\relax

\newtheoremstyle{ExerciseStyle}{\topsep}{\topsep}%%% space between body and thm
		{}                      %%% Thm body font
		{}                              %%% Indent amount (empty = no indent)
		{\bfseries}            %%% Thm head font
		{}                              %%% Punctuation after thm head
		{3em}                           %%% Space after thm head
		{{#1}~\thmnumber{#2}\thmnote{ \bfseries(#3)}}%%% Thm head spec
\theoremstyle{ExerciseStyle}
\newtheorem{br}{In-class Problem}

\newenvironment{sketch}
 {\begin{proof}[Sketch of Proof]}
 {\end{proof}}


\newcommand{\gt}{>}
\newcommand{\lt}{<}
\newcommand{\N}{\mathbb N}
\newcommand{\Q}{\mathbb Q}
\newcommand{\Z}{\mathbb Z}
\newcommand{\C}{\mathbb C}
\newcommand{\R}{\mathbb R}
\renewcommand{\H}{\mathbb{H}}
\newcommand{\lcm}{\operatorname{lcm}}
\newcommand{\nequiv}{\not\equiv}
\newcommand{\ord}{\operatorname{ord}}
\newcommand{\ds}{\displaystyle}
\newcommand{\floor}[1]{\left\lfloor #1\right\rfloor}
\newcommand{\legendre}[2]{\left(\frac{#1}{#2}\right)}



%%%%%%%%%%%%



\title{Introduction to quadratic reciprocity}
\begin{document}
\begin{abstract}
\end{abstract}
\maketitle

%%%%%%%%%%%%%%%%%%%%%%%%%%

\begin{obj}
    \item Use quadratic reciprocity to find the Legendre symbol of an integer modulo $p$.
    \item Characterize the primes where $3,-3,5,-5,7,-7$ are quadratic residues.
\end{obj}


\begin{pre}
    \item[Reading:] Scanned notes, Pommersheim-Marks-Flapan Section 11.2 and part of 11.3.
    \item[Turn in:] \emph{Use the results from the reading} to find $\legendre{-3}{71}.$

     \begin{solution}
     	From Theorem 4.5\ref{legendre-mult}, $\legendre{-3}{71}=\legendre{-1}{71}\legendre{3}{71}.$
     	Since $71\equiv3\pmod{4},$ from \nameref{thm:residue-neg1}, $\legendre{-1}{71}=-1,$
	and from \hyperref[quad-rec-useful-form]{quadratic reciprocity}$\legendre{3}{71}=-\legendre{71}{3}.$ Putting this together, we get
	
		\begin{align*}
		 	\legendre{-3}{71}&=\legendre{-1}{71}\legendre{3}{71}=(-1)\left(-\legendre{71}{3}\right)\\
				&=\legendre{71}{3}=\legendre{2}{3}=-1
		\end{align*}
	from Theorem 4.5\ref{legendre-respects-mod} and the fact that $2$ is a quadratic nonresidue modulo $3.$
     \end{solution}
\end{pre}


%%%%%%%%%%%%%%%%%%%%%%%%%%
\subsection{Quadratic reciprocity (20 minutes)}
%%%%%%%%%%%%%%%%%%%%%%%%%%

\begin{theorem}[Quadratic Reciprocity (first statement)]\label{quad-rec-useful-form}
	Let $p$ and $q$ be distinct primes.  
	\begin{enumerate}[label=(\alph*)]
		\item If $p\equiv 1 \pmod{4}$ or $q\equiv 1\pmod{4},$ then $\legendre{p}{q}=\legendre{q}{p}.$
 		\item If $p\equiv q \equiv 3 \pmod{4},$ then $\legendre{p}{q}=-\legendre{q}{p}.$
	\end{enumerate}
\end{theorem}

It will take time to prove the lemmas for quadratic reciprocity, but we can immediately prove some results. 
The first is that this initial statement is equivalent to the more standard statement. This is our first example where one statement of a result (\nameref{quad-rec-useful-form}) is more useful in calculations and proofs, but the proof involves proving an equivalent statement (\nameref{quad-rec-standard-form}).

\begin{theorem}[Law of Quadratic Reciprocity]\label{quad-rec-standard-form} (Theorem 4.9)
	Let $p$ and $q$ be distinct primes.  Then
	\[
		\legendre{p}{q}\legendre{q}{p}=(-1)^{(p-1)/2\ (q-1)/2}=
			\begin{cases}
 				1, & \textnormal{ if } p\equiv 1\pmod{4} \textnormal{ or } q\equiv 1\pmod{4}\\
				-1, & \textnormal{ if } p\equiv q\equiv 3\pmod{4}.
			\end{cases}
	\]
\end{theorem}

\begin{remark}
	We can quickly prove the second equality, 
		\[(-1)^{(p-1)/2\ (q-1)/2}=
			\begin{cases}
 				1, & \textnormal{ if } p\equiv 1\pmod{4} \textnormal{ or } q\equiv 1\pmod{4}\\
				-1, & \textnormal{ if } p\equiv q\equiv 3\pmod{4}.
			\end{cases}\]
	If $p\equiv 1\pmod{4}$ or $q\equiv 1\pmod{4},$ then one of $\frac{p-1}{2}$ and $\frac{q-1}{2}$ is even. Thus $(-1)^{(p-1)/2\ (q-1)/2}=1.$ If $p\equiv q\equiv 3\pmod{4},$ then $\frac{p-1}{2}$ and $\frac{q-1}{2}$ are both odd. Thus $(-1)^{(p-1)/2\ (q-1)/2}=-1.$
\end{remark}

\begin{proposition}
 \nameref{quad-rec-useful-form} is equivalent to \nameref{quad-rec-standard-form}. That is, \nameref{quad-rec-useful-form} implies \nameref{quad-rec-standard-form} and \nameref{quad-rec-standard-form} implies \nameref{quad-rec-useful-form} .
\end{proposition}
\begin{proof}
	We will first show that \nameref{quad-rec-standard-form} implies \nameref{quad-rec-useful-form}.
	
	Let $p$ and $q$ be distinct primes.  Then \nameref{quad-rec-standard-form} says that 
	\[
		\legendre{p}{q}\legendre{q}{p}=
			\begin{cases}
 				1, & \textnormal{ if } p\equiv 1\pmod{4} \textnormal{ or } q\equiv 1\pmod{4}\\
				-1, & \textnormal{ if } p\equiv q\equiv 3\pmod{4}.
			\end{cases}
	\]
	Since $\legendre{p}{q}=\pm 1$ and $\legendre{q}{p}=\pm 1,$ we know that $\legendre{p}{q}\legendre{q}{p}=1$ if and only if $\legendre{p}{q}=\legendre{q}{p}.$ Thus, when $p\equiv 1\pmod{4}$ or $q\equiv 1\pmod{4},$ $\legendre{p}{q}=\legendre{q}{p}.$ Similarly, $\legendre{p}{q}\legendre{q}{p}=-1$ if and only if $\legendre{p}{q}=-\legendre{q}{p}.$ Thus, when $p\equiv q\equiv 3\pmod{4},$ $\legendre{p}{q}=-\legendre{q}{p}.$ 
	
	The other direction is on next homework assignment.
\end{proof}

Notice that we also proved the converse of \nameref{quad-rec-useful-form}:
\begin{corollary}\label{cor:quad-rec}
	Let $p$ and $q$ be distinct primes.  
	\begin{enumerate}[label=(\alph*)]
		\item If $\legendre{p}{q}=\legendre{q}{p},$ then $p\equiv 1 \pmod{4}$ or $q\equiv 1\pmod{4}.$
 		\item If $\legendre{p}{q}=-\legendre{q}{p},$ then $p\equiv q \equiv 3 \pmod{4}.$
	\end{enumerate}
\end{corollary}


%%%%%%%%%%%%%%%%%%%%%%%%%%
\subsection{Quadratic reciprocity practice (30 minutes)}
%%%%%%%%%%%%%%%%%%%%%%%%%%

\begin{br}[Strayer Chapter 4, Exercise 35]
	Let $p$ be an odd prime number. Prove the following statements the following provided outlines, which will help solve the next problem, as well.
	
	\begin{enumerate}[label=(\alph*)]
	\setcounter{enumi}{1}
		\item $\legendre{3}{p}=1$ if and only if $p\equiv \pm1\pmod{12}.$
		\item $\legendre{-3}{p}=1$ if and only if $p\equiv 1\pmod{6}.$	
	\end{enumerate}
\end{br}

\begin{proof}
	\begin{enumerate}[label=(\alph*)]
	\setcounter{enumi}{1}
		\item Since $3\equiv \answer{3}\pmod{4},$\footnote{In this problem, this step is repetitive, but it is needed when $p\neq3$.} we need two cases for \nameref{quad-rec-useful-form}. 
			\begin{enumerate}[label=(\roman*)]
 				\item If $p\equiv 1\pmod{4},$ then $\legendre{3}{p}=\answer{\legendre{p}{3}}$ by \nameref{quad-rec-useful-form}, and $\legendre{p}{3}=1$ if and only if $p\equiv\answer{1\pmod{3}}$. Then $p\equiv \answer{1}\pmod{12},$ and this is the unique equivalence class modulo $12$ by the Chinese Remainder Theorem.
		
				\item If $p\equiv 3\equiv -1\pmod{4},$ then $\legendre{3}{p}=\answer{-\legendre{p}{3}}$ by \nameref{quad-rec-useful-form}, and $\legendre{p}{3}=-1$ if and only if $p\equiv\answer{ 2\equiv-1\pmod{3}}$. Then $p\equiv \answer{-1}\pmod{12},$ and this is the unique equivalence class modulo $12$ by the Chinese Remainder Theorem.
			\end{enumerate}
		Therefore, $\legendre{3}{p}=1$ if and only if $p\equiv \pm1\pmod{12}.$ 
		
		\item From Theorem 4.25\ref{legendre-mult}, $\legendre{-3}{p}= \answer{\legendre{-1}{p}\legendre{3}{p}}.$
		Again, we have two cases.
			\begin{enumerate}[label=(\roman*)]
 				\item If $p\equiv 1\pmod{4},$ then $\legendre{-1}{p}=\answer{1}$ by \nameref{thm:residue-neg1} and
				$\legendre{3}{p}=\answer{\legendre{p}{3}}$ by \nameref{quad-rec-useful-form}. Thus, $\legendre{-3}{p}=\answer{\legendre{p}{3}}=1$ if and only if $p\equiv \answer{1\pmod{3}}.$ Then $p\equiv \answer{1}\pmod{12},$ and this is the unique equivalence class modulo $12$ by the Chinese Remainder Theorem.
		
		\item If $p\equiv 3\equiv -1\pmod{4},$ then $\legendre{-1}{p}=\answer{-1}$ by \nameref{thm:residue-neg1} and
				$\legendre{3}{p}=\answer{-\legendre{p}{3}}$ by \nameref{quad-rec-useful-form}. Thus, $\legendre{-3}{p}=\answer{\legendre{p}{3}}=1$ if and only if $p\equiv \answer{1\pmod{3}}.$ Then $p\equiv \answer{7}\pmod{12},$ and this is the unique equivalence class modulo $12$ by the Chinese Remainder Theorem.
	\end{enumerate}
	Therefore, $\legendre{-3}{p}=1$ if and only if $p\equiv \answer{1,7}\pmod{12},$ which is equivalent to $p\equiv 1\pmod{6}.$ \qedhere
	\end{enumerate}
\end{proof}

\begin{br}[Strayer Chapter 4, Exercise 36]
	Find congruences characterizing all prime numbers $p$ for which the following integers are quadratic residues modulo $p,$ as done in the previous exercise. 
	
	Outline is provided for the first part. Outline for second part added after class.
	\begin{enumerate}[label=(\alph*)]
		\item $5$ 
		\item $-5$	
		\item $7$ 
		\item $-7$	
	\end{enumerate}
\end{br}

\begin{solution}
 	\begin{enumerate}[label=(\alph*)]
		\item Since $5\equiv\answer{1}\pmod{4},$ $\answer{\legendre{5}{p}=\legendre{p}{5}}$ by \nameref{quad-rec-useful-form}. Then $\legendre{5}{p}=\answer{\legendre{p}{5}}=1$ if and only if $p\equiv\answer{ \pm1\pmod{5}}$.

		\item From Theorem 4.25\ref{legendre-mult}, $\legendre{-5}{p}= \answer{\legendre{-1}{p}\legendre{5}{p}}=\answer{\legendre{-1}{p}\legendre{p}{5}}$ by \nameref{quad-rec-useful-form}. Again, we have two cases.
			
			\begin{enumerate}[label=(\roman*)]
 				\item If $p\equiv 1\pmod{4},$ then $\legendre{-1}{p}=\answer{1}$ by \nameref{thm:residue-neg1} and
				$\legendre{5}{p}=\answer{\legendre{p}{5}}$ by \nameref{quad-rec-useful-form}. Thus, $\legendre{-5}{p}=\answer{\legendre{p}{5}}=1$ if and only if $p\equiv \answer{\pm1\pmod{5}}.$ When $p\equiv \answer{1}\pmod{5},$ we have $p\equiv\answer{1}\pmod{\answer{20}}$ and when $p\equiv \answer{-1}\pmod{5},$ we have $p\equiv\answer{9}\pmod{\answer{20}}$. In each case, there is a unique equivalence class modulo $\answer{20}$ by the Chinese Remainder Theorem.
		
		\item If $p\equiv 3\pmod{4},$ then $\legendre{-1}{p}=\answer{-1}$ by \nameref{thm:residue-neg1} and
				$\legendre{5}{p}=\answer{\legendre{p}{5}}$ by \nameref{quad-rec-useful-form}. Thus, $\legendre{-5}{p}=\answer{-\legendre{p}{5}}=1$ if and only if $p\equiv \answer{\pm 2}\pmod{5}.$ When $p\equiv \answer{2}\pmod{5},$ we have $p\equiv\answer{7}\pmod{\answer{20}}$ and when $p\equiv \answer{3}\pmod{5},$ we have $p\equiv\answer{3}\pmod{\answer{20}}$. In each case, there is a unique equivalence class modulo $\answer{20}$ by the Chinese Remainder Theorem.
	\end{enumerate}
	Therefore, $\legendre{-5}{p}=1$ if and only if $p\equiv \answer{1,3,7,9}\pmod{\answer{20}}.$\qedhere
	\end{enumerate}
\end{solution}



\end{document}
