\documentclass[letterpaper, 11 pt]{ximera}
\usepackage{amssymb, latexsym, amsmath, amsthm, graphicx, amsthm,alltt,color, listings,multicol,hyperref,xr-hyper,aliascnt,enumitem}
\usepackage{xfrac}


\usepackage{parskip}
\usepackage{graphicx}
\usepackage[,margin=0.7in]{geometry}
\setlength{\textheight}{8.5in}
 
\usepackage{tkz-euclide}
%\usetkzobj{all}
\tikzstyle geometryDiagrams=[rounded corners=.5pt,ultra thick,color=black]
\colorlet{penColor}{black} % Color of a curve in a plot


\usepackage{subcaption}
\usepackage{float}
\usepackage{fancyhdr}
\usepackage{pdfpages}
\newcounter{includepdfpage}


\newcommand{\semester}{%
  \ifcase\month
  \or Spring %1
  \or Spring %2
  \or Spring %3
  \or Spring %4
  \or Spring  %5
  \or Fall %8
  \or Fall %9
  \or Fall %10
  \or Fall %11
  \or Fall %12
  \fi
}
\usepackage{currfile}
\usepackage{xstring}


\lhead{\large{Number Theory: MAT-255}}
%Put your Document Title (Camp: Topic) Here
\chead{}
\rhead{\semester 24}
\lfoot{}
\cfoot{}
\rfoot{Page \thepage}
\renewcommand\headrulewidth{0pt}
\renewcommand\footrulewidth{0pt}

\headheight 50pt
\headsep 30pt

\author{Claire Merriman}
\date{Spring 2024}


%%%%%%%%%%%%%%%%%%%%%
% Counters and autoref for unnumbered environments
%%%%%%%%%%%%%%%%%%%%%
\theoremstyle{plain}


\newtheorem*{namedthm}{Theorem}
\newcounter{thm}%makes pointer correct
\providecommand{\thmname}{Proposition}

\makeatletter
\NewDocumentEnvironment{thm*}{o}
 {%
  \IfValueTF{#1}
    {\namedthm[#1]\refstepcounter{thm}\def\@currentlabel{(#1)}}%
    {\namedthm}%
 }
 {%
  \endnamedthm
 }
\makeatother


\newtheorem*{namedprop}{Proposition}
\newcounter{prop}%makes pointer correct
\providecommand{\propname}{Proposition}

\makeatletter
\NewDocumentEnvironment{prop*}{o}
 {%
  \IfValueTF{#1}
    {\namedprop[#1]\refstepcounter{prop}\def\@currentlabel{(#1)}}%
    {\namedprop}%
 }
 {%
  \endnamedprop
 }
\makeatother

\newtheorem*{namedlem}{Lemma}
\newcounter{lem}%makes pointer correct
\providecommand{\lemname}{Lemma}

\makeatletter
\NewDocumentEnvironment{lem*}{o}
 {%
  \IfValueTF{#1}
    {\namedlem[#1]\refstepcounter{lem}\def\@currentlabel{(#1)}}%
    {\namedlem}%
 }
 {%
  \endnamedlem
 }
\makeatother

\newtheorem*{namedcor}{Corollary}
\newcounter{cor}%makes pointer correct
\providecommand{\corname}{Corollary}

\makeatletter
\NewDocumentEnvironment{cor*}{o}
 {%
  \IfValueTF{#1}
    {\namedcor[#1]\refstepcounter{cor}\def\@currentlabel{(#1)}}%
    {\namedcor}%
 }
 {%
  \endnamedcor
 }
\makeatother

\theoremstyle{definition}

\newtheorem*{innerrem}{Remark}
\newcounter{rem}%makes pointer correct
\providecommand{\remname}{Remark}

\makeatletter
\NewDocumentEnvironment{rem}{o}
 {%
  \IfValueTF{#1}
    {\innerrem[#1]\refstepcounter{rem}\def\@currentlabel{(#1)}}%
    {\innerrem}%
 }
 {%
  \endinnerrem
 }
\makeatother

\newtheorem*{innerdefn}{Definition}%%placeholder
\newcounter{defn}%makes pointer correct
\providecommand{\defnname}{Definition}

\makeatletter
\NewDocumentEnvironment{defn}{o}
 {%
  \IfValueTF{#1}
    {\innerdefn[#1]\refstepcounter{defn}\def\@currentlabel{(#1)}}%
    {\innerdefn}%
 }
 {%
  \endinnerdefn
 }
\makeatother

\newtheorem*{scratch}{Scratch Work}


\newtheorem*{namedconj}{Conjecture}
\newcounter{conj}%makes pointer correct
\providecommand{\conjname}{Conjecture}
\makeatletter
\NewDocumentEnvironment{conj}{o}
 {%
  \IfValueTF{#1}
    {\innerconj[#1]\refstepcounter{conj}\def\@currentlabel{(#1)}}%
    {\innerconj}%
 }
 {%
  \endinnerconj
 }
\makeatother

\newtheorem*{poll}{Poll question}
\newtheorem{tps}{Think-Pair-Share}[section]
%\newtheorem{br}{In-class Problem}[section]
\newtheorem*{cs}{Crowd Sourced Proof}

\newlist{checklist}{itemize}{2}
\setlist[checklist]{label=$\square$}

\newenvironment{obj}{
	\textbf{Learning Objectives.} By the end of class, students will be able to:
		\begin{itemize}}
		{\!.\end{itemize}
		}

\newenvironment{pre}{
	\begin{description}
	}{
	\end{description}
}


\newcounter{br}%makes pointer correct
\providecommand{\brname}{In-class Problem}

\newenvironment{br}[1][2in]%
{%Env start code
\problemEnvironmentStart{#1}{In-class Problem}
\refstepcounter{br}
}
{%Env end code
\problemEnvironmentEnd
}

\newcounter{ex}%makes pointer correct
\providecommand{\exname}{Homework Problem}
\newenvironment{ex}[1][2in]%
{%Env start code
\problemEnvironmentStart{#1}{Homework Problem}
\refstepcounter{ex}
}
{%Env end code
\problemEnvironmentEnd
}



\newenvironment{sketch}
 {\begin{proof}[Sketch of Proof]}
 {\end{proof}}
%\newenvironment{hint}
%  {\begin{proof}[Hint]}
%  {\end{proof}}

\newcommand{\gt}{>}
\newcommand{\lt}{<}
\newcommand{\N}{\mathbb N}
\newcommand{\Q}{\mathbb Q}
\newcommand{\Z}{\mathbb Z}
\newcommand{\C}{\mathbb C}
\newcommand{\R}{\mathbb R}
\renewcommand{\H}{\mathbb{H}}
\newcommand{\lcm}{\operatorname{lcm}}
\newcommand{\nequiv}{\not\equiv}
\newcommand{\ord}{\operatorname{ord}}
\newcommand{\ds}{\displaystyle}
\newcommand{\floor}[1]{\left\lfloor #1\right\rfloor}
\newcommand{\legendre}[2]{\left(\frac{#1}{#2}\right)}



%%%%%%%%%%%%




\title{Other Results from Strayer}


\begin{document}

\begin{axiom}[Well Ordering Principle]\label{well-order}
    Every nonempty set of positive integers contains a least element.
\end{axiom}
    \section*{Divisibility facts}\label{sec:additional-div}

    \begin{lem*}[Proposition 1.2]\label{lem:linear-combo}
         Let $a,b,c,d\in\Z.$ If $c\mid a$ and $c\mid d,$ then $c\mid ma+nb.$
    \end{lem*}


    \begin{prop*}[Proposition 1.10]\label{prop:div-gcd-rel-prime}
        Let $a,b\in\Z$ with $(a,b)=d.$ Then $(\tfrac{a}{d},\tfrac{b}{d})=1.$
    \end{prop*}


    \begin{lem*}[Lemma 1.12]\label{lem:gcd-remainders}
     If $a,b\in\Z,$ $a\geq b\gt 0,$ and $a=bq+r$ with $q,r\in|Z,$ then $(a,b)=(b,r).$
    \end{lem*}
    

\section*{Prime facts}\label{sec:additional-primes}

\begin{lem*}[Lemma 1.14]\label{lem:irreducible-prime}
    Let $a,b,p\in\Z$ with $p$ prime. If $p\mid ab,$ then $p\mid a$ or $p\mid b.$
\end{lem*}

\begin{cor*}[Corollary 1.15]\label{cor:irreducible-prime} Let $a_1,a_2,\dots,a_n,p\in\Z$ with $p$ prime. If $p\mid a_1a_2\cdots a_n,$ then $p\mid a_i$ for some $i.$
\end{cor*}

\begin{prop*}[Proposition 1.17]\label{prop:form-lcm-gcd}
 Let $a,b\in\Z$ with $a,b\gt 1.$ Write $a=p_1^{a_1}p_2^{a_2}\cdots  p_n^{a_n}$ and $b=p_1^{b_1}p_2^{b_2}\cdots p_n^{b_n}$ where $p_1,p_2,\dots,p_n$ are distinct primes and ${a_1},{a_2}\cdots,{a_n},{b_1},{b_2},\cdots,{b_n}$ are nonnegative integers (possibly zero). Then
        \[(a,b)=p_1^{\min\{a_1,b_1\}}p_2^{\min\{a_2,b_2\}}\cdots p_n^{\min\{a_n,b_n\}}\]
        and 
        \[[a,b]=p_1^{\max\{a_1,b_1\}}p_2^{\max\{a_2,b_2\}}\cdots p_n^{\max\{a_n,b_n\}}.\]
\end{prop*}

\begin{thm*}[Theorem 1.19]\label{thm:prod-lcm-gcd} Let $a,b\in\Z$ with $a,b\gt 0.$ Then $(a,b)[a,b]=ab.$
\end{thm*}

\section*{Congruences}

\begin{prop*}[Proposition 2.1]\label{prop:equiv-rel}
    Let $a,b,c,d,m\in\Z$ with $m>0,$ then:
        \begin{enumerate}
            \item\label{equiv-reflect} $a\equiv a \pmod{m}$
            
            \item\label{equiv-sym} $a\equiv b \pmod{m}$ implies $b\equiv a \pmod{m}$

            \item\label{equiv-trans} $a\equiv b \pmod{m}$ and $b\equiv c \pmod{m}$ implies $a\equiv c \pmod{m}$
\end{enumerate}
\end{prop*}


\begin{prop*}[Proposition 2.4]\label{prop:add-mult}
    Let $a,b,c,d,m\in\Z$ with $m>0,$ then:
    
    \begin{enumerate}[label=(\alph*)]
        \item\label{equiv-add} $a\equiv b \pmod{m}$ and $c\equiv d \pmod{m}$ implies $a+c \equiv b+d \pmod{m}$ 
        \item\label{equiv-multiply} $a\equiv b\pmod{m}$ and $c\equiv d \pmod{m}$ implies $ac\equiv bd \pmod{m}$.
    \end{enumerate}
\end{prop*}


\begin{prop*}[Proposition 2.5]\label{prop:equiv-gcd}
    Let $a,b,c,m\in\Z$ with $m>0.$ Then $ca\equiv cb\pmod{m}$ if and only if $a\equiv b\pmod{\tfrac{m}{(a,m)}}.$
\end{prop*}


\begin{lem*}[Chapter 2, Exercise 9]\label{ex-equiv-upmod}
    Let $a,b,c,m\in\Z$ with $m>0.$ If $a\equiv b \pmod{m}$ then $ac\equiv bc \pmod{mc}$ for $c>0$.
\end{lem*}


\begin{cor*}[Corollary 2.15]\label{cor:a_power_prime_mod}
    Let $p$ be a prime number and let $a\in\Z.$ Then $a^p\equiv a\pmod{p}.$
\end{cor*}

\section*{The Euler Phi-Function}

\begin{thm*}[Theorem 3.3]\label{thm:phi-prime-power}
    Let $p$ be prime and let $a\in\Z$ with $a>0.$ Then $\phi(p^a)=p^a-p^{a-1}=p^{a-1}(p-1).$
\end{thm*}
\section*{Diophantine equations}
\begin{thm*}[Theorem 6.2]\label{thm:linear-dioph}
Let $ax+by=c$ be a linear Diophantine equation in two variables $x$ and $y$ and let $d=(a,b).$ If $d\nmid c,$ then the equation has no solutions. If $d\mid c$ then there are infinitely many solutions of the form \[
            x=x_0+\frac{b}{d}n, y=y_0-\frac{a}{d}n, \text{ for } n\in\Z.
        \]
\end{thm*}


\end{document}