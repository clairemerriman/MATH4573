\documentclass{ximera}
\usepackage{amssymb, latexsym, amsmath, amsthm, graphicx, amsthm,alltt,color, listings,multicol,xr-hyper,hyperref,aliascnt,enumitem}
\usepackage{xfrac}

\usepackage{parskip}
\usepackage[,margin=0.7in]{geometry}
\setlength{\textheight}{8.5in}

\usepackage{epstopdf}

\DeclareGraphicsExtensions{.eps}
\usepackage{tikz}


\usepackage{tkz-euclide}
%\usetkzobj{all}
\tikzstyle geometryDiagrams=[rounded corners=.5pt,ultra thick,color=black]
\colorlet{penColor}{black} % Color of a curve in a plot


\usepackage{subcaption}
\usepackage{float}
\usepackage{fancyhdr}
\usepackage{pdfpages}
\newcounter{includepdfpage}
\usepackage{makecell}


\usepackage{currfile}
\usepackage{xstring}




\graphicspath{  
{./otherDocuments/}
}

\author{Claire Merriman}
\newcommand{\classday}[1]{\def\classday{#1}}

%%%%%%%%%%%%%%%%%%%%%
% Counters and autoref for unnumbered environments
% Not needed??
%%%%%%%%%%%%%%%%%%%%%
\theoremstyle{plain}


\newtheorem*{namedthm}{Theorem}
\newcounter{thm}%makes pointer correct
\providecommand{\thmname}{Theorem}

\makeatletter
\NewDocumentEnvironment{thm*}{o}
 {%
  \IfValueTF{#1}
    {\namedthm[#1]\refstepcounter{thm}\def\@currentlabel{(#1)}}%
    {\namedthm}%
 }
 {%
  \endnamedthm
 }
\makeatother


\newtheorem*{namedprop}{Proposition}
\newcounter{prop}%makes pointer correct
\providecommand{\propname}{Proposition}

\makeatletter
\NewDocumentEnvironment{prop*}{o}
 {%
  \IfValueTF{#1}
    {\namedprop[#1]\refstepcounter{prop}\def\@currentlabel{(#1)}}%
    {\namedprop}%
 }
 {%
  \endnamedprop
 }
\makeatother

\newtheorem*{namedlem}{Lemma}
\newcounter{lem}%makes pointer correct
\providecommand{\lemname}{Lemma}

\makeatletter
\NewDocumentEnvironment{lem*}{o}
 {%
  \IfValueTF{#1}
    {\namedlem[#1]\refstepcounter{lem}\def\@currentlabel{(#1)}}%
    {\namedlem}%
 }
 {%
  \endnamedlem
 }
\makeatother

\newtheorem*{namedcor}{Corollary}
\newcounter{cor}%makes pointer correct
\providecommand{\corname}{Corollary}

\makeatletter
\NewDocumentEnvironment{cor*}{o}
 {%
  \IfValueTF{#1}
    {\namedcor[#1]\refstepcounter{cor}\def\@currentlabel{(#1)}}%
    {\namedcor}%
 }
 {%
  \endnamedcor
 }
\makeatother

\theoremstyle{definition}
\newtheorem*{annotation}{Annotation}
\newtheorem*{rubric}{Rubric}

\newtheorem*{innerrem}{Remark}
\newcounter{rem}%makes pointer correct
\providecommand{\remname}{Remark}

\makeatletter
\NewDocumentEnvironment{rem}{o}
 {%
  \IfValueTF{#1}
    {\innerrem[#1]\refstepcounter{rem}\def\@currentlabel{(#1)}}%
    {\innerrem}%
 }
 {%
  \endinnerrem
 }
\makeatother

\newtheorem*{innerdefn}{Definition}%%placeholder
\newcounter{defn}%makes pointer correct
\providecommand{\defnname}{Definition}

\makeatletter
\NewDocumentEnvironment{defn}{o}
 {%
  \IfValueTF{#1}
    {\innerdefn[#1]\refstepcounter{defn}\def\@currentlabel{(#1)}}%
    {\innerdefn}%
 }
 {%
  \endinnerdefn
 }
\makeatother

\newtheorem*{scratch}{Scratch Work}


\newtheorem*{namedconj}{Conjecture}
\newcounter{conj}%makes pointer correct
\providecommand{\conjname}{Conjecture}
\makeatletter
\NewDocumentEnvironment{conj}{o}
 {%
  \IfValueTF{#1}
    {\innerconj[#1]\refstepcounter{conj}\def\@currentlabel{(#1)}}%
    {\innerconj}%
 }
 {%
  \endinnerconj
 }
\makeatother

\newtheorem*{poll}{Poll question}
\newtheorem{tps}{Think-Pair-Share}[section]


\newenvironment{obj}{
	\textbf{Learning Objectives.} By the end of class, students will be able to:
		\begin{itemize}}
		{\!.\end{itemize}
		}

\newenvironment{pre}{
	\begin{description}
	}{
	\end{description}
}


\newcounter{ex}%makes pointer correct
\providecommand{\exname}{Homework Problem}
\newenvironment{ex}[1][2in]%
{%Env start code
\problemEnvironmentStart{#1}{Homework Problem}
\refstepcounter{ex}
}
{%Env end code
\problemEnvironmentEnd
}

\newcommand{\inlineAnswer}[2][2 cm]{
    \ifhandout{\pdfOnly{\rule{#1}{0.4pt}}}
    \else{\answer{#2}}
    \fi
}


\ifhandout
\newenvironment{shortAnswer}[1][
    \vfill]
        {% Begin then result
        #1
            \begin{freeResponse}
            }
    {% Environment Ending Code
    \end{freeResponse}
    }
\else
\newenvironment{shortAnswer}[1][]
        {\begin{freeResponse}
            }
    {% Environment Ending Code
    \end{freeResponse}
    }
\fi

\let\question\relax
\let\endquestion\relax

\newtheoremstyle{ExerciseStyle}{\topsep}{\topsep}%%% space between body and thm
		{}                      %%% Thm body font
		{}                              %%% Indent amount (empty = no indent)
		{\bfseries}            %%% Thm head font
		{}                              %%% Punctuation after thm head
		{3em}                           %%% Space after thm head
		{{#1}~\thmnumber{#2}\thmnote{ \bfseries(#3)}}%%% Thm head spec
\theoremstyle{ExerciseStyle}
\newtheorem{br}{In-class Problem}

\newenvironment{sketch}
 {\begin{proof}[Sketch of Proof]}
 {\end{proof}}


\newcommand{\gt}{>}
\newcommand{\lt}{<}
\newcommand{\N}{\mathbb N}
\newcommand{\Q}{\mathbb Q}
\newcommand{\Z}{\mathbb Z}
\newcommand{\C}{\mathbb C}
\newcommand{\R}{\mathbb R}
\renewcommand{\H}{\mathbb{H}}
\newcommand{\lcm}{\operatorname{lcm}}
\newcommand{\nequiv}{\not\equiv}
\newcommand{\ord}{\operatorname{ord}}
\newcommand{\ds}{\displaystyle}
\newcommand{\floor}[1]{\left\lfloor #1\right\rfloor}
\newcommand{\legendre}[2]{\left(\frac{#1}{#2}\right)}



%%%%%%%%%%%%



\title{Sums and Differences of Squares}
\begin{document}
\begin{abstract}
\end{abstract}
\maketitle

%%%%%%%%%%%%%%%%%%%%%%%%%%

\begin{pre}
    \item[Reading and Turn in:] \href{https://ximera.osu.edu/elementarynt/MATH4573Online/ElementaryNumberTheory/MATH4573Online/April10/April10}{This fill-in-the-blank version} of the proof of Sum of Three Squares
\end{pre}


\begin{br}[Chapter 6, Exercise 34]
	Prove that a positive integer can be written as the difference of two squares of integers if and only if it is not of the form $4n + 2$ for some $n\in\Z.$
	\begin{solution}
	\begin{description}
	 	\item[($\Rightarrow$)] We will show that if a positive integer can be written as the difference of two squares of integers, then it is not of the form $4n + 2$ for some $n\in\Z.$

	 	(Proof)
	  
		\item[($\Leftarrow$)] We will show that any positive integer not of the form $4n + 2$ for some $n\in\Z$ can be written as the difference of two squares of integers. 
	  
		First, we will show that if $a$ and $b$ are positive integers that can be written as the difference of two squares of integers, then so can $ab.$
		
		(Proof)
	

		Now we will show that every odd prime can be written as the difference of two squares of integers. Let $p$ be an odd prime. Then $p=x^2-y^2=(x-y)(x+y)$ when $x=\answer{\frac{p+1}{2}}$ and $y=\answer{\frac{p-1}{2}}.$
		Therefore every odd number can be written as the difference of two squares since \pdfOnly{\rule{2 cm}{.4pt}.} %\onlineOnly{Think about why this is true.}
	
		It remains to show that every positive integer of the form $4n$ for some $n\in\Z$ can be written as the difference of two squares of integers. Why is this the only remaining case?
	
		Similar to the odd prime case, $4n=x^2-y^2=(x-y)(x+y)$ when $x=\answer{n+1}$ and $y=\answer{n-1}$ or when $x=\answer{\frac{n}{2}-2}$ and $y=\answer{\frac{n}{2}+2}$ (solutions are not unique).
	\end{description}
\end{solution}
\end{br}

If there is time

\begin{br}[Chapter 6, Exercise 14d]
	Let $x,y,z$ be a primitive Pythagorean triple with $y$ even. Prove that $x+y\equiv x-y\equiv 1,7\pmod{8}.$

	\begin{hint}
		First show that $x+y\equiv x-y\pmod{8}.$
	\end{hint}


	\begin{solution}
		Let $x,y,z$ be a primitive Pythagorean triple with $y$ even. 
		Then from \nameref{thm:form-pyth-trip}, there exist positive integers $m,n$ with $m>n, (m,n)=1$ and exactly one of $m,n$ even such that 
		\begin{align*}
			x&=m^2-n^2\\
			y&=2mn\\
			z&=m^2+n^2.
		\end{align*}
		Thus, $4\mid y$ and $y\equiv -y\pmod{8}.$ Adding $x$ to both sides of the congruence gives $x+y\equiv x-y\pmod{8}.$

		Since exactly one of $m,n$ even, $x\equiv 0-1, 4-1, 1-0, 1-4\pmod{8}.$ When $x\equiv \pm 1\pmod{8},$ $4\mid mn$ and thus $y=2mn\equiv 0\pmod{8}.$ Thus,
		$x+y\equiv x-y\equiv \pm 1\pmod{8}.$ 
		When $x\equiv \pm 3\pmod{8},$ $4\nmid mn$ and thus $y=2mn\equiv 4\pmod{8}.$ Thus,
		$x+y\equiv x-y\equiv \pm 1\pmod{8}.$ 
	\end{solution}
\end{br}



\end{document}
