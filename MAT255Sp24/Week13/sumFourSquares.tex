\documentclass{ximera}

%\newtheorem{theorem}{Theorem}%[section] % reset theorem numbering for each section
%\newtheorem*{theorem*}{Theorem}%[section] % reset theorem numbering for each section
% \newtheorem{prop}[theorem]{Proposition}
% \newtheorem{lem}[theorem]{Lemma}
% \newtheorem{ex}{Example}


\title{Sums of four squares}  
\begin{document}  
\begin{abstract}  We prove that all integers can be written as the sum of four squares.
\end{abstract}
\maketitle  

We now prove that all positive integers can be written as the sum of four squares. The new few results are similar to the proof of the sum of two squares. Some of these calculations are even more involved, but still use multiplication and factoring. I will upload a scan of the sums of squares section from \emph{Elementary Number Theory} by James K. Strayer. All of the missing calculations are expanding and refactoring polynomial expression. 

\begin{theorem}[Euler]
 Let $n_1,n_2\in\mathbb{Z}$ with $n_1,n_2>0$. If $n_1$ and $n_2$ are expressible as the sum of four squares, then so is $n_1n_2$. 
\end{theorem}
\begin{proof}
 Let $a,b,c,d,w,x,y,u\in\mathbb{Z}$ such that \[n_1=a^2+b^2+c^2+d^2\] and \[n_1=w^2+x^2+y^2+z^2.\]  Then \[n_1n_2=(aw+bx+cy+dz)^2+(-ax+bw-cz+dy)^2+(-ay+bz+cw-dx)^2+(-az-by+cx+dw)^2 \] as desired.
\end{proof}

\begin{theorem}[Also Euler]
 If $p$ is an odd prime number, then there exist $x,y\in\mathbb{Z}$ such that $x^2+y^2+1=kp$ for some $k\in\mathbb{Z}$ with $0<k<p$.
\end{theorem}
\begin{proof}
 We consider two cases.
 
{\bf Case 1: $p\equiv 1\pmod 4$}. Then $\left(\frac{-1}{p}\right)=1$, so there exists $x\in\mathbb{Z}$ with $0<x\leq\frac{p-1}{2}$ such that $x^2\equiv -1 \pmod p$. Then, $p\mid x^2+1$, and we have that $x^2+1=kp$ for some $k\in\mathbb{Z}$. Thus, we found $x$ and $y=\answer{0}$. Since $x^2+1$ and $p$ are positive, so is $k$. Also, \[kp=x^2+1<\left(\frac{p}{2}\right)^2+1<p^2\] implies $k<p$.

{\bf Case 2: $p\equiv 3\pmod 4$}. Let $a$ be the least positive quadratic nonresidue modulo $p$. Note that $a\geq2$. Then \[\left(\frac{-a}{p}\right)=\left(\frac{-1}{p}\right)\left(\frac{a}{p}\right)=(-1)(-1)=1\] and so there exists $x\in\mathbb{Z}$ with $0<x\leq\frac{p-1}{2}$ such that $x^2\equiv -a \pmod p$. Now, $a-1$ is positive and less than $a$, then $a-1$ is a quadratic residue modulo $p$. Thus, there exists $y\in\mathbb{Z}$ with $0<y\leq\frac{p-1}{2}$ such that $y^2\equiv a-1 \pmod p$. Thus, \[x^2+y^2+1\equiv (-a)+(a-1)+1\equiv 0 \pmod p\] or, equivalently, $x^2+y^2+1=kp$ for some $k\in\mathbb{Z}$. Again, $k>0$. Furthermore, \[kp=x^2+y^2+1<\left(\frac{p}{2}\right)^2+\left(\frac{p}{2}\right)^2+1<p^2\] implies $k<p$. \qedhere
\end{proof}

We will prove that every prime number can be written as the sum of four squares. 
\begin{theorem}[Lagrange, 1770]
 All prime numbers can be written as the sum of four squares.
\end{theorem}
\begin{proof}
 When $p=2=1^2+1^2+0^2+0^2$, we are done. In fact, we can also writes a prime $p$ where $p\equiv 1 \pmod 4$ as $p=x^2+y^2+0^2+0^2$, but the following method works for all odd primes. 
 
 Let $m$ be the least positive integer where $x^2+y^2+z^2+w^2=mp$ and $0<m<p$. We want to show that $m=1$. To get a contradiction, assume $m>1$. We consider two cases.

{\bf Case 1:} $m$ even. There are three posibilities: $w,x,y,z$ are all even;  $w,x,y,z$ are all odd; two of $w,x,y,z$ are odd and the other two are even. In all three cases, we can assume $w\equiv x \pmod 2$ and $y\equiv z \pmod 2$. 
Then $\frac{w+x}{2},\frac{w-x}{2},\frac{y+z}{2},\frac{y-z}{2}$ are integers and \[\left(\frac{w+x}{2}\right)^2+\left(\frac{w-x}{2}\right)^2+\left(\frac{y+z}{2}\right)^2+\left(\frac{y-z}{2}\right)^2=\frac{mp}{2}\] which contradicts the fact that $m$ is minimal.

{\bf Case 2:} $m$ is odd.  

Then $m\geq 3.$ Let $a,b,c,d\in\mathbb{Z}$ such that 
\begin{align*}
&a\equiv w\pmod m, \frac{-m}{2}<a<\frac{m}{2}\\
&b\equiv x\pmod m, \frac{-m}{2}<b<\frac{m}{2}\\
&c\equiv y\pmod m, \frac{-m}{2}<c<\frac{m}{2}\\
&d\equiv z\pmod m, \frac{-m}{2}<d<\frac{m}{2}.
\end{align*}
Then $a^2+b^2+c^2+d^2\equiv w^2+x^2+y^2+z^2\equiv mp\equiv 0\pmod m$ and so there exists $k\in\mathbb{Z}$ with $k>0$ such that $a^2+b^2+c^2+d^2=km$.
Now \[(a^2+b^2+c^2+d^2)( w^2+x^2+y^2+z^2)=(km)(mp)=km^2p.\]
By theorem 2 from today, we can rewrite $a^2+b^2+c^2+d^2)( w^2+x^2+y^2+z^2)$ as the sum of four squares 
$(aw+bx+cy+dz)^2+(-ax+bw-cz+dy)^2+(-ay+bz+cw-dx)^2+(-az-by+cx+dw)^2=km^2p$. Since $a\equiv w\pmod m, b\equiv x\pmod m, c\equiv y\pmod m, \frac{-m}{2}<c<\frac{m}{2}, d\equiv z\pmod m, \frac{-m}{2}<d<\frac{m}{2},$ we have 
\begin{align*}
 aw+bx+cy+dz\equiv w^2+x^2+y^2+z^2\equiv 0 \pmod m\\
-ax+bw-cz+dy\equiv -wx+xw-yz+zy\equiv 0\pmod m\\
-ay+bz+cw-dx\equiv -wy+yw+xz-zx\equiv 0\pmod m\\
-az-by+cx+dw\equiv -wx+zw-xy+yx\equiv 0\pmod m
\end{align*}
Let $W=\frac{w^2+x^2+y^2+z^2}{m}, X=\frac{-wx+xw-yz+zy}{m}, Y=\frac{-wy+yw+xz-zx}{m}, Z=\frac{-wx+zw-xy+yx}{m}$. Then $W^2+X^2+Y^2+Z^2=\frac{km^2p}{m^2}=kp.$ Since $\frac{-m}{2}<a,b,c,d<\frac{m}{2},$ then $a^2,b^2,c^2,d^2<\frac{m^2}{4}$. Thus, \[km=a^2+b^2+c^2+d^2<\frac{4m^2}{4}\] and $k<m$. This contradicts that $m$ is the smallest such integers.

Thus, $m=1, w^2+x^2+y^2+z^2=p$.
\end{proof}

\begin{theorem}[Lagrange] All positive integers can be written as the sum of four squares.
\end{theorem}
\begin{proof}
 Let $n\in\mathbb{Z}$ with $n>1$. If $n=1=1^2+0^2+0^2+0^2$. If $n>1$, then $n$ is the product of primes by the Fundamental Theorem of Arithmetic. By the previous theorem, every prime number can be written as the sum of four squares. By theorem 2 from today, $n$ is can be written by the sum of four squares.
\end{proof}

We finish this section with a few famous problems.
\begin{example}[Waring's Problem, 1770]
 Let $k\in\mathbb{Z}$ with $k>0$. Does there exist a minimum integer $g(k)$ such that every positive integer can be written as the sum of at most $g(k)$ nonnegative integers to the $k^{th}$ power?
 \end{example}
 For example, $g(1)=\answer{1}$. Today we showed that $g(2)=\answer{4}$. The next step would be to find if $g(3)$ exists and what it equals.
 
 
\begin{theorem}[Hilbert, 1906]
 Let $k\in\mathbb{Z}$ with $k>0$. There exists a minimal integer $g(k)$ such that every positive integer can be written as the sum of at most $g(k)$ nonnegative integers to the $k^{th}$ power.
\end{theorem}
The proof of Hilbert's theorem does not provide a formula for $g(k)$, merely proves it exists. Numerical evidence suggests $g(k)=\left\lfloor\left(\frac{3}{2}\right)^k\right\rfloor+2^k-2$. It's been proven that there are only finitely many (or 0) $k$ where the formula does not hold, and the formula holds when $k\leq 471,600,000$. Thus, $g(3)=\answer{9},g(4)=19,$ and $g(5)=37$. Proofs of these facts come from analytic number theory.
\end{document}