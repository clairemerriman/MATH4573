\documentclass{ximera}
<<<<<<< Updated upstream
\usepackage{amssymb, latexsym, amsmath, amsthm, graphicx, amsthm,alltt,color, listings,multicol,xr-hyper,hyperref,aliascnt,enumitem}
=======
\usepackage{amssymb, latexsym, amsmath, amsthm, graphicx, amsthm,alltt,color, listings,multicol,hyperref}
\usepackage[capitalise,nameinlink]{cleveref}
>>>>>>> Stashed changes
\usepackage{xfrac}

\usepackage{parskip}
\usepackage[,margin=0.7in]{geometry}
\setlength{\textheight}{8.5in}

\usepackage{epstopdf}

\DeclareGraphicsExtensions{.eps}
\usepackage{tikz}


\usepackage{tkz-euclide}
%\usetkzobj{all}
\tikzstyle geometryDiagrams=[rounded corners=.5pt,ultra thick,color=black]
\colorlet{penColor}{black} % Color of a curve in a plot


\usepackage{subcaption}
\usepackage{float}
\usepackage{fancyhdr}
\usepackage{pdfpages}
\newcounter{includepdfpage}
\usepackage{makecell}


\usepackage{currfile}
\usepackage{xstring}




\graphicspath{  
{./otherDocuments/}
}

\author{Claire Merriman}
\newcommand{\classday}[1]{\def\classday{#1}}

%%%%%%%%%%%%%%%%%%%%%
% Counters and autoref for unnumbered environments
% Not needed??
%%%%%%%%%%%%%%%%%%%%%
<<<<<<< Updated upstream
\theoremstyle{plain}


\newtheorem*{namedthm}{Theorem}
\newcounter{thm}%makes pointer correct
\providecommand{\thmname}{Theorem}
=======

\crefname{problem}{problem}{problems}


% \theoremstyle{plain}


% \newtheorem*{namedthm}{Theorem}
% \newcounter{thm}%makes pointer correct
% \providecommand{\thmname}{Theorem}
>>>>>>> Stashed changes

\makeatletter
\NewDocumentEnvironment{thm*}{o}
 {%
  \IfValueTF{#1}
    {\namedthm[#1]\refstepcounter{thm}\def\@currentlabel{(#1)}}%
    {\namedthm}%
 }
 {%
  \endnamedthm
 }
\makeatother


\newtheorem*{namedprop}{Proposition}
\newcounter{prop}%makes pointer correct
\providecommand{\propname}{Proposition}

\makeatletter
\NewDocumentEnvironment{prop*}{o}
 {%
  \IfValueTF{#1}
    {\namedprop[#1]\refstepcounter{prop}\def\@currentlabel{(#1)}}%
    {\namedprop}%
 }
 {%
  \endnamedprop
 }
\makeatother

\newtheorem*{namedlem}{Lemma}
\newcounter{lem}%makes pointer correct
\providecommand{\lemname}{Lemma}

\makeatletter
\NewDocumentEnvironment{lem*}{o}
 {%
  \IfValueTF{#1}
    {\namedlem[#1]\refstepcounter{lem}\def\@currentlabel{(#1)}}%
    {\namedlem}%
 }
 {%
  \endnamedlem
 }
\makeatother

\newtheorem*{namedcor}{Corollary}
\newcounter{cor}%makes pointer correct
\providecommand{\corname}{Corollary}

\makeatletter
\NewDocumentEnvironment{cor*}{o}
 {%
  \IfValueTF{#1}
    {\namedcor[#1]\refstepcounter{cor}\def\@currentlabel{(#1)}}%
    {\namedcor}%
 }
 {%
  \endnamedcor
 }
\makeatother

\theoremstyle{definition}
\newtheorem*{annotation}{Annotation}
\newtheorem*{rubric}{Rubric}

\newtheorem*{innerrem}{Remark}
\newcounter{rem}%makes pointer correct
\providecommand{\remname}{Remark}

\makeatletter
\NewDocumentEnvironment{rem}{o}
 {%
  \IfValueTF{#1}
    {\innerrem[#1]\refstepcounter{rem}\def\@currentlabel{(#1)}}%
    {\innerrem}%
 }
 {%
  \endinnerrem
 }
\makeatother

\newtheorem*{innerdefn}{Definition}%%placeholder
\newcounter{defn}%makes pointer correct
\providecommand{\defnname}{Definition}

\makeatletter
\NewDocumentEnvironment{defn}{o}
 {%
  \IfValueTF{#1}
    {\innerdefn[#1]\refstepcounter{defn}\def\@currentlabel{(#1)}}%
    {\innerdefn}%
 }
 {%
  \endinnerdefn
 }
\makeatother

\newtheorem*{scratch}{Scratch Work}


\newtheorem*{namedconj}{Conjecture}
\newcounter{conj}%makes pointer correct
\providecommand{\conjname}{Conjecture}
\makeatletter
\NewDocumentEnvironment{conj}{o}
 {%
  \IfValueTF{#1}
    {\innerconj[#1]\refstepcounter{conj}\def\@currentlabel{(#1)}}%
    {\innerconj}%
 }
 {%
  \endinnerconj
 }
\makeatother

\newtheorem*{poll}{Poll question}
\newtheorem{tps}{Think-Pair-Share}[section]


\newenvironment{obj}{
	\textbf{Learning Objectives.} By the end of class, students will be able to:
		\begin{itemize}}
		{\!.\end{itemize}
		}

<<<<<<< Updated upstream
\newenvironment{pre}{
	\begin{description}
	}{
	\end{description}
}
=======

\ifinstructornotes
\newenvironment{pre}
  {{\textbf Reading assignment:}
  \begin{description}
    }{
	\end{description}
  }
\else
\newenvironment{pre}{ 
  \begin{trivlist}
  \item[]}
  {\end{trivlist}}
\fi
>>>>>>> Stashed changes


\newcounter{ex}%makes pointer correct
\providecommand{\exname}{Homework Problem}
\newenvironment{ex}[1][2in]%
{%Env start code
\problemEnvironmentStart{#1}{Homework Problem}
\refstepcounter{ex}
}
{%Env end code
\problemEnvironmentEnd
}

\newcommand{\inlineAnswer}[2][2 cm]{
    \ifhandout{\pdfOnly{\rule{#1}{0.4pt}}}
    \else{\answer{#2}}
    \fi
}


\ifhandout
\newenvironment{shortAnswer}[1][
    \vfill]
        {% Begin then result
        #1
            \begin{freeResponse}
            }
    {% Environment Ending Code
    \end{freeResponse}
    }
\else
\newenvironment{shortAnswer}[1][]
        {\begin{freeResponse}
            }
    {% Environment Ending Code
    \end{freeResponse}
    }
\fi

\let\question\relax
\let\endquestion\relax

\newtheoremstyle{ExerciseStyle}{\topsep}{\topsep}%%% space between body and thm
		{}                      %%% Thm body font
		{}                              %%% Indent amount (empty = no indent)
		{\bfseries}            %%% Thm head font
		{}                              %%% Punctuation after thm head
		{3em}                           %%% Space after thm head
		{{#1}~\thmnumber{#2}\thmnote{ \bfseries(#3)}}%%% Thm head spec
\theoremstyle{ExerciseStyle}
\newtheorem{br}{In-class Problem}

\newenvironment{sketch}
 {\begin{proof}[Sketch of Proof]}
 {\end{proof}}


\newcommand{\gt}{>}
\newcommand{\lt}{<}
\newcommand{\N}{\mathbb N}
\newcommand{\Q}{\mathbb Q}
\newcommand{\Z}{\mathbb Z}
\newcommand{\C}{\mathbb C}
\newcommand{\R}{\mathbb R}
\renewcommand{\H}{\mathbb{H}}
\newcommand{\lcm}{\operatorname{lcm}}
\newcommand{\nequiv}{\not\equiv}
\newcommand{\ord}{\operatorname{ord}}
\newcommand{\ds}{\displaystyle}
\newcommand{\floor}[1]{\left\lfloor #1\right\rfloor}
\newcommand{\legendre}[2]{\left(\frac{#1}{#2}\right)}



%%%%%%%%%%%%



\title{Pythagorean triples}
\begin{document}
\begin{abstract}
\end{abstract}
\maketitle

\begin{obj}
    \item Define a nonlinear Diophantine equation
    \item Define a primitive Pythagorean triple
	\item Prove the formula for generating primitive Pythagorean triples
\end{obj}

One of the most famous math equations is $x^2+y^2=z^2$, probably because we learn it in high school. We are going to classify all integer solutions to the equation.
 
\begin{definition}
	A triple $(x,y,z)$ of positive integers satisfying the Diophantine equation $x^2+y^2=z^2$ is called \emph{Pythagorean triple}.
\end{definition}
 

\begin{onlineOnly}
	Select the Pythagorean triples:
  
	\begin{selectAll}
	 	\choice[correct] {3,4,5}
	 	\choice[correct]{5,12,13}
	 	\choice{-3,4,5}
	 	\choice[correct]{6,8,10}
	 	\choice{0,1,1}
	\end{selectAll}
\end{onlineOnly}

 
It is actually possible to classify all Pythagorean triples, just like we did for linear Diophantine equations in two variables. To simplify this process, we will work with $x,y,z>0$, and $(x,y,z)=1$. For any given solution of this form, we have that $(-x,y,z),(x,-y,z),(x,y,-z),(-x,-y,z),(x,-y,-z),(-x,y,-z),$ and $(-x,-y,-z)$ are also solutions to the Diophantine equation, as is $(nx,ny,nz)$ for any integer $n$. Thus, we call such a solution a \emph{primitive Pythagorean triple}.  We call $(0,n,\pm n)$ and $(n,0,\pm n)$ the \emph{trival solutions}.
 
\begin{theorem}
For a primitive Pythagorean triple $(x,y,z)$, exactly one of $x$ and $y$ is even.
\end{theorem}
\begin{proof}
 If $x$ and $y$ are both even, then $z$ must also be even, contradicting that $(x,y,z)=1$.
  
 If $x$ and $y$ are both odd, then $z$ is even. Now we can work modulo $4$ to get a contradiction. Since $x$ and $y$ are odd, we have that $x^2\equiv y^2\equiv \answer{1}\pmod 4$. Since $z$ is even, we have that $z^2\equiv \answer{0}\pmod 4$, but $x^2+y^2\equiv \answer{2}\pmod 4$.
  
 Thus, the only remaining option is exactly one of $x$ and $y$ is even.
\end{proof}

\begin{theorem}[Theorem 6.3]\label{thm:form-pyth-trip}
	There are infinitely many primitive Pythagorean triples $x,y,z$ with $y$ even. Furthermore, they are given precisely by the equations
   \begin{align*}
	x&=m^2-n^2\\
	y&=2mn\\
	z&=m^2+n^2
   \end{align*}
   where $m,n\in\mathbb{Z}, m>n>0, (m,n)=1$ and exactly one of $m$ and $n$ is even.
\end{theorem}

\begin{example}

\begin{enumerate}
	\item $m=2$ and $n=1$ satisfy the conditions of $m$ and $n$ in the theorem. This gives $x=\answer{3}, y=\answer{4},z=\answer{5}$.
	\item $m=3$ and $n=2$ gives $x=\answer{5},y=\answer{12}, z=\answer{13}$.
	\item Try with your own values of $m$ and $n$.
   \end{enumerate}
\end{example}
   \begin{proof}
	We first show that given a primitive Pythagorean triple with $y$ even, there exist $m$ and $n$ as described. Since $y$ is  even, $y$ and $z$ are both odd. Moreover, $(x,y)=1,(y,z)=1,$ and $(x,z)=1$. Now,
	\[y^2=z^2-x^2=(x+z)(z-x)\] implies that \[\left(\frac{y}{2}\right)^2=\frac{(x+z)}{2}\frac{(z-x)}{2}.\]
	 
	To show, $\left(\frac{(x+z)}{2},\frac{(z-x)}{2}\right)=1$, let $\left(\frac{(x+z)}{2},\frac{(z-x)}{2}\right)=d.$ Then $d\mid\frac{z+x}{2}$ and $d\mid\frac{z-x}{2}$. Thus, $d\mid\frac{z+x}{2}+\frac{z-x}{2}=z$ and $d\mid\frac{z+x}{2}-\frac{z-x}{2}=x$. Since $(x,z)=1$, we have that $d=1$. Thus, $\frac{(x+z)}{2}$ and $\frac{(z-x)}{2}$ are perfect squares.
	 
	Let \begin{align*}m^2=\frac{(x+z)}{2},\quad n^2=\frac{(z-x)}{2}.\end{align*}
	Then $m>n>0, (m,n)=1, m^2-n^2=x, 2mn=y,$ and $m^2+n^2=z.$ Also, $(m,n)=1$ implies that not both $m$ and $n$ are both even. If both $m$ and $n$  are odd, we have that $z$ and $x$ are both even, but $(x,z)=1$. This proves that every primitive Pythagorean triple has this form.
	 
	Now we prove that given any such $m$ and $n$, we have a primitive Pythagorean triple. First, $(m^2-n^2)^2+
   (2mn)^2=m^4-2m^2n^2+n^4+4m^2n^2=(m^2+n^2)^2.$ We need to show that $(x,y,z)=1$. Let $(x,y,z)=d$. Since exactly one of $m$ and $n$ is even, we have that $x$ and $z$ are both odd. Then $d$ is odd, and thus $d=1$ or $d$ is divisible by some odd prime $p$. Assume that $p\mid d$. Thus, $p\mid x$ and $p\mid z$. Thus, $p\mid z+x$ and $p\mid z-x$. Thus, $p\mid (m^2+n^2)+(m^2-n^2)=2m^2$ and $p\mid (m^2+n^2)-(m^2-n^2)=2n^2.$ Since $p$ is odd, we have that $p\mid m^2$ and $p\mid n^2$, but  $(m,n)=1$, so $d=1$.
\end{proof}
%%%%%%%%%%%%%%%%%%%%%%%%%
\end{document}