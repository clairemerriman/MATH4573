\documentclass{ximera}
<<<<<<< Updated upstream
\usepackage{amssymb, latexsym, amsmath, amsthm, graphicx, amsthm,alltt,color, listings,multicol,xr-hyper,hyperref,aliascnt,enumitem}
=======
\usepackage{amssymb, latexsym, amsmath, amsthm, graphicx, amsthm,alltt,color, listings,multicol,hyperref}
\usepackage[capitalise,nameinlink]{cleveref}
>>>>>>> Stashed changes
\usepackage{xfrac}

\usepackage{parskip}
\usepackage[,margin=0.7in]{geometry}
\setlength{\textheight}{8.5in}

\usepackage{epstopdf}

\DeclareGraphicsExtensions{.eps}
\usepackage{tikz}


\usepackage{tkz-euclide}
%\usetkzobj{all}
\tikzstyle geometryDiagrams=[rounded corners=.5pt,ultra thick,color=black]
\colorlet{penColor}{black} % Color of a curve in a plot


\usepackage{subcaption}
\usepackage{float}
\usepackage{fancyhdr}
\usepackage{pdfpages}
\newcounter{includepdfpage}
\usepackage{makecell}


\usepackage{currfile}
\usepackage{xstring}




\graphicspath{  
{./otherDocuments/}
}

\author{Claire Merriman}
\newcommand{\classday}[1]{\def\classday{#1}}

%%%%%%%%%%%%%%%%%%%%%
% Counters and autoref for unnumbered environments
% Not needed??
%%%%%%%%%%%%%%%%%%%%%
<<<<<<< Updated upstream
\theoremstyle{plain}


\newtheorem*{namedthm}{Theorem}
\newcounter{thm}%makes pointer correct
\providecommand{\thmname}{Theorem}
=======

\crefname{problem}{problem}{problems}


% \theoremstyle{plain}


% \newtheorem*{namedthm}{Theorem}
% \newcounter{thm}%makes pointer correct
% \providecommand{\thmname}{Theorem}
>>>>>>> Stashed changes

\makeatletter
\NewDocumentEnvironment{thm*}{o}
 {%
  \IfValueTF{#1}
    {\namedthm[#1]\refstepcounter{thm}\def\@currentlabel{(#1)}}%
    {\namedthm}%
 }
 {%
  \endnamedthm
 }
\makeatother


\newtheorem*{namedprop}{Proposition}
\newcounter{prop}%makes pointer correct
\providecommand{\propname}{Proposition}

\makeatletter
\NewDocumentEnvironment{prop*}{o}
 {%
  \IfValueTF{#1}
    {\namedprop[#1]\refstepcounter{prop}\def\@currentlabel{(#1)}}%
    {\namedprop}%
 }
 {%
  \endnamedprop
 }
\makeatother

\newtheorem*{namedlem}{Lemma}
\newcounter{lem}%makes pointer correct
\providecommand{\lemname}{Lemma}

\makeatletter
\NewDocumentEnvironment{lem*}{o}
 {%
  \IfValueTF{#1}
    {\namedlem[#1]\refstepcounter{lem}\def\@currentlabel{(#1)}}%
    {\namedlem}%
 }
 {%
  \endnamedlem
 }
\makeatother

\newtheorem*{namedcor}{Corollary}
\newcounter{cor}%makes pointer correct
\providecommand{\corname}{Corollary}

\makeatletter
\NewDocumentEnvironment{cor*}{o}
 {%
  \IfValueTF{#1}
    {\namedcor[#1]\refstepcounter{cor}\def\@currentlabel{(#1)}}%
    {\namedcor}%
 }
 {%
  \endnamedcor
 }
\makeatother

\theoremstyle{definition}
\newtheorem*{annotation}{Annotation}
\newtheorem*{rubric}{Rubric}

\newtheorem*{innerrem}{Remark}
\newcounter{rem}%makes pointer correct
\providecommand{\remname}{Remark}

\makeatletter
\NewDocumentEnvironment{rem}{o}
 {%
  \IfValueTF{#1}
    {\innerrem[#1]\refstepcounter{rem}\def\@currentlabel{(#1)}}%
    {\innerrem}%
 }
 {%
  \endinnerrem
 }
\makeatother

\newtheorem*{innerdefn}{Definition}%%placeholder
\newcounter{defn}%makes pointer correct
\providecommand{\defnname}{Definition}

\makeatletter
\NewDocumentEnvironment{defn}{o}
 {%
  \IfValueTF{#1}
    {\innerdefn[#1]\refstepcounter{defn}\def\@currentlabel{(#1)}}%
    {\innerdefn}%
 }
 {%
  \endinnerdefn
 }
\makeatother

\newtheorem*{scratch}{Scratch Work}


\newtheorem*{namedconj}{Conjecture}
\newcounter{conj}%makes pointer correct
\providecommand{\conjname}{Conjecture}
\makeatletter
\NewDocumentEnvironment{conj}{o}
 {%
  \IfValueTF{#1}
    {\innerconj[#1]\refstepcounter{conj}\def\@currentlabel{(#1)}}%
    {\innerconj}%
 }
 {%
  \endinnerconj
 }
\makeatother

\newtheorem*{poll}{Poll question}
\newtheorem{tps}{Think-Pair-Share}[section]


\newenvironment{obj}{
	\textbf{Learning Objectives.} By the end of class, students will be able to:
		\begin{itemize}}
		{\!.\end{itemize}
		}

<<<<<<< Updated upstream
\newenvironment{pre}{
	\begin{description}
	}{
	\end{description}
}
=======

\ifinstructornotes
\newenvironment{pre}
  {{\textbf Reading assignment:}
  \begin{description}
    }{
	\end{description}
  }
\else
\newenvironment{pre}{ 
  \begin{trivlist}
  \item[]}
  {\end{trivlist}}
\fi
>>>>>>> Stashed changes


\newcounter{ex}%makes pointer correct
\providecommand{\exname}{Homework Problem}
\newenvironment{ex}[1][2in]%
{%Env start code
\problemEnvironmentStart{#1}{Homework Problem}
\refstepcounter{ex}
}
{%Env end code
\problemEnvironmentEnd
}

\newcommand{\inlineAnswer}[2][2 cm]{
    \ifhandout{\pdfOnly{\rule{#1}{0.4pt}}}
    \else{\answer{#2}}
    \fi
}


\ifhandout
\newenvironment{shortAnswer}[1][
    \vfill]
        {% Begin then result
        #1
            \begin{freeResponse}
            }
    {% Environment Ending Code
    \end{freeResponse}
    }
\else
\newenvironment{shortAnswer}[1][]
        {\begin{freeResponse}
            }
    {% Environment Ending Code
    \end{freeResponse}
    }
\fi

\let\question\relax
\let\endquestion\relax

\newtheoremstyle{ExerciseStyle}{\topsep}{\topsep}%%% space between body and thm
		{}                      %%% Thm body font
		{}                              %%% Indent amount (empty = no indent)
		{\bfseries}            %%% Thm head font
		{}                              %%% Punctuation after thm head
		{3em}                           %%% Space after thm head
		{{#1}~\thmnumber{#2}\thmnote{ \bfseries(#3)}}%%% Thm head spec
\theoremstyle{ExerciseStyle}
\newtheorem{br}{In-class Problem}

\newenvironment{sketch}
 {\begin{proof}[Sketch of Proof]}
 {\end{proof}}


\newcommand{\gt}{>}
\newcommand{\lt}{<}
\newcommand{\N}{\mathbb N}
\newcommand{\Q}{\mathbb Q}
\newcommand{\Z}{\mathbb Z}
\newcommand{\C}{\mathbb C}
\newcommand{\R}{\mathbb R}
\renewcommand{\H}{\mathbb{H}}
\newcommand{\lcm}{\operatorname{lcm}}
\newcommand{\nequiv}{\not\equiv}
\newcommand{\ord}{\operatorname{ord}}
\newcommand{\ds}{\displaystyle}
\newcommand{\floor}[1]{\left\lfloor #1\right\rfloor}
\newcommand{\legendre}[2]{\left(\frac{#1}{#2}\right)}



%%%%%%%%%%%%



\title{Sums of squares}
\begin{document}
\begin{abstract}
\end{abstract}
\maketitle

%%%%%%%%%%%%%%%%%%%%%%%%%%

\begin{pre}
    \item[Reading] None
\end{pre}

The first result will prove which primes can be written as the sum of two squares. Note $1^2+1^2=2,$ and if $a$ is a positive integer such that $a\equiv 3 \pmod 4$, then $a$ cannot be written as the sum of two squares.


\begin{proposition}\label{prop:prod-sum-squares}
 Let $m,n\in\Z$ with $m,n>0$. If $m$ and $n$ can be written as the sums of two squares of integers, then $mn$ can be written as the sum of two squares of integers.

 \begin{proof}
	Let $m,n\in\Z$ with $m,n>0$ and assume that there exists $a,b,c,d\in\Z$ such that $m=a^2+b^2$ and $n=c^2+d^2.$ Then 
	\begin{align*}
		mn=(a^2+b^2)(c^2+d^2)&=a^2c^2+b^2c^2+a^2d^2+b^2d^2\\
		&=a^2c^2+2abcd+b^2d^2+a^2d^2-2abcd+b^2c^2\\
		&=(ac+bd)^2+(ad-bc)^2.\qedhere
	\end{align*}
 \end{proof}
\end{proposition}

We need two lemmas to prove 
\begin{theorem}\label{thm:express-sum-sqrs}
	Let $n\in\mathbb{Z}$ with $n>0$. Then $n$ is expressible as the sum of two squares if and only if every prime factor congruent to $3$ modulo  $4$ occurs to an even power in the prime factorization of $n$.
\end{theorem}

\begin{lemma}\label{lem:prime-sum-sqrs}
	If $p$ is a prime such that $p\equiv 1\pmod{4},$ then there are integers $x,y$ such that $x^2+y^2=kp$ for some $k\in\Z$ with $0<k<p.$

	\begin{proof}
		Since $p\equiv 1 \pmod 4$, we have that $\left(\frac{-1}{p}\right)=1$. Thus, there exists $x\in\mathbb{Z}$ with $0<x\leq\frac{p-1}{2}$ such that $x^2\equiv -1 \pmod p$. Then, $p\mid x^2+1$, and we have that $x^2+1=kp$ for some $k\in\mathbb{Z}$. Thus, we found $x$ and $y=1$. Since $x^2+1$ and $p$ are positive, so is $k$. Also, \[kp=x^2+y^2<\left(\frac{p}{2}\right)^2+1<p^2\] implies $k<p$.
	\end{proof}
\end{lemma}

\begin{proposition}\label{prop:primes-sum-sqrs}
	A prime $p$ can be written as the sum of two squares if and only if $p=2$ or $p\equiv 1 \pmod4$.
   
   \begin{proof}
	   If $p\equiv 3 \pmod 4$, then $p$ cannot be written as the sum of two squares. Since the squares modulo $4$ are $0$ and $1$, the integers that can be written as a sum of two squares are congruent to $0^2+0^2\equiv 0 \pmod 4, 1^2+0^2\equiv 1 \pmod 4$ or $1^2+1^2\equiv 2 \pmod 4$, so no integer that is congruent to $3\pmod 4$ can be written as the sum of two squares. Thus, if $p$ can be written as the sum of two squares, $p\nequiv 3 \pmod 4.$ That is, $p=2$ or $p\equiv 1 \pmod 4$.
	
	   We will prove the other direction with two cases. When $p=2$, then $1^2+1^2=2$. It remains to show that every prime $p\equiv 1 \pmod 4$ can be written as the sum of two squares.
	
	   Let $p\equiv 1\pmod 4$, and let $m$ be the least integer such that there exists $x,y\in\mathbb{Z}$ with $x^2+y^2=mp$ and $0<m<p$ as in the previous theorem. We show that $m=1$. Assume, by way of contradiction, that $m>1$. Let $a,b\in\mathbb{Z}$ such that \[a\equiv x\pmod m,\quad \frac{-m}{2}<a\leq\frac{m}{2}\] and \[b\equiv y\pmod m,\quad \frac{-m}{2}<b\leq\frac{m}{2}.\] Then \[a^2+b^2\equiv x^2+y^2=mp\equiv 0\pmod m,\] and so there exists $k\in\mathbb{Z}$ with $k>0$ such that $a^2+b^2=km$. (Why?)
	 
	   Now, \[(a^2+b^2)(x^2+y^2)=(km)(mp)=km^2p.\] By \nameref{lem:prime-sum-sqrs}, $(a^2+b^2)(x^2+y^2)=(ax+by)^2+(ay-bx)^2$, so $(ax+by)^2+(ay-bx)^2=km^2p$. Since $a\equiv x\pmod m$ and $b\equiv y\pmod m$, \[ax+by\equiv x^2+y^2\equiv 0\pmod m\] and \[ay-bx\equiv xy-yx\equiv 0\pmod m\] so $\frac{ax+by}{m},\frac{ay-bx}{m}\in\mathbb{Z}$ and \[\left(\frac{ax+by}{m}\right)^2+\left(\frac{ay-bx}{m}\right)^2=\frac{km^2p}{m^2}=kp.\]  Now, $\frac{-m}{2}<a\leq\frac{m}{2}$ and $\frac{-m}{2}<b\leq\frac{m}{2}$  imply that $a^2\leq\frac{m^2}{4}$ and $b^2\leq\frac{m^2}{4}$. Thus, $km=a^2+b^2\leq\frac{m^2}{2}$. Thus, $k\leq \frac{m}{2}<m$, but this contradicts that $m$ is the smallest such integer.
		
	   Thus, $m=1$ and $p$ can be written as the sum of two squares of integers.
   \end{proof}
   \end{proposition}


\end{document}
