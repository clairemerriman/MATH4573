\documentclass{ximera}
<<<<<<< Updated upstream
\usepackage{amssymb, latexsym, amsmath, amsthm, graphicx, amsthm,alltt,color, listings,multicol,xr-hyper,hyperref,aliascnt,enumitem}
=======
\usepackage{amssymb, latexsym, amsmath, amsthm, graphicx, amsthm,alltt,color, listings,multicol,hyperref}
\usepackage[capitalise,nameinlink]{cleveref}
>>>>>>> Stashed changes
\usepackage{xfrac}

\usepackage{parskip}
\usepackage[,margin=0.7in]{geometry}
\setlength{\textheight}{8.5in}

\usepackage{epstopdf}

\DeclareGraphicsExtensions{.eps}
\usepackage{tikz}


\usepackage{tkz-euclide}
%\usetkzobj{all}
\tikzstyle geometryDiagrams=[rounded corners=.5pt,ultra thick,color=black]
\colorlet{penColor}{black} % Color of a curve in a plot


\usepackage{subcaption}
\usepackage{float}
\usepackage{fancyhdr}
\usepackage{pdfpages}
\newcounter{includepdfpage}
\usepackage{makecell}


\usepackage{currfile}
\usepackage{xstring}




\graphicspath{  
{./otherDocuments/}
}

\author{Claire Merriman}
\newcommand{\classday}[1]{\def\classday{#1}}

%%%%%%%%%%%%%%%%%%%%%
% Counters and autoref for unnumbered environments
% Not needed??
%%%%%%%%%%%%%%%%%%%%%
<<<<<<< Updated upstream
\theoremstyle{plain}


\newtheorem*{namedthm}{Theorem}
\newcounter{thm}%makes pointer correct
\providecommand{\thmname}{Theorem}
=======

\crefname{problem}{problem}{problems}


% \theoremstyle{plain}


% \newtheorem*{namedthm}{Theorem}
% \newcounter{thm}%makes pointer correct
% \providecommand{\thmname}{Theorem}
>>>>>>> Stashed changes

\makeatletter
\NewDocumentEnvironment{thm*}{o}
 {%
  \IfValueTF{#1}
    {\namedthm[#1]\refstepcounter{thm}\def\@currentlabel{(#1)}}%
    {\namedthm}%
 }
 {%
  \endnamedthm
 }
\makeatother


\newtheorem*{namedprop}{Proposition}
\newcounter{prop}%makes pointer correct
\providecommand{\propname}{Proposition}

\makeatletter
\NewDocumentEnvironment{prop*}{o}
 {%
  \IfValueTF{#1}
    {\namedprop[#1]\refstepcounter{prop}\def\@currentlabel{(#1)}}%
    {\namedprop}%
 }
 {%
  \endnamedprop
 }
\makeatother

\newtheorem*{namedlem}{Lemma}
\newcounter{lem}%makes pointer correct
\providecommand{\lemname}{Lemma}

\makeatletter
\NewDocumentEnvironment{lem*}{o}
 {%
  \IfValueTF{#1}
    {\namedlem[#1]\refstepcounter{lem}\def\@currentlabel{(#1)}}%
    {\namedlem}%
 }
 {%
  \endnamedlem
 }
\makeatother

\newtheorem*{namedcor}{Corollary}
\newcounter{cor}%makes pointer correct
\providecommand{\corname}{Corollary}

\makeatletter
\NewDocumentEnvironment{cor*}{o}
 {%
  \IfValueTF{#1}
    {\namedcor[#1]\refstepcounter{cor}\def\@currentlabel{(#1)}}%
    {\namedcor}%
 }
 {%
  \endnamedcor
 }
\makeatother

\theoremstyle{definition}
\newtheorem*{annotation}{Annotation}
\newtheorem*{rubric}{Rubric}

\newtheorem*{innerrem}{Remark}
\newcounter{rem}%makes pointer correct
\providecommand{\remname}{Remark}

\makeatletter
\NewDocumentEnvironment{rem}{o}
 {%
  \IfValueTF{#1}
    {\innerrem[#1]\refstepcounter{rem}\def\@currentlabel{(#1)}}%
    {\innerrem}%
 }
 {%
  \endinnerrem
 }
\makeatother

\newtheorem*{innerdefn}{Definition}%%placeholder
\newcounter{defn}%makes pointer correct
\providecommand{\defnname}{Definition}

\makeatletter
\NewDocumentEnvironment{defn}{o}
 {%
  \IfValueTF{#1}
    {\innerdefn[#1]\refstepcounter{defn}\def\@currentlabel{(#1)}}%
    {\innerdefn}%
 }
 {%
  \endinnerdefn
 }
\makeatother

\newtheorem*{scratch}{Scratch Work}


\newtheorem*{namedconj}{Conjecture}
\newcounter{conj}%makes pointer correct
\providecommand{\conjname}{Conjecture}
\makeatletter
\NewDocumentEnvironment{conj}{o}
 {%
  \IfValueTF{#1}
    {\innerconj[#1]\refstepcounter{conj}\def\@currentlabel{(#1)}}%
    {\innerconj}%
 }
 {%
  \endinnerconj
 }
\makeatother

\newtheorem*{poll}{Poll question}
\newtheorem{tps}{Think-Pair-Share}[section]


\newenvironment{obj}{
	\textbf{Learning Objectives.} By the end of class, students will be able to:
		\begin{itemize}}
		{\!.\end{itemize}
		}

<<<<<<< Updated upstream
\newenvironment{pre}{
	\begin{description}
	}{
	\end{description}
}
=======

\ifinstructornotes
\newenvironment{pre}
  {{\textbf Reading assignment:}
  \begin{description}
    }{
	\end{description}
  }
\else
\newenvironment{pre}{ 
  \begin{trivlist}
  \item[]}
  {\end{trivlist}}
\fi
>>>>>>> Stashed changes


\newcounter{ex}%makes pointer correct
\providecommand{\exname}{Homework Problem}
\newenvironment{ex}[1][2in]%
{%Env start code
\problemEnvironmentStart{#1}{Homework Problem}
\refstepcounter{ex}
}
{%Env end code
\problemEnvironmentEnd
}

\newcommand{\inlineAnswer}[2][2 cm]{
    \ifhandout{\pdfOnly{\rule{#1}{0.4pt}}}
    \else{\answer{#2}}
    \fi
}


\ifhandout
\newenvironment{shortAnswer}[1][
    \vfill]
        {% Begin then result
        #1
            \begin{freeResponse}
            }
    {% Environment Ending Code
    \end{freeResponse}
    }
\else
\newenvironment{shortAnswer}[1][]
        {\begin{freeResponse}
            }
    {% Environment Ending Code
    \end{freeResponse}
    }
\fi

\let\question\relax
\let\endquestion\relax

\newtheoremstyle{ExerciseStyle}{\topsep}{\topsep}%%% space between body and thm
		{}                      %%% Thm body font
		{}                              %%% Indent amount (empty = no indent)
		{\bfseries}            %%% Thm head font
		{}                              %%% Punctuation after thm head
		{3em}                           %%% Space after thm head
		{{#1}~\thmnumber{#2}\thmnote{ \bfseries(#3)}}%%% Thm head spec
\theoremstyle{ExerciseStyle}
\newtheorem{br}{In-class Problem}

\newenvironment{sketch}
 {\begin{proof}[Sketch of Proof]}
 {\end{proof}}


\newcommand{\gt}{>}
\newcommand{\lt}{<}
\newcommand{\N}{\mathbb N}
\newcommand{\Q}{\mathbb Q}
\newcommand{\Z}{\mathbb Z}
\newcommand{\C}{\mathbb C}
\newcommand{\R}{\mathbb R}
\renewcommand{\H}{\mathbb{H}}
\newcommand{\lcm}{\operatorname{lcm}}
\newcommand{\nequiv}{\not\equiv}
\newcommand{\ord}{\operatorname{ord}}
\newcommand{\ds}{\displaystyle}
\newcommand{\floor}[1]{\left\lfloor #1\right\rfloor}
\newcommand{\legendre}[2]{\left(\frac{#1}{#2}\right)}



%%%%%%%%%%%%



\title{Proving Gauss's Lemma and the Quadratic Residue of $2$}
\begin{document}
\begin{abstract}
\end{abstract}
\maketitle

\begin{obj}
    \item Prove \nameref{lem:gauss}
	\item Classify when $2$ is a quadratic residue modulo a prime.
\end{obj}


\begin{pre}
    \item[Read:] The remainder of Strayer Section 4.2. For a slower approach, \cite{theo-gauss-lem} 
	is on Moodle.
    \item[Turn in] The $p\equiv 3\pmod{9}$
	 case of Strayer Theorem 4.8 (scanned notes Theorem 11.4.3/Theorem 11.4.4)

	 \begin{solution}
		Mirroring Strayer's argument for $p\equiv 1\pmod{8}:$

		If $p\equiv 3\pmod{8},$ then there exists $k\in\Z$ such that $p=8k+3.$ Then 
			\begin{align*}
				\frac{p-1}{2}-\floor{ \frac{p}{4}}
				&=\frac{8k+3-1}{2}-\floor{\frac{8k+3}{4}}
				=4k+1-(2k)=2k+1
				\equiv 1\pmod{2},
			\end{align*}
		and 
			\begin{align*}
				\frac{p^2-1}{8}=\frac{(8k+3)^2-1}{8}=\frac{64k^2+48k+9-1}{8}=8k^2+6k+1\equiv 1\pmod{2}.
			\end{align*}
		Thus, $\frac{p-1}{2}-\floor{ \frac{p}{4}}\equiv \frac{p^2-1}{8}\pmod{8}.$
	 \end{solution}
\end{pre}


\begin{remark}
	\nameref{lem:gauss} is often stated as:

	Let $p$ be an odd prime number and like $a\in\Z$ with $p\nmid a$. Let $n$ be the number of least absolute residues of the integers $a,2a,3a,\dots,\frac{p-1}{2}$ modulo $p$ that are negative. Then \[\legendre{a}{p}=(-1)^n.\]
\end{remark}


\begin{lemma}\label{lem:residues-gauss-lem}
	Let $p$ be an odd prime number and like $a\in\Z$ with $p\nmid a.$ Consider  
	\[a,2a,3a,\dots,\frac{p-1}{2}a,\frac{p+1}{2}a,\dots,(p-1)a.\] The least absolute residues of $ak$ and $a(p-k)$ differ by a negative sign. In other words, 
	\[ak\equiv -a(p-k)\pmod{p}.\]
	
	Furthermore, for each $k=1,2,\dots,\frac{p-1}{2},$ the exactly one of $k$ and $-k$ is a least absolute residue of $\{a,2a,3a,\dots,\frac{p-1}{2}a\}.$
\end{lemma}

\begin{proof}
	Let $p$ be an odd prime number and let $a\in\Z$ with $p\nmid a.$  Then \[ak\equiv-ap+ak\equiv -a(p-k)\pmod{p}.\]
	
	Then \[\{a,2a,3a,\dots,\frac{p-1}{2}a,\frac{p+1}{2}a,\dots,(p-1)a\}\] is a reduced residue system modulo $p$ by \nameref{lem:reduced-sys}. From Chapter 2, \nameref{ch2-ex74}, \[\left\{\frac{-(p-1)}{2}, \frac{-(p-3)}{2},\dots, -2,-1,1,2,\dots,\frac{p-2}{2},\frac{p-1}{2}\right\}\]
	is a reduced residue system modulo $p$. Thus, every element of $\{a,2a,\dots,\frac{p-1}{2}a,\frac{p+1}{2}a,\dots,(p-1)a\}$ is congruent to exactly one of $\left\{\frac{-(p-1)}{2}, \dots, -1,1,\dots,\frac{p-2}{2},\frac{p-1}{2}\right\}.$ That is, for each $k=1,2,\dots,\frac{p-1}{2},$ both $k$ and $-k$ are least absolute residue of $\{a,2a,3a,\dots,\frac{p-1}{2}a,\frac{p+1}{2}a,\dots,(p-1)a\}.$
	
	If $k$ is the least absolute remainder of $aj$ modulo $p$ for some $j=1,2,\dots, \frac{p-1}{2},$ then $-k$ is the absolute least residue of $a(p-j)$ modulo $p$ and $p-j=\frac{p+1}{2},\frac{p+3}{2},\dots,p-1.$ Thus, $-k$ is not an absolute least residue of $\{a,2a,3a,\dots,\frac{p-1}{2}a\}.$ Since there are $\frac{p-1}{2}$ elements of $\{a,2a,3a,\dots,\frac{p-1}{2}a\},$ there must be $\frac{p-1}{2}$ distinct absolute least residues modulo $p.$ Thus, for each $k=1,2,\dots,\frac{p-1}{2},$ exactly on of $k$ and $-k$ is an absolute least residue of $\{a,2a,3a,\dots,\frac{p-1}{2}a\}.$
\end{proof}

\begin{br}
	Check \hyperref[lem:residues-gauss-lem]{Lemma 1} for 
	\begin{enumerate}
		\item $a=3, p=7$
		\item $a=5, p=11$
		\item $a=6, p=11$
	\end{enumerate}

	\begin{solution}
		\begin{enumerate}

		\item $a=3, p=7$ 
			\begin{align*}
				3\pmod{7},
				\ 3(2)\equiv -1\pmod{7},
				\ 3(3)\equiv 2\pmod{7},\\
				\ 3(4)\equiv -2\pmod{7},
				\ 3(5)\equiv 1\pmod{7},
				\ 3(6)\equiv -3\pmod{7},
			\end{align*}
		\item $a=5, p=11$
			\begin{align*}
				5\pmod{11},
				\ 5(2)\equiv -1\pmod{11},
				\ 5(3)\equiv 4\pmod{11},\\
				\ 5(4)\equiv -2\pmod{11},
				\ 5(5)\equiv 3\pmod{11},\\
				\ 5(6)\equiv -3\pmod{11},
				\ 5(7)\equiv -2\pmod{11},
				\ 5(8)\equiv -4\pmod{11},\\
				\ 5(9)\equiv 1\pmod{11},
				\ 5(10)\equiv -5\pmod{11},
			\end{align*}
		\item $a=11, p=23$
			\begin{align*}
				11\pmod{23},
				\ 11(2)\equiv -1\pmod{23},
				\ 11(3)\equiv 10\pmod{23},\\
				\ 11(4)\equiv -2\pmod{23},
				\ 11(5)\equiv 9\pmod{23},
				\ 11(6)\equiv -3\pmod{23},\\
				\ 11(7)\equiv 8\pmod{23},
				\ 11(8)\equiv -4\pmod{23},
				\ 11(9)\equiv 7\pmod{23},\\
				\ 11(10)\equiv -5\pmod{23},
				\ 11(11)\equiv 6\pmod{23},\\
				\ 11(12)\equiv -6\pmod{23},
				\ 11(13)\equiv 5\pmod{23},\\
				\ 11(14)\equiv -7\pmod{23},
				\ 11(15)\equiv 4\pmod{23},
				\ 11(16)\equiv -8\pmod{23},\\
				\ 11(17)\equiv 3\pmod{23},
				\ 11(18)\equiv -9\pmod{23},
				\ 11(19)\equiv 2\pmod{23},\\
				\ 11(20)\equiv -10\pmod{23},
				\ 11(21)\equiv 1\pmod{23},
				\ 11(22)\equiv -11\pmod{23},
			\end{align*}
		\end{enumerate}
	\end{solution}
\end{br}

We now prove \nameref{lem:gauss}.


\begin{proof}
 	Let $r_1,r_2,\dots r_n$ be the least nonnegative residues of the integers $a,2a,\dots,\frac{p-1}{2}a$ that are greater than $\frac{p}{2}$ and $s_1,s_2,\dots,s_m$ be the least nonnegative residues that are less that $\frac{p}{2}$. Note that no $r_i$ or $s_j$ is $0,$ since $p$ does not divide any of $a,2a,\dots \frac{p-1}{2}$. Consider the $\frac{p-1}{2}$ integers given by 
		\[p-r_1,p-r_2,\dots,p-r_n,s_1,s_2,\dots,s_m.\]
 	We want to show that these integers are the integers from $1$ to $\frac{p-1}{2},$ inclusive, in some order. Since each integer is less than or equal to $\frac{p-1}{2}$, it suffices to show that no two of these integers are congruent modulo $p$. 
 
	If $p-r_i\equiv p-r_j \pmod p$ for some $i\neq j$, then $r_i\equiv r_j \pmod p$, but this implies that there exists some $k_i,k_j\in\mathbb{Z}$ such that $r_i=k_ia\equiv k_ja=r_j\pmod p$ with $k_i\neq k_j$ and $1\leq k_i,k_j\leq\answer{\frac{p-1}{2}}
	$. Since 
	\begin{multipleChoice}
 		\choice[correct] {$p\nmid a$}
 		\choice {$p\mid a$}
	\end{multipleChoice}
 	we know that the multiplicative inverse of $a$ modulo $p$ 
	\begin{multipleChoice}
 		\choice[correct] {exists}
 		\choice {does not exist}
	\end{multipleChoice}
 	and thus $k_i\equiv k_j \pmod p$, a contradiction. Thus, no two of the first $n$ integers are congruent modulo $p$. 
 
 	Similarly, no two of the second $m$ integers are congruent. Now, if $p-r_ij \pmod p$, for some $i$ and $j$, then $-r_ij \pmod p$. Thus, there exists $k_i,k_j\in\mathbb{Z}$ such that $-r_i=-k_ia\equiv k_ja=s_j\pmod p$ with $k_i\neq k_j$ and $1\leq k_i,k_j\leq\frac{p-1}{2}$. Since $p\nmid a$, we know that the multiplicative inverse of $a$ modulo $p$ exists, and thus $-k_i\equiv k_j \pmod p$, a contradiction.
	Thus, the $\frac{p-1}{2}$ integers $p-r_1,p-r_2,\dots,p-r_n,s_1,s_2,\dots,s_m$ are the integers $1,2,\dots,\frac{p-1}{2}$ in some order. 

	Then, 
		\[
			(p-r_1)(p-r_2)\cdots(p-r_n)s_1s_2\cdots s_m
			\equiv\frac{p-1}{2}! \pmod p
		\]
	implies that  
		\[
			(-1)^nr_1r_2\cdots r_ns-1s_2\cdots s_m
			\equiv\frac{p-1}{2}! \pmod p. 
		\]
	
	By the definition of $r_i$ and $s_j$, we have 
		\[
			(-1)^na(2a)(3a)\cdots(\frac{p-1}{2}a)
			\equiv\frac{p-1}{2}! \pmod p. 
		\]
	
	By reordering, we have 
		\[
			(-1)^na^{\frac{p-1}{2}}\frac{p-1}{2}!\equiv\frac{p-1}{2}! \pmod p. 
		\]
	
	Thus, $(-1)^na^{\frac{p-1}{2}}\equiv 1 \pmod p$, and $a^{\frac{p-1}{2}}\equiv (-1)^n \pmod p$. By Euler's criterion, we get that $\left(\frac{a}{p}\right)\equiv(-1)^n \pmod p$. Since both sides of the congruence must be $\pm1,$ we have $\left(\frac{a}{p}\right)=(-1)^n $.
\end{proof}


We are going to prove a result about $\left(\frac{2}{p}\right)$ before our next technical lemma.

\begin{theorem}
 Let $p$ be an odd prime. Then 
	\begin{equation*}
	\left(\frac{2}{p}\right)=(-1)^{\frac{p^2-1}{8}}=
		\begin{cases}
		1& if\ p\equiv 1,7 \pmod 8\\
		-1 & if\ p\equiv 3,5 \pmod 8.
		\end{cases}
	\end{equation*}

	\begin{proof}
	By Gauss's Lemma, we have that $\left(\frac{2}{p}\right)=(-1)^n,$ where $n$ is the number of least positive residues of the integers $2,2*2,2*3,\dots,\frac{p-1}{2}$ that are greater than $\frac{p}{2}$. Let $k\in\mathbb{Z}$ with $1\leq k\leq \frac{p-1}{2}$. Then $2k<{\frac{p}{2}}
	$ if and only  if $k<\frac{p}{4};$ so $\floor{\frac{p}{4}}
	$ of the integers $2,2*2,2*3,\dots,\frac{p-1}{2}$ that are less than $\frac{p}{2}$, where $\floor{\cdot}$ is the greatest integer (or floor) function. So, $\frac{p-1}{2}-\floor{\frac{p}{4}}$  of these integers are greater than $\frac{p}{2}$, from which 
			\[
				\left(\frac{2}{p}\right)=
				(-1)^{\frac{p-1}{2}-\floor{\frac{p}{4}}} 
			\]
		by Gauss's Lemma. For the first equality, it suffices to show that 
			\[
				\frac{p-1}{2}-\floor{\frac{p}{4}}
				\equiv \frac{p^2-1}{8} \pmod 2. 
			\]
	
	If $p\equiv 1 \pmod 8$, the $p=8k+1$ for some $k\in\mathbb{Z}$. That gives us
			\[
				\frac{p-1}{2}-\floor{\frac{p}{4}}
				=\frac{(8k+1)-1}{2}-\floor{\frac{8k+1}{4}}
				=4k-2k
				=2k\equiv 0 \pmod 2 
			\]
		
		and
			\[
				\frac{p^2-1}{8}
				=\frac{(8k+1)^2-1}{8}=8k^2+2k\equiv 0\pmod 2.
			\]
	Thus,  holds when $p\equiv 1 \pmod 8$. The rest of the cases are left as homework.
	\end{proof}
\end{theorem}


\end{document}
