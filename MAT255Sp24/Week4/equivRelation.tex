\documentclass{ximera}
\usepackage{amssymb, latexsym, amsmath, amsthm, graphicx, amsthm,alltt,color, listings,multicol,xr-hyper,hyperref,aliascnt,enumitem}
\usepackage{xfrac}

\usepackage{parskip}
\usepackage[,margin=0.7in]{geometry}
\setlength{\textheight}{8.5in}

\usepackage{epstopdf}

\DeclareGraphicsExtensions{.eps}
\usepackage{tikz}


\usepackage{tkz-euclide}
%\usetkzobj{all}
\tikzstyle geometryDiagrams=[rounded corners=.5pt,ultra thick,color=black]
\colorlet{penColor}{black} % Color of a curve in a plot


\usepackage{subcaption}
\usepackage{float}
\usepackage{fancyhdr}
\usepackage{pdfpages}
\newcounter{includepdfpage}
\usepackage{makecell}


\usepackage{currfile}
\usepackage{xstring}




\graphicspath{  
{./otherDocuments/}
}

\author{Claire Merriman}
\newcommand{\classday}[1]{\def\classday{#1}}

%%%%%%%%%%%%%%%%%%%%%
% Counters and autoref for unnumbered environments
% Not needed??
%%%%%%%%%%%%%%%%%%%%%
\theoremstyle{plain}


\newtheorem*{namedthm}{Theorem}
\newcounter{thm}%makes pointer correct
\providecommand{\thmname}{Theorem}

\makeatletter
\NewDocumentEnvironment{thm*}{o}
 {%
  \IfValueTF{#1}
    {\namedthm[#1]\refstepcounter{thm}\def\@currentlabel{(#1)}}%
    {\namedthm}%
 }
 {%
  \endnamedthm
 }
\makeatother


\newtheorem*{namedprop}{Proposition}
\newcounter{prop}%makes pointer correct
\providecommand{\propname}{Proposition}

\makeatletter
\NewDocumentEnvironment{prop*}{o}
 {%
  \IfValueTF{#1}
    {\namedprop[#1]\refstepcounter{prop}\def\@currentlabel{(#1)}}%
    {\namedprop}%
 }
 {%
  \endnamedprop
 }
\makeatother

\newtheorem*{namedlem}{Lemma}
\newcounter{lem}%makes pointer correct
\providecommand{\lemname}{Lemma}

\makeatletter
\NewDocumentEnvironment{lem*}{o}
 {%
  \IfValueTF{#1}
    {\namedlem[#1]\refstepcounter{lem}\def\@currentlabel{(#1)}}%
    {\namedlem}%
 }
 {%
  \endnamedlem
 }
\makeatother

\newtheorem*{namedcor}{Corollary}
\newcounter{cor}%makes pointer correct
\providecommand{\corname}{Corollary}

\makeatletter
\NewDocumentEnvironment{cor*}{o}
 {%
  \IfValueTF{#1}
    {\namedcor[#1]\refstepcounter{cor}\def\@currentlabel{(#1)}}%
    {\namedcor}%
 }
 {%
  \endnamedcor
 }
\makeatother

\theoremstyle{definition}
\newtheorem*{annotation}{Annotation}
\newtheorem*{rubric}{Rubric}

\newtheorem*{innerrem}{Remark}
\newcounter{rem}%makes pointer correct
\providecommand{\remname}{Remark}

\makeatletter
\NewDocumentEnvironment{rem}{o}
 {%
  \IfValueTF{#1}
    {\innerrem[#1]\refstepcounter{rem}\def\@currentlabel{(#1)}}%
    {\innerrem}%
 }
 {%
  \endinnerrem
 }
\makeatother

\newtheorem*{innerdefn}{Definition}%%placeholder
\newcounter{defn}%makes pointer correct
\providecommand{\defnname}{Definition}

\makeatletter
\NewDocumentEnvironment{defn}{o}
 {%
  \IfValueTF{#1}
    {\innerdefn[#1]\refstepcounter{defn}\def\@currentlabel{(#1)}}%
    {\innerdefn}%
 }
 {%
  \endinnerdefn
 }
\makeatother

\newtheorem*{scratch}{Scratch Work}


\newtheorem*{namedconj}{Conjecture}
\newcounter{conj}%makes pointer correct
\providecommand{\conjname}{Conjecture}
\makeatletter
\NewDocumentEnvironment{conj}{o}
 {%
  \IfValueTF{#1}
    {\innerconj[#1]\refstepcounter{conj}\def\@currentlabel{(#1)}}%
    {\innerconj}%
 }
 {%
  \endinnerconj
 }
\makeatother

\newtheorem*{poll}{Poll question}
\newtheorem{tps}{Think-Pair-Share}[section]


\newenvironment{obj}{
	\textbf{Learning Objectives.} By the end of class, students will be able to:
		\begin{itemize}}
		{\!.\end{itemize}
		}

\newenvironment{pre}{
	\begin{description}
	}{
	\end{description}
}


\newcounter{ex}%makes pointer correct
\providecommand{\exname}{Homework Problem}
\newenvironment{ex}[1][2in]%
{%Env start code
\problemEnvironmentStart{#1}{Homework Problem}
\refstepcounter{ex}
}
{%Env end code
\problemEnvironmentEnd
}

\newcommand{\inlineAnswer}[2][2 cm]{
    \ifhandout{\pdfOnly{\rule{#1}{0.4pt}}}
    \else{\answer{#2}}
    \fi
}


\ifhandout
\newenvironment{shortAnswer}[1][
    \vfill]
        {% Begin then result
        #1
            \begin{freeResponse}
            }
    {% Environment Ending Code
    \end{freeResponse}
    }
\else
\newenvironment{shortAnswer}[1][]
        {\begin{freeResponse}
            }
    {% Environment Ending Code
    \end{freeResponse}
    }
\fi

\let\question\relax
\let\endquestion\relax

\newtheoremstyle{ExerciseStyle}{\topsep}{\topsep}%%% space between body and thm
		{}                      %%% Thm body font
		{}                              %%% Indent amount (empty = no indent)
		{\bfseries}            %%% Thm head font
		{}                              %%% Punctuation after thm head
		{3em}                           %%% Space after thm head
		{{#1}~\thmnumber{#2}\thmnote{ \bfseries(#3)}}%%% Thm head spec
\theoremstyle{ExerciseStyle}
\newtheorem{br}{In-class Problem}

\newenvironment{sketch}
 {\begin{proof}[Sketch of Proof]}
 {\end{proof}}


\newcommand{\gt}{>}
\newcommand{\lt}{<}
\newcommand{\N}{\mathbb N}
\newcommand{\Q}{\mathbb Q}
\newcommand{\Z}{\mathbb Z}
\newcommand{\C}{\mathbb C}
\newcommand{\R}{\mathbb R}
\renewcommand{\H}{\mathbb{H}}
\newcommand{\lcm}{\operatorname{lcm}}
\newcommand{\nequiv}{\not\equiv}
\newcommand{\ord}{\operatorname{ord}}
\newcommand{\ds}{\displaystyle}
\newcommand{\floor}[1]{\left\lfloor #1\right\rfloor}
\newcommand{\legendre}[2]{\left(\frac{#1}{#2}\right)}



%%%%%%%%%%%%



\title{Equivalence Relations}
\begin{document}
\begin{abstract}
\end{abstract}
\maketitle

\begin{obj}
  \item Prove a given set is an equivalence relation
\end{obj}


\begin{instructorNotes}

\begin{pre}
  \item[Reading] Strayer, Appendix B
  \item[Turn in] Let $R$ be the equivalence relation on $\R$
  defined by
  \[
 [a]=\{b\in\R:\sin(a)=\sin(b) \textnormal{ and }\cos(a)=\cos(b)\}.\]
   Prove that $R$ is an equivalence relation on $\R.$
 Describe the equivalence classes on $\R$
 
 \begin{solution}
      Since $\sin(a)=\sin(a)$ and $\cos(a)=\cos(a),$ the relation $R$ is reflexive.

      If $\sin(a)=\sin(b)$ and $\cos(a)=\cos(b),$ the $\sin(b)=\sin(a)$ and $\cos(b)=\cos(a),$ so the relation is symmetric.

      If $\sin(a)=\sin(b)$ and $\cos(a)=\cos(b),$ $\sin(b)=\sin(c)$ and $\cos(b)=\cos(c),$ then $\sin(a)=\sin(c)$ and $\cos(a)=\cos(c)$ is transitive.

      Note that $\sin(a)=\sin(b)$ if $b=a+2\pi k$ or $b=-a+\pi+2\pi k$ for some $k\in\Z,$ and  $\cos(a)=\cos(b)$ if $b=a+2\pi k$ or $b=-a+2\pi k$ for some $k\in\Z.$ These conditions are both true with $b=a+2\pi k$. Thus, for $a\in[0,2\pi),$
      \[
          [a]=\{\dots,a-4\pi,a-2\pi,a,a+2\pi,a+4\pi,\dots\}.
      \]
     \end{solution}
\end{pre}
  
\end{instructorNotes}
%%%%%%%%%%%%%%%%%%%%%%%%%%

\begin{br}
    Prove that \[[a]=\{b\in\Z: 3\mid(a-b)\}\] is an equivalence relation on $\Z.$
  \end{br}
  
  \begin{proof}
    Let $a,b\in\Z$. We must show that the relation is reflexive, symmetric, and transitive.
  
     To show the relation is reflexive, we must show $a\in \{b\in\Z: 3\mid(a-b)\}.$ Since $\answer{3\mid a-a=0}$,\\
     $a\in \{b\in\Z: 3\mid(a-b)\}.$
  
     To show the relation is symmetric, we must show that if $x\in \{b\in\Z: 3\mid(a-b)\},$ then $a\in \{b\in\Z: 3\mid(x-b)\}.$ If $x\in \{b\in\Z: 3\mid(a-b)\},$ then $\answer{\textnormal{there exists $k\in\Z$ such that $3k=a-x$}}$. Therefore, $\answer{-3k=b-a}$
     and $a\in \{b\in\Z: 3\mid(x-b)\}.$
  
     To show the relation is transitive, we must show that if $x\in \{b\in\Z: 3\mid(a-b)\}$ and $y\in \{b\in\Z: 3\mid(x-b)\},$ then $x\in \{b\in\Z: 3\mid(a-b)\}.$ If $x\in \{b\in\Z: 3\mid(a-b)\},$ then $\answer{\textnormal{there exists $k\in\Z$ such that $3k=a-x$}}$. 
     Similarly, if $y\in \{b\in\Z: 3\mid(x-b)\},$ then $\answer{\textnormal{there exists $m\in\Z$ such that $3m=x-y$}}$.
     Therefore, $\answer{3(m+k)=a-k}$
     and $y\in \{b\in\Z: 3\mid(a-b)\}.$ $\answer{\textnormal{Since the relation is reflexive, symmetric, and transitive, it is an equivalence relation.}}$
  \end{proof}

%%%%%%%%%%%%%%%%%%%%%%%%%%




\end{document}
