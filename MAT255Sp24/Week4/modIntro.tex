\documentclass{ximera}
\usepackage{amssymb, latexsym, amsmath, amsthm, graphicx, amsthm,alltt,color, listings,multicol,xr-hyper,hyperref,aliascnt,enumitem}
\usepackage{xfrac}

\usepackage{parskip}
\usepackage[,margin=0.7in]{geometry}
\setlength{\textheight}{8.5in}

\usepackage{epstopdf}

\DeclareGraphicsExtensions{.eps}
\usepackage{tikz}


\usepackage{tkz-euclide}
%\usetkzobj{all}
\tikzstyle geometryDiagrams=[rounded corners=.5pt,ultra thick,color=black]
\colorlet{penColor}{black} % Color of a curve in a plot


\usepackage{subcaption}
\usepackage{float}
\usepackage{fancyhdr}
\usepackage{pdfpages}
\newcounter{includepdfpage}
\usepackage{makecell}


\usepackage{currfile}
\usepackage{xstring}




\graphicspath{  
{./otherDocuments/}
}

\author{Claire Merriman}
\newcommand{\classday}[1]{\def\classday{#1}}

%%%%%%%%%%%%%%%%%%%%%
% Counters and autoref for unnumbered environments
% Not needed??
%%%%%%%%%%%%%%%%%%%%%
\theoremstyle{plain}


\newtheorem*{namedthm}{Theorem}
\newcounter{thm}%makes pointer correct
\providecommand{\thmname}{Theorem}

\makeatletter
\NewDocumentEnvironment{thm*}{o}
 {%
  \IfValueTF{#1}
    {\namedthm[#1]\refstepcounter{thm}\def\@currentlabel{(#1)}}%
    {\namedthm}%
 }
 {%
  \endnamedthm
 }
\makeatother


\newtheorem*{namedprop}{Proposition}
\newcounter{prop}%makes pointer correct
\providecommand{\propname}{Proposition}

\makeatletter
\NewDocumentEnvironment{prop*}{o}
 {%
  \IfValueTF{#1}
    {\namedprop[#1]\refstepcounter{prop}\def\@currentlabel{(#1)}}%
    {\namedprop}%
 }
 {%
  \endnamedprop
 }
\makeatother

\newtheorem*{namedlem}{Lemma}
\newcounter{lem}%makes pointer correct
\providecommand{\lemname}{Lemma}

\makeatletter
\NewDocumentEnvironment{lem*}{o}
 {%
  \IfValueTF{#1}
    {\namedlem[#1]\refstepcounter{lem}\def\@currentlabel{(#1)}}%
    {\namedlem}%
 }
 {%
  \endnamedlem
 }
\makeatother

\newtheorem*{namedcor}{Corollary}
\newcounter{cor}%makes pointer correct
\providecommand{\corname}{Corollary}

\makeatletter
\NewDocumentEnvironment{cor*}{o}
 {%
  \IfValueTF{#1}
    {\namedcor[#1]\refstepcounter{cor}\def\@currentlabel{(#1)}}%
    {\namedcor}%
 }
 {%
  \endnamedcor
 }
\makeatother

\theoremstyle{definition}
\newtheorem*{annotation}{Annotation}
\newtheorem*{rubric}{Rubric}

\newtheorem*{innerrem}{Remark}
\newcounter{rem}%makes pointer correct
\providecommand{\remname}{Remark}

\makeatletter
\NewDocumentEnvironment{rem}{o}
 {%
  \IfValueTF{#1}
    {\innerrem[#1]\refstepcounter{rem}\def\@currentlabel{(#1)}}%
    {\innerrem}%
 }
 {%
  \endinnerrem
 }
\makeatother

\newtheorem*{innerdefn}{Definition}%%placeholder
\newcounter{defn}%makes pointer correct
\providecommand{\defnname}{Definition}

\makeatletter
\NewDocumentEnvironment{defn}{o}
 {%
  \IfValueTF{#1}
    {\innerdefn[#1]\refstepcounter{defn}\def\@currentlabel{(#1)}}%
    {\innerdefn}%
 }
 {%
  \endinnerdefn
 }
\makeatother

\newtheorem*{scratch}{Scratch Work}


\newtheorem*{namedconj}{Conjecture}
\newcounter{conj}%makes pointer correct
\providecommand{\conjname}{Conjecture}
\makeatletter
\NewDocumentEnvironment{conj}{o}
 {%
  \IfValueTF{#1}
    {\innerconj[#1]\refstepcounter{conj}\def\@currentlabel{(#1)}}%
    {\innerconj}%
 }
 {%
  \endinnerconj
 }
\makeatother

\newtheorem*{poll}{Poll question}
\newtheorem{tps}{Think-Pair-Share}[section]


\newenvironment{obj}{
	\textbf{Learning Objectives.} By the end of class, students will be able to:
		\begin{itemize}}
		{\!.\end{itemize}
		}

\newenvironment{pre}{
	\begin{description}
	}{
	\end{description}
}


\newcounter{ex}%makes pointer correct
\providecommand{\exname}{Homework Problem}
\newenvironment{ex}[1][2in]%
{%Env start code
\problemEnvironmentStart{#1}{Homework Problem}
\refstepcounter{ex}
}
{%Env end code
\problemEnvironmentEnd
}

\newcommand{\inlineAnswer}[2][2 cm]{
    \ifhandout{\pdfOnly{\rule{#1}{0.4pt}}}
    \else{\answer{#2}}
    \fi
}


\ifhandout
\newenvironment{shortAnswer}[1][
    \vfill]
        {% Begin then result
        #1
            \begin{freeResponse}
            }
    {% Environment Ending Code
    \end{freeResponse}
    }
\else
\newenvironment{shortAnswer}[1][]
        {\begin{freeResponse}
            }
    {% Environment Ending Code
    \end{freeResponse}
    }
\fi

\let\question\relax
\let\endquestion\relax

\newtheoremstyle{ExerciseStyle}{\topsep}{\topsep}%%% space between body and thm
		{}                      %%% Thm body font
		{}                              %%% Indent amount (empty = no indent)
		{\bfseries}            %%% Thm head font
		{}                              %%% Punctuation after thm head
		{3em}                           %%% Space after thm head
		{{#1}~\thmnumber{#2}\thmnote{ \bfseries(#3)}}%%% Thm head spec
\theoremstyle{ExerciseStyle}
\newtheorem{br}{In-class Problem}

\newenvironment{sketch}
 {\begin{proof}[Sketch of Proof]}
 {\end{proof}}


\newcommand{\gt}{>}
\newcommand{\lt}{<}
\newcommand{\N}{\mathbb N}
\newcommand{\Q}{\mathbb Q}
\newcommand{\Z}{\mathbb Z}
\newcommand{\C}{\mathbb C}
\newcommand{\R}{\mathbb R}
\renewcommand{\H}{\mathbb{H}}
\newcommand{\lcm}{\operatorname{lcm}}
\newcommand{\nequiv}{\not\equiv}
\newcommand{\ord}{\operatorname{ord}}
\newcommand{\ds}{\displaystyle}
\newcommand{\floor}[1]{\left\lfloor #1\right\rfloor}
\newcommand{\legendre}[2]{\left(\frac{#1}{#2}\right)}



%%%%%%%%%%%%



\title{Introduction to modular arithmetic}
\begin{document}
\begin{abstract}
\end{abstract}
\maketitle

%%%%%%%%%%%%%%%%%%%%%%%%%%
\begin{obj}
\item Prove that congruence modulo $m$ is an equivalence relation on $\Z$.
\item Define a complete residue system.
\item Practice using modular arithmetic.
\end{obj}


\begin{instructorNotes}
    
\begin{pre}
    \item[Read] Strayer, Section 2.1 through Example 1.
    \item[Turn in] The book concludes the section with a caution about division. It states that $6a\equiv 6b \pmod 3$ for all integers $a$ and $b$. Explain why this is true.
    
    
    \begin{solution}
        Since $3\mid 6a-6b=3(2a-2b),$ $6a\equiv 6b \pmod 3$ for all integers $a$ and $b$.
    \end{solution}
    \end{pre}

\end{instructorNotes}



\begin{defn}[divisibility definition of $a\equiv b \pmod{m}$]\label{defn:mod-divides} Let $a,b,m\in\Z$ with $m>0.$
We say that $a$ is \emph{congruent to $b$ modulo $m$} and write $a \equiv b \pmod{m}$ if $m\mid b-a$, and $m$ is said to be the \emph{modulus of the congruence}. The notation $a\not\equiv b\pmod m$ means $a$ is not congruent to $b$ modulo $m,$ or $a$ is \emph{incongruent to $b$ modulo $m$.} 
\end{defn}

\begin{defn}[remainder definition of $a\equiv b \pmod{m}$]\label{defn:mod-remainder} Let $a,b,m\in\Z$ with $m>0.$ We say that $a$ is congruent to $b$ modulo $m$ if $a$ and $b$ have the same remainder when divided by $m$. 
\end{defn}
Be careful with this idea and negative values. Make sure you understand why $-2\equiv 1\pmod{3}$ or $-10\equiv 4\pmod{7}$.

\begin{proposition}[Definitions of congruence modulo $m$ are equivalent]\label{prop:mod-defn:equiv}
These two definitions are equivalent. That is, for $a,b,m\in\Z$ with $m>0,$ $m\mid b-a$ if and only if $a$ and $b$ have the same remainder when divided by $m$.
\end{proposition}

\begin{proof}
Let $a,b,m\in\Z$ with $m> 0.$ By the \nameref{div-alg}, there exists $q_1,q_2,r_1,r_2\in\Z$ such that \begin{align*}
                    a & q_1 m +r_1, 0\leq r_1< m, \textnormal{ and }\\
                    b & q_2 m +r_2, 0\leq r_2< m.
                \end{align*}
            
            
                If $m\mid b-a,$ then by definition, there exists $k\in\Z$ such that $mk=b-a.$ Thus, $mk=q_2 m+r_2-q_1 m-r_1.$ Rearranging, we get 
                $m(k-q_2+q_1)=r_2-r_1$ and $m\mid r_2-r_1.$ Since 
                $0\leq r_1< m, 0\leq r_2< m,$ we have 
                $-m< r_2-r_1< m.$ Thus, $r_2-r_1=0,$ so $a$ and $b$ have the same remainder when divided by $m$.
            

            
                In the other direction, if $r_1=r_2,$ then $a-b=q_1 m-q_2 m=m(q_1-q_2).$ Thus, $m\mid a-b.$
\end{proof}

\begin{example}
 We will eventually find a function that generates all integers solutions to the equation $a^2+b^2=c^2$ (this can be done with only divisibility, so feel free to try for yourself after class).
 
 Modular arithmetic allows us to say a few things about solutions. 
 \begin{description}
 \item[First, let's look at $\!\pmod 2$.] Note that $0^2\equiv 0 \pmod 2$ and $1^2\equiv 1 \pmod 2$.

\begin{description}
 \item[Case 1: $c^2\equiv 0 \pmod 2$] In this case, $c\equiv 0\pmod 2$ and either $1^2+1^2\equiv 0\pmod 2$ or $0^2+0^2\equiv 0\pmod 2$. So, we know $a\equiv b\pmod 2.$ \emph{(Note: $\!\pmod 4$ will eliminate the $a\equiv b\equiv 1 \pmod 2$ case)}
  \item[Case 2: $c^2\equiv 1 \pmod 2$] In this case, $c\equiv 1\pmod 2$ and either $0^2+1^2\equiv 1\pmod 2$. So, we know $a\not\equiv b\pmod 2.$
\end{description}

\item[Let's start with $\!\pmod 3$.] Note that $0^2\equiv 0 \pmod 3$, $1^2\equiv 1 \pmod 3$, and $2^2\equiv 1 \pmod 3$.
\end{description} 

\begin{description}
 \item[Case 1: $c^2\equiv 0 \pmod 3$.] In this case, $c\equiv 0\pmod 3$ and $0^2+0^2\equiv 0\pmod 3$. So, we know $a\equiv b\equiv c\equiv 0\pmod 3.$ 
  \item[Case 2: $c^2\equiv 1 \pmod 3$.] In this case, $c$ could be $1$ or $2$ modulo $3$. We also know $0^2+1^2\equiv 1\pmod 3$, so  $a\not\equiv b\pmod 3.$
  \item[Case 3: $c^2\equiv 2 \pmod 3$] has no solutions.
\end{description}

So at least one of $a,b,c$ is even, and at least one is divisible by $3$.
\end{example}

We can use the idea of congruences to simplify divisibility arguments, as well as nonlinear Diophantine equations.






\end{document}
