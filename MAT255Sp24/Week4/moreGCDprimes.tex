\documentclass{ximera}
\usepackage{amssymb, latexsym, amsmath, amsthm, graphicx, amsthm,alltt,color, listings,multicol,xr-hyper,hyperref,aliascnt,enumitem}
\usepackage{xfrac}

\usepackage{parskip}
\usepackage[,margin=0.7in]{geometry}
\setlength{\textheight}{8.5in}

\usepackage{epstopdf}

\DeclareGraphicsExtensions{.eps}
\usepackage{tikz}


\usepackage{tkz-euclide}
%\usetkzobj{all}
\tikzstyle geometryDiagrams=[rounded corners=.5pt,ultra thick,color=black]
\colorlet{penColor}{black} % Color of a curve in a plot


\usepackage{subcaption}
\usepackage{float}
\usepackage{fancyhdr}
\usepackage{pdfpages}
\newcounter{includepdfpage}
\usepackage{makecell}


\usepackage{currfile}
\usepackage{xstring}




\graphicspath{  
{./otherDocuments/}
}

\author{Claire Merriman}
\newcommand{\classday}[1]{\def\classday{#1}}

%%%%%%%%%%%%%%%%%%%%%
% Counters and autoref for unnumbered environments
% Not needed??
%%%%%%%%%%%%%%%%%%%%%
\theoremstyle{plain}


\newtheorem*{namedthm}{Theorem}
\newcounter{thm}%makes pointer correct
\providecommand{\thmname}{Theorem}

\makeatletter
\NewDocumentEnvironment{thm*}{o}
 {%
  \IfValueTF{#1}
    {\namedthm[#1]\refstepcounter{thm}\def\@currentlabel{(#1)}}%
    {\namedthm}%
 }
 {%
  \endnamedthm
 }
\makeatother


\newtheorem*{namedprop}{Proposition}
\newcounter{prop}%makes pointer correct
\providecommand{\propname}{Proposition}

\makeatletter
\NewDocumentEnvironment{prop*}{o}
 {%
  \IfValueTF{#1}
    {\namedprop[#1]\refstepcounter{prop}\def\@currentlabel{(#1)}}%
    {\namedprop}%
 }
 {%
  \endnamedprop
 }
\makeatother

\newtheorem*{namedlem}{Lemma}
\newcounter{lem}%makes pointer correct
\providecommand{\lemname}{Lemma}

\makeatletter
\NewDocumentEnvironment{lem*}{o}
 {%
  \IfValueTF{#1}
    {\namedlem[#1]\refstepcounter{lem}\def\@currentlabel{(#1)}}%
    {\namedlem}%
 }
 {%
  \endnamedlem
 }
\makeatother

\newtheorem*{namedcor}{Corollary}
\newcounter{cor}%makes pointer correct
\providecommand{\corname}{Corollary}

\makeatletter
\NewDocumentEnvironment{cor*}{o}
 {%
  \IfValueTF{#1}
    {\namedcor[#1]\refstepcounter{cor}\def\@currentlabel{(#1)}}%
    {\namedcor}%
 }
 {%
  \endnamedcor
 }
\makeatother

\theoremstyle{definition}
\newtheorem*{annotation}{Annotation}
\newtheorem*{rubric}{Rubric}

\newtheorem*{innerrem}{Remark}
\newcounter{rem}%makes pointer correct
\providecommand{\remname}{Remark}

\makeatletter
\NewDocumentEnvironment{rem}{o}
 {%
  \IfValueTF{#1}
    {\innerrem[#1]\refstepcounter{rem}\def\@currentlabel{(#1)}}%
    {\innerrem}%
 }
 {%
  \endinnerrem
 }
\makeatother

\newtheorem*{innerdefn}{Definition}%%placeholder
\newcounter{defn}%makes pointer correct
\providecommand{\defnname}{Definition}

\makeatletter
\NewDocumentEnvironment{defn}{o}
 {%
  \IfValueTF{#1}
    {\innerdefn[#1]\refstepcounter{defn}\def\@currentlabel{(#1)}}%
    {\innerdefn}%
 }
 {%
  \endinnerdefn
 }
\makeatother

\newtheorem*{scratch}{Scratch Work}


\newtheorem*{namedconj}{Conjecture}
\newcounter{conj}%makes pointer correct
\providecommand{\conjname}{Conjecture}
\makeatletter
\NewDocumentEnvironment{conj}{o}
 {%
  \IfValueTF{#1}
    {\innerconj[#1]\refstepcounter{conj}\def\@currentlabel{(#1)}}%
    {\innerconj}%
 }
 {%
  \endinnerconj
 }
\makeatother

\newtheorem*{poll}{Poll question}
\newtheorem{tps}{Think-Pair-Share}[section]


\newenvironment{obj}{
	\textbf{Learning Objectives.} By the end of class, students will be able to:
		\begin{itemize}}
		{\!.\end{itemize}
		}

\newenvironment{pre}{
	\begin{description}
	}{
	\end{description}
}


\newcounter{ex}%makes pointer correct
\providecommand{\exname}{Homework Problem}
\newenvironment{ex}[1][2in]%
{%Env start code
\problemEnvironmentStart{#1}{Homework Problem}
\refstepcounter{ex}
}
{%Env end code
\problemEnvironmentEnd
}

\newcommand{\inlineAnswer}[2][2 cm]{
    \ifhandout{\pdfOnly{\rule{#1}{0.4pt}}}
    \else{\answer{#2}}
    \fi
}


\ifhandout
\newenvironment{shortAnswer}[1][
    \vfill]
        {% Begin then result
        #1
            \begin{freeResponse}
            }
    {% Environment Ending Code
    \end{freeResponse}
    }
\else
\newenvironment{shortAnswer}[1][]
        {\begin{freeResponse}
            }
    {% Environment Ending Code
    \end{freeResponse}
    }
\fi

\let\question\relax
\let\endquestion\relax

\newtheoremstyle{ExerciseStyle}{\topsep}{\topsep}%%% space between body and thm
		{}                      %%% Thm body font
		{}                              %%% Indent amount (empty = no indent)
		{\bfseries}            %%% Thm head font
		{}                              %%% Punctuation after thm head
		{3em}                           %%% Space after thm head
		{{#1}~\thmnumber{#2}\thmnote{ \bfseries(#3)}}%%% Thm head spec
\theoremstyle{ExerciseStyle}
\newtheorem{br}{In-class Problem}

\newenvironment{sketch}
 {\begin{proof}[Sketch of Proof]}
 {\end{proof}}


\newcommand{\gt}{>}
\newcommand{\lt}{<}
\newcommand{\N}{\mathbb N}
\newcommand{\Q}{\mathbb Q}
\newcommand{\Z}{\mathbb Z}
\newcommand{\C}{\mathbb C}
\newcommand{\R}{\mathbb R}
\renewcommand{\H}{\mathbb{H}}
\newcommand{\lcm}{\operatorname{lcm}}
\newcommand{\nequiv}{\not\equiv}
\newcommand{\ord}{\operatorname{ord}}
\newcommand{\ds}{\displaystyle}
\newcommand{\floor}[1]{\left\lfloor #1\right\rfloor}
\newcommand{\legendre}[2]{\left(\frac{#1}{#2}\right)}



%%%%%%%%%%%%



\title{More facts about greatest common divisor and primes}
\begin{document}
\begin{abstract}
\end{abstract}
\maketitle

%%%%%%%%%%%%%%%%%%%%%%%%%%
%%%%%%%%%%%%%%%%%%%%%%%%%%

\begin{obj}
    \item Find the solutions to a specific Diophantine equation in three variables
    \item Prove that when a Diophantine equation in three variables has a solutions, it has infinitely many.
\end{obj}

 
 \begin{instructorNotes}
    \begin{pre}
        \item[Reading]  Strayer Section 1.5.
        
        \item[Turn in] 
        \begin{enumerate}
         \item  The proof of Theorem 1.19 ends with ``the cases $a=1$ and $b>1,$ $a>1$ and $b=1,$ and $a=b=1$ are easily checked and are left as exercises. Do this.
        \item For Corollary 1.20, the book states ``The (extremely easy) proof is left as an exercise for the reader." Complete this proof.\end{enumerate}
        
        \begin{solution}
         
        \begin{enumerate}
         \item When $a=1$ and $b>1,$ then $(a,b)=1$ and $[a,b]=b$. Then $(a,b)[a,b]=b=ab$. Similarly for $a>1, b=1$. When $a=b=1,$ then $(a,b)=[a,b]=1$ and $(a,b)[a,b]=1=ab.$
         \item From Theorem 1.19, we know that $\gcd(a,b)\lcm[a,b]=ab$. Since $\gcd(a,b),\lcm[a,b],$ and $ab$ are all positive, $\lcm[a,b]=\frac{ab}{\gcd(a,b)}$ if and only if $\gcd(a,b)=1$.
        \end{enumerate}
        \end{solution}
        \end{pre}
 \end{instructorNotes}


\begin{proposition}\label{prop:3diophantine}
    Let $a,b,c,d\in\Z$ and let $ax+by+cz=d$ be a linear Diophantine equation. If $(a,b,c)\nmid d$, then the equation has no solutions. If $(a,b,c)\mid d,$ then there are infinitely many solutions.
\end{proposition}

\begin{br}
Find integral solutions to the Diophantine equation \[8x_1-4x_2+6x_3=6.\]

\begin{enumerate}
    \item Since $(8,-4,6)=2,$ solutions exist
    \item The linear Diophantine equation $8x_1-4x_2=4y$ has infinitely many solutions for all $y\in\Z$ by Theorem \ref{thm:linear-dioph}. Substituting into the original Diophantine equation gives $4y+6x_3=6,$ which has infinitely many solutions by Theorem \ref{thm:linear-dioph}, since $(4,6)=2\mid 6$. Find them.
     
    \begin{solution}
        By inspection, $y=0,x_3=1$ is a particular solution. Then by Theorem \ref{thm:linear-dioph}, the solutions have the form 
        \begin{align*}
            y&=0+\frac{6n}{2},\quad x_3=1-\frac{4n}{2}, \quad \textnormal{ or }\\
            y&=0+3n,\quad x_3=1-2n, \qquad n\in\Z. \qedhere
        \end{align*}
        \end{solution}
    \item  For a particular value of $y$, the Diophantine equation $8x_1-4x_2=0$ has solutions, find them. 
    \begin{solution}
        By inspection, $x_1=1,x_2=2$ is a particular solution. Then by Theorem \ref{thm:linear-dioph}, the solutions have the form 
        \begin{align*}
            x_1&=1+\frac{-4m}{4},\quad x_2=2-\frac{8m}{4}, \quad \textnormal{ or }\\
            x_1&=1-m,\quad x_2=2-2m, \qquad m\in\Z. \qedhere
        \end{align*}
        \end{solution}
    \item Then $x_1=1-m, x_2=2-2m,x_3=1$ for $m\in\Z$.
\end{enumerate}
\end{br}


\begin{proof}[Proof of Proposition \ref{prop:3diophantine}]
    Let $a,b,c,d\in\Z$ and let $ax+by+cz=d$ be a linear Diophantine equation. If $(a,b,c)\mid d,$ let $e=(a,b).$ 
    Then \begin{equation}\label{eq:interDioph1}ax+by=ew\end{equation} has a solution for all $w\in\Z$ by Theorem \ref{thm:linear-dioph}.
    Similarly, the linear Diophantine equation \begin{equation}\label{eq:interDioph2}
        ew+cz=d
    \end{equation} has infinitely many solutions by Theorem \ref{thm:linear-dioph}, since $(e,c)=(a,b,c)$ by the Lemma \ref{lem:gcd_3case} and $(a,b,c)\mid d$ by assumption. These solutions have the form 
    \[w=w_0+\frac{cn}{(a,b,c)},\quad z=z_0-\frac{en}{(a,b,c)},\qquad n\in\Z,\]
    where $w_0,z_0$ is a particular solution. Let $x_0,y_0$ be a particular solution to \[ax+by=ew_0.\] Then the general solution is 
    \[x=x_0+\frac{bm}{e},\quad y=y_0-\frac{am}{e},\qquad m\in\Z.\]

    To verify that these formulas for $x,y,$ and $z$ give solutions to $ax+by+cz=d$, we substitute into equation \ref{eq:interDioph2} then \ref{eq:interDioph1}
    \begin{align*}
        e\left(w_0+\frac{cn}{(a,b,c)}\right)+c\left(z_0-\frac{en}{(a,b,c)}\right)&=d\\
        e w_0 + c z_0 & =d\\
        a\left(x_0+\frac{bm}{e}\right)+b\left(y_0-\frac{am}{e}\right)+cz _0 &=d\\
        ax_0+by_0+c z_0&=d.
    \end{align*}

    When $(a,b,c)\nmid d$, $\frac{a}{(a,b,c)},\frac{b}{(a,b,c)},\frac{c}{(a,b,c)}\in\Z$ by definition, but $\frac{d}{(a,b,c)}$ is not an integer. Therefore, there are no integers such that \[\frac{a}{(a,b,c)}x+\frac{b}{(a,b,c)}y+\frac{c}{(a,b,c)}z=\frac{d}{(a,b,c)}. \qedhere\] 
\end{proof}

%%%%%%%%%%%%%%%%%%%%%%%%%


\end{document}
