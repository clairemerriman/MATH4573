\documentclass{ximera}
\usepackage{amssymb, latexsym, amsmath, amsthm, graphicx, amsthm,alltt,color, listings,multicol,xr-hyper,hyperref,aliascnt,enumitem}
\usepackage{xfrac}

\usepackage{parskip}
\usepackage[,margin=0.7in]{geometry}
\setlength{\textheight}{8.5in}

\usepackage{epstopdf}

\DeclareGraphicsExtensions{.eps}
\usepackage{tikz}


\usepackage{tkz-euclide}
%\usetkzobj{all}
\tikzstyle geometryDiagrams=[rounded corners=.5pt,ultra thick,color=black]
\colorlet{penColor}{black} % Color of a curve in a plot


\usepackage{subcaption}
\usepackage{float}
\usepackage{fancyhdr}
\usepackage{pdfpages}
\newcounter{includepdfpage}
\usepackage{makecell}


\usepackage{currfile}
\usepackage{xstring}




\graphicspath{  
{./otherDocuments/}
}

\author{Claire Merriman}
\newcommand{\classday}[1]{\def\classday{#1}}

%%%%%%%%%%%%%%%%%%%%%
% Counters and autoref for unnumbered environments
% Not needed??
%%%%%%%%%%%%%%%%%%%%%
\theoremstyle{plain}


\newtheorem*{namedthm}{Theorem}
\newcounter{thm}%makes pointer correct
\providecommand{\thmname}{Theorem}

\makeatletter
\NewDocumentEnvironment{thm*}{o}
 {%
  \IfValueTF{#1}
    {\namedthm[#1]\refstepcounter{thm}\def\@currentlabel{(#1)}}%
    {\namedthm}%
 }
 {%
  \endnamedthm
 }
\makeatother


\newtheorem*{namedprop}{Proposition}
\newcounter{prop}%makes pointer correct
\providecommand{\propname}{Proposition}

\makeatletter
\NewDocumentEnvironment{prop*}{o}
 {%
  \IfValueTF{#1}
    {\namedprop[#1]\refstepcounter{prop}\def\@currentlabel{(#1)}}%
    {\namedprop}%
 }
 {%
  \endnamedprop
 }
\makeatother

\newtheorem*{namedlem}{Lemma}
\newcounter{lem}%makes pointer correct
\providecommand{\lemname}{Lemma}

\makeatletter
\NewDocumentEnvironment{lem*}{o}
 {%
  \IfValueTF{#1}
    {\namedlem[#1]\refstepcounter{lem}\def\@currentlabel{(#1)}}%
    {\namedlem}%
 }
 {%
  \endnamedlem
 }
\makeatother

\newtheorem*{namedcor}{Corollary}
\newcounter{cor}%makes pointer correct
\providecommand{\corname}{Corollary}

\makeatletter
\NewDocumentEnvironment{cor*}{o}
 {%
  \IfValueTF{#1}
    {\namedcor[#1]\refstepcounter{cor}\def\@currentlabel{(#1)}}%
    {\namedcor}%
 }
 {%
  \endnamedcor
 }
\makeatother

\theoremstyle{definition}
\newtheorem*{annotation}{Annotation}
\newtheorem*{rubric}{Rubric}

\newtheorem*{innerrem}{Remark}
\newcounter{rem}%makes pointer correct
\providecommand{\remname}{Remark}

\makeatletter
\NewDocumentEnvironment{rem}{o}
 {%
  \IfValueTF{#1}
    {\innerrem[#1]\refstepcounter{rem}\def\@currentlabel{(#1)}}%
    {\innerrem}%
 }
 {%
  \endinnerrem
 }
\makeatother

\newtheorem*{innerdefn}{Definition}%%placeholder
\newcounter{defn}%makes pointer correct
\providecommand{\defnname}{Definition}

\makeatletter
\NewDocumentEnvironment{defn}{o}
 {%
  \IfValueTF{#1}
    {\innerdefn[#1]\refstepcounter{defn}\def\@currentlabel{(#1)}}%
    {\innerdefn}%
 }
 {%
  \endinnerdefn
 }
\makeatother

\newtheorem*{scratch}{Scratch Work}


\newtheorem*{namedconj}{Conjecture}
\newcounter{conj}%makes pointer correct
\providecommand{\conjname}{Conjecture}
\makeatletter
\NewDocumentEnvironment{conj}{o}
 {%
  \IfValueTF{#1}
    {\innerconj[#1]\refstepcounter{conj}\def\@currentlabel{(#1)}}%
    {\innerconj}%
 }
 {%
  \endinnerconj
 }
\makeatother

\newtheorem*{poll}{Poll question}
\newtheorem{tps}{Think-Pair-Share}[section]


\newenvironment{obj}{
	\textbf{Learning Objectives.} By the end of class, students will be able to:
		\begin{itemize}}
		{\!.\end{itemize}
		}

\newenvironment{pre}{
	\begin{description}
	}{
	\end{description}
}


\newcounter{ex}%makes pointer correct
\providecommand{\exname}{Homework Problem}
\newenvironment{ex}[1][2in]%
{%Env start code
\problemEnvironmentStart{#1}{Homework Problem}
\refstepcounter{ex}
}
{%Env end code
\problemEnvironmentEnd
}

\newcommand{\inlineAnswer}[2][2 cm]{
    \ifhandout{\pdfOnly{\rule{#1}{0.4pt}}}
    \else{\answer{#2}}
    \fi
}


\ifhandout
\newenvironment{shortAnswer}[1][
    \vfill]
        {% Begin then result
        #1
            \begin{freeResponse}
            }
    {% Environment Ending Code
    \end{freeResponse}
    }
\else
\newenvironment{shortAnswer}[1][]
        {\begin{freeResponse}
            }
    {% Environment Ending Code
    \end{freeResponse}
    }
\fi

\let\question\relax
\let\endquestion\relax

\newtheoremstyle{ExerciseStyle}{\topsep}{\topsep}%%% space between body and thm
		{}                      %%% Thm body font
		{}                              %%% Indent amount (empty = no indent)
		{\bfseries}            %%% Thm head font
		{}                              %%% Punctuation after thm head
		{3em}                           %%% Space after thm head
		{{#1}~\thmnumber{#2}\thmnote{ \bfseries(#3)}}%%% Thm head spec
\theoremstyle{ExerciseStyle}
\newtheorem{br}{In-class Problem}

\newenvironment{sketch}
 {\begin{proof}[Sketch of Proof]}
 {\end{proof}}


\newcommand{\gt}{>}
\newcommand{\lt}{<}
\newcommand{\N}{\mathbb N}
\newcommand{\Q}{\mathbb Q}
\newcommand{\Z}{\mathbb Z}
\newcommand{\C}{\mathbb C}
\newcommand{\R}{\mathbb R}
\renewcommand{\H}{\mathbb{H}}
\newcommand{\lcm}{\operatorname{lcm}}
\newcommand{\nequiv}{\not\equiv}
\newcommand{\ord}{\operatorname{ord}}
\newcommand{\ds}{\displaystyle}
\newcommand{\floor}[1]{\left\lfloor #1\right\rfloor}
\newcommand{\legendre}[2]{\left(\frac{#1}{#2}\right)}



%%%%%%%%%%%%



\title{Prime factorizations}
\begin{document}
\begin{abstract}
\end{abstract}
\maketitle

\begin{obj}
    \item State and prove facts about prime factorizations using the Fundamental Theorem of Arithmetic
    \item Prove there are infinitely many primes of the form $4n+3$.
\end{obj}
%%%%%%%%%%%%%%%%%%%%%%%%%%
% Note on $m^4-n^4=(m^2-n^2)(m^2+n^2)$: In order to sho w this is not prime, must prove that the factors cannot be $1$ and the number itself. Hint: show that if one of the factors is $1$ the other is $1$ or $0$ (or $-1$).


\begin{corollary}\label{cor:lcm-gcd}
 Let $a,b\in\Z$ with $a,b>0$. Then $[a,b]=ab$ if and only if $(a,b)=1$.
\end{corollary}
A note on ``if and only if" proofs: 
\begin{itemize}
 \item You can do two directions: 
\begin{itemize}
 \item  If $[a,b]=ab$, then $(a,b)=1$.
 \item  If $(a,b)=1$, then $[a,b]=ab$.
\end{itemize}
\item Sometimes you can string together a series of ``if and only if statements." Definitions are always ``if and only if," even though rarely stated that way. For example, an integer $n$ is even if and only if there exist an integer $m$ such that $n=2m$:
\begin{itemize}
 \item An integer $n$ is even if and only if $2\mid n$ (definition of even) 
 \item if and only if there exist an integer $m$ such that $n=2m$ (definition of $2\mid n$).
\end{itemize}
\end{itemize}

\begin{thm*}[Dirichlet's Theorem]\label{thm:dirichlet}
    Let $a,b\in\Z$ with $a,b>0$ and $(a,b)=1$. Then the arithmetic progression \[a,a+b, a+2b, \dots, a+nb,\dots\]
    contains infinitely many primes.
\end{thm*}


\begin{remark}
    Surprisingly, this proof involves complex analysis. The statement that there are infinitely many prime numbers is the case $a=b=1$.
\end{remark}

\begin{warning}
    You may not use this result to prove special cases, ie, specific values of $a$ and $b$.
\end{warning}


\begin{lemma}\label{lem:prod-goodprimes}
 If $a,b\in\Z$ such that $a=4m+1$ and $b=4n+1$ for some integers $m$ and $n$, then $ab$ can also be written in that form.
\begin{proof}
 Let $a=4m+1$ and $b=4n+1$ for some integers $m$ and $n$. Then 
\begin{align*}
ab&=(4m+1)(4n+1)\\
&=16mn+4m+4n+1\\
&=4(4mn+m+n)+1.\qedhere
\end{align*}
\end{proof}
\end{lemma}

\begin{prop*}[Proposition 1.22]\label{prop:inf-badprimes}
 There are infinitely may prime numbers expressible in the form $4n+3$ where $n$ is a nonnegative integer.
\end{prop*}
\begin{proof}
 (Similar to the proof that there are infinitely many prime numbers). Assume, by way of contradiction, that there are only finitely many prime numbers of the form $4n+3$, say $p_0=3, p_1,p_2,\dots, p_r$, where the $p_i$ are distinct. Let $N=4p_1 p_2 \cdots p_r+3$. If every prime factor of $N$ has the form $4n+1$, then so does $N$, by repeated applications of \nameref{lem:prod-goodprimes}. Thus, one of the prime factors of $N$, say $p$, have the for $4n+3$. We consider two cases:
 
\begin{description}
 \item[Case 1, $p=3$:] If $p=3$, then $p\mid N-3$ by linear combinations. Then $p\mid 4 p_1p_2\cdots p_r$. Then by \nameref{cor:irreducible-prime}, either $3\mid 4$ or $3\mid p_1p_2\cdots p_r$. This implies that $p\mid p_i$ for some $i=1,2,\dots,r$. However, $p_1,p_2,\dots,p_r$ are distinct primes not equal to $3$, so this is not possible. Therefore, $p\neq 3$.
 
 \item[Case 2, $p=p_i$ for some $i=1,2,\dots, r$:] If $p=p_i$, then $p\mid N-4p_1p_2\cdots p_r$ by linear combinations. Then $p\mid 3$. However, $p_1,p_2,\dots,p_r$ are distinct primes not equal to $3$, so this is not possible. Therefore, $p\neq p_i$ for $i=1,2,\dots,r$.
\end{description}
Therefore, $N$ has a prime divisor of the form $4n+3$ which is not on the list $p_0,p_1,\dots,p_r$, which contradicts the assumption that $p_0,p_1,\dots,p_r$ are all primes of this form. Thus, there are infinitely many primes of the form $4n+3$.
\end{proof}

%%%%%%%%%%%%%%%%%%%%%%%%%%


\end{document}
