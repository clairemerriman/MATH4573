\documentclass{../ximera}
\usepackage{amssymb, latexsym, amsmath, amsthm, graphicx, amsthm,alltt,color, listings,multicol,xr-hyper,hyperref,aliascnt,enumitem}
\usepackage{xfrac}

\usepackage{parskip}
\usepackage[,margin=0.7in]{geometry}
\setlength{\textheight}{8.5in}

\usepackage{epstopdf}

\DeclareGraphicsExtensions{.eps}
\usepackage{tikz}


\usepackage{tkz-euclide}
%\usetkzobj{all}
\tikzstyle geometryDiagrams=[rounded corners=.5pt,ultra thick,color=black]
\colorlet{penColor}{black} % Color of a curve in a plot


\usepackage{subcaption}
\usepackage{float}
\usepackage{fancyhdr}
\usepackage{pdfpages}
\newcounter{includepdfpage}
\usepackage{makecell}


\usepackage{currfile}
\usepackage{xstring}




\graphicspath{  
{./otherDocuments/}
}

\author{Claire Merriman}
\newcommand{\classday}[1]{\def\classday{#1}}

%%%%%%%%%%%%%%%%%%%%%
% Counters and autoref for unnumbered environments
% Not needed??
%%%%%%%%%%%%%%%%%%%%%
\theoremstyle{plain}


\newtheorem*{namedthm}{Theorem}
\newcounter{thm}%makes pointer correct
\providecommand{\thmname}{Theorem}

\makeatletter
\NewDocumentEnvironment{thm*}{o}
 {%
  \IfValueTF{#1}
    {\namedthm[#1]\refstepcounter{thm}\def\@currentlabel{(#1)}}%
    {\namedthm}%
 }
 {%
  \endnamedthm
 }
\makeatother


\newtheorem*{namedprop}{Proposition}
\newcounter{prop}%makes pointer correct
\providecommand{\propname}{Proposition}

\makeatletter
\NewDocumentEnvironment{prop*}{o}
 {%
  \IfValueTF{#1}
    {\namedprop[#1]\refstepcounter{prop}\def\@currentlabel{(#1)}}%
    {\namedprop}%
 }
 {%
  \endnamedprop
 }
\makeatother

\newtheorem*{namedlem}{Lemma}
\newcounter{lem}%makes pointer correct
\providecommand{\lemname}{Lemma}

\makeatletter
\NewDocumentEnvironment{lem*}{o}
 {%
  \IfValueTF{#1}
    {\namedlem[#1]\refstepcounter{lem}\def\@currentlabel{(#1)}}%
    {\namedlem}%
 }
 {%
  \endnamedlem
 }
\makeatother

\newtheorem*{namedcor}{Corollary}
\newcounter{cor}%makes pointer correct
\providecommand{\corname}{Corollary}

\makeatletter
\NewDocumentEnvironment{cor*}{o}
 {%
  \IfValueTF{#1}
    {\namedcor[#1]\refstepcounter{cor}\def\@currentlabel{(#1)}}%
    {\namedcor}%
 }
 {%
  \endnamedcor
 }
\makeatother

\theoremstyle{definition}
\newtheorem*{annotation}{Annotation}
\newtheorem*{rubric}{Rubric}

\newtheorem*{innerrem}{Remark}
\newcounter{rem}%makes pointer correct
\providecommand{\remname}{Remark}

\makeatletter
\NewDocumentEnvironment{rem}{o}
 {%
  \IfValueTF{#1}
    {\innerrem[#1]\refstepcounter{rem}\def\@currentlabel{(#1)}}%
    {\innerrem}%
 }
 {%
  \endinnerrem
 }
\makeatother

\newtheorem*{innerdefn}{Definition}%%placeholder
\newcounter{defn}%makes pointer correct
\providecommand{\defnname}{Definition}

\makeatletter
\NewDocumentEnvironment{defn}{o}
 {%
  \IfValueTF{#1}
    {\innerdefn[#1]\refstepcounter{defn}\def\@currentlabel{(#1)}}%
    {\innerdefn}%
 }
 {%
  \endinnerdefn
 }
\makeatother

\newtheorem*{scratch}{Scratch Work}


\newtheorem*{namedconj}{Conjecture}
\newcounter{conj}%makes pointer correct
\providecommand{\conjname}{Conjecture}
\makeatletter
\NewDocumentEnvironment{conj}{o}
 {%
  \IfValueTF{#1}
    {\innerconj[#1]\refstepcounter{conj}\def\@currentlabel{(#1)}}%
    {\innerconj}%
 }
 {%
  \endinnerconj
 }
\makeatother

\newtheorem*{poll}{Poll question}
\newtheorem{tps}{Think-Pair-Share}[section]


\newenvironment{obj}{
	\textbf{Learning Objectives.} By the end of class, students will be able to:
		\begin{itemize}}
		{\!.\end{itemize}
		}

\newenvironment{pre}{
	\begin{description}
	}{
	\end{description}
}


\newcounter{ex}%makes pointer correct
\providecommand{\exname}{Homework Problem}
\newenvironment{ex}[1][2in]%
{%Env start code
\problemEnvironmentStart{#1}{Homework Problem}
\refstepcounter{ex}
}
{%Env end code
\problemEnvironmentEnd
}

\newcommand{\inlineAnswer}[2][2 cm]{
    \ifhandout{\pdfOnly{\rule{#1}{0.4pt}}}
    \else{\answer{#2}}
    \fi
}


\ifhandout
\newenvironment{shortAnswer}[1][
    \vfill]
        {% Begin then result
        #1
            \begin{freeResponse}
            }
    {% Environment Ending Code
    \end{freeResponse}
    }
\else
\newenvironment{shortAnswer}[1][]
        {\begin{freeResponse}
            }
    {% Environment Ending Code
    \end{freeResponse}
    }
\fi

\let\question\relax
\let\endquestion\relax

\newtheoremstyle{ExerciseStyle}{\topsep}{\topsep}%%% space between body and thm
		{}                      %%% Thm body font
		{}                              %%% Indent amount (empty = no indent)
		{\bfseries}            %%% Thm head font
		{}                              %%% Punctuation after thm head
		{3em}                           %%% Space after thm head
		{{#1}~\thmnumber{#2}\thmnote{ \bfseries(#3)}}%%% Thm head spec
\theoremstyle{ExerciseStyle}
\newtheorem{br}{In-class Problem}

\newenvironment{sketch}
 {\begin{proof}[Sketch of Proof]}
 {\end{proof}}


\newcommand{\gt}{>}
\newcommand{\lt}{<}
\newcommand{\N}{\mathbb N}
\newcommand{\Q}{\mathbb Q}
\newcommand{\Z}{\mathbb Z}
\newcommand{\C}{\mathbb C}
\newcommand{\R}{\mathbb R}
\renewcommand{\H}{\mathbb{H}}
\newcommand{\lcm}{\operatorname{lcm}}
\newcommand{\nequiv}{\not\equiv}
\newcommand{\ord}{\operatorname{ord}}
\newcommand{\ds}{\displaystyle}
\newcommand{\floor}[1]{\left\lfloor #1\right\rfloor}
\newcommand{\legendre}[2]{\left(\frac{#1}{#2}\right)}



%%%%%%%%%%%%



\title{Introduction to quadratic residues}
\begin{document}
\begin{abstract}
\end{abstract}
\maketitle

%%%%%%%%%%%%%%%%%%%%%%%%%%

\begin{obj}
    \item Define a quadratic residue modulo $m$
    \item Prove that the quadratic congruence $x^2\equiv a\pmod{p}$ has zero or one solution modulo a prime when $p\nmid a$
    \item Use the solution to a quadratic congruence modulo a prime to find the other solution
\end{obj}


\begin{pre}
    \item[Reading:] Strayer Section 4.1
    \item[Turn in:] Exercise 3
     Find all incongruent solutions of the quadratic congruence $x^2\equiv 1\pmod{8}.$ Is it not true that quadratic congruences have either no solutions or exactly two incongruent solutions? Explain.

     \begin{solution}
        As we have seen on many previous questions, $x^2\equiv 1\pmod{8}$ for all odd numbers. So there are $4$ incongruent solutions modulo $8$, which is not a contradiction because $8$ is not an odd prime number.
     \end{solution}
\end{pre}

%%%%%%%%%%%%%%%%%%%%%%%%%%
\subsection{Finish proof of the existence of primitive roots modulo a prime (10 minutes)}
%%%%%%%%%%%%%%%%%%%%%%%%%%

%%%%%%%%%%%%%%%%%%%%%%%%%%
\subsection{Quadratic residues (40 minutes)}
%%%%%%%%%%%%%%%%%%%%%%%%%%


\begin{definition}[quadratic residue]\label{defn:quad-residue}
    Let $a,m\in\Z$ with $m>0$ and $(a,m)=1.$ The $a$ is said to be a \emph{quadratic residue modulo $m$} if the quadratic congruence $x^2\equiv a\pmod{m}$ is solvable in $\Z.$ Otherwise, $a$ is said to be a \emph{quadratic nonresidue modulo $m$}.
\end{definition}

\begin{remark}
    When finding squares modulo $m,$ we only need to check up to $\frac{m}{2},$ since $(-a)^2=a^2$ and $m-a\equiv -a\pmod{m}$
\end{remark}

\begin{br}
    Find all incongruent quadratic residues and nonresidues modulo $2,3,4,5,6,7,8,$ and $9$.
    % \begin{enumerate}
    %     \item $3$
    %     \item $4$
    %     \item $5$
    %     \item $7$
    %     \item $6$
    %     \item $8$
    %     \item $9$
    %     \item $10$
    % \end{enumerate}


    \begin{solution}
        I also included solutions modulo $10,11,12$

        \begin{tabular}{p{1.5cm}|p{4cm}p{3cm}p{3cm}}
            Modulus & least nonnegative reduced residues & quadratic residues & quadratic nonresidues \\\hline
            $2$ & $1$   
                & $1$ 
                & N/A \\
            $3$ & $\answer{1,2}$ 
                & $\answer{1}$
                & $\answer{2}$\\
            $4$ & $\answer{1,3}$ 
                & $\answer{1}$
                & $\answer{3}$\\
            $5$ & $\answer{1,2,3,4}$
                & $\answer{1,4}$
                & $\answer{2,3}$\\
            $6$ & $\answer{1,5}$
                & $\answer{1}$
                & $\answer{5}$\\
            $7$ & $\answer{1,2,3,4,5}$
                & $\answer{1,2,4}$
                & $\answer{3,5,6}$\\
            $8$ & $\answer{1,3,5,7}$
                & $\answer{1}$
                & $\answer{3,5,7}$\\
            $9$ & $\answer{1,2,4,5,7,8}$
                & $\answer{1,4,7}$
                & $\answer{2,4,8}$\\
            $10$ & $\answer{1,3,7,9}$
                & $\answer{1,9}$
                & $\answer{3,7}$\\
            $11$ & $\answer{1,2,3,4,5,6,7,8,9,10}$
                & $\answer{1,3,4,5,9}$
                & $\answer{2,6,7,8,10}$\\
            $12$ & $\answer{1,5,7,11}$
                & $\answer{1}$
                & $\answer{5,7,11}$\\
        \end{tabular}
    \end{solution}
\end{br}

\begin{lem*}[Generalized Porism 4.2]\label{lem:roots-pairs}
    Let $a,m\in\Z$ with $m>0$ and $(a,m)=1.$ If the quadratic congruence $x^2\equiv a\pmod{m}$ is solvable, say with $x=x_0,$ then  $m-x_0$ is also a solution. If $m\gt 2,$ then $x_0\not\equiv m-x_0\pmod{m},$ and solutions occur in pairs.
\end{lem*}

\begin{proof}
    Let $a,m\in\Z$ with $m>0$ and $(a,m)=1.$ If the quadratic congruence $x^2\equiv a\pmod{m}$ is solvable, say with $x=x_0.$ Then 
    \[(m-x_0)^2\equiv (-x_0)^2\equiv x_0^2\equiv a\pmod{m}.\]
    
    If $x_0\equiv m-x_0\pmod{m},$ then $2x_0\equiv m\equiv 0 \pmod{m}$ and $m\mid 2x_0$ by definition. Since $(a,m)=1,$ it must be that $(x_0,m)=1$ since $(x_0,m)\mid(a,m).$ Thus, $m\mid 2,$ so $m=2.$ Therefore, when  $m\gt 2,$ then $x_0\not\equiv m-x_0\pmod{m},$ and solutions occur in pairs.
\end{proof}


\begin{remark}
    Since $x_0\equiv m-x_0\pmod{m}$ implies $x_0\equiv \frac{m}{2},$ we can say that if $x^2\equiv a\pmod{m}$ is solvable and $\frac{m}{2}$ is \emph{not} a solution, then solutions occur in pairs.
\end{remark}

\begin{prop*}[Proposition 4.1]\label{prop:number-sqrts}
    Let $p$ be an odd prime number and let $a\in\Z$ with $p\mid a.$ Then the quadratic congruence $x^2\equiv a\pmod{p}$ has either no solutions or exactly two incongruent solutions modulo $p$.
\end{prop*}

\begin{proof}
	Let $p$ be an odd prime number and let $a\in\Z$ with $p\mid a.$ Consider the quadratic congruence $x^2\equiv a\pmod{p}.$ If no solutions exist, we are done.
	
	If solutions to the quadratic congruence exist, then \nameref{lem:roots-pairs} says that there are at least two solutions, since $p>2.$ \nameref{thm:lagrange} says that there are at most two solutions to $x^2-a\equiv 0\pmod{p}$ and therefore $x^2\equiv a\pmod{p}.$ Thus, there are exactly two incongruent solutions modulo $p.$
\end{proof}

\begin{prop*}[Proposition 4.3]\label{prop:number-quad-residues}
	Let $p$ be an odd prime number. Then there are exactly $\frac{p-1}{2}$ incongruent quadratic residues modulo $p$ and exactly $\frac{p-1}{2}$ incongruent quadratic nonresidues modulo $p.$
\end{prop*}
\begin{proof}
	Consider the $p-1$ quadratic congruences
	\begin{align*}
 		x^2&\equiv 1\pmod{p}\\
		x^2&\equiv 2\pmod{p}\\
		&\vdots\\
		x^2&\equiv p-1\pmod{p}.
	\end{align*}
	Since each congruence has either zero or two incongruent solutions modulo $p$ by \nameref{prop:number-sqrts}, and no integer is a solution to more than one of the congruences, exactly half are solvable. Therefore,  there are exactly $\frac{p-1}{2}$ incongruent quadratic residues modulo $p$ and exactly $\frac{p-1}{2}$ incongruent quadratic nonresidues modulo $p.$
\end{proof}

%%%%%%%%%%%%%%%%%%%%%%%%%%


\end{document}
