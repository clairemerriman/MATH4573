\documentclass{ximera}
<<<<<<< Updated upstream
\usepackage{amssymb, latexsym, amsmath, amsthm, graphicx, amsthm,alltt,color, listings,multicol,xr-hyper,hyperref,aliascnt,enumitem}
=======
\usepackage{amssymb, latexsym, amsmath, amsthm, graphicx, amsthm,alltt,color, listings,multicol,hyperref}
\usepackage[capitalise,nameinlink]{cleveref}
>>>>>>> Stashed changes
\usepackage{xfrac}

\usepackage{parskip}
\usepackage[,margin=0.7in]{geometry}
\setlength{\textheight}{8.5in}

\usepackage{epstopdf}

\DeclareGraphicsExtensions{.eps}
\usepackage{tikz}


\usepackage{tkz-euclide}
%\usetkzobj{all}
\tikzstyle geometryDiagrams=[rounded corners=.5pt,ultra thick,color=black]
\colorlet{penColor}{black} % Color of a curve in a plot


\usepackage{subcaption}
\usepackage{float}
\usepackage{fancyhdr}
\usepackage{pdfpages}
\newcounter{includepdfpage}
\usepackage{makecell}


\usepackage{currfile}
\usepackage{xstring}




\graphicspath{  
{./otherDocuments/}
}

\author{Claire Merriman}
\newcommand{\classday}[1]{\def\classday{#1}}

%%%%%%%%%%%%%%%%%%%%%
% Counters and autoref for unnumbered environments
% Not needed??
%%%%%%%%%%%%%%%%%%%%%
<<<<<<< Updated upstream
\theoremstyle{plain}


\newtheorem*{namedthm}{Theorem}
\newcounter{thm}%makes pointer correct
\providecommand{\thmname}{Theorem}
=======

\crefname{problem}{problem}{problems}


% \theoremstyle{plain}


% \newtheorem*{namedthm}{Theorem}
% \newcounter{thm}%makes pointer correct
% \providecommand{\thmname}{Theorem}
>>>>>>> Stashed changes

\makeatletter
\NewDocumentEnvironment{thm*}{o}
 {%
  \IfValueTF{#1}
    {\namedthm[#1]\refstepcounter{thm}\def\@currentlabel{(#1)}}%
    {\namedthm}%
 }
 {%
  \endnamedthm
 }
\makeatother


\newtheorem*{namedprop}{Proposition}
\newcounter{prop}%makes pointer correct
\providecommand{\propname}{Proposition}

\makeatletter
\NewDocumentEnvironment{prop*}{o}
 {%
  \IfValueTF{#1}
    {\namedprop[#1]\refstepcounter{prop}\def\@currentlabel{(#1)}}%
    {\namedprop}%
 }
 {%
  \endnamedprop
 }
\makeatother

\newtheorem*{namedlem}{Lemma}
\newcounter{lem}%makes pointer correct
\providecommand{\lemname}{Lemma}

\makeatletter
\NewDocumentEnvironment{lem*}{o}
 {%
  \IfValueTF{#1}
    {\namedlem[#1]\refstepcounter{lem}\def\@currentlabel{(#1)}}%
    {\namedlem}%
 }
 {%
  \endnamedlem
 }
\makeatother

\newtheorem*{namedcor}{Corollary}
\newcounter{cor}%makes pointer correct
\providecommand{\corname}{Corollary}

\makeatletter
\NewDocumentEnvironment{cor*}{o}
 {%
  \IfValueTF{#1}
    {\namedcor[#1]\refstepcounter{cor}\def\@currentlabel{(#1)}}%
    {\namedcor}%
 }
 {%
  \endnamedcor
 }
\makeatother

\theoremstyle{definition}
\newtheorem*{annotation}{Annotation}
\newtheorem*{rubric}{Rubric}

\newtheorem*{innerrem}{Remark}
\newcounter{rem}%makes pointer correct
\providecommand{\remname}{Remark}

\makeatletter
\NewDocumentEnvironment{rem}{o}
 {%
  \IfValueTF{#1}
    {\innerrem[#1]\refstepcounter{rem}\def\@currentlabel{(#1)}}%
    {\innerrem}%
 }
 {%
  \endinnerrem
 }
\makeatother

\newtheorem*{innerdefn}{Definition}%%placeholder
\newcounter{defn}%makes pointer correct
\providecommand{\defnname}{Definition}

\makeatletter
\NewDocumentEnvironment{defn}{o}
 {%
  \IfValueTF{#1}
    {\innerdefn[#1]\refstepcounter{defn}\def\@currentlabel{(#1)}}%
    {\innerdefn}%
 }
 {%
  \endinnerdefn
 }
\makeatother

\newtheorem*{scratch}{Scratch Work}


\newtheorem*{namedconj}{Conjecture}
\newcounter{conj}%makes pointer correct
\providecommand{\conjname}{Conjecture}
\makeatletter
\NewDocumentEnvironment{conj}{o}
 {%
  \IfValueTF{#1}
    {\innerconj[#1]\refstepcounter{conj}\def\@currentlabel{(#1)}}%
    {\innerconj}%
 }
 {%
  \endinnerconj
 }
\makeatother

\newtheorem*{poll}{Poll question}
\newtheorem{tps}{Think-Pair-Share}[section]


\newenvironment{obj}{
	\textbf{Learning Objectives.} By the end of class, students will be able to:
		\begin{itemize}}
		{\!.\end{itemize}
		}

<<<<<<< Updated upstream
\newenvironment{pre}{
	\begin{description}
	}{
	\end{description}
}
=======

\ifinstructornotes
\newenvironment{pre}
  {{\textbf Reading assignment:}
  \begin{description}
    }{
	\end{description}
  }
\else
\newenvironment{pre}{ 
  \begin{trivlist}
  \item[]}
  {\end{trivlist}}
\fi
>>>>>>> Stashed changes


\newcounter{ex}%makes pointer correct
\providecommand{\exname}{Homework Problem}
\newenvironment{ex}[1][2in]%
{%Env start code
\problemEnvironmentStart{#1}{Homework Problem}
\refstepcounter{ex}
}
{%Env end code
\problemEnvironmentEnd
}

\newcommand{\inlineAnswer}[2][2 cm]{
    \ifhandout{\pdfOnly{\rule{#1}{0.4pt}}}
    \else{\answer{#2}}
    \fi
}


\ifhandout
\newenvironment{shortAnswer}[1][
    \vfill]
        {% Begin then result
        #1
            \begin{freeResponse}
            }
    {% Environment Ending Code
    \end{freeResponse}
    }
\else
\newenvironment{shortAnswer}[1][]
        {\begin{freeResponse}
            }
    {% Environment Ending Code
    \end{freeResponse}
    }
\fi

\let\question\relax
\let\endquestion\relax

\newtheoremstyle{ExerciseStyle}{\topsep}{\topsep}%%% space between body and thm
		{}                      %%% Thm body font
		{}                              %%% Indent amount (empty = no indent)
		{\bfseries}            %%% Thm head font
		{}                              %%% Punctuation after thm head
		{3em}                           %%% Space after thm head
		{{#1}~\thmnumber{#2}\thmnote{ \bfseries(#3)}}%%% Thm head spec
\theoremstyle{ExerciseStyle}
\newtheorem{br}{In-class Problem}

\newenvironment{sketch}
 {\begin{proof}[Sketch of Proof]}
 {\end{proof}}


\newcommand{\gt}{>}
\newcommand{\lt}{<}
\newcommand{\N}{\mathbb N}
\newcommand{\Q}{\mathbb Q}
\newcommand{\Z}{\mathbb Z}
\newcommand{\C}{\mathbb C}
\newcommand{\R}{\mathbb R}
\renewcommand{\H}{\mathbb{H}}
\newcommand{\lcm}{\operatorname{lcm}}
\newcommand{\nequiv}{\not\equiv}
\newcommand{\ord}{\operatorname{ord}}
\newcommand{\ds}{\displaystyle}
\newcommand{\floor}[1]{\left\lfloor #1\right\rfloor}
\newcommand{\legendre}[2]{\left(\frac{#1}{#2}\right)}



%%%%%%%%%%%%



\title{Existence of primitive roots modulo a prime}
\begin{document}
\begin{abstract}
\end{abstract}
\maketitle

%%%%%%%%%%%%%%%%%%%%%%%%%%

\begin{obj}
    \item Find the number of roots of unity modulo $m$
    \item Prove primitive roots exist modulo a prime
\end{obj}

We will now prove the existence of primivitive roots modulo a prime combining the two methods from the reading: we will show that when $d\mid p-1,$ there are $\phi(d)$ incongruent integers of order $d$ modulo $p,$ like Strayer. However, we will prove this using the method from Reading Lemma 10.3.4 instead of results from Chapter 3.

\begin{theorem}\label{thm:number-element-order-d}
    Let $p$ be a prime and let $d\in\Z$ with $d>0$ and $d\mid p-1.$ Then there are exactly $\phi(d)$ incongruent integers of order $d$ modulo $p.$
    \begin{proof}
        Let $p$ be a prime and let $d\in\Z$ with $d>0$ and $d\mid p-1.$ First we will prove the theorem for $d=q^s$ modulo $p$ where $q$ is prime and $s$ is a nonnegative integer.
    
        By \cref{prop:roots-unity}, there are exactly $q^s$ incongruent solutions to 
        \begin{equation}\label{eq:unity-2a}
            x^{q^s}\equiv 1\pmod{p}
        \end{equation} and exactly $q^{s-1}$ incongruent solutions to 
        \begin{equation}\label{eq:unity-2b}
            x^{q^{s-1}}\equiv 1\pmod{p}.
        \end{equation}
        Since $(x^{q^{s-1}})^q=x^{q^s},$ all solutions to \eqref{eq:unity-2b} are solutions to \eqref{eq:unity-2a}. 
        Thus, there are exactly $q^s-q^{s-1}=q^{s-1}(q-1)$ integers $a$ where $a^{q^s}\equiv 1\pmod{p}$ and $a^{q^{s-1}}\not\equiv 1\pmod{p}.$ Thus, by \cref{prop:order_divides_phi}, $\ord_p a \mid q^s$ and $\ord_p a\nmid q^{s-1}.$ Since $q$ is prime, $\ord_p a =q^s.$ By \nameref{thm:phi-prime-power}, $\phi(q^s)=q^s-q^{s-1}=q^{s-1}(q-1),$ so we have shown there are $\phi(q^s)$ incongruent integers with order $q^s$ modulo $p$.
    
        Now we will prove the general case. Let 
        \[d=q_1^{s_1}q_2^{s_2}\cdots q_k^{s_k}\]
        for distinct primes $q_1,q_2,\dots,q_k$ and positive integers $s_1,s_2,\dots,s_k.$ Let $a_1,a_2,\dots,a_k$ be elements of order $q_1^{s_1},q_2^{s_2},\dots, q_k^{s_k}$ respectively.
        %Since there are $\phi(q_1^{s_1})$ options for $a_1,$ $\phi(q_2^{s_2})$ options for $a_2,$ etc there are $\phi(q_1^{s_1})\phi(q_2^{s_2})\cdots\phi(q_k^{s_k})=\phi(d)$ ways to form $a=a_1a_2\cdots a_k.$ 
        Consider $a=a_1a_2\cdots a_k$ and $a^2, a^3,\dots,a^d$. By Homework 6, Problem 6, $a$ has order $q_1^{s_1}q_2^{s_2}\cdots q_k^{s_k}=d.$ 
        By \cref{prop:roots-unity}, there are exactly $d$ solutions to $x^d\equiv 1\pmod{p}$. 
        Thus, $a, a^2,\dots,a^d$ are all incongruent solutions to $x^d\equiv 1\pmod{p}$ by \cref{prop:order_divides_phi}. 
        By \cref{prop:compare-order},
        $\ord_p a^i=\frac{d}{(d,i)}=d$ if and only if $(d,i)=1.$ Since there are $\phi(d)$ such integers $i,$ there are in fact $\phi(d)$ incongruent integers with order $d$ modulo $p.$
    \end{proof}
\end{theorem}





\begin{corollary}\label{cor:number-prime-roots}
    Let $p$ be prime. There are exactly $\phi(p-1)$ primitive roots modulo $p.$
\end{corollary}


%%%%%%%%%%%%%%%%%%%%%%%%%%


\end{document}
