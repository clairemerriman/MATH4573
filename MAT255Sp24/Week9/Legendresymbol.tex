\documentclass{ximera}
<<<<<<< Updated upstream
\usepackage{amssymb, latexsym, amsmath, amsthm, graphicx, amsthm,alltt,color, listings,multicol,xr-hyper,hyperref,aliascnt,enumitem}
=======
\usepackage{amssymb, latexsym, amsmath, amsthm, graphicx, amsthm,alltt,color, listings,multicol,hyperref}
\usepackage[capitalise,nameinlink]{cleveref}
>>>>>>> Stashed changes
\usepackage{xfrac}

\usepackage{parskip}
\usepackage[,margin=0.7in]{geometry}
\setlength{\textheight}{8.5in}

\usepackage{epstopdf}

\DeclareGraphicsExtensions{.eps}
\usepackage{tikz}


\usepackage{tkz-euclide}
%\usetkzobj{all}
\tikzstyle geometryDiagrams=[rounded corners=.5pt,ultra thick,color=black]
\colorlet{penColor}{black} % Color of a curve in a plot


\usepackage{subcaption}
\usepackage{float}
\usepackage{fancyhdr}
\usepackage{pdfpages}
\newcounter{includepdfpage}
\usepackage{makecell}


\usepackage{currfile}
\usepackage{xstring}




\graphicspath{  
{./otherDocuments/}
}

\author{Claire Merriman}
\newcommand{\classday}[1]{\def\classday{#1}}

%%%%%%%%%%%%%%%%%%%%%
% Counters and autoref for unnumbered environments
% Not needed??
%%%%%%%%%%%%%%%%%%%%%
<<<<<<< Updated upstream
\theoremstyle{plain}


\newtheorem*{namedthm}{Theorem}
\newcounter{thm}%makes pointer correct
\providecommand{\thmname}{Theorem}
=======

\crefname{problem}{problem}{problems}


% \theoremstyle{plain}


% \newtheorem*{namedthm}{Theorem}
% \newcounter{thm}%makes pointer correct
% \providecommand{\thmname}{Theorem}
>>>>>>> Stashed changes

\makeatletter
\NewDocumentEnvironment{thm*}{o}
 {%
  \IfValueTF{#1}
    {\namedthm[#1]\refstepcounter{thm}\def\@currentlabel{(#1)}}%
    {\namedthm}%
 }
 {%
  \endnamedthm
 }
\makeatother


\newtheorem*{namedprop}{Proposition}
\newcounter{prop}%makes pointer correct
\providecommand{\propname}{Proposition}

\makeatletter
\NewDocumentEnvironment{prop*}{o}
 {%
  \IfValueTF{#1}
    {\namedprop[#1]\refstepcounter{prop}\def\@currentlabel{(#1)}}%
    {\namedprop}%
 }
 {%
  \endnamedprop
 }
\makeatother

\newtheorem*{namedlem}{Lemma}
\newcounter{lem}%makes pointer correct
\providecommand{\lemname}{Lemma}

\makeatletter
\NewDocumentEnvironment{lem*}{o}
 {%
  \IfValueTF{#1}
    {\namedlem[#1]\refstepcounter{lem}\def\@currentlabel{(#1)}}%
    {\namedlem}%
 }
 {%
  \endnamedlem
 }
\makeatother

\newtheorem*{namedcor}{Corollary}
\newcounter{cor}%makes pointer correct
\providecommand{\corname}{Corollary}

\makeatletter
\NewDocumentEnvironment{cor*}{o}
 {%
  \IfValueTF{#1}
    {\namedcor[#1]\refstepcounter{cor}\def\@currentlabel{(#1)}}%
    {\namedcor}%
 }
 {%
  \endnamedcor
 }
\makeatother

\theoremstyle{definition}
\newtheorem*{annotation}{Annotation}
\newtheorem*{rubric}{Rubric}

\newtheorem*{innerrem}{Remark}
\newcounter{rem}%makes pointer correct
\providecommand{\remname}{Remark}

\makeatletter
\NewDocumentEnvironment{rem}{o}
 {%
  \IfValueTF{#1}
    {\innerrem[#1]\refstepcounter{rem}\def\@currentlabel{(#1)}}%
    {\innerrem}%
 }
 {%
  \endinnerrem
 }
\makeatother

\newtheorem*{innerdefn}{Definition}%%placeholder
\newcounter{defn}%makes pointer correct
\providecommand{\defnname}{Definition}

\makeatletter
\NewDocumentEnvironment{defn}{o}
 {%
  \IfValueTF{#1}
    {\innerdefn[#1]\refstepcounter{defn}\def\@currentlabel{(#1)}}%
    {\innerdefn}%
 }
 {%
  \endinnerdefn
 }
\makeatother

\newtheorem*{scratch}{Scratch Work}


\newtheorem*{namedconj}{Conjecture}
\newcounter{conj}%makes pointer correct
\providecommand{\conjname}{Conjecture}
\makeatletter
\NewDocumentEnvironment{conj}{o}
 {%
  \IfValueTF{#1}
    {\innerconj[#1]\refstepcounter{conj}\def\@currentlabel{(#1)}}%
    {\innerconj}%
 }
 {%
  \endinnerconj
 }
\makeatother

\newtheorem*{poll}{Poll question}
\newtheorem{tps}{Think-Pair-Share}[section]


\newenvironment{obj}{
	\textbf{Learning Objectives.} By the end of class, students will be able to:
		\begin{itemize}}
		{\!.\end{itemize}
		}

<<<<<<< Updated upstream
\newenvironment{pre}{
	\begin{description}
	}{
	\end{description}
}
=======

\ifinstructornotes
\newenvironment{pre}
  {{\textbf Reading assignment:}
  \begin{description}
    }{
	\end{description}
  }
\else
\newenvironment{pre}{ 
  \begin{trivlist}
  \item[]}
  {\end{trivlist}}
\fi
>>>>>>> Stashed changes


\newcounter{ex}%makes pointer correct
\providecommand{\exname}{Homework Problem}
\newenvironment{ex}[1][2in]%
{%Env start code
\problemEnvironmentStart{#1}{Homework Problem}
\refstepcounter{ex}
}
{%Env end code
\problemEnvironmentEnd
}

\newcommand{\inlineAnswer}[2][2 cm]{
    \ifhandout{\pdfOnly{\rule{#1}{0.4pt}}}
    \else{\answer{#2}}
    \fi
}


\ifhandout
\newenvironment{shortAnswer}[1][
    \vfill]
        {% Begin then result
        #1
            \begin{freeResponse}
            }
    {% Environment Ending Code
    \end{freeResponse}
    }
\else
\newenvironment{shortAnswer}[1][]
        {\begin{freeResponse}
            }
    {% Environment Ending Code
    \end{freeResponse}
    }
\fi

\let\question\relax
\let\endquestion\relax

\newtheoremstyle{ExerciseStyle}{\topsep}{\topsep}%%% space between body and thm
		{}                      %%% Thm body font
		{}                              %%% Indent amount (empty = no indent)
		{\bfseries}            %%% Thm head font
		{}                              %%% Punctuation after thm head
		{3em}                           %%% Space after thm head
		{{#1}~\thmnumber{#2}\thmnote{ \bfseries(#3)}}%%% Thm head spec
\theoremstyle{ExerciseStyle}
\newtheorem{br}{In-class Problem}

\newenvironment{sketch}
 {\begin{proof}[Sketch of Proof]}
 {\end{proof}}


\newcommand{\gt}{>}
\newcommand{\lt}{<}
\newcommand{\N}{\mathbb N}
\newcommand{\Q}{\mathbb Q}
\newcommand{\Z}{\mathbb Z}
\newcommand{\C}{\mathbb C}
\newcommand{\R}{\mathbb R}
\renewcommand{\H}{\mathbb{H}}
\newcommand{\lcm}{\operatorname{lcm}}
\newcommand{\nequiv}{\not\equiv}
\newcommand{\ord}{\operatorname{ord}}
\newcommand{\ds}{\displaystyle}
\newcommand{\floor}[1]{\left\lfloor #1\right\rfloor}
\newcommand{\legendre}[2]{\left(\frac{#1}{#2}\right)}



%%%%%%%%%%%%



\title{Legendre symbol}
\begin{document}
\begin{abstract}
\end{abstract}
\maketitle

%%%%%%%%%%%%%%%%%%%%%%%%%%

\begin{obj}
    \item Define the Legendre symbol
    \item Prove basic facts about the Legendre symbol
    \item Use the definition and basic facts to find the Legendre symbol for specific examples
\end{obj}


\begin{pre}
<<<<<<< Updated upstream
    \item[Reading:] Strayer Section 4.2 through Example 4
=======
    \item[Read:] Strayer Section 4.2 through Example 4
>>>>>>> Stashed changes
    \item[Turn in:] Exercise 12
     Use Euler's Criterion to evaluate the following Legendre symbols 
	\begin{enumerate}
 		\item $\left(\frac{11}{23}\right)$
		
		\begin{solution}
 			$\left(\frac{11}{23}\right)\equiv 11^{(23-1)/2}\equiv 11^{11}\pmod{23}$ By Euler's Criterion. Then
			\[11^{11}\equiv (11^{2})^{5}(11)\equiv 6^5(11)\equiv (6^2)(6^3)(11)\equiv (13)(9)(11)\equiv(-90)(11)\equiv -1\pmod{23}\]
		\end{solution}
		
		\item $\left(\frac{-6}{11}\right)$
		
		\begin{solution}
 			$\left(\frac{-6}{11}\right)\equiv (-6)^{(11-1)/2}\equiv (-6)^{5}\pmod{11}$ By Euler's Criterion. Then
			\[(-6)^{5}\equiv ((6)^{2})^{2}(-6)\equiv 3^2(-6)\equiv -54 \equiv 1\pmod{11}\]
		\end{solution}
	\end{enumerate}
 

<<<<<<< Updated upstream
    \end{pre}


%%%%%%%%%%%%%%%%%%%%%%%%%%
\subsection{Quiz (15 minutes)}
%%%%%%%%%%%%%%%%%%%%%%%%%%
Technical difficults with printer and projector.

%%%%%%%%%%%%%%%%%%%%%%%%%%
\subsection{Legendre symbol (35 minutes)}
%%%%%%%%%%%%%%%%%%%%%%%%%%
=======
\end{pre}


>>>>>>> Stashed changes

\begin{definition}[Legendre symbol]\label{defn:legendre}
    Let $p$ be an odd prime number and let $a\in\Z$ with $p\nmid a$. The \emph{Legendre symbol}, denoted $\legendre{a}{p}$, is
        \[
            \legendre{a}{p}=
            \begin{cases}
                1, & \text{ if $a$ is a quadratic residue modulo $p$} \\
                -1, & \text{ if $a$ is a quadratic nonresidue modulo $p$} 
            \end{cases}
        \]
\end{definition}

\begin{theorem}[Euler's Criterion]\label{thm:euler-quads}
    Let $p$ be an odd prime and $a\in\Z$ with $p\nmid a.$ Then \[\legendre{a}{p}\equiv a^{(p-1)/2}\pmod{p}\]
\end{theorem}

We will not prove this today, but we will use it to go over the solution to the reading assignment and to prove the following proposition.

<<<<<<< Updated upstream
\begin{proposition}[Proposition 4.5]\label{prop:legendre-facts}
	Let $p$ be an odd prime number and $a,b\in\Z$ with $p\nmid a$ and $p\nmid b.$ Then 
	\begin{enumerate}[label=(\alph*)]
=======
\begin{proposition}\label{prop:legendre-facts}
	Let $p$ be an odd prime number and $a,b\in\Z$ with $p\nmid a$ and $p\nmid b.$ Then 
	\begin{enumerate}
>>>>>>> Stashed changes
		\item $\legendre{a^2}{p}=1$ \label{squares-are-square}
		\item If $a\equiv b\pmod{p}$ then $\legendre{a}{p}=\legendre{b}{p}$ \label{legendre-respects-mod}
		\item $\legendre{ab}{p}=\legendre{a}{p}\legendre{b}{p}$ \label{legendre-mult}
	\end{enumerate}
\end{proposition}


\begin{proof}
    Let $p$ be an odd prime number and $a,b\in\Z$ with $p\nmid a$ and $p\nmid b.$ Then $a^2$ is a quadratic residue modulo $p,$ by definition, so $\legendre{a^2}{p}=1$ by the definition of the Legendre symbol.

    If $a\equiv b\pmod{p},$ then either both $a$ and $b$ are quadratic residues modulo $p$ or both $a$ and $b$ are quadratic nonresidues modulo $p.$ Thus $\legendre{a}{p}=\legendre{b}{p}.$

    For the last part, \nameref{thm:euler-quads} gives \[\legendre{ab}{p}\equiv (ab)^{(p-1)/2}\equiv (a^{(p-1)/2})(b^{(p-1)/2})\equiv \legendre{a}{p}\legendre{b}{p}\pmod{p}\]
\end{proof}

<<<<<<< Updated upstream
=======
\begin{remark}
	Some sources define $\legendre{a}{p}=0$ when $p\mid a.$ In this case,  Let $p$ be an odd prime and $a\in\Z$.
	If $p\mid a,$ then $a^{(p-1)/2}\equiv 0^{(p-1)/2}\equiv 0\equiv\legendre{a}{p}\pmod{p}.$
\end{remark}


\begin{theorem}\label{thm:residue-neg1}
	Let $p$ be an odd prime number. Then 
	\[
		\legendre{-1}{p}=
			\begin{cases}
 				1, & p\equiv 1\pmod{4}\\
				-1, & p\equiv 3\pmod{4}
			\end{cases}.
	\]
	\begin{proof}
		Let $p$ be an odd prime number. Then from \nameref{thm:euler-quads}, $\legendre{-1}{p}\equiv (-1)^{(p-1)/2}\pmod{p}.$ Since both values are $\pm1,$ we can say $\legendre{-1}{p}=(-1)^{(p-1)/2}.$
	
		If $p\equiv 1\pmod{4},$ then there exists $k\in\Z$ such that $p=4k+1.$ Thus, $\frac{p-1}{2}=2k$ and 
			\[
				\legendre{-1}{p}=(-1)^{(p-1)/2}=(-1)^{2k}=1.
			\]
		
		If $p\equiv 3\pmod{4},$ then there exists $k\in\Z$ such that $p=4k+3.$ Thus, $\frac{p-1}{2}=2k+1$ and 
			\[
				\legendre{-1}{p}=(-1)^{(p-1)/2}=(-1)^{2k+1}=-1.
			\]
	\end{proof}
\end{theorem}


%%%%%%%%%%%%%%%%%%%%%%%%%%
>>>>>>> Stashed changes


\end{document}
