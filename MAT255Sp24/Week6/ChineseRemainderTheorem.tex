\documentclass{ximera}
\usepackage{amssymb, latexsym, amsmath, amsthm, graphicx, amsthm,alltt,color, listings,multicol,xr-hyper,hyperref,aliascnt,enumitem}
\usepackage{xfrac}

\usepackage{parskip}
\usepackage[,margin=0.7in]{geometry}
\setlength{\textheight}{8.5in}

\usepackage{epstopdf}

\DeclareGraphicsExtensions{.eps}
\usepackage{tikz}


\usepackage{tkz-euclide}
%\usetkzobj{all}
\tikzstyle geometryDiagrams=[rounded corners=.5pt,ultra thick,color=black]
\colorlet{penColor}{black} % Color of a curve in a plot


\usepackage{subcaption}
\usepackage{float}
\usepackage{fancyhdr}
\usepackage{pdfpages}
\newcounter{includepdfpage}
\usepackage{makecell}


\usepackage{currfile}
\usepackage{xstring}




\graphicspath{  
{./otherDocuments/}
}

\author{Claire Merriman}
\newcommand{\classday}[1]{\def\classday{#1}}

%%%%%%%%%%%%%%%%%%%%%
% Counters and autoref for unnumbered environments
% Not needed??
%%%%%%%%%%%%%%%%%%%%%
\theoremstyle{plain}


\newtheorem*{namedthm}{Theorem}
\newcounter{thm}%makes pointer correct
\providecommand{\thmname}{Theorem}

\makeatletter
\NewDocumentEnvironment{thm*}{o}
 {%
  \IfValueTF{#1}
    {\namedthm[#1]\refstepcounter{thm}\def\@currentlabel{(#1)}}%
    {\namedthm}%
 }
 {%
  \endnamedthm
 }
\makeatother


\newtheorem*{namedprop}{Proposition}
\newcounter{prop}%makes pointer correct
\providecommand{\propname}{Proposition}

\makeatletter
\NewDocumentEnvironment{prop*}{o}
 {%
  \IfValueTF{#1}
    {\namedprop[#1]\refstepcounter{prop}\def\@currentlabel{(#1)}}%
    {\namedprop}%
 }
 {%
  \endnamedprop
 }
\makeatother

\newtheorem*{namedlem}{Lemma}
\newcounter{lem}%makes pointer correct
\providecommand{\lemname}{Lemma}

\makeatletter
\NewDocumentEnvironment{lem*}{o}
 {%
  \IfValueTF{#1}
    {\namedlem[#1]\refstepcounter{lem}\def\@currentlabel{(#1)}}%
    {\namedlem}%
 }
 {%
  \endnamedlem
 }
\makeatother

\newtheorem*{namedcor}{Corollary}
\newcounter{cor}%makes pointer correct
\providecommand{\corname}{Corollary}

\makeatletter
\NewDocumentEnvironment{cor*}{o}
 {%
  \IfValueTF{#1}
    {\namedcor[#1]\refstepcounter{cor}\def\@currentlabel{(#1)}}%
    {\namedcor}%
 }
 {%
  \endnamedcor
 }
\makeatother

\theoremstyle{definition}
\newtheorem*{annotation}{Annotation}
\newtheorem*{rubric}{Rubric}

\newtheorem*{innerrem}{Remark}
\newcounter{rem}%makes pointer correct
\providecommand{\remname}{Remark}

\makeatletter
\NewDocumentEnvironment{rem}{o}
 {%
  \IfValueTF{#1}
    {\innerrem[#1]\refstepcounter{rem}\def\@currentlabel{(#1)}}%
    {\innerrem}%
 }
 {%
  \endinnerrem
 }
\makeatother

\newtheorem*{innerdefn}{Definition}%%placeholder
\newcounter{defn}%makes pointer correct
\providecommand{\defnname}{Definition}

\makeatletter
\NewDocumentEnvironment{defn}{o}
 {%
  \IfValueTF{#1}
    {\innerdefn[#1]\refstepcounter{defn}\def\@currentlabel{(#1)}}%
    {\innerdefn}%
 }
 {%
  \endinnerdefn
 }
\makeatother

\newtheorem*{scratch}{Scratch Work}


\newtheorem*{namedconj}{Conjecture}
\newcounter{conj}%makes pointer correct
\providecommand{\conjname}{Conjecture}
\makeatletter
\NewDocumentEnvironment{conj}{o}
 {%
  \IfValueTF{#1}
    {\innerconj[#1]\refstepcounter{conj}\def\@currentlabel{(#1)}}%
    {\innerconj}%
 }
 {%
  \endinnerconj
 }
\makeatother

\newtheorem*{poll}{Poll question}
\newtheorem{tps}{Think-Pair-Share}[section]


\newenvironment{obj}{
	\textbf{Learning Objectives.} By the end of class, students will be able to:
		\begin{itemize}}
		{\!.\end{itemize}
		}

\newenvironment{pre}{
	\begin{description}
	}{
	\end{description}
}


\newcounter{ex}%makes pointer correct
\providecommand{\exname}{Homework Problem}
\newenvironment{ex}[1][2in]%
{%Env start code
\problemEnvironmentStart{#1}{Homework Problem}
\refstepcounter{ex}
}
{%Env end code
\problemEnvironmentEnd
}

\newcommand{\inlineAnswer}[2][2 cm]{
    \ifhandout{\pdfOnly{\rule{#1}{0.4pt}}}
    \else{\answer{#2}}
    \fi
}


\ifhandout
\newenvironment{shortAnswer}[1][
    \vfill]
        {% Begin then result
        #1
            \begin{freeResponse}
            }
    {% Environment Ending Code
    \end{freeResponse}
    }
\else
\newenvironment{shortAnswer}[1][]
        {\begin{freeResponse}
            }
    {% Environment Ending Code
    \end{freeResponse}
    }
\fi

\let\question\relax
\let\endquestion\relax

\newtheoremstyle{ExerciseStyle}{\topsep}{\topsep}%%% space between body and thm
		{}                      %%% Thm body font
		{}                              %%% Indent amount (empty = no indent)
		{\bfseries}            %%% Thm head font
		{}                              %%% Punctuation after thm head
		{3em}                           %%% Space after thm head
		{{#1}~\thmnumber{#2}\thmnote{ \bfseries(#3)}}%%% Thm head spec
\theoremstyle{ExerciseStyle}
\newtheorem{br}{In-class Problem}

\newenvironment{sketch}
 {\begin{proof}[Sketch of Proof]}
 {\end{proof}}


\newcommand{\gt}{>}
\newcommand{\lt}{<}
\newcommand{\N}{\mathbb N}
\newcommand{\Q}{\mathbb Q}
\newcommand{\Z}{\mathbb Z}
\newcommand{\C}{\mathbb C}
\newcommand{\R}{\mathbb R}
\renewcommand{\H}{\mathbb{H}}
\newcommand{\lcm}{\operatorname{lcm}}
\newcommand{\nequiv}{\not\equiv}
\newcommand{\ord}{\operatorname{ord}}
\newcommand{\ds}{\displaystyle}
\newcommand{\floor}[1]{\left\lfloor #1\right\rfloor}
\newcommand{\legendre}[2]{\left(\frac{#1}{#2}\right)}



%%%%%%%%%%%%



\title{Chinese Remainder Theorem}
\begin{document}
\begin{abstract}
\end{abstract}
\maketitle

%%%%%%%%%%%%%%%%%%%%%%%%%%

\begin{obj}
\item Solve system of linear equations in one variable.
 \item Prove the Chinese Remainder Theorem.
\end{obj}





\begin{example}
    Consider the system of linear equations 
    \begin{align*}
        x &\equiv 2 \pmod{5}\\
        x &\equiv 3 \pmod{7}\\
        x &\equiv 1 \pmod{8}.
    \end{align*}

    A slow way to find an integer $x$ that satisfies all three congruences is to write out the congruence classes:
    \begin{align*}
        2, 2+5, 2+5(2), \boxed{2+5(3)}, \dots\\
        3, 3+7, \boxed{3+7(2)}, 3+7(3), \dots\\
        1, 1+8, 1+8(2), \boxed{1+8(3)}, \dots
    \end{align*}
    and see what integers are on all three lists. In addition to being tedius, we this doesn't help find \emph{all} such integers.

    To find all such integers, define $M=5(7)(8)=280,$ and $M_1=\frac{M}{5}=7(8),M_2=\frac{M}{7}=5(8),M_3=\frac{M}{8}=5(7).$ Then each $M_i$ is relatively prime to $M$ by construction. Thus, by \nameref{cor:condition-invertible} the congruences
    \begin{align*}
        M_1x_1 & \equiv 1\pmod 5, & 7(8)x_1&\equiv x_1 \equiv 1\pmod 5\\
        M_2x_2 & \equiv 1\pmod 7, & 5(8)x_2 &\equiv 5x_2 \equiv 1\pmod 7\\
        M_3x_3 & \equiv 1\pmod 8, & 5(7)x_3&\equiv 3x_3\equiv 1\pmod 8
    \end{align*}
    have solutions. Thus, $x_1\equiv 1\pmod{5}, x_2\equiv 3\pmod{7},$ and $x_3\equiv 3\pmod{8}.$

    Note that 
    \begin{align*}
        M_1x_1(2)&=56(1)(2)\equiv 2\pmod 5, & M_2&\equiv M_3\equiv 0\pmod{5}\\
        M_2x_2(3)&=40(3)(3)\equiv 3\pmod 7, & M_1&\equiv M_3\equiv 0\pmod{7}\\
        M_3x_3(1)&=35(3)(1)\equiv 1\pmod 8, & M_1&\equiv M_2\equiv 0\pmod{8}
    \end{align*}

    Thus, \[x=M_1x_1(2)+M_2x_2(3)+M_3x_3(1)=56(1)(2)+40(3)(3)+35(3)(1)\]
    is a solution to all three congruences.
\end{example}

\begin{theorem}[Chinese Remainder Theorem]\label{CRT}
    Let $m_1,m_2,\dots m_k$ be pairwise relatively prime positive integers (that is, any pair $\gcd(m_i,m_j)=1$ when $i\neq j$). Let $b_1, b_2,\dots, b_k$ be integers. Then the system of congruences 
   \begin{align*}
    x&\equiv b_1 \pmod{m_1}\\
    x&\equiv b_2 \pmod{m_2}\\
       &\vdots\\
     x&\equiv b_n \pmod{m_k}
   \end{align*}
   has a unique solution modulo $M=m_1m_2\dots m_k$. This solution has the form 
   \[x=M_1x_1b_1+M_2x_2b_2+\cdot+M_kx_kb_k,\] where $M_i=\frac{M}{m_i}$ and $M_i x_i\equiv 1 \pmod{m_i}$.
   
   \begin{proof} Let $m_1,m_2,\dots m_k$ be pairwise relatively prime positive integers.
    We start by constructing a solution modulo $M=m_1m_2\dots m_k$. By construction, $M_i=\frac{M}{m_i}$ is an integer. Since each the $m_i$ are pairwise relatively prime, $\left(M_i, m_i\right)=1$. Thus, by \nameref{cor:condition-invertible}, for each $i$ there is an integer $x_i$ where $M_i x_i\equiv 1 \pmod{m_i}$. Thus $M_i x_i b_i\equiv b_i\pmod{m_i}$. We also have that $(M_i, m_j)=m_j$ when $i\neq j$, so $M_i b_i\equiv 0 \pmod{m_j}$ when $i\neq j$.  Let 
    \[x=M_1x_1b_1+M_2x_2b_2+\cdot+M_kx_kb_k.\] 
    Then $x\equiv M_i x_i b_i\equiv b_i\pmod{m_i}$ for each $i=1,2,\dots,k$ and $x\equiv M_i x_i b_i\equiv 0\pmod{m_j}$ when $i\neq j.$ Thus, we have found a solution to the system of equivalences.
    
    To show the solution is unique modulo $M,$ consider two solutions $x_1,x_2.$ Then $x_1\equiv x_2\pmod{m_i}$ for each $i=1,2,\dots,k.$ Thus $m_i\mid x_2-x_1$. Since $(m_i,m_j)=1$ when $i\neq j,$ $M=[m_1,m_2,\dots,m_k]$ and $M\mid x_2-x_1.$ Thus, $x_1\equiv x_2\pmod M.$ 
   \end{proof}
\end{theorem}  
%%%%%%%%%%%%%%%%%%%%%%%%%%

%%%%%%%%%%%%%%%%%%%%%%%%%


\end{document}
