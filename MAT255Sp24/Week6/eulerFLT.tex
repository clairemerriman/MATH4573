\documentclass{ximera}
\usepackage{amssymb, latexsym, amsmath, amsthm, graphicx, amsthm,alltt,color, listings,multicol,xr-hyper,hyperref,aliascnt,enumitem}
\usepackage{xfrac}

\usepackage{parskip}
\usepackage[,margin=0.7in]{geometry}
\setlength{\textheight}{8.5in}

\usepackage{epstopdf}

\DeclareGraphicsExtensions{.eps}
\usepackage{tikz}


\usepackage{tkz-euclide}
%\usetkzobj{all}
\tikzstyle geometryDiagrams=[rounded corners=.5pt,ultra thick,color=black]
\colorlet{penColor}{black} % Color of a curve in a plot


\usepackage{subcaption}
\usepackage{float}
\usepackage{fancyhdr}
\usepackage{pdfpages}
\newcounter{includepdfpage}
\usepackage{makecell}


\usepackage{currfile}
\usepackage{xstring}




\graphicspath{  
{./otherDocuments/}
}

\author{Claire Merriman}
\newcommand{\classday}[1]{\def\classday{#1}}

%%%%%%%%%%%%%%%%%%%%%
% Counters and autoref for unnumbered environments
% Not needed??
%%%%%%%%%%%%%%%%%%%%%
\theoremstyle{plain}


\newtheorem*{namedthm}{Theorem}
\newcounter{thm}%makes pointer correct
\providecommand{\thmname}{Theorem}

\makeatletter
\NewDocumentEnvironment{thm*}{o}
 {%
  \IfValueTF{#1}
    {\namedthm[#1]\refstepcounter{thm}\def\@currentlabel{(#1)}}%
    {\namedthm}%
 }
 {%
  \endnamedthm
 }
\makeatother


\newtheorem*{namedprop}{Proposition}
\newcounter{prop}%makes pointer correct
\providecommand{\propname}{Proposition}

\makeatletter
\NewDocumentEnvironment{prop*}{o}
 {%
  \IfValueTF{#1}
    {\namedprop[#1]\refstepcounter{prop}\def\@currentlabel{(#1)}}%
    {\namedprop}%
 }
 {%
  \endnamedprop
 }
\makeatother

\newtheorem*{namedlem}{Lemma}
\newcounter{lem}%makes pointer correct
\providecommand{\lemname}{Lemma}

\makeatletter
\NewDocumentEnvironment{lem*}{o}
 {%
  \IfValueTF{#1}
    {\namedlem[#1]\refstepcounter{lem}\def\@currentlabel{(#1)}}%
    {\namedlem}%
 }
 {%
  \endnamedlem
 }
\makeatother

\newtheorem*{namedcor}{Corollary}
\newcounter{cor}%makes pointer correct
\providecommand{\corname}{Corollary}

\makeatletter
\NewDocumentEnvironment{cor*}{o}
 {%
  \IfValueTF{#1}
    {\namedcor[#1]\refstepcounter{cor}\def\@currentlabel{(#1)}}%
    {\namedcor}%
 }
 {%
  \endnamedcor
 }
\makeatother

\theoremstyle{definition}
\newtheorem*{annotation}{Annotation}
\newtheorem*{rubric}{Rubric}

\newtheorem*{innerrem}{Remark}
\newcounter{rem}%makes pointer correct
\providecommand{\remname}{Remark}

\makeatletter
\NewDocumentEnvironment{rem}{o}
 {%
  \IfValueTF{#1}
    {\innerrem[#1]\refstepcounter{rem}\def\@currentlabel{(#1)}}%
    {\innerrem}%
 }
 {%
  \endinnerrem
 }
\makeatother

\newtheorem*{innerdefn}{Definition}%%placeholder
\newcounter{defn}%makes pointer correct
\providecommand{\defnname}{Definition}

\makeatletter
\NewDocumentEnvironment{defn}{o}
 {%
  \IfValueTF{#1}
    {\innerdefn[#1]\refstepcounter{defn}\def\@currentlabel{(#1)}}%
    {\innerdefn}%
 }
 {%
  \endinnerdefn
 }
\makeatother

\newtheorem*{scratch}{Scratch Work}


\newtheorem*{namedconj}{Conjecture}
\newcounter{conj}%makes pointer correct
\providecommand{\conjname}{Conjecture}
\makeatletter
\NewDocumentEnvironment{conj}{o}
 {%
  \IfValueTF{#1}
    {\innerconj[#1]\refstepcounter{conj}\def\@currentlabel{(#1)}}%
    {\innerconj}%
 }
 {%
  \endinnerconj
 }
\makeatother

\newtheorem*{poll}{Poll question}
\newtheorem{tps}{Think-Pair-Share}[section]


\newenvironment{obj}{
	\textbf{Learning Objectives.} By the end of class, students will be able to:
		\begin{itemize}}
		{\!.\end{itemize}
		}

\newenvironment{pre}{
	\begin{description}
	}{
	\end{description}
}


\newcounter{ex}%makes pointer correct
\providecommand{\exname}{Homework Problem}
\newenvironment{ex}[1][2in]%
{%Env start code
\problemEnvironmentStart{#1}{Homework Problem}
\refstepcounter{ex}
}
{%Env end code
\problemEnvironmentEnd
}

\newcommand{\inlineAnswer}[2][2 cm]{
    \ifhandout{\pdfOnly{\rule{#1}{0.4pt}}}
    \else{\answer{#2}}
    \fi
}


\ifhandout
\newenvironment{shortAnswer}[1][
    \vfill]
        {% Begin then result
        #1
            \begin{freeResponse}
            }
    {% Environment Ending Code
    \end{freeResponse}
    }
\else
\newenvironment{shortAnswer}[1][]
        {\begin{freeResponse}
            }
    {% Environment Ending Code
    \end{freeResponse}
    }
\fi

\let\question\relax
\let\endquestion\relax

\newtheoremstyle{ExerciseStyle}{\topsep}{\topsep}%%% space between body and thm
		{}                      %%% Thm body font
		{}                              %%% Indent amount (empty = no indent)
		{\bfseries}            %%% Thm head font
		{}                              %%% Punctuation after thm head
		{3em}                           %%% Space after thm head
		{{#1}~\thmnumber{#2}\thmnote{ \bfseries(#3)}}%%% Thm head spec
\theoremstyle{ExerciseStyle}
\newtheorem{br}{In-class Problem}

\newenvironment{sketch}
 {\begin{proof}[Sketch of Proof]}
 {\end{proof}}


\newcommand{\gt}{>}
\newcommand{\lt}{<}
\newcommand{\N}{\mathbb N}
\newcommand{\Q}{\mathbb Q}
\newcommand{\Z}{\mathbb Z}
\newcommand{\C}{\mathbb C}
\newcommand{\R}{\mathbb R}
\renewcommand{\H}{\mathbb{H}}
\newcommand{\lcm}{\operatorname{lcm}}
\newcommand{\nequiv}{\not\equiv}
\newcommand{\ord}{\operatorname{ord}}
\newcommand{\ds}{\displaystyle}
\newcommand{\floor}[1]{\left\lfloor #1\right\rfloor}
\newcommand{\legendre}[2]{\left(\frac{#1}{#2}\right)}



%%%%%%%%%%%%



<<<<<<< Updated upstream
\title{Euler's Theorem and Fermat's Little Theorem}
=======
\title{Euler's Theorem and Fermat's Little Theorem}\label{sec:FlT-eulerFLT}
>>>>>>> Stashed changes
\begin{document}
\begin{abstract}
\end{abstract}
\maketitle

%%%%%%%%%%%%%%%%%%%%%%%%%%
%%%%%%%%%%%%%%%%%%%%%%%%%%
\begin{obj}
	\item Define and find a reduced residue system modulo $m$
	\item Define the Euler $\phi$-function $\phi(n)$
	\item Prove Euler's Generalization of Fermat's Little Theorem
\end{obj}


<<<<<<< Updated upstream
\begin{instructorNotes}

\begin{pre} \item[Read] Strayer, Section 2.5
=======

\begin{pre} 
    \item[Read] Strayer, Section 2.5
>>>>>>> Stashed changes

    \item[Turn in]  
    
    Exercise 50. Prove that $9^{10} = 1\pmod{11}$ by following the steps of the proof of Fermat's Little Theorem.
    
    \begin{solution}
<<<<<<< Updated upstream
        Consider the $10$ integers given by $9,2(9), 3(9),\dots, 9(10).$ Note that $11\mid 9i$ for $i=1,2,\dots,10$ since $11$ is prime and $11\nmid 10$ and $11\nmid i.$ By \nameref{cor:condition-invertible}, since $(9,11)=1$ if $9i\equiv 9j\pmod{11}$ implies $i\equiv j\pmod{11}.$ Therefore, no two of $9,2(9), 3(9),\dots, 9(10)$ are congruent modulo $11.$ So the least nonnegative residues modulo $11$ of the integers $9,2(9), 3(9),\dots, 9(10),$ taken in some order, must be $1,2,\dots, p-1.$ Then \[(9)(2(9)) (3(9))\cdots (9(10))\equiv (1)(2)\cdots (10)\pmod{11}\] or, equivalently, \[9^{10}10!\equiv 10!\pmod{11}.\] By \nameref{Wilson}, the congruence above becomes $-9^{10}\equiv -1\pmod{11},$ which is equivalent to $9^{10}\equiv 1\pmod{11}.$
    \end{solution}
\end{pre}
    
\end{instructorNotes}
=======
        Consider the $10$ integers given by $9,2(9), 3(9),\dots, 9(10).$ Note that $11\mid 9i$ for $i=1,2,\dots,10$ since $11$ is prime and $11\nmid 10$ and $11\nmid i.$ By \cref{cor:condition-invertible}, since $(9,11)=1$ if $9i\equiv 9j\pmod{11}$ implies $i\equiv j\pmod{11}.$ Therefore, no two of $9,2(9), 3(9),\dots, 9(10)$ are congruent modulo $11.$ So the least nonnegative residues modulo $11$ of the integers $9,2(9), 3(9),\dots, 9(10),$ taken in some order, must be $1,2,\dots, p-1.$ Then \[(9)(2(9)) (3(9))\cdots (9(10))\equiv (1)(2)\cdots (10)\pmod{11}\] or, equivalently, \[9^{10}10!\equiv 10!\pmod{11}.\] By \nameref{Wilson}, the congruence above becomes $-9^{10}\equiv -1\pmod{11},$ which is equivalent to $9^{10}\equiv 1\pmod{11}.$
    \end{solution}
\end{pre}
    
>>>>>>> Stashed changes


\begin{instructorNotes}
    There are several different ways to present the material in Sections 2.4 through 2.6. In class, we will do the other order: Fermat's Little Theorem to prove Wilson's Theorem. I will keep the result numbering from the book, so they will be out of order.
\end{instructorNotes}

\begin{defn}[reduced residue system modulo $m$]\label{defn:reduced-res-sys}
    Let $m$ be a positive integer. We say that $\{r_1,r_2,\dots,r_k\}$ is a \emph{reduced residue system modulo $m$} if 
    \begin{itemize}
        \item $(r_i,m)=1$ for all $i=1,2,\dots,k,$
        \item $r_i\not \equiv r_j \pmod {m}$ when $i\neq j,$
        \item for all $a\in\Z$ with $(a,m)=1,$ $a\equiv r_1\pmod{p}$ for some $i=1,2,\dots,k.$ 
    \end{itemize}
\end{defn}

\begin{example}\label{example:reduced-sys}
    
    \begin{itemize}
        \item The sets $\{1,2,3,4,5,6\}$ and $\{5,10,15,20,25,30,35\}$ are both reduced residue systems modulo $7.$
        
        \item If $p$ is prime, then $\{1,2,\dots,p-1\}$ is a complete residue system modulo $p.$ If $p\neq 5,$ $\{5,10,\dots, 5(p-1)\}$ is a complete residue system modulo $p.$
        
        \item The sets $\{1,5,7,11\}$ and $\{5,25,35,55\}$ are both reduced residue systems modulo $12.$
    \end{itemize}
\end{example}


\begin{lemma}\label{lem:reduced-sys}
    Let $m$ be a positive integer and let $\{r_1,r_2,\dots,r_k\}$ be a reduced residue system modulo $m.$ If $a\in\Z$ with $(a,m)=1,$ then $\{ar_1,ar_2,\dots,ar_k\}$ is a reduced residue system modulo $m.$
\end{lemma}
This result is also implicitly used in the proof of \nameref{FlT} since $\{1,2,\dots,p-1\}$ is a reduced residue system.

\begin{proof}
<<<<<<< Updated upstream
    Let $\{r_1,r_2,\dots,r_k\}$ be a reduced residue system modulo $m$ and $a\in\Z$ with $(a,m)=1.$ Since $\{r_1,r_2,\dots,r_k\}$ and $\{ar_1,ar_2,\dots,ar_k\}$ have the same number of elements, it suffices to show that $(ar_i,m)=1$ and  $ar_i\not\equiv ar_j\pmod{m}$ for $i\neq j.$ If there exist some prime $p$ such that $p\mid(ar_i,m)$ then $p\mid ar_i$ and $p\mid m$ by \autoref{defn:gcd}. By \nameref{lem:irreducible-prime}, $p\mid a$ or $p\mid r_i$, so either $p\mid (a,m)$ or $p\mid (r_i,m).$ which is a contradiction. Thus, $(ar_i,m)=1.$

    By \nameref{prop-equiv-gcd}, $ar_i\equiv ar_j\pmod{m}$ if and only $r_i\equiv r_j \pmod{\tfrac{m}{(a,m)}}.$ Since $(a,m)=1,$ $ar_i\not\equiv ar_j\pmod{m}$ when $i\neq j$.
=======
    Let $\{r_1,r_2,\dots,r_k\}$ be a reduced residue system modulo $m$ and $a\in\Z$ with $(a,m)=1.$ Since $\{r_1,r_2,\dots,r_k\}$ and $\{ar_1,ar_2,\dots,ar_k\}$ have the same number of elements, it suffices to show that $(ar_i,m)=1$ and  $ar_i\not\equiv ar_j\pmod{m}$ for $i\neq j.$ If there exist some prime $p$ such that $p\mid(ar_i,m)$ then $p\mid ar_i$ and $p\mid m$ by definition of \nameref{defn:gcd}. By \nameref{lem:irreducible-prime}, $p\mid a$ or $p\mid r_i$, so either $p\mid (a,m)$ or $p\mid (r_i,m).$ which is a contradiction. Thus, $(ar_i,m)=1.$

    By \nameref{prop:equiv-gcd}, $ar_i\equiv ar_j\pmod{m}$ if and only $r_i\equiv r_j \pmod{\tfrac{m}{(a,m)}}.$ Since $(a,m)=1,$ $ar_i\not\equiv ar_j\pmod{m}$ when $i\neq j$.
>>>>>>> Stashed changes
\end{proof}


\begin{defn}[Euler $\phi$-function]\label{defn:phi-fn}
    Let $n$ be a positive integer. The \emph{Euler $\phi$-function} $\phi(n)$ is \[\phi(n)=\#\{a\in\Z : a>0 \textnormal{ and } (a,m)=1\}.\]
\end{defn}


\begin{remark}
    For a positive integer $m,$ $\phi(m)$ is the number of reduced residues modulo $m$
\end{remark}
\begin{example}\label{example:phi}
    
    \begin{itemize}
        \item $\phi(7)=6$
        
        \item If $p$ is prime, $\phi(p)=p-1$
        
        \item $\phi(12)=4$
    \end{itemize}
\end{example}

\begin{theorem}[Euler's Generalization of \nameref{FlT}]\label{thm:euler-FlT}
    Let $a,m\in\Z$ with $m>0.$ If $(a,m)=1,$ then \[a^{\phi(m)}\equiv 1\pmod{m}.\]
\end{theorem}



\begin{corollary}[Fermat's Little Theorem]\label{FlT}
    Let $p$ be prime and $a\in\Z.$ If $p\nmid a,$ then \[a^{p-1}\equiv 1\pmod{p}.\]

    \begin{proof}
        Let $p$ be prime and $a\in\Z,$ then $(a,p)=1$ if and only if $p\nmid a.$ Since $\phi(p)=p-1,$ $a^{p-1}\equiv 1\pmod{p}.$
    \end{proof}
\end{corollary}

\begin{warning}
    The converse of both of these theorems is false. The easiest example is $1^k\equiv 1\pmod{m}$ for all positive integers $k, m$. Also note that $2^{341}\equiv 2\pmod{341}$. Since $(2,341)=1,$ there exists an integer $a$ such that $2a\equiv 1\pmod{341}.$ Thus \[a2^{341}\equiv (2a)2^{340}\equiv 2^{340}\equiv 2a\equiv 1\pmod{341}.\]
    However, $341=(11)(31).$
\end{warning}

\begin{proof}[Proof of \nameref{thm:euler-FlT}]
<<<<<<< Updated upstream
    Let $m$ be a positive integer and let $\{r_1,r_2,\dots,r_{\phi(m)}\}$ be a reduced residue system modulo $m.$ If $a\in\Z$ with $(a,m)=1,$ then $\{ar_1,ar_2,\dots,ar_{\phi(m)}\}$ is a reduced residue system modulo $m$ by \nameref{lem:reduced-sys}. Thus, for all $i=1,2,\dots, \phi(m),$ then $r_i\equiv a r_j\pmod{m}$ for some $j=1,2,\dots,\phi(m).$ Thus \[r_1 r_2\cdots r_{\phi(min)}\equiv ar_1 ar_2\cdots ar_{\phi(min)}\equiv a^{\phi(m)}r_1 r_2\cdots r_{\phi(m)} \pmod{m}.\]
=======
    Let $m$ be a positive integer and let $\{r_1,r_2,\dots,r_{\phi(m)}\}$ be a reduced residue system modulo $m.$ If $a\in\Z$ with $(a,m)=1,$ then $\{ar_1,ar_2,\dots,ar_{\phi(m)}\}$ is a reduced residue system modulo $m$ by \cref{lem:reduced-sys}. Thus, for all $i=1,2,\dots, \phi(m),$ then $r_i\equiv a r_j\pmod{m}$ for some $j=1,2,\dots,\phi(m).$ Thus \[r_1 r_2\cdots r_{\phi(min)}\equiv ar_1 ar_2\cdots ar_{\phi(min)}\equiv a^{\phi(m)}r_1 r_2\cdots r_{\phi(m)} \pmod{m}.\]
>>>>>>> Stashed changes

    Since $(r_i,m)=1,$ there exists $x_i\in\Z$ such that $r_i x_i\equiv 1\pmod{m}.$ Thus, 
    \begin{align*}
        r_1 x_1 r_2 x_2\cdots r_{\phi(min)} x_{\phi(m)}&\equiv a^{\phi(m)} r_1 x_1 r_2 x_2\cdots r_{\phi(min)} x_{\phi(m)} \pmod{m}\\
        1\equiv a^{\phi(m)} \pmod{m}.\qedhere
    \end{align*}
\end{proof}



\end{document}
