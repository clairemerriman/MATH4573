\documentclass{ximera}
\usepackage{amssymb, latexsym, amsmath, amsthm, graphicx, amsthm,alltt,color, listings,multicol,xr-hyper,hyperref,aliascnt,enumitem}
\usepackage{xfrac}

\usepackage{parskip}
\usepackage[,margin=0.7in]{geometry}
\setlength{\textheight}{8.5in}

\usepackage{epstopdf}

\DeclareGraphicsExtensions{.eps}
\usepackage{tikz}


\usepackage{tkz-euclide}
%\usetkzobj{all}
\tikzstyle geometryDiagrams=[rounded corners=.5pt,ultra thick,color=black]
\colorlet{penColor}{black} % Color of a curve in a plot


\usepackage{subcaption}
\usepackage{float}
\usepackage{fancyhdr}
\usepackage{pdfpages}
\newcounter{includepdfpage}
\usepackage{makecell}


\usepackage{currfile}
\usepackage{xstring}




\graphicspath{  
{./otherDocuments/}
}

\author{Claire Merriman}
\newcommand{\classday}[1]{\def\classday{#1}}

%%%%%%%%%%%%%%%%%%%%%
% Counters and autoref for unnumbered environments
% Not needed??
%%%%%%%%%%%%%%%%%%%%%
\theoremstyle{plain}


\newtheorem*{namedthm}{Theorem}
\newcounter{thm}%makes pointer correct
\providecommand{\thmname}{Theorem}

\makeatletter
\NewDocumentEnvironment{thm*}{o}
 {%
  \IfValueTF{#1}
    {\namedthm[#1]\refstepcounter{thm}\def\@currentlabel{(#1)}}%
    {\namedthm}%
 }
 {%
  \endnamedthm
 }
\makeatother


\newtheorem*{namedprop}{Proposition}
\newcounter{prop}%makes pointer correct
\providecommand{\propname}{Proposition}

\makeatletter
\NewDocumentEnvironment{prop*}{o}
 {%
  \IfValueTF{#1}
    {\namedprop[#1]\refstepcounter{prop}\def\@currentlabel{(#1)}}%
    {\namedprop}%
 }
 {%
  \endnamedprop
 }
\makeatother

\newtheorem*{namedlem}{Lemma}
\newcounter{lem}%makes pointer correct
\providecommand{\lemname}{Lemma}

\makeatletter
\NewDocumentEnvironment{lem*}{o}
 {%
  \IfValueTF{#1}
    {\namedlem[#1]\refstepcounter{lem}\def\@currentlabel{(#1)}}%
    {\namedlem}%
 }
 {%
  \endnamedlem
 }
\makeatother

\newtheorem*{namedcor}{Corollary}
\newcounter{cor}%makes pointer correct
\providecommand{\corname}{Corollary}

\makeatletter
\NewDocumentEnvironment{cor*}{o}
 {%
  \IfValueTF{#1}
    {\namedcor[#1]\refstepcounter{cor}\def\@currentlabel{(#1)}}%
    {\namedcor}%
 }
 {%
  \endnamedcor
 }
\makeatother

\theoremstyle{definition}
\newtheorem*{annotation}{Annotation}
\newtheorem*{rubric}{Rubric}

\newtheorem*{innerrem}{Remark}
\newcounter{rem}%makes pointer correct
\providecommand{\remname}{Remark}

\makeatletter
\NewDocumentEnvironment{rem}{o}
 {%
  \IfValueTF{#1}
    {\innerrem[#1]\refstepcounter{rem}\def\@currentlabel{(#1)}}%
    {\innerrem}%
 }
 {%
  \endinnerrem
 }
\makeatother

\newtheorem*{innerdefn}{Definition}%%placeholder
\newcounter{defn}%makes pointer correct
\providecommand{\defnname}{Definition}

\makeatletter
\NewDocumentEnvironment{defn}{o}
 {%
  \IfValueTF{#1}
    {\innerdefn[#1]\refstepcounter{defn}\def\@currentlabel{(#1)}}%
    {\innerdefn}%
 }
 {%
  \endinnerdefn
 }
\makeatother

\newtheorem*{scratch}{Scratch Work}


\newtheorem*{namedconj}{Conjecture}
\newcounter{conj}%makes pointer correct
\providecommand{\conjname}{Conjecture}
\makeatletter
\NewDocumentEnvironment{conj}{o}
 {%
  \IfValueTF{#1}
    {\innerconj[#1]\refstepcounter{conj}\def\@currentlabel{(#1)}}%
    {\innerconj}%
 }
 {%
  \endinnerconj
 }
\makeatother

\newtheorem*{poll}{Poll question}
\newtheorem{tps}{Think-Pair-Share}[section]


\newenvironment{obj}{
	\textbf{Learning Objectives.} By the end of class, students will be able to:
		\begin{itemize}}
		{\!.\end{itemize}
		}

\newenvironment{pre}{
	\begin{description}
	}{
	\end{description}
}


\newcounter{ex}%makes pointer correct
\providecommand{\exname}{Homework Problem}
\newenvironment{ex}[1][2in]%
{%Env start code
\problemEnvironmentStart{#1}{Homework Problem}
\refstepcounter{ex}
}
{%Env end code
\problemEnvironmentEnd
}

\newcommand{\inlineAnswer}[2][2 cm]{
    \ifhandout{\pdfOnly{\rule{#1}{0.4pt}}}
    \else{\answer{#2}}
    \fi
}


\ifhandout
\newenvironment{shortAnswer}[1][
    \vfill]
        {% Begin then result
        #1
            \begin{freeResponse}
            }
    {% Environment Ending Code
    \end{freeResponse}
    }
\else
\newenvironment{shortAnswer}[1][]
        {\begin{freeResponse}
            }
    {% Environment Ending Code
    \end{freeResponse}
    }
\fi

\let\question\relax
\let\endquestion\relax

\newtheoremstyle{ExerciseStyle}{\topsep}{\topsep}%%% space between body and thm
		{}                      %%% Thm body font
		{}                              %%% Indent amount (empty = no indent)
		{\bfseries}            %%% Thm head font
		{}                              %%% Punctuation after thm head
		{3em}                           %%% Space after thm head
		{{#1}~\thmnumber{#2}\thmnote{ \bfseries(#3)}}%%% Thm head spec
\theoremstyle{ExerciseStyle}
\newtheorem{br}{In-class Problem}

\newenvironment{sketch}
 {\begin{proof}[Sketch of Proof]}
 {\end{proof}}


\newcommand{\gt}{>}
\newcommand{\lt}{<}
\newcommand{\N}{\mathbb N}
\newcommand{\Q}{\mathbb Q}
\newcommand{\Z}{\mathbb Z}
\newcommand{\C}{\mathbb C}
\newcommand{\R}{\mathbb R}
\renewcommand{\H}{\mathbb{H}}
\newcommand{\lcm}{\operatorname{lcm}}
\newcommand{\nequiv}{\not\equiv}
\newcommand{\ord}{\operatorname{ord}}
\newcommand{\ds}{\displaystyle}
\newcommand{\floor}[1]{\left\lfloor #1\right\rfloor}
\newcommand{\legendre}[2]{\left(\frac{#1}{#2}\right)}



%%%%%%%%%%%%



\title{Wilson's Theorem}
\begin{document}
\begin{abstract}
\end{abstract}
\maketitle

%%%%%%%%%%%%%%%%%%%%%%%%%%

\begin{obj}
    \item Characterize when $a$ is its own inverse modulo a prime.

    \item Prove Wilson's Theorem and its converse
\end{obj}


\begin{pre}
    \item[Reading] Strayer, Section 2.4

    \item[Turn in]
    Does this match with your conjecture from Exercise 5? If not, what is the difference?
\end{pre}




\begin{lemma}\label{lem:sqrt1}
    Let $p$ be a prime number and $a\in\Z.$ Then $a$ is its own inverse modulo $m$ if and only if $a\equiv \pm 1\pmod{p}.$
\end{lemma}

\begin{proof}
    Let $p$ be a prime number and $a\in\Z.$ Then $a$ is its own inverse modulo $m$ if and only if $a^2\equiv 1 \pmod{p}$ if and only if $p\mid a^2-1=(a-1)(a+1).$ Since $p$ is prime, $p\mid a-1$ or $a+1$ by \nameref{lem:irreducible-prime}. Thus, $a\equiv\pm 1\pmod p.$
\end{proof}

\begin{corollary}\label{cor:sqrt1}
    Let $p$ be a prime. Then $x^2\equiv 1\pmod{p}$ if and only if $x\equiv \pm 1\pmod{p}.$
\end{corollary}

\begin{remark}
    It is important to note why we require $p$ is prime. \nameref{lem:irreducible-prime} is only true for primes: 
    \begin{itemize}
        \item $8\mid ab$ is true when $8\mid a,$ $8\mid b,$ $4\mid a$ and $2\mid b,$ or $2\mid a$ and $4\mid b.$
    \end{itemize}
    Let $a=2k+1$ for some integer $k.$ Then 
    \[a^2=4k^2+4k+1=4k(k+1)+1.\]
    Since either $k$ or $k+1$ is even, $a^2=8m+1$ for some $m\in\Z.$ Thus, $a^2\equiv 1\pmod 8$ for all odd integers $a\in\Z.$

    \begin{itemize}
        \item When $a\equiv 1\pmod 8,$ then $8\mid (a-1).$
        \item When $a\equiv 3\pmod 8,$ then $8k=a-3$ for some $k\in\Z.$ Thus $2\mid (a-1)$ and $4\mid (a+1)$.
        \item When $a\equiv 5\pmod 8,$ then $8k=a-5$ for some $k\in\Z.$ Thus $4\mid (a-1)$ and $2\mid (a+1)$.
        \item When $a\equiv 7\pmod 8,$ then $8\mid (a+1).$
    \end{itemize}
\end{remark}


\begin{theorem}[Wilson's Theorem]\label{Wilson}
    Let $p$ be a prime number. Then \[(p-1)!\equiv -1\pmod{p}.\]
\end{theorem}

\begin{proof}
    When $p=2,$ $(2-1)!=1\equiv -1 \pmod{2}.$ Now consider $p$ an odd prime. By \nameref{cor:condition-invertible}, each $a=1,2,\dots,p-1$ has a unique multiplicative inverse modulo $p.$ \nameref{lem:sqrt1} says the only elements that are their own multiplicative inverse are $1$ and $p-1$. Thus $(p-2)!$ is the product of $1$ and $\frac{p-3}{2}$ pairs of $a,a^\prime$ where $a a^\prime\equiv 1\pmod{p}.$ Therefore, 
    \begin{align*}
        (p-2)! & \equiv 1\pmod{p}\\
        (p-1)! & \equiv p-1\equiv -1 \pmod{p}.\qedhere
    \end{align*}
\end{proof}

Wilson's Theorem is normally stated as above, but the converse is also true. It can also be a (very ineffective) prime test.
\begin{proposition}[Converse of Wilson's Theorem]\label{Wilson-converse}
    Let $n$ be a positive integer. If $(n-1)!\equiv 1\pmod{n},$ then $n$ is prime.
\end{proposition}

\begin{proof}
    Let $a$ and $b$ be positive integers where $ab=n.$ It suffices to show that if $1\leq a < n,$ then $a=1.$ If $a=n,$ then $b=1.$ If  $1\leq a < n,$ then $a\mid (n-1)!$ by the definition of factorial. Then $(n-1)!\equiv -1\pmod{n}$ implies $a\mid (n-1)!+1$ by transitivity of division. Thus, $a\mid (n-1)!+1-(n-1)!=1$ by linear combination and $a=1.$ Therefore, the only positive factors of $n$ are $1$ and $n,$ so $n$ is prime.
\end{proof}


\begin{br}[Part of Strayer, Chapter 2 Exercise 47]
    Let $p$ be an odd prime. Use (a) $\left(\left(\frac{p-1}{2}\right)!\right)\equiv (-1)^{(p+1)/2} \pmod{p}$ to show
    \begin{enumerate}[label=(\alph*)]
        \setcounter{enumi}{1}
        \item If $p\equiv 1\pmod{4},$ then $\left(\left(\frac{p-1}{2}\right)!\right)^2\equiv -1 \pmod{p}$
        
        \item If $p\equiv 3\pmod{4},$ then $\left(\left(\frac{p-1}{2}\right)!\right)^2\equiv 1 \pmod{p}$
    \end{enumerate}

    \begin{solution}

        \begin{enumerate}[label=(\alph*)]
            \setcounter{enumi}{1}
            \item Let $p$ be a prime with $p\equiv 1 \pmod 4.$ Then $p=4k+1$ for some $k\in\Z.$ From part (a), 
            \[\left(\left(\frac{p-1}{2}\right)!\right)\equiv (-1)^{(p+1)/2} \equiv (-1)^{(4k+1+1)/2}\equiv (-1)^{2k+1}\equiv -1 \pmod{p}.\]

            \item Let $p$ be a prime with $p\equiv 3 \pmod 4.$ Then $p=4k+3$ for some $k\in\Z.$ From part (a), 
            \[\left(\left(\frac{p-1}{2}\right)!\right)\equiv (-1)^{(p+1)/2} \equiv (-1)^{(4k+3+1)/2}\equiv (-1)^{2k+2}\equiv 1 \pmod{p}.\]
            
        \end{enumerate}
        
    \end{solution}
\end{br}


% \begin{theorem}[On Paper 2, Polynomial Factorization option]
%     Let $p$ be a prime number. The congruence $x^2\equiv -1 \pmod p$ has solutions if and only if $p=2$ or $p\equiv 1 \pmod 4.$
% \end{theorem}

%%%%%%%%%%%%%%%%%%%%%%%%%%


\end{document}
