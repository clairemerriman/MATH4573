\documentclass{ximera}
\usepackage{amssymb, latexsym, amsmath, amsthm, graphicx, amsthm,alltt,color, listings,multicol,hyperref,xr-hyper,aliascnt,enumitem}
\usepackage{xfrac}


\usepackage{parskip}
\usepackage{graphicx}
\usepackage[,margin=0.7in]{geometry}
\setlength{\textheight}{8.5in}
 
\usepackage{tkz-euclide}
%\usetkzobj{all}
\tikzstyle geometryDiagrams=[rounded corners=.5pt,ultra thick,color=black]
\colorlet{penColor}{black} % Color of a curve in a plot


\usepackage{subcaption}
\usepackage{float}
\usepackage{fancyhdr}
\usepackage{pdfpages}
\newcounter{includepdfpage}


\newcommand{\semester}{%
  \ifcase\month
  \or Spring %1
  \or Spring %2
  \or Spring %3
  \or Spring %4
  \or Spring  %5
  \or Fall %8
  \or Fall %9
  \or Fall %10
  \or Fall %11
  \or Fall %12
  \fi
}
\usepackage{currfile}
\usepackage{xstring}


\lhead{\large{Number Theory: MAT-255}}
%Put your Document Title (Camp: Topic) Here
\chead{}
\rhead{\semester 24}
\lfoot{}
\cfoot{}
\rfoot{Page \thepage}
\renewcommand\headrulewidth{0pt}
\renewcommand\footrulewidth{0pt}

\headheight 50pt
\headsep 30pt

\author{Claire Merriman}
\date{Spring 2024}


%%%%%%%%%%%%%%%%%%%%%
% Counters and autoref for unnumbered environments
%%%%%%%%%%%%%%%%%%%%%
\theoremstyle{plain}


\newtheorem*{namedthm}{Theorem}
\newcounter{thm}%makes pointer correct
\providecommand{\thmname}{Proposition}

\makeatletter
\NewDocumentEnvironment{thm*}{o}
 {%
  \IfValueTF{#1}
    {\namedthm[#1]\refstepcounter{thm}\def\@currentlabel{(#1)}}%
    {\namedthm}%
 }
 {%
  \endnamedthm
 }
\makeatother


\newtheorem*{namedprop}{Proposition}
\newcounter{prop}%makes pointer correct
\providecommand{\propname}{Proposition}

\makeatletter
\NewDocumentEnvironment{prop*}{o}
 {%
  \IfValueTF{#1}
    {\namedprop[#1]\refstepcounter{prop}\def\@currentlabel{(#1)}}%
    {\namedprop}%
 }
 {%
  \endnamedprop
 }
\makeatother

\newtheorem*{namedlem}{Lemma}
\newcounter{lem}%makes pointer correct
\providecommand{\lemname}{Lemma}

\makeatletter
\NewDocumentEnvironment{lem*}{o}
 {%
  \IfValueTF{#1}
    {\namedlem[#1]\refstepcounter{lem}\def\@currentlabel{(#1)}}%
    {\namedlem}%
 }
 {%
  \endnamedlem
 }
\makeatother

\newtheorem*{namedcor}{Corollary}
\newcounter{cor}%makes pointer correct
\providecommand{\corname}{Corollary}

\makeatletter
\NewDocumentEnvironment{cor*}{o}
 {%
  \IfValueTF{#1}
    {\namedcor[#1]\refstepcounter{cor}\def\@currentlabel{(#1)}}%
    {\namedcor}%
 }
 {%
  \endnamedcor
 }
\makeatother

\theoremstyle{definition}

\newtheorem*{innerrem}{Remark}
\newcounter{rem}%makes pointer correct
\providecommand{\remname}{Remark}

\makeatletter
\NewDocumentEnvironment{rem}{o}
 {%
  \IfValueTF{#1}
    {\innerrem[#1]\refstepcounter{rem}\def\@currentlabel{(#1)}}%
    {\innerrem}%
 }
 {%
  \endinnerrem
 }
\makeatother

\newtheorem*{innerdefn}{Definition}%%placeholder
\newcounter{defn}%makes pointer correct
\providecommand{\defnname}{Definition}

\makeatletter
\NewDocumentEnvironment{defn}{o}
 {%
  \IfValueTF{#1}
    {\innerdefn[#1]\refstepcounter{defn}\def\@currentlabel{(#1)}}%
    {\innerdefn}%
 }
 {%
  \endinnerdefn
 }
\makeatother

\newtheorem*{scratch}{Scratch Work}


\newtheorem*{namedconj}{Conjecture}
\newcounter{conj}%makes pointer correct
\providecommand{\conjname}{Conjecture}
\makeatletter
\NewDocumentEnvironment{conj}{o}
 {%
  \IfValueTF{#1}
    {\innerconj[#1]\refstepcounter{conj}\def\@currentlabel{(#1)}}%
    {\innerconj}%
 }
 {%
  \endinnerconj
 }
\makeatother

\newtheorem*{poll}{Poll question}
\newtheorem{tps}{Think-Pair-Share}[section]
%\newtheorem{br}{In-class Problem}[section]
\newtheorem*{cs}{Crowd Sourced Proof}

\newlist{checklist}{itemize}{2}
\setlist[checklist]{label=$\square$}

\newenvironment{obj}{
	\textbf{Learning Objectives.} By the end of class, students will be able to:
		\begin{itemize}}
		{\!.\end{itemize}
		}

\newenvironment{pre}{
	\begin{description}
	}{
	\end{description}
}


\newcounter{br}%makes pointer correct
\providecommand{\brname}{In-class Problem}

\newenvironment{br}[1][2in]%
{%Env start code
\problemEnvironmentStart{#1}{In-class Problem}
\refstepcounter{br}
}
{%Env end code
\problemEnvironmentEnd
}

\newcounter{ex}%makes pointer correct
\providecommand{\exname}{Homework Problem}
\newenvironment{ex}[1][2in]%
{%Env start code
\problemEnvironmentStart{#1}{Homework Problem}
\refstepcounter{ex}
}
{%Env end code
\problemEnvironmentEnd
}



\newenvironment{sketch}
 {\begin{proof}[Sketch of Proof]}
 {\end{proof}}
%\newenvironment{hint}
%  {\begin{proof}[Hint]}
%  {\end{proof}}

\newcommand{\gt}{>}
\newcommand{\lt}{<}
\newcommand{\N}{\mathbb N}
\newcommand{\Q}{\mathbb Q}
\newcommand{\Z}{\mathbb Z}
\newcommand{\C}{\mathbb C}
\newcommand{\R}{\mathbb R}
\renewcommand{\H}{\mathbb{H}}
\newcommand{\lcm}{\operatorname{lcm}}
\newcommand{\nequiv}{\not\equiv}
\newcommand{\ord}{\operatorname{ord}}
\newcommand{\ds}{\displaystyle}
\newcommand{\floor}[1]{\left\lfloor #1\right\rfloor}
\newcommand{\legendre}[2]{\left(\frac{#1}{#2}\right)}



%%%%%%%%%%%%



%Imports for cross references
\externaldocument{otherResults}
\externaldocument{Week1Spring24}

\StrBetween*[1,1]{\currfilename}{Week}{Sp}[\week]

\title{Week \week--MAT-255 Number Theory}

\begin{document}
%\maketitle
%\tableofcontents
%%%%%%%%%%%%%%%%%%%%%%%%%%
%%%%%%%%%%%%%%%%%%%%%%%%%%
\section{Monday, January 22: Division algorithm and quantifiers}
%%%%%%%%%%%%%%%%%%%%%%%%%%
%%%%%%%%%%%%%%%%%%%%%%%%%%

\begin{obj}
\item  Understand universal and existential quantifiers
\item  Negate statements using quantifiers
\item Negate conditional statements using quantifiers
\item Prove existence and uniqueness for the Division Algorithm
\end{obj}

\begin{pre}
 \item[Read] Ernst \href{https://danaernst.com/IBL-IntroToProof/pretext/sec_Intro_to_Logic.html}{Section 2.2} and \href{https://danaernst.com/IBL-IntroToProof/pretext/sec_Introduction_to_Quantification.html}{Section 2.4}
 
 \item[Turn in] 
 
\begin{itemize}
 
\item Ernst, Problem 2.59.
Both of the following sentences are propositions. Decide whether each is true or false. What would it take to justify your answers?
\begin{enumerate}%[label=\alph*.]
 \item For all $x\in\R,$ $x^2-4=0$.
 
\begin{solution}
False--find a counterexample.
\end{solution}
 \item There exists $x\in\R$ such that $x^2-4=0$.
\begin{solution}
True--find a solution $x$.
\end{solution}
\end{enumerate}

\item  Ernst Problem 2.64.
 Suppose the universe of discourse is the set of real numbers and consider the predicate $F(x,y):=``x=y^2"$.
Interpret the meaning of each of the following statements.
\begin{enumerate}%[label=\alph*.]
 \item There exists $x$ such that there exists $y$ such that $F(x,y)$.
\begin{solution}
There exists $x$ such that for some $y,$ $x=y^2.$ 
\end{solution}

 \item There exists $y$ such that there exists $x$ such that $F(x,y)$.
  
\begin{solution}
There exists $y\in\R$ such that for some $x\in\R,$ $x=y^2.$ 
\end{solution}

\item For all $y$, for all $x$, $F(x,y)$.
 
\begin{solution}
For all real numbers $x$ and $y,$ $x=y^2.$ 
\end{solution}
 \end{enumerate}

\end{itemize}
\end{pre}
Go over reading assignment at the start of class.

%%%%%%%%%%%%%%%%%%%%%%%%%%
\subsection{Division Algorithm (45 minutes)}
%%%%%%%%%%%%%%%%%%%%%%%%%%
Section 1.1 introduces the division algorithm, which will come up repeatedly throughout the semester, as well as the definition of divisors from last class.

\begin{thm*}[Division Algorithm]\label{div-alg}( Theorem 1.4)
 Let $a,b\in\Z$ with $b>0$. Then there exists a unique $q,r\in\Z$ such that \[a=bq+r, \quad 0\leq r <b.\]
\end{thm*}

Before proving this theorem, let's think about division with remainders, ie long division. The quotient $q$ should be the largest integer such that $bq\leq a$. If we divide both sides by $b$, we have $q\leq\frac{a}{b}$. We have a function to find the greatest integer less than or equal to $\frac{a}{b},$ namely $q=\left\lfloor\frac{a}{b}\right\rfloor$. If we rearrange the equation $a=bq+r,$ we gave $r=a-bq$. This is our scratch work for existence.

\begin{proof} Let $a,b\in\Z$ with $b>0$.
Define $q=\left\lfloor\frac{a}{b}\right\rfloor$ and $r=a-b\left\lfloor\frac{a}{b}\right\rfloor$. Then $a=bq+r$ by rearranging the equation. 
Now we need to show $0\leq r<b$. 

Since $x-1<\lfloor x\rfloor\leq x$ by \nameref{lem:floor-inter}, we have \[\frac{a}{b}-1<\left\lfloor\frac{a}{b}\right\rfloor\leq\frac{a}{b}.\] 
Multiplying all terms by $-b$, we get 
 \[-a+b>-b\left\lfloor\frac{a}{b}\right\rfloor\geq-a.\]
 Adding $a$ to every term gives \[b>a-b\left\lfloor\frac{a}{b}\right\rfloor\geq 0.\] 
By the definition of $r$, we have shown $0\leq r <b$.

Finally, we need to show that $q$ and $r$ are unique.
Assume there exist $q_1,q_2,r_1,r_2\in\Z$ with \[a=bq_1+r_1, \quad 0\leq r_1<b\]
 \[a=bq_2+r_2, \quad 0\leq r_2<b.\]
 We need to show $q_1=q_2$ and $r_1=r_2$. We can subtract the two equations from each other. 
 
\begin{align*}
  a&=bq_1+r_1, \\
\underline{ -(a}&\underline{=bq_2+r_2)}, \\
 0&=bq_1+r_1-bq_2-r_2=b(q_1-q_2)+(r_1-r_2) . 
\end{align*}

Rearranging, we get $b(q_1-q_2)=r_2-r_1$. Thus, $b\mid r_2-r_1$. From rearranging the inequalities:
\begin{align*}
 & 0\leq r_2<b\\
- & \underline{b< -r_1\leq 0}\\
 -&b<r_2-r_1<b.
\end{align*}
Thus, the only way $b\mid r_2-r_1$ is that $r_2-r_1=0$ and thus $r_1=r_2$. Now, $0=b(q_1-q_2)+(r_1-r_2)$ becomes $0=b(q_1-q_2)$. Since we assumed $b>0$, we have that $q_1-q_2=0$. 
\end{proof}



\begin{br}
 Use the \nameref{div-alg} on $a=47, b=6$ and $a=281, b=13$.
\end{br}
\begin{solution}
For $a=47, b=6$, we have that $a=(7)6+5, q=7, r=5$.
For $a=281, b=13$, we have that $a=(21)13+8, q=21, r=8$.
\end{solution}

\begin{corollary}
 Let $a,b\in\Z$ with $b\neq0$. Then there exists a unique $q,r\in\Z$ such that \[a=bq+r, \quad 0\leq r <|b|.\]
\end{corollary}
One proof method is using an existing proof as a guide.

\begin{br} [Exercise Set 1.1, Exercise 15a] Let $a$ and $b$ be nonzero integers. Prove that there exists a unique $q,r\in\Z$ such that 
  \[a=bq+r, \quad 0\leq r <|b|.\]
(Outline updated from class)
  \begin{enumerate}
    	\item Use the \nameref{div-alg} to prove this statement as a corollary. That is, use the \emph{conclusion} of the \nameref{div-alg} as part of the proof.  Use the following outline:
    	\begin{enumerate}
		\item  Let $a$ and $b$ be nonzero integers. Since $|b|>0$, the \nameref{div-alg} says that there exist unique $p,s\in\Z$ such that $\answer{a=p|b|+s}$ and $\answer{0\leq s<|b|}$.
      		\item There are two cases:
      		\begin{enumerate}
        			\item When $\answer{b>0}$, the conditions are already met and $\answer{r=s$ and $q=p}$.
        			\item Otherwise, $\answer{b<0}$, $r=\answer{s}$ and $q=\answer{-b}$.
      		\end{enumerate}
      		\item Since both cases used that the $p,s$ are unique, then $q,r$ are also unique
	\end{enumerate}
    	\item Use the \emph{proof} of the \nameref{div-alg} as a template to prove this statement. That is, repeat the steps, adjusting as necessary, but do not use the conclusion.
    	\begin{enumerate}
    		\item In the proof of the \nameref{div-alg}, we let $q=\lfloor\frac{a}{b}\rfloor$. Here we have two cases:
    		\begin{enumerate}
      			\item When $\answer{b>0}$, $q=\answer{\lfloor\frac{a}{b}\rfloor}$ and $r=\answer{a-bq}.$
      			\item When $\answer{b<0}$, $q=\answer{-\lfloor\frac{a}{b}\rfloor}$ and $r=\answer{a-bq}.$   
		\end{enumerate}
    		\item Follow the steps of the \emph{proof} of the \nameref{div-alg} to finish the proof.
    	\end{enumerate}

\end{enumerate}

\begin{solution}
 Problem on Homework 2. You only need to provide one proof on Homework 2.
\end{solution}
\end{br}
%%%%%%%%%%%%%%%%%%%%%%%%%%
%%%%%%%%%%%%%%%%%%%%%%%%%%
\section{Wednesday, January 24: Primes}
%%%%%%%%%%%%%%%%%%%%%%%%%%
%%%%%%%%%%%%%%%%%%%%%%%%%%

\begin{obj}
\item  Every integer greater than 1 has a prime divisor.
\item  Prove that there are infinitely many prime numbers
\end{obj}

\begin{pre}
 \item[Read] Strayer, Section 1.2
 \item[Turn in] 
\begin{itemize}
 \item The proof method for Euclid's infinitude of primes is an important method. Summarize this method in your own words.
 
\begin{solution}
 Summaries will vary
\end{solution}
 \item Identify any other new proof methods in this section
 
\begin{solution}
 Proof by construction may be new to some students. Students also identified: 
\begin{itemize}
 \item Introducing a variable to aid in proof
 \item Without loss of generality
 
\end{itemize}
\end{solution}
 \item Exercise 22. Prove that 2 is the only even prime number.
 
\begin{solution}
 Assume that there exists another even prime number, call it $p$. Then there exists $2\mid p$ by the definition of even, but that implies that $p=2$ by the definition of prime. Thus, $2$ is the only even prime number.
\end{solution}
\end{itemize}
\end{pre}
%%%%%%%%%%%%%%%%%%%%%%%%%


%%%%%%%%%%%%%%%%%%%%%%%%%
\subsection{Primes (50 minutes)}
%%%%%%%%%%%%%%%%%%%%%%%%%%
\begin{defn}[prime and composite]
An integer $p>1$ is \emph{prime} if the only positive divisors of $p$ are $1$ and itself. An integer $n$ which is not prime is \emph{composite}. 
\end{defn}

Why is $1$ not prime?

%It has to do with units and invertibility. The number $1$ holds a special place. It is the multiplicative identity, i.e., anything multiplied by $1$ is just that thing again. Something is said to be invertible in a ``group" (more on that later) if there exist something, which, when multiplied to it, gives you $1$. How many invertible elements are there among the integers? Just two. $1$ and $-1$. And that's the key. If we want to extend our results from positive integers to non-zero integers, we often just need to take into account $\pm 1$. That sounds obvious, but it turns out to be surprisingly critical and yet non-intuitive when we start moving from real integers to complex ones.

\begin{lem*}[Lemma 1.5]\label{lem:prime-composite}
 Every integer greater than 1 has a prime divisor.
\end{lem*}

We will not go over this proof in class.

\begin{proof} 
 Assume by contradiction that there exists $n\in\Z$ greater than 1 with no prime divisor. By the \nameref{well-order}, we may assume $n$ is the least such integer. By definition, $n\mid n$, so $n$ is not prime. Thus, $n$ is composite and there exists $a,b\in\Z$ such that $n=ab$ and $1<a<n$, $1<b<n$. Since $a<n$, then it has a prime divisor $p$. But since $p\mid a$ and $p\mid n$, $p\mid n$. This contradicts our assumption, so no such integer exists.
\end{proof}


\begin{thm*}[Euclid's Infinitude of Primes]\label{thm:inf-primes}(Theorem 1.6)
 There are infinitely many prime numbers.
\end{thm*}
\begin{proof}
 Assume by way of contradiction, that there are only finitely many prime numbers, so $p_1,p_2,\dots,p_n$. Consider the number $N=p_1p_2\cdots p_n +1$. Now $N$ has a prime divisor, say, $p$, by \nameref{lem:prime-composite}. So $p=p_i$ for some $i$, $i=1,2,\dots,n$. Then $p\mid N-p_1p_2\dots p_n$, which implies that $p\mid 1$, a contradiction. Hence, there are infinitely many prime numbers.
\end{proof}

Another important fact is there are arbitrarily large sequences of composite numbers. Put another way, there are arbitrarily large gaps in the primes. Another important proof method, which is a \emph{constructive proof}:

\begin{prop*}[Proposition 1.8]\label{prop:gaps-primes}
 For any positive integer $n$, there are at least $n$ consecutive positive integers.
\end{prop*}
\begin{proof}
 Given the positive integer $n$, consider the $n$ consecutive positive integers \[(n+1)!+2, (n+1)!+3,\dots, (n+1)!+n+1.\]
 Let $i$ be a positive integer such that $2\leq i\leq n+1$. Since $i\mid (n+1)!$ and $i\mid i$, we have \[i\mid(n+1)! +i,\quad 2\leq i\leq n+1\] by linear combination (\nameref{lem:linear-combo}). So each of the $n$ consecutive positive integers is composite.
\end{proof}

  \begin{br} Let $n$ be a positive integer with $n\neq 1$. Prove that if $n^2+1$ is prime, then $n^2+1$ can be written in the form $4k+1$ with $k\in\Z$.
  \begin{solution}
   Assume that $n$ is a positive integer, $n\neq 1,$ and $n^2+1$ is prime. If $n$ is odd, then $n^2$ is odd, which would imply $n^2+1=2,$ the only even prime. However, $n\neq 1$ by assumption. Thus, $n$ is even. 

    By definition of even, there exists $j\in\Z$ such that $n=2k$ and $n^2=4j^2$. Thus, $n^2+1=4k+1$ when $k=j^2.$
  \end{solution}
\end{br}

  \begin{br}
  Prove or disprove the following conjecture, which is similar to Conjecture 1:\\
      \textbf{Conjecture:} There are infinitely many prime number $p$ for which $p+2$ and $p+4$ are also prime numbers.

  
  \begin{solution}
    On Homework 2.
  \end{solution}
    \end{br}



%%%%%%%%%%%%%%%%%%%%%%%%%%
\section{Friday, January 26: Quiz 1, Induction, Greatest Common Divisors}
%%%%%%%%%%%%%%%%%%%%%%%%%%
%%%%%%%%%%%%%%%%%%%%%%%%%%

\begin{obj}
\item  Understand induction
\item Prove basic facts about the greatest common divisor
\end{obj}

\begin{pre}
 \item[Read] Strayer Appendix A.1: The First Principle of Mathematical Induction  or Ernst \href{https://danaernst.com/IBL-IntroToProof/pretext/sec_Intro_to_Induction.html}{Section 4.1} and \href{https://danaernst.com/IBL-IntroToProof/pretext/sec_More_on_Induction.html}{Section 4.2}
 
 \item[Turn in] Strayer Exercise Set A, Exercise 1a. If $n$ is a positive integer, then 
 \[1^2+2^2+3^2+\cdots+n^2=\frac{n(n+1)(2n+1)}{6}.\]
 
\begin{proof}
 We proceed by induction. The base case is $n=1$. Since $1^2=\frac{1(1+1)(2*1+1)}{6},$ we are done.
 
 Now assume that if $k \geq 1$ and for $n = k,$  \[1^2+2^2+3^2+\cdots+k^2=\frac{k(k+1)(2k+1)}{6}.\]
 Adding $(k+1)^2$ to both sides gives, 
 \begin{align*}
 1^2+2^2+3^2+\cdots+k^2+(k+1)^2&=\frac{k(k+1)(2k+1)}{6}+(k+1)^2\\
 &=\frac{k(k+1)(2k+1)+6(k+1)^2}{6}\\
 &=\frac{(k+1)[k(2k+1)+6(k+1)]}{6}\\
 &=\frac{(k+1)[2k^2+k+6k+6]}{6}
 \\
 &=\frac{(k+1)(k+2)(2k+3)}{6}.\end{align*}
 So the desired statement is true for $n = k + 1$. By the first principle of mathematical induction, the desired statement is true for all positive integers, and the proof is complete.
\end{proof}
\end{pre}


%%%%%%%%%%%%%%%%%%%%%%%%%
\subsection{Quiz (10 minutes)}
%%%%%%%%%%%%%%%%%%%%%%%%%%
%%%%%%%%%%%%%%%%%%%%%%%%%%%
\subsection{Greatest common divisor (20 min)} 
%%%%%%%%%%%%%%%%%%%%%%%%%%%

\begin{defn}[greatest common divisor]\label{defn:gcd} 
  If $a\mid b$ and $a\mid c$ then $a$ is a \emph{common divisor} of $b$ and $c$.

  If at least one of $b$ and $c$ is not $0$, the greatest (positive) number among their common divisors  is called the \emph{greatest common divisor of $a$ and $b$} and is denoted $gcd(a,b)$ or just $(a,b)$. 
 
  If $gcd(a,b)=1$, we say that $a$ and $b$ are \emph{relatively prime}.

  If we want the greatest common divisor of several integers at once we denote that by $\gcd(b_1,b_2,b_3,\dots,b_n)$.
\end{defn}

For example, $\gcd(4,8)$ is $4$ but $\gcd(4,6,8)$ is $2$.

The GCD always exists when at least one of the integers is nonzero. How to show this: $1$ is always a divisor, and no divisor can be larger than the maximum of $|a|,|b|$. So there is a finite number of divisors, thus there is a maximum.


\begin{prop*}[Proposition 1.11]\label{Bezout}
 Let $a, b\in \Z$ with $a$ and $b$ not both zero. Then 
 \[\{(a,b)=\min\{ma+nb:m,n\in\Z, ma+nb>0\}.\]
\end{prop*}
This proof brings together definitions (of gcd), previous results (\nameref{div-alg}, factors of linear combinations), the well-ordering principle, and some methods for minimum and maximum/greatest.
\begin{proof}
 Since $a,b\in\Z$ are not both zero, at least one of $1a+0b, -1a+0b, 0a+1b, 0a+(-1)b$ is in $\{ma+nb:m,n\in\Z, ma+nb>0\}$. Therefore, the set is nonempty and has a minimal element by the \nameref{well-order}. Call this element $d$, and $d=xa+yb$ for some $x,y\in \Z$.
 
First we will show that $d\mid a$. By the \nameref{div-alg}, there exist unique $q,r\in\Z$ such that $a=qd+r$ with $0\leq r<d$. Then, \[r=a-qd=a-q(xa+yb)=(1-qx)a-qyb,\] so $r$ is an integral linear combination of $a$ and $b$.  Since $d$ is the least positive such integer, $r=0$ and $d\mid a$. Similarly, $d\mid b$. 

It remains to show that $d$ is the \emph{greatest} common divisor of $a$ and$b$. Let $c$be any common divisor of $a$  and $b$. Then $c\mid ax+by=d$, so $c\mid d$.
\end{proof}

Since we assume $a$ and $b$ are not both zero, we could also simplify the first sentence using \emph{without loss of generality}. Since there is no difference between $a$ and $b$, we can assume $a\neq0$.

%%%%%%%%%%%%%%%%%%%%%%%%%
\subsection{More induction (15 minutes)}
%%%%%%%%%%%%%%%%%%%%%%%%%%


\begin{br}
Theorems in Ernst \href{https://danaernst.com/IBL-IntroToProof/pretext/sec_Intro_to_Induction.html}{Section 4.1} 
 

\begin{thm*}[Ernst Theorem 4.5]
For all $n\in\mathbb{N}$, 3 divides $4^{n}-1$.
\end{thm*}
\begin{solution}
We proceed by induction.  When $n=1,$ $3\mid 4^n-1=3$. Thus, the statement is true for $n=1.$

Now assume $k\geq 1$ and the desired statement is true for $n=k$. Then the induction hypothesis is \[3\mid 4^k-1.\]
By the definition of \nameref{defn:divides}, there exists $m\in\Z$ such that $3m=4^k-1.$ In other words, $3m+1=4^k$. Multiplying both sides by $4$ gives $12m+4=4^{k+1}$. Rewriting this equation gives $3(4m+1)=4^{k+1}-1$. Thus, $3\mid 4^{k+1}-1$, and the desired statement is true for $n=k+1$. By the (first) principle of mathematical induction, the statement is true for all positive integers, and the proof is complete.
\end{solution}

 \begin{thm*}[Ernst Theorem 4.7]
 Let $p_{1}, p_{2}, \ldots, p_{n}$ be $n$ distinct points arranged on a circle.  Then the number of line segments joining all pairs of points is $\frac{n^{2}-n}{2}$.
 \end{thm*}
\begin{solution}
 We proceed by induction. When $n=1$, there is only one point, so there are no lines connecting pairs of points. Additionally, $\frac{1^2-1}{2}=0$.\footnote{Alternately, you could use $n=2$ for the base case. Then there is one line connecting the only pair of points and $\frac{2^2-2}{2}=1$}
 
 Now assume $k\geq 1$ and the desired statement is true for $n=k$. Then the induction hypothesis is for $k$ distinct points arranged in a circle, the number of line segments joining all pairs of points is $\frac{k^{2}-k}{2}$. Adding a $(k+1)^{st}$ point on the circle will add an additional $k$ line segments joining pairs of points, one for each existing point. Note that 
 \[ 
 	\frac{k^{2}-k}{2}+k=\frac{k^{2}+k}{2}=
	\frac{k^2+k+k+1-(k+1)}{2}=\frac{(k+1)^{2}-(k+1)}{2}\qedhere
 \]
\end{solution}
\end{br}

\begin{br}
[Strayer Exercise 1]. Use the first principle of mathematical induction to prove each statement.
\begin{enumerate}%[label=\emph{(\alph*)}]
\addtocounter{enumi}{1}
  \item If $n$ is a positive integer, then 
  \[1^3+2^3+3^3+\cdots+n^3=\frac{n^2(n+1)^2}{4}.\]
  
\begin{solution}
 	We proceed by induction.  When $n=1,$ $1^3=\frac{1^2(1+1)^2}{4}$. Thus, the statement is true for $n=1.$
                        
            
                        Now assume $k\geq 1$ and the desired statement is true for $n=k$. Then the induction hypothesis is 
                            \[1^3+2^3+3^3+\cdots+k^3=\frac{k^2(k+1)^2}{4}\]
                        Adding $(k+1)^3$ to both sides gives
                            \begin{align*}
                                 1^3+2^3+3^3+\cdots+k^3+(k+1)^3 &=\frac{k^2(k+1)^2+4(k+1)^3}{4}\\
                                 &= \frac{(k+1)^2(k^2+4(k+1))}{4}\\
                                 &= \frac{(k+1)^2(k+2)^2}{4}\\
                            \end{align*}
                        
                        Thus, the desired statement is true for $n=k+1$. By the (first) principle of mathematical induction, the statement is true for all positive integers, and the proof is complete.
\end{solution}
 \item If $n$ is an integer with $n\geq 5,$ then \[2^n>n^2.\]
 
\begin{solution}
                            We proceed by induction with base case $n=5$. When $n=5, 32=2^5\gt 5^2=25$. Thus, the statement is true for $n=1.$
                        
            
                        Now assume $k\geq 1$ and \[2^k\gt k^2\] is true for $n=k$. Multiplying both sides of the inequality by $2$ gives $2^{k+1} \gt 2k^2$. Notice that $2k^2 \gt k^2 +2k+1$ when $(k-1)^2\gt 0$, which is true for all $k\geq 5$. Thus 
                            \[ 2^{k+1} \gt 2k^2 \gt (k+1)^2.\]
                        Thus, the desired statement is true for $n=k+1$. By the (first) principle of mathematical induction, the statement is true for all positive integers, and the proof is complete.
\end{solution}
\end{enumerate}
\end{br}







%\begin{br} Let $a_1,a_2,\dots,a_n\in\Z$ with $a_1\neq 0$. Prove that \[\gcd(a_1,\dots,a_n)=\gcd(\gcd(a_1,a_2,a_3,\dots,a_{n-1}),a_n).\]
%\end{br}

\end{document}

%%%%%%%%%%%%%%%%%%%%%%%%%%%
