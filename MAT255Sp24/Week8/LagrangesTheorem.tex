\documentclass{ximera}
\usepackage{amssymb, latexsym, amsmath, amsthm, graphicx, amsthm,alltt,color, listings,multicol,xr-hyper,hyperref,aliascnt,enumitem}
\usepackage{xfrac}

\usepackage{parskip}
\usepackage[,margin=0.7in]{geometry}
\setlength{\textheight}{8.5in}

\usepackage{epstopdf}

\DeclareGraphicsExtensions{.eps}
\usepackage{tikz}


\usepackage{tkz-euclide}
%\usetkzobj{all}
\tikzstyle geometryDiagrams=[rounded corners=.5pt,ultra thick,color=black]
\colorlet{penColor}{black} % Color of a curve in a plot


\usepackage{subcaption}
\usepackage{float}
\usepackage{fancyhdr}
\usepackage{pdfpages}
\newcounter{includepdfpage}
\usepackage{makecell}


\usepackage{currfile}
\usepackage{xstring}




\graphicspath{  
{./otherDocuments/}
}

\author{Claire Merriman}
\newcommand{\classday}[1]{\def\classday{#1}}

%%%%%%%%%%%%%%%%%%%%%
% Counters and autoref for unnumbered environments
% Not needed??
%%%%%%%%%%%%%%%%%%%%%
\theoremstyle{plain}


\newtheorem*{namedthm}{Theorem}
\newcounter{thm}%makes pointer correct
\providecommand{\thmname}{Theorem}

\makeatletter
\NewDocumentEnvironment{thm*}{o}
 {%
  \IfValueTF{#1}
    {\namedthm[#1]\refstepcounter{thm}\def\@currentlabel{(#1)}}%
    {\namedthm}%
 }
 {%
  \endnamedthm
 }
\makeatother


\newtheorem*{namedprop}{Proposition}
\newcounter{prop}%makes pointer correct
\providecommand{\propname}{Proposition}

\makeatletter
\NewDocumentEnvironment{prop*}{o}
 {%
  \IfValueTF{#1}
    {\namedprop[#1]\refstepcounter{prop}\def\@currentlabel{(#1)}}%
    {\namedprop}%
 }
 {%
  \endnamedprop
 }
\makeatother

\newtheorem*{namedlem}{Lemma}
\newcounter{lem}%makes pointer correct
\providecommand{\lemname}{Lemma}

\makeatletter
\NewDocumentEnvironment{lem*}{o}
 {%
  \IfValueTF{#1}
    {\namedlem[#1]\refstepcounter{lem}\def\@currentlabel{(#1)}}%
    {\namedlem}%
 }
 {%
  \endnamedlem
 }
\makeatother

\newtheorem*{namedcor}{Corollary}
\newcounter{cor}%makes pointer correct
\providecommand{\corname}{Corollary}

\makeatletter
\NewDocumentEnvironment{cor*}{o}
 {%
  \IfValueTF{#1}
    {\namedcor[#1]\refstepcounter{cor}\def\@currentlabel{(#1)}}%
    {\namedcor}%
 }
 {%
  \endnamedcor
 }
\makeatother

\theoremstyle{definition}
\newtheorem*{annotation}{Annotation}
\newtheorem*{rubric}{Rubric}

\newtheorem*{innerrem}{Remark}
\newcounter{rem}%makes pointer correct
\providecommand{\remname}{Remark}

\makeatletter
\NewDocumentEnvironment{rem}{o}
 {%
  \IfValueTF{#1}
    {\innerrem[#1]\refstepcounter{rem}\def\@currentlabel{(#1)}}%
    {\innerrem}%
 }
 {%
  \endinnerrem
 }
\makeatother

\newtheorem*{innerdefn}{Definition}%%placeholder
\newcounter{defn}%makes pointer correct
\providecommand{\defnname}{Definition}

\makeatletter
\NewDocumentEnvironment{defn}{o}
 {%
  \IfValueTF{#1}
    {\innerdefn[#1]\refstepcounter{defn}\def\@currentlabel{(#1)}}%
    {\innerdefn}%
 }
 {%
  \endinnerdefn
 }
\makeatother

\newtheorem*{scratch}{Scratch Work}


\newtheorem*{namedconj}{Conjecture}
\newcounter{conj}%makes pointer correct
\providecommand{\conjname}{Conjecture}
\makeatletter
\NewDocumentEnvironment{conj}{o}
 {%
  \IfValueTF{#1}
    {\innerconj[#1]\refstepcounter{conj}\def\@currentlabel{(#1)}}%
    {\innerconj}%
 }
 {%
  \endinnerconj
 }
\makeatother

\newtheorem*{poll}{Poll question}
\newtheorem{tps}{Think-Pair-Share}[section]


\newenvironment{obj}{
	\textbf{Learning Objectives.} By the end of class, students will be able to:
		\begin{itemize}}
		{\!.\end{itemize}
		}

\newenvironment{pre}{
	\begin{description}
	}{
	\end{description}
}


\newcounter{ex}%makes pointer correct
\providecommand{\exname}{Homework Problem}
\newenvironment{ex}[1][2in]%
{%Env start code
\problemEnvironmentStart{#1}{Homework Problem}
\refstepcounter{ex}
}
{%Env end code
\problemEnvironmentEnd
}

\newcommand{\inlineAnswer}[2][2 cm]{
    \ifhandout{\pdfOnly{\rule{#1}{0.4pt}}}
    \else{\answer{#2}}
    \fi
}


\ifhandout
\newenvironment{shortAnswer}[1][
    \vfill]
        {% Begin then result
        #1
            \begin{freeResponse}
            }
    {% Environment Ending Code
    \end{freeResponse}
    }
\else
\newenvironment{shortAnswer}[1][]
        {\begin{freeResponse}
            }
    {% Environment Ending Code
    \end{freeResponse}
    }
\fi

\let\question\relax
\let\endquestion\relax

\newtheoremstyle{ExerciseStyle}{\topsep}{\topsep}%%% space between body and thm
		{}                      %%% Thm body font
		{}                              %%% Indent amount (empty = no indent)
		{\bfseries}            %%% Thm head font
		{}                              %%% Punctuation after thm head
		{3em}                           %%% Space after thm head
		{{#1}~\thmnumber{#2}\thmnote{ \bfseries(#3)}}%%% Thm head spec
\theoremstyle{ExerciseStyle}
\newtheorem{br}{In-class Problem}

\newenvironment{sketch}
 {\begin{proof}[Sketch of Proof]}
 {\end{proof}}


\newcommand{\gt}{>}
\newcommand{\lt}{<}
\newcommand{\N}{\mathbb N}
\newcommand{\Q}{\mathbb Q}
\newcommand{\Z}{\mathbb Z}
\newcommand{\C}{\mathbb C}
\newcommand{\R}{\mathbb R}
\renewcommand{\H}{\mathbb{H}}
\newcommand{\lcm}{\operatorname{lcm}}
\newcommand{\nequiv}{\not\equiv}
\newcommand{\ord}{\operatorname{ord}}
\newcommand{\ds}{\displaystyle}
\newcommand{\floor}[1]{\left\lfloor #1\right\rfloor}
\newcommand{\legendre}[2]{\left(\frac{#1}{#2}\right)}



%%%%%%%%%%%%



\title{Lagrange's Theorem}
\begin{document}
\begin{abstract}
\end{abstract}
\maketitle

%%%%%%%%%%%%%%%%%%%%%%%%%%
\begin{obj}
    \item Prove Lagrange's Theorem
\end{obj}


\begin{pre}
    \item[Read:] Strayer Section 5.2
    \item[Turn in:] 
    
    \begin{enumerate}
        \item Exercise 10a: Determine the number of incongruent primitive roots modulo $41$
        \begin{solution}
            Since $41$ is prime, \nameref{read-thm:number-prime-roots} says there are $\phi(41)=40$ primitive roots modulo $41.$
        \end{solution}
        \item Exercise 11a: Find all incongruent integers having order $6$ modulo $31.$
        \begin{solution}
                From Appendix E, Table 3, $3$ is a primitive root modulo $31.$ By \nameref{prop:compare-order}, the elements of order $6$ modulo $31$ are those where 
                \[6=\ord_{31}(3^i)=\frac{\phi(31)}{(\phi(31),i)}=\frac{30}{5}.\] 
                The positive integers less than $31$ where $(30,i)=5$ are $i=5,25.$ So the elements of order $6$ are $3^5, 3^{25}.$

                The problem does not ask for the least nonnegative residues. However, we can also find those:
                    \[
                        3^5\equiv (-4)(9)\equiv -5\equiv 26\pmod{31}
                    \]
                    \[
                        3^{25}\equiv (-5)^{5}\equiv (-6)^2(-5)\equiv -25\equiv 6\pmod{31}
                    \]
            \end{solution}
    \end{enumerate}
\end{pre}

The goal is to finish proving the \nameref{thm:prime-roots} with a look at polynomials.

\begin{theorem}[Lagrange]\label{thm:lagrange}
    Let $p$ be a prime number and let 
    \[f(x)=a_n x^n +a_{n-1} x^{n-1}+\cdots +a_1 x+a_0\]
    for integers $a_0,a_1,\dots,a_n.$ Let $d$ be the greatest integer such that $a_d\not\equiv 0\pmod{p}t$ then $d$ is the \emph{degree of $f(x)$ modulo $p.$}
    Then the congruence 
    \[f(x)\equiv 0\pmod{p}\]
    has at most $d$ incongruent solutions. We call these solutions \emph{roots of $f(x)$ modulo $p.$}
    

    \begin{proof}[Proof from class]
        We proceed by induction on the degree $d.$
        
        First, for degree $d=0,$ note that $f(x)\equiv a_0\not\equiv 0\pmod{p}$ by assumption, so $f(x)\equiv 0\pmod{p}$ for $0$ integers.

            
        \begin{description}
                
            \item[Base Case: $d=1$.] 
            Then $f(x)\equiv a_1 x+a_0\pmod{p}$. Since $a_1\not\equiv 0\pmod{p}$ by assumption, $p\nmid a_1.$ Since $p$ is prime, $(a_1,p)=1.$ Thus, by \cref{cor:condition-invertible}, there is a unique solution modulo $p$ to $a_1\not\equiv 0\pmod{p}.$

            \item[Induction Hypothesis:]
            Assume that for all $k\lt d,$ if $f(x)$ has degree $k$ modulo $p,$ then 
            \[f(x)\equiv a_k x^k +a_{k-1}x^{k-1}+\cdots+a_1x+a_0\equiv 0 \pmod{p}\]
            has at most $k$ incongruent solutions.
        \end{description}

        We will proceed by contradiction. That is, assume that there exists $f(x)$ with degree $d$ modulo $p$ and at least $d+1$ roots modulo $p.$ Call these roots $r_1,r_2,\dots,r_d,r_{d+1}.$ Consider the polynomial \[g(x)=a_d(x-r_1)(x-r_2)\cdots (x-r_d).\]
        Then $f(x)$ and $g(x)$ have the same leading term modulo $p.$ The polynomial $h(x)=f(x)-g(x)$ is either the $0$ polynomial or it has degree less than $d$ modulo $p.$

        If $h(x)$ is the $0$ polynomial, then 
        \[h(r_{1})\equiv h(r_2)\equiv\cdots\equiv h(r_{d+1})\equiv 0\pmod{p}\]
        and 
        \[f(r_{1})\equiv f(r_2)\equiv\cdots\equiv f(r_{d+1})\equiv 0\pmod{p}\]
        implies
        \[g(r_{1})\equiv g(r_2)\equiv\cdots\equiv g(r_{d+1})\equiv 0\pmod{p}.\]
        That is, 
        \[a_d(r_{d+1}-r_1)(r_{d+1}-r_2)\cdots(r_{d+1}-r_d)\equiv 0\pmod{p}.\]
        Since $p$ is prime, repeated applications of \cref{prop:zero-divisors} gives that one of $a_d,r_{d+1}-r_1,r_{d+1}-r_2,\dots,r_{d+1}-r_d$ is $0$ modulo $p.$ Now, $a_d\not\equiv0\pmod{p}$ by assumption, and the $r_i$ are distinct modulo $p,$ so we have a contradiction. Thus, $h(x)$ is not the $0$ polynomial. 

        Since $r_1,r_2,\dots,r_d$ are roots of both $f(x)$ and $g(x),$ they are also roots of $h(x).$ This contradicts the induction hypothesis, since $h(x)$ has degree less than $d$ by construction.

        Thus, $f(x)$ has at most $d$ incongruent solution modulo $p.$
    \end{proof}

    \begin{proof}[Modified proof from Strayer]
        We proceed by induction on the degree $d.$
        
        First, for degree $d=0,$ note that $f(x)\equiv a_0\not\equiv 0\pmod{p}$ by assumption, so $f(x)\equiv 0\pmod{p}$ for $0$ integers.
            
        \begin{description}
                
            \item[Base Case: $d=1$.] Then $f(x)\equiv a_1 x+a_0\pmod{p}$. Since $a_1\not\equiv 0\pmod{p}$ by assumption, $p\nmid a_1.$ Since $p$ is prime, $(a_1,p)=1.$ Thus, by \nameref{cor:condition-invertible}, there is a unique solution modulo $p$ to $a_1\not\equiv 0\pmod{p}.$

            \item[Induction Hypothesis:]
            Assume that for all $k\lt d,$ if $f(x)$ has degree $k$ modulo $p,$ then 
            \[f(x)\equiv a_k x^k +a_{k-1}x^{k-1}+\cdots+a_1x+a_0\equiv 0 \pmod{p}\]
            has at most $k$ incongruent solutions.
        \end{description}
        If the congruence $f(x)\equiv 0\pmod{p}$ has no solutions we are done. Otherwise, assume that there exists at least one solution, say $a.$ Dividing $f(x)$ by $(x-a)$ gives 
        \[f(x)\equiv (x-a)q(x)\pmod{p}\]
        where $q(x)$ is a polynomial of degree $d-1$ modulo $p.$
        Since $q(x)$ has at most $d-1$ roots modulo $p$ by the induction hypothesis, there are at most $d-1$ incongruent additional roots of $f(x)$ modulo $p$. Thus, there are a total of at most $d$ incongruent roots modulo $p.$
    \end{proof}
\end{theorem}

\begin{proposition}\label{prop:roots-unity}
    Let $p$ be prime and $m$ a positive integer where $m\mid p-1$. Then 
    \[
        x^m\equiv 1\pmod{p}
    \]
    has $m$ incongruent solutions modulo $p.$

    \begin{proof}
        Let $p$ be prime and $m$ a positive integer where $m\mid p-1$. Then there exists $k\in\Z$ such that $mk=p-1.$
        Then 
        \[
            x^{p-1}-1=(x^m-1)(x^{(k-1)m}+x^{(k-2)m}+\cdots+x^{2m}+x^m+1)
        \]
        By \nameref{FlT}, there are $p-1$ incongruent solutions to $x^{p-1}-1\equiv0\pmod{p},$ namely $1,2,\dots,p-1.$ We will show that $m$ of these are solutions to 
        $x^m-1\equiv 0\pmod{p}$ and the rest are solutions to $x^{(k-1)m}+x^{(k-2)m}+\cdots+x^{2m}+x^m+1\equiv 0\pmod{p}.$

        By \nameref{thm:lagrange}, there are at most $(k-1)m$ solutions to $x^{(k-1)m}+x^{(k-2)m}+\cdots+x^{2m}+x^m+1\equiv 0\pmod{p}.$ Thus, there are at least $p-1-(k-1)m=m$ incongruent solutions to $x^m-1\equiv 0\pmod{p}.$ 
        Since there are also at least $m$ incongruent solutions to $x^m-1\equiv 0\pmod{p}$ by \nameref{thm:lagrange}, there are exactly $m$ incongruent solutions to $x^m-1\equiv 0\pmod{p}$ and thus $x^m\equiv 1\pmod{p}.$
    \end{proof}
\end{proposition}

\begin{definition}[Roots of unity]\label{def:root-unity}
    
    Let $p$ be prime and $m$ a positive integer. We call the solutions to  
        \[x^m\equiv 1\pmod{p}\]
    the \emph{$m^{th}$ roots of unity modulo $p.$}
\end{definition}


\begin{br}\label{br:condition-root-unity}
    Let $p$ be prime, $m$ a positive integer, and $d=(m,p-1).$ Prove that $a^m\equiv 1\pmod{p}$ if and only if $a^d\equiv 1\pmod{p}.$


    \begin{solution}
        Let $p$ be prime, $m$ a positive integer, and $d=(m,p-1).$ Let $a\in\Z$. If $p\mid a,$ then $a^i\answer{\equiv 0\pmod{p}}$ for all positive integers $i$. 
        Thus, we are only considering $a\in\Z$ such that $p\nmid a.$
        Otherwise, $a^{p-1}\equiv 1\pmod{p}$ by $\answer{\textnormal{\nameref{FlT}}}$.
        
        By \cref{prop:order_divides_phi}, $a^m\equiv 1\pmod{p}$ if and only if $\answer{\ord_p a \mid m}$. Similarly, $\answer{a^{p-1}\equiv 1\pmod{p}}$ if and only if $\answer{\ord_p a \mid p-1}$. Thus, $\answer{\ord_p a}$ is a common divisor of $\answer{m}$ and $\answer{p-1}$. Combining \cref{lem:gcd_mult} and \cref{lem:gcd_trans} gives $\ord_p a$ is a common divisor of   $\answer{m}$ and $\answer{p-1}$ if and only if $\ord_p a\mid d$. One final application of \cref{prop:order_divides_phi} gives $\answer{\ord_p a\mid d}$ if and only if $\answer{a^d\equiv 1\pmod{p}}$.
    \end{solution}
\end{br}

\begin{br}\label{br:more-roots-unity}
    Let $p$ be prime and $m$ a positive integer. Prove that 
    \[
        x^m\equiv 1\pmod{p}
    \]
    has exactly $(m,p-1)$ incongruent solutions modulo $p.$


    \begin{proof}
        Let $p$ be prime, $m$ a positive integer, and $d=(m,p-1).$ 
        By \autoref{br:condition-root-unity}, $x^m\equiv 1\pmod{p}$ if and only if $x^d\equiv 1\pmod{p}$. By \cref{prop:roots-unity} there are exactly $d$ solutions to $x^d\equiv 1\pmod{p}.$ Thus, there are exactly $d$ solutions to $x^m\equiv 1\pmod{p}.$
    \end{proof}
\end{br}

%%%%%%%%%%%%%%%%%%%%%%%%%%


\end{document}
