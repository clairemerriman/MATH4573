\documentclass{ximera}
\usepackage{amssymb, latexsym, amsmath, amsthm, graphicx, amsthm,alltt,color, listings,multicol,xr-hyper,hyperref,aliascnt,enumitem}
\usepackage{xfrac}

\usepackage{parskip}
\usepackage[,margin=0.7in]{geometry}
\setlength{\textheight}{8.5in}

\usepackage{epstopdf}

\DeclareGraphicsExtensions{.eps}
\usepackage{tikz}


\usepackage{tkz-euclide}
%\usetkzobj{all}
\tikzstyle geometryDiagrams=[rounded corners=.5pt,ultra thick,color=black]
\colorlet{penColor}{black} % Color of a curve in a plot


\usepackage{subcaption}
\usepackage{float}
\usepackage{fancyhdr}
\usepackage{pdfpages}
\newcounter{includepdfpage}
\usepackage{makecell}


\usepackage{currfile}
\usepackage{xstring}




\graphicspath{  
{./otherDocuments/}
}

\author{Claire Merriman}
\newcommand{\classday}[1]{\def\classday{#1}}

%%%%%%%%%%%%%%%%%%%%%
% Counters and autoref for unnumbered environments
% Not needed??
%%%%%%%%%%%%%%%%%%%%%
\theoremstyle{plain}


\newtheorem*{namedthm}{Theorem}
\newcounter{thm}%makes pointer correct
\providecommand{\thmname}{Theorem}

\makeatletter
\NewDocumentEnvironment{thm*}{o}
 {%
  \IfValueTF{#1}
    {\namedthm[#1]\refstepcounter{thm}\def\@currentlabel{(#1)}}%
    {\namedthm}%
 }
 {%
  \endnamedthm
 }
\makeatother


\newtheorem*{namedprop}{Proposition}
\newcounter{prop}%makes pointer correct
\providecommand{\propname}{Proposition}

\makeatletter
\NewDocumentEnvironment{prop*}{o}
 {%
  \IfValueTF{#1}
    {\namedprop[#1]\refstepcounter{prop}\def\@currentlabel{(#1)}}%
    {\namedprop}%
 }
 {%
  \endnamedprop
 }
\makeatother

\newtheorem*{namedlem}{Lemma}
\newcounter{lem}%makes pointer correct
\providecommand{\lemname}{Lemma}

\makeatletter
\NewDocumentEnvironment{lem*}{o}
 {%
  \IfValueTF{#1}
    {\namedlem[#1]\refstepcounter{lem}\def\@currentlabel{(#1)}}%
    {\namedlem}%
 }
 {%
  \endnamedlem
 }
\makeatother

\newtheorem*{namedcor}{Corollary}
\newcounter{cor}%makes pointer correct
\providecommand{\corname}{Corollary}

\makeatletter
\NewDocumentEnvironment{cor*}{o}
 {%
  \IfValueTF{#1}
    {\namedcor[#1]\refstepcounter{cor}\def\@currentlabel{(#1)}}%
    {\namedcor}%
 }
 {%
  \endnamedcor
 }
\makeatother

\theoremstyle{definition}
\newtheorem*{annotation}{Annotation}
\newtheorem*{rubric}{Rubric}

\newtheorem*{innerrem}{Remark}
\newcounter{rem}%makes pointer correct
\providecommand{\remname}{Remark}

\makeatletter
\NewDocumentEnvironment{rem}{o}
 {%
  \IfValueTF{#1}
    {\innerrem[#1]\refstepcounter{rem}\def\@currentlabel{(#1)}}%
    {\innerrem}%
 }
 {%
  \endinnerrem
 }
\makeatother

\newtheorem*{innerdefn}{Definition}%%placeholder
\newcounter{defn}%makes pointer correct
\providecommand{\defnname}{Definition}

\makeatletter
\NewDocumentEnvironment{defn}{o}
 {%
  \IfValueTF{#1}
    {\innerdefn[#1]\refstepcounter{defn}\def\@currentlabel{(#1)}}%
    {\innerdefn}%
 }
 {%
  \endinnerdefn
 }
\makeatother

\newtheorem*{scratch}{Scratch Work}


\newtheorem*{namedconj}{Conjecture}
\newcounter{conj}%makes pointer correct
\providecommand{\conjname}{Conjecture}
\makeatletter
\NewDocumentEnvironment{conj}{o}
 {%
  \IfValueTF{#1}
    {\innerconj[#1]\refstepcounter{conj}\def\@currentlabel{(#1)}}%
    {\innerconj}%
 }
 {%
  \endinnerconj
 }
\makeatother

\newtheorem*{poll}{Poll question}
\newtheorem{tps}{Think-Pair-Share}[section]


\newenvironment{obj}{
	\textbf{Learning Objectives.} By the end of class, students will be able to:
		\begin{itemize}}
		{\!.\end{itemize}
		}

\newenvironment{pre}{
	\begin{description}
	}{
	\end{description}
}


\newcounter{ex}%makes pointer correct
\providecommand{\exname}{Homework Problem}
\newenvironment{ex}[1][2in]%
{%Env start code
\problemEnvironmentStart{#1}{Homework Problem}
\refstepcounter{ex}
}
{%Env end code
\problemEnvironmentEnd
}

\newcommand{\inlineAnswer}[2][2 cm]{
    \ifhandout{\pdfOnly{\rule{#1}{0.4pt}}}
    \else{\answer{#2}}
    \fi
}


\ifhandout
\newenvironment{shortAnswer}[1][
    \vfill]
        {% Begin then result
        #1
            \begin{freeResponse}
            }
    {% Environment Ending Code
    \end{freeResponse}
    }
\else
\newenvironment{shortAnswer}[1][]
        {\begin{freeResponse}
            }
    {% Environment Ending Code
    \end{freeResponse}
    }
\fi

\let\question\relax
\let\endquestion\relax

\newtheoremstyle{ExerciseStyle}{\topsep}{\topsep}%%% space between body and thm
		{}                      %%% Thm body font
		{}                              %%% Indent amount (empty = no indent)
		{\bfseries}            %%% Thm head font
		{}                              %%% Punctuation after thm head
		{3em}                           %%% Space after thm head
		{{#1}~\thmnumber{#2}\thmnote{ \bfseries(#3)}}%%% Thm head spec
\theoremstyle{ExerciseStyle}
\newtheorem{br}{In-class Problem}

\newenvironment{sketch}
 {\begin{proof}[Sketch of Proof]}
 {\end{proof}}


\newcommand{\gt}{>}
\newcommand{\lt}{<}
\newcommand{\N}{\mathbb N}
\newcommand{\Q}{\mathbb Q}
\newcommand{\Z}{\mathbb Z}
\newcommand{\C}{\mathbb C}
\newcommand{\R}{\mathbb R}
\renewcommand{\H}{\mathbb{H}}
\newcommand{\lcm}{\operatorname{lcm}}
\newcommand{\nequiv}{\not\equiv}
\newcommand{\ord}{\operatorname{ord}}
\newcommand{\ds}{\displaystyle}
\newcommand{\floor}[1]{\left\lfloor #1\right\rfloor}
\newcommand{\legendre}[2]{\left(\frac{#1}{#2}\right)}



%%%%%%%%%%%%



\title{Primitive roots modulo a prime}
\begin{document}
\begin{abstract}
\end{abstract}
\maketitle

%%%%%%%%%%%%%%%%%%%%%%%%%%

\begin{obj}
    \item Find the order of an element modulo $m$ using primitive roots
\end{obj}

\begin{instructorNotes}
\begin{pre}
    \item[Reading] Uploaded notes
    \item[Turn in] For each result in the scanned notes, identify the result in our textbook. If it is a special case of the theorem in the textbook, (ie, the reading only proves the theorem for primes or $d=q^s$), also note this.

\end{pre}
    
\end{instructorNotes}
%%%%%%%%%%%%%%%%%%%%%%%%%%


\begin{definition}[primitive root]\label{defn:prime-root}
    Let $r,m\in\Z$ with $m>0$ and $(r,m)=1.$ Then $r$ is said to be a \emph{primitive root modulo $m$} if $\ord_m r=\phi(r).$
\end{definition}

We saw in the reading that primitive roots always exist modulo a prime. What about composites?
\begin{example}
    
    \begin{itemize}
        \item Since $\phi(4)=2,$ and $\ord_4 3=2,$ $3$ is a primite root modulo $4.$ The powers $\{3^1, 3^2\}$ are a reduced residue system modulo $4.$
        \item Since $\phi(6)=\phi(3)\phi(2)=2$ and $\ord_6 5=2,$ $5$ is a primitive root modulo $6.$ The powers $\{5^1, 5^2\}$ are a reduced residue system modulo $6.$
        \item There are no primitive roots modulo $8.$ By \nameref{thm:phi-prime-power}, $\phi(8)=4.$ Since every odd number squares to $1$ modulo $8,$ $\ord_8 1=1$ and $\ord_8 3=\ord_8 5=\ord_8 7=2.$ 
        \item Since $\phi(9)=3^1(3-1)=6$ by \nameref{thm:phi-prime-power}, we check:

        \begin{align*}
            2^1&=1, &2^2&=4, &2^3&=8, &2^4&\equiv7\pmod{9}, &2^5&\equiv 5\pmod{9},&2^6&\equiv 1\pmod{9}
        \end{align*}
        So $2$ is a primite root modulo $9,$ but are there more?
        \begin{align*}
            4^1&=4, &4^2&=2^4\equiv 7\pmod{9}, &4^3&=2^6\equiv1\pmod{9}
        \end{align*}
        We can also use exponent rules and \nameref{prop:exponents_mod_order} to simplify some calculuations. For example, $5\equiv 2^5\pmod{9},$ so $5^i\equiv 2^{5i}\equiv 2^j\pmod{9}$ if and only if $5i\equiv j\pmod{6}.$

        \begin{align*}
            5^1&\equiv 5\pmod{9}, 
            &5^2&\equiv 2^{10}\equiv 2^4\equiv 7\pmod{9},
            &5^3&\equiv 2^{15}\equiv 2^3\equiv 8\pmod{9}, \\
            5^4&\equiv 2^{20}\equiv 2^2\equiv 4\pmod{9}, 
            &5^5&\equiv 2^{25}\equiv 2^1\equiv 2\pmod{9},
            &5^6&\equiv 1\pmod{9},
        \end{align*}
        \begin{align*}
            7^1&\equiv (-2)\equiv 7\pmod{9}, 
            &7^2&\equiv (-2)^2\equiv 4\pmod{9},
            &7^3&\equiv (-2)^{3}\equiv -8 \equiv 1\pmod{9}
        \end{align*}
        \begin{align*}
            \ord_9(1)&=1\\
            \ord_9(2)&=\ord_9(5)=6\\
            \ord_9(4)&=\ord_9(7)=3\\
            \ord_9 (8)&=2
        \end{align*}
    \end{itemize}
\end{example}


\begin{proposition}\label{prop:prime-roots-generate}
    Let $r$ be a primitive root modulo $m.$ Then \[\{r,r^2,\dots,r^{\phi(m)}\}\]
    is a set of reduced residues modulo $m.$
\end{proposition}
This is the general version of \nameref{read-prop:prime-root-generates}, using exponents $1,2,\dots,\phi(m)$ instead of $0,1,\dots,\phi(m)-1$. Since Strayer's statement of \nameref{prop:exponents_mod_order} is already stated and proved for composites, and both lists have the same number of elements, the only changes to the proof is replacing $p-1$ with $\phi(m).$ Note $a^0\equiv a^{\phi(m)}\equiv 1\pmod{m}$ when $(a,m)=1.$


\begin{proposition}\label{prop:compare-order}
    Let $a,m\in\Z$ with $m>0$ and $(a,m)=1.$ If $i$ is a positive integer, then 
    \[\ord_m(a^i)=\frac{\ord_m a}{\gcd(\ord_m a,i)}.\]
\end{proposition}

\begin{br}
    Use only the results through \nameref{prop:prime-roots-generate}/\nameref{read-lem:order-multiplicative} to prove the primitive root version:


    \begin{proposition}
        Let $r,m\in\Z$ with $m>0$ and $r$ a primitive root modulo $m.$
        If $i$ is a positive integer, then 
    \[\ord_m(r^i)=\frac{\phi(m)}{\gcd(\phi(m),i)}.\]
    
    \begin{proof}
        Let $i,r,m\in\Z$ with $i,m\gt 0$ and $r$ a primitive root modulo $m.$ Then $\ord_m r=\phi(m)$ by \hyperref[defn:prime-root]{definition}. Let $d=(\phi(m),i).$ Then there exists positive integers $j,k$ such that $\phi(m)=dj,i=dk$ and $(j,k)=1$ by \nameref{prop:div-gcd-rel-prime}. Then using the proceding equations and exponent rules, we find
            \[
                (a^i)^j=(a^{dk})^{\phi(m)/d}=(a^{\phi(m)})^k\equiv 1\pmod{m}
            \]
        since $a^{\phi(m)}\equiv 1\pmod{p}$ by \hyperref[defn:primeRoot]{definition}. \nameref{prop:order_divides_phi} says that $\ord_p(a^i)\mid j.$

        Since $a^{i\ord_p(a^i)}\equiv(a^i)^{\ord_p(a^i)}\equiv 1\pmod{p}$ by \hyperref[defn:order]{definition of order}, \nameref{prop:order_divides_phi} says that $\ord_p a\mid i \ord_p(a^i).$ Since $\ord_p a=\phi(m)=dj$ and $i=dk,$ we have $dj\mid dk \ord_p(a^i)$ which simplifies to $j\mid k \ord_p(a^i)$. Since $(j,k)=1,$ we can conclude $j\mid \ord_p(a^i)$. 

        Since $\ord_p(a^i)\mid j, j\mid \ord_p(a^i)$ and both values are positive, we can conclude that $\ord_p(a^i)=j.$ Finally, we have 
            \[
                \ord_p(a^i)=j=\frac{\phi(m)}{d}=\frac{\phi(m)}{(\phi(m),i)}.
            \]
    \end{proof}
\end{proposition}
\end{br}


Exercises cited in the reading, also on Homework 6:



\begin{br}
    Prove the following statement, which is the converse of \nameref{read-prop:prime-root-generates}:

    Let $p$ be prime, and let $a\in\Z.$ If every $b\in\Z$ such that $p\nmid b$ is congruent to a power of $a$ modulo $p,$ then ${a}$ is a primitive root modulo $p$.
    
\end{br}

\begin{br}
    Prove the following generalization of \nameref{read-lem:order-multiplicative}
    
    
    \begin{lemma}
        Let $n\in\Z$ and let $x_1,x_2,\dots,x_m$ be reduced residues modulo $n$.  Suppose that for all $i\neq j,$ $\ord_n(x_i)$ and $\ord_n(x_j)$ are relatively prime. Then \[\ord_n(x_1 x_2\cdots x_m)=(\ord_n x_1)(\ord_n x_2)\cdots (\ord_n x_m).\]
    \end{lemma}
\end{br}



\end{document}
