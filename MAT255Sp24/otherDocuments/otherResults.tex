\documentclass[letterpaper, 11 pt]{ximera}
\usepackage{amssymb, latexsym, amsmath, amsthm, graphicx, amsthm,alltt,color, listings,multicol,xr-hyper,hyperref,aliascnt,enumitem}
\usepackage{xfrac}

\usepackage{parskip}
\usepackage[,margin=0.7in]{geometry}
\setlength{\textheight}{8.5in}

\usepackage{epstopdf}

\DeclareGraphicsExtensions{.eps}
\usepackage{tikz}


\usepackage{tkz-euclide}
%\usetkzobj{all}
\tikzstyle geometryDiagrams=[rounded corners=.5pt,ultra thick,color=black]
\colorlet{penColor}{black} % Color of a curve in a plot


\usepackage{subcaption}
\usepackage{float}
\usepackage{fancyhdr}
\usepackage{pdfpages}
\newcounter{includepdfpage}
\usepackage{makecell}


\usepackage{currfile}
\usepackage{xstring}




\graphicspath{  
{./otherDocuments/}
}

\author{Claire Merriman}
\newcommand{\classday}[1]{\def\classday{#1}}

%%%%%%%%%%%%%%%%%%%%%
% Counters and autoref for unnumbered environments
% Not needed??
%%%%%%%%%%%%%%%%%%%%%
\theoremstyle{plain}


\newtheorem*{namedthm}{Theorem}
\newcounter{thm}%makes pointer correct
\providecommand{\thmname}{Theorem}

\makeatletter
\NewDocumentEnvironment{thm*}{o}
 {%
  \IfValueTF{#1}
    {\namedthm[#1]\refstepcounter{thm}\def\@currentlabel{(#1)}}%
    {\namedthm}%
 }
 {%
  \endnamedthm
 }
\makeatother


\newtheorem*{namedprop}{Proposition}
\newcounter{prop}%makes pointer correct
\providecommand{\propname}{Proposition}

\makeatletter
\NewDocumentEnvironment{prop*}{o}
 {%
  \IfValueTF{#1}
    {\namedprop[#1]\refstepcounter{prop}\def\@currentlabel{(#1)}}%
    {\namedprop}%
 }
 {%
  \endnamedprop
 }
\makeatother

\newtheorem*{namedlem}{Lemma}
\newcounter{lem}%makes pointer correct
\providecommand{\lemname}{Lemma}

\makeatletter
\NewDocumentEnvironment{lem*}{o}
 {%
  \IfValueTF{#1}
    {\namedlem[#1]\refstepcounter{lem}\def\@currentlabel{(#1)}}%
    {\namedlem}%
 }
 {%
  \endnamedlem
 }
\makeatother

\newtheorem*{namedcor}{Corollary}
\newcounter{cor}%makes pointer correct
\providecommand{\corname}{Corollary}

\makeatletter
\NewDocumentEnvironment{cor*}{o}
 {%
  \IfValueTF{#1}
    {\namedcor[#1]\refstepcounter{cor}\def\@currentlabel{(#1)}}%
    {\namedcor}%
 }
 {%
  \endnamedcor
 }
\makeatother

\theoremstyle{definition}
\newtheorem*{annotation}{Annotation}
\newtheorem*{rubric}{Rubric}

\newtheorem*{innerrem}{Remark}
\newcounter{rem}%makes pointer correct
\providecommand{\remname}{Remark}

\makeatletter
\NewDocumentEnvironment{rem}{o}
 {%
  \IfValueTF{#1}
    {\innerrem[#1]\refstepcounter{rem}\def\@currentlabel{(#1)}}%
    {\innerrem}%
 }
 {%
  \endinnerrem
 }
\makeatother

\newtheorem*{innerdefn}{Definition}%%placeholder
\newcounter{defn}%makes pointer correct
\providecommand{\defnname}{Definition}

\makeatletter
\NewDocumentEnvironment{defn}{o}
 {%
  \IfValueTF{#1}
    {\innerdefn[#1]\refstepcounter{defn}\def\@currentlabel{(#1)}}%
    {\innerdefn}%
 }
 {%
  \endinnerdefn
 }
\makeatother

\newtheorem*{scratch}{Scratch Work}


\newtheorem*{namedconj}{Conjecture}
\newcounter{conj}%makes pointer correct
\providecommand{\conjname}{Conjecture}
\makeatletter
\NewDocumentEnvironment{conj}{o}
 {%
  \IfValueTF{#1}
    {\innerconj[#1]\refstepcounter{conj}\def\@currentlabel{(#1)}}%
    {\innerconj}%
 }
 {%
  \endinnerconj
 }
\makeatother

\newtheorem*{poll}{Poll question}
\newtheorem{tps}{Think-Pair-Share}[section]


\newenvironment{obj}{
	\textbf{Learning Objectives.} By the end of class, students will be able to:
		\begin{itemize}}
		{\!.\end{itemize}
		}

\newenvironment{pre}{
	\begin{description}
	}{
	\end{description}
}


\newcounter{ex}%makes pointer correct
\providecommand{\exname}{Homework Problem}
\newenvironment{ex}[1][2in]%
{%Env start code
\problemEnvironmentStart{#1}{Homework Problem}
\refstepcounter{ex}
}
{%Env end code
\problemEnvironmentEnd
}

\newcommand{\inlineAnswer}[2][2 cm]{
    \ifhandout{\pdfOnly{\rule{#1}{0.4pt}}}
    \else{\answer{#2}}
    \fi
}


\ifhandout
\newenvironment{shortAnswer}[1][
    \vfill]
        {% Begin then result
        #1
            \begin{freeResponse}
            }
    {% Environment Ending Code
    \end{freeResponse}
    }
\else
\newenvironment{shortAnswer}[1][]
        {\begin{freeResponse}
            }
    {% Environment Ending Code
    \end{freeResponse}
    }
\fi

\let\question\relax
\let\endquestion\relax

\newtheoremstyle{ExerciseStyle}{\topsep}{\topsep}%%% space between body and thm
		{}                      %%% Thm body font
		{}                              %%% Indent amount (empty = no indent)
		{\bfseries}            %%% Thm head font
		{}                              %%% Punctuation after thm head
		{3em}                           %%% Space after thm head
		{{#1}~\thmnumber{#2}\thmnote{ \bfseries(#3)}}%%% Thm head spec
\theoremstyle{ExerciseStyle}
\newtheorem{br}{In-class Problem}

\newenvironment{sketch}
 {\begin{proof}[Sketch of Proof]}
 {\end{proof}}


\newcommand{\gt}{>}
\newcommand{\lt}{<}
\newcommand{\N}{\mathbb N}
\newcommand{\Q}{\mathbb Q}
\newcommand{\Z}{\mathbb Z}
\newcommand{\C}{\mathbb C}
\newcommand{\R}{\mathbb R}
\renewcommand{\H}{\mathbb{H}}
\newcommand{\lcm}{\operatorname{lcm}}
\newcommand{\nequiv}{\not\equiv}
\newcommand{\ord}{\operatorname{ord}}
\newcommand{\ds}{\displaystyle}
\newcommand{\floor}[1]{\left\lfloor #1\right\rfloor}
\newcommand{\legendre}[2]{\left(\frac{#1}{#2}\right)}



%%%%%%%%%%%%


\lhead{\large{Number Theory: MAT-255}}
%Put your Document Title (Camp: Topic) Here
\chead{}
\rhead{Spring 2024}
\lfoot{}
\cfoot{}
\rfoot{Page \thepage}
\renewcommand\headrulewidth{0pt}
\renewcommand\footrulewidth{0pt}

\headheight 50pt
\headsep 30pt




%%%%%%%%%%%%%%%%%%%%%
% Create a chapter divider where there is not an intro file
% Should be a place holder until there is a file
%%%%%%%%%%%%%%%%%%%%%
\newcommand{\chapter}[1]{\addtocounter{titlenumber}{1}%
{\flushleft\LARGE\sffamily\bfseries\thetitlenumber\hspace{1em}#1 \par }%
{\vskip .6em\noindent\textit\theabstract\setcounter{problem}{0}\setcounter{sectiontitlenumber}{0}}%
\par\vspace{2em}
\phantomsection\addcontentsline{toc}{section}{\textbf{\thetitlenumber\hspace{1em}#1}}%
}

<<<<<<< Updated upstream

=======
\makeatletter
\renewcommand\chapterstyle{%
  \def\activitystyle{activity-chapter}
  \def\maketitle{%
    \addtocounter{titlenumber}{1}%
        {\flushleft\small\sffamily\bfseries\@pretitle\par\vspace{-1.5em}}%
        {\flushleft\LARGE\sffamily\bfseries\thetitlenumber\hspace{1em}\@title \par }%
        {\vskip .6em\noindent\textit\theabstract\setcounter{problem}{0}\setcounter{sectiontitlenumber}{0}}%
        \par\vspace{2em}
        \phantomsection\addcontentsline{toc}{section}{\textbf{\thetitlenumber\hspace{1em}\@title}}%
        \let\section\subsection
        \let\subsection\subsubsection
    }}

\renewcommand\sectionstyle{%
    \def\activitystyle{activity-section}
    \def\maketitle{%
        \addtocounter{sectiontitlenumber}{1}
        {\flushleft\small\sffamily\bfseries\@pretitle\par\vspace{-1.5em}}%
        {\flushleft\Large\sffamily\bfseries\thetitlenumber.\thesectiontitlenumber\hspace{1em}\@title \par}%
        {\vskip .6em\noindent\textit\theabstract}%
        \par\vspace{2em}
        \phantomsection\addcontentsline{toc}{subsection}{\thetitlenumber.\thesectiontitlenumber\hspace{1em}\@title}%
        \let\section\subsubsection
        \let\subsection\subsubsubsection
    }}

\makeatother
>>>>>>> Stashed changes

%%%%%%%%%%%%%%%%%%%%%
% Create handoutstyle for in class assignments
%%%%%%%%%%%%%%%%%%%%%
<<<<<<< Updated upstream
\makeatletter
 \newcommand\handoutstyle{%
    \addtocounter{titlenumber}{1}%
    \phantomsection\addcontentsline{toc}{section}{\@date}%
        \setcounter{br}{0}}
%
%
%\newcommand{\handoutTitle}{\title[%
%\textnormal{\large \scshape MAT-255-- Number Theory \hfill Spring 2024 \hfill In Class Work \classday}%
%
%\textnormal{\large
%Your Name: \hrulefill \quad Group Members:\hrulefill \quad \hrulefill}%
%\vspace{-5em}]{}}
=======
\newcommand{\handoutTitle}{
    \title[%
        \textnormal{\large \scshape MAT-255-- Number Theory \hfill Spring 2024 \hfill In Class Work \classday}%
        
        \textnormal{\large
        Your Name: \hrulefill \quad Group Members:\hrulefill \quad \hrulefill}%
        \vspace{-5em}]{}
    }


\makeatletter
\newcommand\handoutstyle{%
\def\activitystyle{activity-handout}
\def\maketitle{
    \renewcommand{\handoutTitle}{\title{In Class \classday}}
    \phantomsection\addcontentsline{toc}{subsection}{\@date}%
      \setcounter{br}{0}
      \setcounter{theorem}{0}
       \setcounter{proposition}{0}
       \setcounter{lemma}{0}}
   }
>>>>>>> Stashed changes

\newcommand{\handoutAbstract}{\begin{abstract}
\end{abstract}}

\makeatother


\title{Other Results from Strayer and Homework Assignments}


\begin{document}

\begin{abstract}
    Most of these results are covered in the readings from \emph{Elementary Number Theory} by James K. Strayer in Spring 2024, and referenced in these notes. Additionally, some results were proved on homework assignments and not listed in other places in the notes.
    All of the results in this section are standard elementary number theory and presented without proof. 
\end{abstract}

\maketitle

\begin{axiom}[Well Ordering Principle]\label{well-order}
    Every nonempty set of positive integers contains a least element.
\end{axiom}

\section{Divisibility facts}\label{sec:additional-div}

    \begin{lemma}[Proposition 1.2]\label{lem:linear-combo}
         Let $a,b,c,d\in\Z.$ If $c\mid a$ and $c\mid d,$ then $c\mid ma+nb.$
    \end{lemma}


    \begin{proposition}[Proposition 1.10]\label{prop:div-gcd-rel-prime}
        Let $a,b\in\Z$ with $(a,b)=d.$ Then $(\tfrac{a}{d},\tfrac{b}{d})=1.$
    \end{proposition}


    \begin{lemma}[Lemma 1.12]\label{lem:gcd-remainders}
     If $a,b\in\Z,$ $a\geq b\gt 0,$ and $a=bq+r$ with $q,r\in|Z,$ then $(a,b)=(b,r).$
    \end{lemma}


    \begin{proposition}[Homework 3, Problem 4]\label{prop:common-div-gcd}
        Let $a_1,\dots,a_n\in\Z$ with $a_1\neq 0$ and let $d=(a_1,\dots,a_n).$ Then $c\in\Z$ is a common divisor of $a_1,\dots,a_n$ if and only if $c\mid d.$ 
    \end{proposition}
    

\section{Prime facts}\label{sec:additional-primes}

\begin{lemma}[Lemma 1.14]\label{lem:irreducible-prime}
    Let $a,b,p\in\Z$ with $p$ prime. If $p\mid ab,$ then $p\mid a$ or $p\mid b.$
\end{lemma}

\begin{corollary}[Corollary 1.15]\label{cor:irreducible-prime} Let $a_1,a_2,\dots,a_n,p\in\Z$ with $p$ prime. If $p\mid a_1a_2\cdots a_n,$ then $p\mid a_i$ for some $i.$
\end{corollary}

\begin{proposition}[Proposition 1.17]\label{prop:form-lcm-gcd}
 Let $a,b\in\Z$ with $a,b\gt 1.$ Write $a=p_1^{a_1}p_2^{a_2}\cdots  p_n^{a_n}$ and $b=p_1^{b_1}p_2^{b_2}\cdots p_n^{b_n}$ where $p_1,p_2,\dots,p_n$ are distinct primes and ${a_1},{a_2}\cdots,{a_n},{b_1},{b_2},\cdots,{b_n}$ are nonnegative integers (possibly zero). Then
        \[(a,b)=p_1^{\min\{a_1,b_1\}}p_2^{\min\{a_2,b_2\}}\cdots p_n^{\min\{a_n,b_n\}}\]
        and 
        \[[a,b]=p_1^{\max\{a_1,b_1\}}p_2^{\max\{a_2,b_2\}}\cdots p_n^{\max\{a_n,b_n\}}.\]
\end{proposition}

\begin{theorem}[Theorem 1.19]\label{thm:prod-lcm-gcd} Let $a,b\in\Z$ with $a,b\gt 0.$ Then $(a,b)[a,b]=ab.$
\end{theorem}

\section{Congruences}

% \begin{proposition}[Proposition 2.1]\label{prop:equiv-rel}
%     Let $a,b,c,d,m\in\Z$ with $m>0,$ then:
%         \begin{enumerate}
%             \item\label{equiv-reflect} $a\equiv a \pmod{m}$
            
%             \item\label{equiv-sym} $a\equiv b \pmod{m}$ implies $b\equiv a \pmod{m}$

%             \item\label{equiv-trans} $a\equiv b \pmod{m}$ and $b\equiv c \pmod{m}$ implies $a\equiv c \pmod{m}$
% \end{enumerate}
% \end{proposition}


% \begin{proposition}[Proposition 2.4]\label{prop:add-mult}
%     Let $a,b,c,d,m\in\Z$ with $m>0,$ then:
    
%     \begin{enumerate}[label=(\alph*)]
%         \item\label{equiv-add} $a\equiv b \pmod{m}$ and $c\equiv d \pmod{m}$ implies $a+c \equiv b+d \pmod{m}$ 
%         \item\label{equiv-multiply} $a\equiv b\pmod{m}$ and $c\equiv d \pmod{m}$ implies $ac\equiv bd \pmod{m}$.
%     \end{enumerate}
% \end{proposition}


\begin{proposition}[Proposition 2.5]\label{prop:equiv-gcd}
    Let $a,b,c,m\in\Z$ with $m>0.$ Then $ca\equiv cb\pmod{m}$ if and only if $a\equiv b\pmod{\tfrac{m}{(a,m)}}.$
\end{proposition}


\begin{lemma}[Chapter 2, Exercise 9]\label{ex:equiv-upmod}
    Let $a,b,c,m\in\Z$ with $m>0.$ If $a\equiv b \pmod{m}$ then $ac\equiv bc \pmod{mc}$ for $c>0$.
\end{lemma}


\begin{proposition}[Homework 4, Problem 9]\label{prop:zero-divisors}
    Let $p$ be prime, then $ax\equiv 0\pmod{p}$ implies $a\equiv 0\pmod{p}$ or $x\equiv 0\pmod{p}.$ 

    Furthermore, for a composite integer $m,$ $ax\equiv 0\pmod{m}$ does \emph{not} imply either $a\equiv 0\pmod{m}$ or $x\equiv 0\pmod{m}.$ 
\end{proposition}

\begin{corollary}[Corollary 2.15]\label{cor:a_power_prime_mod}
    Let $p$ be a prime number and let $a\in\Z.$ Then $a^p\equiv a\pmod{p}.$
\end{corollary}

\section{The Euler Phi-Function}

\begin{theorem}[Theorem 3.3]\label{thm:phi-prime-power}
    Let $p$ be prime and let $a\in\Z$ with $a>0.$ Then $\phi(p^a)=p^a-p^{a-1}=p^{a-1}(p-1).$
\end{theorem}



\end{document}