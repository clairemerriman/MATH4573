\documentclass[letterpaper, 11 pt]{ximera}
\usepackage{amssymb, latexsym, amsmath, amsthm, graphicx, amsthm,alltt,color, listings,multicol,hyperref,xr-hyper,aliascnt,enumitem}
\usepackage{xfrac}


\usepackage{parskip}
\usepackage{graphicx}
\usepackage[,margin=0.7in]{geometry}
\setlength{\textheight}{8.5in}
 
\usepackage{tkz-euclide}
%\usetkzobj{all}
\tikzstyle geometryDiagrams=[rounded corners=.5pt,ultra thick,color=black]
\colorlet{penColor}{black} % Color of a curve in a plot


\usepackage{subcaption}
\usepackage{float}
\usepackage{fancyhdr}
\usepackage{pdfpages}
\newcounter{includepdfpage}


\newcommand{\semester}{%
  \ifcase\month
  \or Spring %1
  \or Spring %2
  \or Spring %3
  \or Spring %4
  \or Spring  %5
  \or Fall %8
  \or Fall %9
  \or Fall %10
  \or Fall %11
  \or Fall %12
  \fi
}
\usepackage{currfile}
\usepackage{xstring}


\lhead{\large{Number Theory: MAT-255}}
%Put your Document Title (Camp: Topic) Here
\chead{}
\rhead{\semester 24}
\lfoot{}
\cfoot{}
\rfoot{Page \thepage}
\renewcommand\headrulewidth{0pt}
\renewcommand\footrulewidth{0pt}

\headheight 50pt
\headsep 30pt

\author{Claire Merriman}
\date{Spring 2024}


%%%%%%%%%%%%%%%%%%%%%
% Counters and autoref for unnumbered environments
%%%%%%%%%%%%%%%%%%%%%
\theoremstyle{plain}


\newtheorem*{namedthm}{Theorem}
\newcounter{thm}%makes pointer correct
\providecommand{\thmname}{Proposition}

\makeatletter
\NewDocumentEnvironment{thm*}{o}
 {%
  \IfValueTF{#1}
    {\namedthm[#1]\refstepcounter{thm}\def\@currentlabel{(#1)}}%
    {\namedthm}%
 }
 {%
  \endnamedthm
 }
\makeatother


\newtheorem*{namedprop}{Proposition}
\newcounter{prop}%makes pointer correct
\providecommand{\propname}{Proposition}

\makeatletter
\NewDocumentEnvironment{prop*}{o}
 {%
  \IfValueTF{#1}
    {\namedprop[#1]\refstepcounter{prop}\def\@currentlabel{(#1)}}%
    {\namedprop}%
 }
 {%
  \endnamedprop
 }
\makeatother

\newtheorem*{namedlem}{Lemma}
\newcounter{lem}%makes pointer correct
\providecommand{\lemname}{Lemma}

\makeatletter
\NewDocumentEnvironment{lem*}{o}
 {%
  \IfValueTF{#1}
    {\namedlem[#1]\refstepcounter{lem}\def\@currentlabel{(#1)}}%
    {\namedlem}%
 }
 {%
  \endnamedlem
 }
\makeatother

\newtheorem*{namedcor}{Corollary}
\newcounter{cor}%makes pointer correct
\providecommand{\corname}{Corollary}

\makeatletter
\NewDocumentEnvironment{cor*}{o}
 {%
  \IfValueTF{#1}
    {\namedcor[#1]\refstepcounter{cor}\def\@currentlabel{(#1)}}%
    {\namedcor}%
 }
 {%
  \endnamedcor
 }
\makeatother

\theoremstyle{definition}

\newtheorem*{innerrem}{Remark}
\newcounter{rem}%makes pointer correct
\providecommand{\remname}{Remark}

\makeatletter
\NewDocumentEnvironment{rem}{o}
 {%
  \IfValueTF{#1}
    {\innerrem[#1]\refstepcounter{rem}\def\@currentlabel{(#1)}}%
    {\innerrem}%
 }
 {%
  \endinnerrem
 }
\makeatother

\newtheorem*{innerdefn}{Definition}%%placeholder
\newcounter{defn}%makes pointer correct
\providecommand{\defnname}{Definition}

\makeatletter
\NewDocumentEnvironment{defn}{o}
 {%
  \IfValueTF{#1}
    {\innerdefn[#1]\refstepcounter{defn}\def\@currentlabel{(#1)}}%
    {\innerdefn}%
 }
 {%
  \endinnerdefn
 }
\makeatother

\newtheorem*{scratch}{Scratch Work}


\newtheorem*{namedconj}{Conjecture}
\newcounter{conj}%makes pointer correct
\providecommand{\conjname}{Conjecture}
\makeatletter
\NewDocumentEnvironment{conj}{o}
 {%
  \IfValueTF{#1}
    {\innerconj[#1]\refstepcounter{conj}\def\@currentlabel{(#1)}}%
    {\innerconj}%
 }
 {%
  \endinnerconj
 }
\makeatother

\newtheorem*{poll}{Poll question}
\newtheorem{tps}{Think-Pair-Share}[section]
%\newtheorem{br}{In-class Problem}[section]
\newtheorem*{cs}{Crowd Sourced Proof}

\newlist{checklist}{itemize}{2}
\setlist[checklist]{label=$\square$}

\newenvironment{obj}{
	\textbf{Learning Objectives.} By the end of class, students will be able to:
		\begin{itemize}}
		{\!.\end{itemize}
		}

\newenvironment{pre}{
	\begin{description}
	}{
	\end{description}
}


\newcounter{br}%makes pointer correct
\providecommand{\brname}{In-class Problem}

\newenvironment{br}[1][2in]%
{%Env start code
\problemEnvironmentStart{#1}{In-class Problem}
\refstepcounter{br}
}
{%Env end code
\problemEnvironmentEnd
}

\newcounter{ex}%makes pointer correct
\providecommand{\exname}{Homework Problem}
\newenvironment{ex}[1][2in]%
{%Env start code
\problemEnvironmentStart{#1}{Homework Problem}
\refstepcounter{ex}
}
{%Env end code
\problemEnvironmentEnd
}



\newenvironment{sketch}
 {\begin{proof}[Sketch of Proof]}
 {\end{proof}}
%\newenvironment{hint}
%  {\begin{proof}[Hint]}
%  {\end{proof}}

\newcommand{\gt}{>}
\newcommand{\lt}{<}
\newcommand{\N}{\mathbb N}
\newcommand{\Q}{\mathbb Q}
\newcommand{\Z}{\mathbb Z}
\newcommand{\C}{\mathbb C}
\newcommand{\R}{\mathbb R}
\renewcommand{\H}{\mathbb{H}}
\newcommand{\lcm}{\operatorname{lcm}}
\newcommand{\nequiv}{\not\equiv}
\newcommand{\ord}{\operatorname{ord}}
\newcommand{\ds}{\displaystyle}
\newcommand{\floor}[1]{\left\lfloor #1\right\rfloor}
\newcommand{\legendre}[2]{\left(\frac{#1}{#2}\right)}



%%%%%%%%%%%%



%Imports for cross references
\externaldocument{otherResults}
\externaldocument{Week1Spring24}
\externaldocument{Week2Spring24}
\externaldocument{Week3Spring24}
\externaldocument{Week4Spring24}
\externaldocument{Week5Spring24}
\externaldocument{Week6Spring24}

\StrBetween*[1,1]{\currfilename}{Week}{Sp}[\week]

\title{Week \week--MAT-255 Number Theory}

\begin{document}
%%%%%%%%%%%%%%%%%%%%%%%%%
\section{Monday, February 26: Modular arithmetic calculations}
%%%%%%%%%%%%%%%%%%%%%%%%%%

\begin{obj}
\item Use Wilson's Theorem to find the least nonnegative residue modulo a prime
\item Use Fermat's Little Theorem to find the least nonnegative residue modulo a prime
%\item Use Euler's Theorem to find the least nonnegative residue modulo a composite
\end{obj}


\begin{pre}
    \item[Reading] None
\end{pre}

%%%%%%%%%%%%%%%%%%%%%%%%%%
\subsection{Multiplicative inverses using Wilson's Theorem (30 min)}
%%%%%%%%%%%%%%%%%%%%%%%%%%
First, some important algebra  for multiplicative inverses
\begin{example}
\begin{enumerate}
 \item  Let $n$ be an odd positive integer. Then \[2\left(\frac{n+1}{2}\right)=n+1\equiv 1\pmod{n}.\] So $\frac{n+1}{2}$ is the multiplicative inverse of $2$ modulo $n$. 
\begin{tps}
 Why is $\left(\frac{n+1}{2},n\right)=1$?
\end{tps}

We also have that $n-\frac{n+1}{2}= \frac{n-1}{2}$ and $n-2\equiv -2\pmod{n},$ so $\frac{n-1}{2}$ is the multiplicative inverse of $n-2$ modulo $n.$ Another way to see this is 
\[-2\left(\frac{n-1}{2}\right)=-n+1\equiv 1\pmod{n}.\] 

\item Let $m$ and $n$ be positive integers such that $n\equiv 1\pmod m.$ Then there exists $k\in\Z$ such that $n=mk+1$ by the \nameref{defn:mod-equiv-all}. Then 
\[-m\left(\frac{n-1}{m}\right)=-n+1\equiv 1\pmod{n}.\]

\item Let $m$ and $n$ be positive integers such that $n\equiv -1\pmod m.$ Then there exists $k\in\Z$ such that $n=mk-1$ by definition. Then 
\[m\left(\frac{n+1}{m}\right)=n+1\equiv 1\pmod{n}.\]
\end{enumerate}

\end{example}


\begin{example}
\begin{enumerate}
 \item  Practice with Wilson's Theorem: 
 Find  $\frac{31!}{23!} \pmod{11}.$
 \begin{align*}
 x\equiv \frac{31!}{23!}	 & \equiv & 24(25)(26)(27)(28)(29)(30)(31) \pmod{11}\\
 			 & \equiv & 2(3)(4)(5)(6)(7)(8)(9) \pmod{11}.
\end{align*}
Then $-x\equiv 10! \equiv -1\pmod{11}.$ Therefore, $x\equiv 1\pmod{11}.$

\item Let $p$ be an odd prime $p,$ then $2(p-3)!\equiv -1 \pmod{p}$.
 
\begin{proof}
 Let $p$ be an odd prime, then $(p-1)!\equiv -1\pmod{p}$ by Wilson's Theorem. Multiplying both sides of the congruence by $-1$ gives $(p-2)!\equiv 1\pmod{p}.$ Since $(p-2)!=(p-3)!(p-2)$ by the definition of factorial, $p-2\equiv -2\pmod{p}$ is the multiplicative inverse of $(p-3)!$ modulo $p.$ Thus, 
\begin{align*}
 -2(p-3)!	&\equiv 1\pmod{p}\\
 2(p-3)!	 & \equiv & -1\pmod{p}.\qedhere
\end{align*}
\end{proof}
\end{enumerate}

\end{example}


%%%%%%%%%%%%%%%%%%%%%%%%%%
\subsection{Finding least nonnegative residue Fermat's Little Theorem (20 min)}
%%%%%%%%%%%%%%%%%%%%%%%%%%
\begin{example}
\begin{enumerate}
 \item  Find the least nonnegative residue of $29^{202}$ modulo $13$. 
 
 First, note that $29\equiv 3\pmod{13}$ and $202=12(10)+82=12(10)+12(6)+10=12(16)+10.$  Thus,
 \[29^{202}\equiv3^{202}\equiv (3^{12})^{16}  3^{10}\equiv 1^{16} 3^{10}\pmod{13}\]
 From here,  we have two  options: 
\begin{description}
 \item[Keep reducing:] For this problem, this is the easier method: \[3^{10}\equiv (3^3)^3 3\equiv (27)^3 3\equiv 3\pmod{13}.\]
 \item[Find inverse:] Note that $3^{12}\equiv 1\pmod{13},$ so $3^{10}$ is the multiplicative inverse of $3^2\equiv 9\pmod{13}.$ Since $9(3)\equiv 1\pmod{13},$ $3^{10}\equiv 1\pmod{13}.$
\end{description}

\item Find the least nonnegative residue of $71^{71}$ modulo $17$. 

First, note that $71\equiv 3\pmod{17}$ and $71=8(8)+7.$  Thus,
 \[71^{71}\equiv3^{71}\equiv (3^{8})^{8}  3^{7}\equiv 1^{8} 3^{7}\pmod{17}\]
Then \[3^7\equiv 3^{3}(3^3)(3)\equiv 10(10)(3)\equiv 10(-4)\equiv -6\equiv 11\pmod{17}.\]
\end{enumerate}
\end{example}

\begin{cor*}[Corollary 2.14]\label{cor:inv-fermat}
 Let $p$ be a prime. If $a\in\Z$ with $p\nmid a,$ then $a^{p-2}$  is the multiplicative inverse of $a$ modulo $p$.
\end{cor*}

\begin{tps} Prove:
  Let $p$ be a prime. If $a,k\in\Z$ with $p\nmid a$ and $0\leq k<p,$ then $a^{p-k}$  is the multiplicative inverse of $a^k$ modulo $p$.
  
  
\begin{proof}
 Let $p$ be a prime. If $a\in\Z$ with $p\nmid a,$ then by Fermat's Little Theorem, $a^{p-1}\equiv 1\pmod{p.}$ If $k\in\Z$ with $0\leq k<p,$ then $a^{p-1}=a^{p-k}a^k.$ Thus, $a^{p-k}a^k\equiv 1\pmod{p}.$
\end{proof}
\end{tps}

\begin{example}
 Find all incongruent solutions to $9x\equiv 21\pmod{23}$.
 
Since$(9,23)=1,$ there is only one  incongruent solution  modulo $23.$ By \nameref{cor:inv-fermat}, $9^{21}$ is the multiplicative inverse of $9$ modulo ${23}$. Thus, $x\equiv 21(9^{21})\pmod{23.}$

Alternately, $3^{20}$ is the multiplicative inverse of $3^2$ modulo ${23},$ so $x\equiv 21(3^{20})\equiv  (3^{21})7\pmod{23}.$ Since $3^{21}$ is the multiplicative inverse of $3$ modulo $23,$ so $3^{21}\equiv 8\pmod{23}.$ Thus, $x\equiv 7(8)\equiv 10\pmod{23}.$
\end{example}

\begin{example}
 Let $p$ be prime and $a,b\in\Z$ with $p\nmid a$ and $p\nmid b.$ Then $a^p\equiv b^p\pmod{p}$ if and only if $a\equiv b\pmod{p}$.
 
 
\begin{proof}
  Let $p$ be prime and $a,b\in\Z$ with $p\nmid a$ and $p\nmid b.$ 
  
  ($\Leftarrow$) If $a\equiv  b\pmod{p},$ then $a^p\equiv b^p\pmod{p}$ by repeated applications of Proposition 2.4.
  
  ($\Rightarrow$)  If $a^p\equiv  b^p\pmod{p},$ then by Fermat's Little Theorem, 
  \[a\equiv a^{p-1}a\equiv b^{p-1}b\equiv b\pmod p. \]
\end{proof}
\begin{warning}
 This statement is only true for primes. Since
\[
 1^2\equiv 3^2\equiv 5^2\equiv 7^2\pmod{8},\quad 2^2\equiv 6^2\pmod{8},
 \]
 
\[
 1^8\equiv 3^8\equiv 5^8\equiv 7^8\pmod{8},\quad  2^8\equiv 6^8\pmod{8}.
\]
\end{warning}
\end{example}

%%%%%%%%%%%%%%%%%%%%%%%%%
\section{Wednesday, February 28: Modular arithmetic calculations and the $\phi-$ function}
%%%%%%%%%%%%%%%%%%%%%%%%%%

\begin{obj}
\item Use Euler's Theorem to find the least nonnegative residue modulo a composite
\item Use Euler's Theorem to find the multiplicative inverse of an integer modulo $m$
\item Prove $\phi(4)\phi(5)=\phi(20)$ using an outline that mirrors the proof that $\phi(m)\phi(n)=\phi(mn)$ when $(m,n)=1$
\end{obj}


\begin{pre}
    \item[Reading] None
\end{pre}

%%%%%%%%%%%%%%%%%%%%%%%%%%
\subsection{Multiplicative inverses using Euler's Extension of Fermat's Little Theorem (30 min)}
%%%%%%%%%%%%%%%%%%%%%%%%%%

\begin{example}\label{ex-euler-solve-cong}
 
\begin{enumerate}
 \item Find the least nonnegative residue of $29^{202}$ modulo $20$. 
 
 The integers $1,3,7,9,11,13,17,19$ are relatively prime to $20.$ Thus $\phi(20)=8.$ Also note that $29\equiv 9\pmod{20}$ and $202=8(25)+2,$ so 
 \[29^{202}\equiv 9^{202}\equiv(9^8)^{25} 9^2\equiv 1^{25} 9^2\equiv 1\pmod{20}\] 
 
 \item Find the least nonnegative residue of $71^{71}$ modulo $16$.
 
 The integers $1,3,5,7,9,11,13,15$ are relatively prime to $16.$ Thus $\phi(16)=8.$ Also note that $71\equiv 7\pmod{16}$ and $71=8(8)+7,$ so 
  \[71^{71}\equiv 7^{71}\equiv(7^8)^{8} 7^7\equiv 1^{8} 7^7\pmod{16}\] 
Since $7^8\equiv 7^7 7\equiv 1\pmod{16},$ $7^7$ is the multiplicative inverse of $7$ mulod $16.$

Using the Euclidean algorithm, 
\begin{align*}
 16	& = 7(2) + 2,	&2&=16+7(-2)\\
 7	& = 2(3) + 1,	&1&=7-2(3)=7-(16+7(-2))(3)=16(-3)+7(7)
\end{align*}
Thus, $7(7)\equiv 1\pmod{16},$ and $7^7\equiv 7\pmod{16}.$
\end{enumerate}
\end{example}

\begin{cor*}[Corollary 2.19]\label{cor:inv-euler}
 Let $a,m\in\Z$ with $m>0.$  If $(a,m)=1,$ then $a^{\phi(m)-1}$ is the multiplicative inverse of $a$ modulo $m$.
\end{cor*}

\begin{example}
 Find all incongruent solutions to $9x\equiv 21\pmod{25}$.
 
The only positive integers less than $25$ that are \emph{not} relatively prime to $25$ are $5,10,15,20$. Thus, $\phi(25)=24-4=20.$

Since$(9,25)=1,$ there is only one  incongruent solution  modulo $25.$ By \nameref{cor:inv-euler}, $9^{19}$ is the multiplicative inverse of $9$ modulo ${25}$. Thus, $x\equiv 21(9^{19})\pmod{25.}$

Alternately, $3^{18}$ is the multiplicative inverse of $3^2$ modulo ${25},$ so $x\equiv 21(3^{18})\equiv  (3^{19})7\pmod{25}.$
\end{example}

The previous example does not ask for the least nonnegative residue, but let's find it anyway.

\begin{example}
  Find the least nonnegative residue of $(9^{19})21$ modulo $25$.

  First, note that $(9^{19})21=(3^2)^19(21)$. From here there are two options:

  \begin{description}
    \item[Factor $21$:]
      \begin{align*}
        (9^{19})21\equiv (3^2)^19(3)(7)
        \equiv (3^{39})(7)
        \equiv (3^{20})(3^{19})(7) \pmod{25}
      \end{align*}
      By \nameref{thm:euler-FlT}, $3^{20}\equiv 1\pmod{25}$ and by \nameref{cor:inv-euler}, $3^{19}$ is the multiplicative inverse of $3$ moddulo $25.$ Since $3(-8)\equiv -24\equiv 1\pmod{25},$ $3^{19}\equiv -8\pmod{25.}$ Thus, 
      \begin{align*}
        (9^{19})21\equiv (-8)(7)\equiv -56\equiv 19 \pmod{25}.
      \end{align*}

    \item[Using $21\equiv -4\pmod{25}$:] 
      \begin{align*}
        (9^{19})21\equiv (3^2)^19(-4)
        \equiv (3^{38})(-4)
        \equiv (3^{20})(3^{18})(-4) \pmod{25}
      \end{align*}
      Since $3^{20}=3^{18}(3^2)\equiv 1\pmod{25}$ by \nameref{thm:euler-FlT}, $3^{18}$ is the multiplicative inverse of $3^2=9$ modulo $25.$ Since $9(-11)\equiv -99\equiv 1\pmod{25},$ we have $3^{18}\equiv -11\pmod{25}.$ Thus, 
      \begin{align*}
        (9^{19})21\equiv (-11)(-4)\equiv 44 \equiv 19 \pmod{25}.
      \end{align*}
  \end{description}
  
\end{example}

\begin{br}\label{br:modpq}
  Let $p,q$ be distinct primes. Prove that $p^{q-1}+q^{p-1}\equiv 1 \pmod{pq}.$
 
 
  \begin{proof} Let $p,q$ be distinct primes. 
   Then $\answer{q^{p-1}\equiv 1}\pmod{p}$ and  $\answer{p^{q-1}\equiv 1}\pmod{q}$ by Fermat's Little Theorem, and $\answer{p^{q-1}\equiv 1}\equiv0\pmod{p}$ and  $\answer{q^{p-1}\equiv 1}\equiv 0\pmod{q}$ by $\answer{\textnormal{definition}}.$
   
   Thus, $p^{q-1}+q^{p-1}\equiv \answer{1} \pmod{p}$ and $p^{q-1}+q^{p-1}\equiv \answer{1} \pmod{p}$ by $\answer{\textnormal{modular addition}}.$

   (Finish proof using definition of congruence modulo $p$ and $q$)
  \end{proof}
\end{br}

%%%%%%%%%%%%%%%%%%%%%%%%%%
\subsection{The Euler $\phi$-function (20 min)}\label{sec-phi-start}
%%%%%%%%%%%%%%%%%%%%%%%%%%

We will also find a formula for $\phi(n)$ in general. The following exercise will outline the general proof:

\begin{br}\label{br:multiplicative-proof}
  Let us prove that $\phi(20)=\phi(4)\phi(5)$. First, note that $\phi(4)=\answer{2}$ and $\phi(5)=\answer{4}$, so $\phi(20)=\answer{8}$.
  \begin{enumerate}
    \item A number $a$ is relatively prime to $20$ if and only if $a$ is relatively prime to $\answer{4}$ and $\answer{5}$ %(first blank should be smaller than second blank for the automatic grading to work, both should be relevant to what we are trying to show). 
    
    \item  
    We can partition the positive integers less that $20$ into 
    \begin{align*}
      & 0\equiv\answer{4}\equiv\answer{8}\equiv\answer{12}\equiv\answer{16}\pmod 4\\
      & 1\equiv\answer{5}\equiv\answer{9}\equiv\answer{13}\equiv\answer{17}\pmod 4\\
      & 2\equiv\answer{6}\equiv\answer{10}\equiv\answer{14}\equiv\answer{18}\pmod 4\\
      & 3\equiv\answer{7}\equiv\answer{11}\equiv\answer{15}\equiv\answer{19}\pmod 4
    \end{align*}
 
    For any $b$ in the range $0,1,2,3$, define $s_b$ to be the number of integers $a$ in the range $0,1,2,\dots, 19$ such that $a\equiv b \pmod 4$ and $\gcd(a,20)=1$. Thus, $s_0=\answer{0}, s_1=\answer{4}, s_2=\answer{0}$, and $s_3=\answer{4}$.
 
    We can see that when $(b,4)=1$, $s_b=\phi(\answer{5})$ and when $(b,4)>1$, $s_b=\answer{0}$.
 
    \item $\phi(20)=s_0+s_1+s_2+s_3$. Why? 
 
    \begin{solution}
      All of the positive integers less than or equal to $20$ is in exactly one of the congruence classes above. The $s_i$ count how many integers in each congruence class are relatively prime to $20$. If we add them up, we have counted all positive integers less than or equal to $20$.
    \end{solution}
 
    \item We have seen that $\phi(20)=s_0+s_1+s_2+s_3$, that when $(b,4)=1$, $s_b=\phi(5)$, and that when $(b,4)>1$, $s_b=0$. Thus, we can say that $\phi(20)=0+\phi(\answer{5})+0+\phi(\answer{5})$. To finish the 	``proof" we show that there are $\phi(\answer{4})$ integers $b$ where $(b,4)=1$. 
 
    \begin{solution}
      There are $\answer{4}$ congruence classes modulo 4. Of these, $\answer{2}=\phi(\answer{4})$ have elements that are relatively prime to $20$. Thus, $\phi(20)=\phi(4)\phi(5)$.
    \end{solution}
  \end{enumerate}
 
\end{br}

\section{Friday, March 1: The $\phi-$ function and a preview of primitive roots}
%%%%%%%%%%%%%%%%%%%%%%%%%%

\begin{obj}
  \item Prove that $\phi(m)\phi(n)=\phi(mn)$ when $(m,n)=1$
\end{obj}


\begin{pre}
    \item[Reading] None
    \item[Turn In] Paper 2
\end{pre}

%%%%%%%%%%%%%%%%%%%%%%%%%%
\subsection{Quiz (10 min)}
%%%%%%%%%%%%%%%%%%%%%%%%%%

%%%%%%%%%%%%%%%%%%%%%%%%%%
\subsection*{The Euler $\phi$-function (20 min)}
%%%%%%%%%%%%%%%%%%%%%%%%%%

We will use \autoref{br:multiplicative-proof} as an outline to prove 

\begin{thm*}[Theorem 3.2]\label{thm:phi-multiplicative}
  Let $m$ and $n$ be positive integers where $(m,n)=1$. Then $\phi(mn)=\phi(m)\phi(n).$
\end{thm*}
maybe works?

\begin{proof} 
  First, we note that an integer $a$ is relatively prime to $mn$ if and only if it is relatively prime to $m$ and $n,$ since $m$ and $n$ (together) have the same prime divisors as $mn$.

  
  We can partition the positive integers less that $mn$ into 
    \begin{alignat*}{8}
      &0  & \equiv & m  & \equiv & 2m & \equiv &\cdots  & \equiv & (n-1)m\pmod{m}\\
      &1  & \equiv & m + 1  & \equiv & 2m + 1 & \equiv &\cdots  & \equiv & (n-1)m + 1\pmod{m}\\
      &2  & \equiv & m + 2  & \equiv & 2m + 2 & \equiv & \cdots  & \equiv & (n-1)m + 2\pmod{m}\\
      & \vdots & & \vdots& & \vdots & & & &  \vdots\\
      &m-1  & \equiv & 2m - 1  & \equiv & 3m -1 & \equiv & \cdots  & \equiv & nm - 1\pmod{m}
    \end{alignat*}
 
    For any $b$ in the range $0,1,2,\dots,m-1$, define $s_b$ to be the number of integers $a$ in the range $0,1,2,\dots, mn-1$ such that $a\equiv b \pmod{m}$ and $\gcd(a,mn)=1$. For each equivalence class $b,$ $\gcd(b,m)\mid km +b$ by linear combination. Thus, $s_b={0}$ if $(b,m)>1.$ If $\gcd(b,m)=1$, the arithmetic progression, \{b, m + b, 2m + b, \dots, (n-1)m+ b\} contains $n$ elements. By \autoref{br:arithmetic-progression}, the arithmetic progression is a \nameref{defn:complete-residue} modulo $n,$ so $\phi(n)$ elements are relatively prime to $n$ and thus $mn.$
 
    Thus, can see that when $(b,m)=1$, $s_b=\phi(n)$ and when $(b,m)>1$, $s_b={0}$.
 
    Since all of the positive integers less than or equal to $mn$ is in exactly one of the congruence classes above and t he $s_i$ count how many integers in each congruence class are relatively prime to $mn$, $phi(mn)=s_0+s_1+\cdots +s_{m-1}.$
 
    Since $\phi(m)$ of the $s_i=\phi(n)$ and the rest are 0, $\phi(mn)=s_0+s_1+\cdots +s_{m-1}=\phi(m)\phi(n).$
 
\end{proof}


\begin{br}\label{br:arithmetic-progression}
  Complete the proof of \nameref{thm:phi-multiplicative} by proving that, if $m, n,$ and $i$ are positive integers with ($m, n) = (m, i) = 1,$ then the integers $i, m + i, 2m +i,..., (n - 1)m +i$ form a complete system of residues modulo $n.$
\end{br}
\end{document}