\documentclass[letterpaper, 11 pt]{ximera}
\usepackage{amssymb, latexsym, amsmath, amsthm, graphicx, amsthm,alltt,color, listings,multicol,hyperref,xr-hyper,aliascnt,enumitem}
\usepackage{xfrac}


\usepackage{parskip}
\usepackage{graphicx}
\usepackage[,margin=0.7in]{geometry}
\setlength{\textheight}{8.5in}
 
\usepackage{tkz-euclide}
%\usetkzobj{all}
\tikzstyle geometryDiagrams=[rounded corners=.5pt,ultra thick,color=black]
\colorlet{penColor}{black} % Color of a curve in a plot


\usepackage{subcaption}
\usepackage{float}
\usepackage{fancyhdr}
\usepackage{pdfpages}
\newcounter{includepdfpage}


\newcommand{\semester}{%
  \ifcase\month
  \or Spring %1
  \or Spring %2
  \or Spring %3
  \or Spring %4
  \or Spring  %5
  \or Fall %8
  \or Fall %9
  \or Fall %10
  \or Fall %11
  \or Fall %12
  \fi
}
\usepackage{currfile}
\usepackage{xstring}


\lhead{\large{Number Theory: MAT-255}}
%Put your Document Title (Camp: Topic) Here
\chead{}
\rhead{\semester 24}
\lfoot{}
\cfoot{}
\rfoot{Page \thepage}
\renewcommand\headrulewidth{0pt}
\renewcommand\footrulewidth{0pt}

\headheight 50pt
\headsep 30pt

\author{Claire Merriman}
\date{Spring 2024}


%%%%%%%%%%%%%%%%%%%%%
% Counters and autoref for unnumbered environments
%%%%%%%%%%%%%%%%%%%%%
\theoremstyle{plain}


\newtheorem*{namedthm}{Theorem}
\newcounter{thm}%makes pointer correct
\providecommand{\thmname}{Proposition}

\makeatletter
\NewDocumentEnvironment{thm*}{o}
 {%
  \IfValueTF{#1}
    {\namedthm[#1]\refstepcounter{thm}\def\@currentlabel{(#1)}}%
    {\namedthm}%
 }
 {%
  \endnamedthm
 }
\makeatother


\newtheorem*{namedprop}{Proposition}
\newcounter{prop}%makes pointer correct
\providecommand{\propname}{Proposition}

\makeatletter
\NewDocumentEnvironment{prop*}{o}
 {%
  \IfValueTF{#1}
    {\namedprop[#1]\refstepcounter{prop}\def\@currentlabel{(#1)}}%
    {\namedprop}%
 }
 {%
  \endnamedprop
 }
\makeatother

\newtheorem*{namedlem}{Lemma}
\newcounter{lem}%makes pointer correct
\providecommand{\lemname}{Lemma}

\makeatletter
\NewDocumentEnvironment{lem*}{o}
 {%
  \IfValueTF{#1}
    {\namedlem[#1]\refstepcounter{lem}\def\@currentlabel{(#1)}}%
    {\namedlem}%
 }
 {%
  \endnamedlem
 }
\makeatother

\newtheorem*{namedcor}{Corollary}
\newcounter{cor}%makes pointer correct
\providecommand{\corname}{Corollary}

\makeatletter
\NewDocumentEnvironment{cor*}{o}
 {%
  \IfValueTF{#1}
    {\namedcor[#1]\refstepcounter{cor}\def\@currentlabel{(#1)}}%
    {\namedcor}%
 }
 {%
  \endnamedcor
 }
\makeatother

\theoremstyle{definition}

\newtheorem*{innerrem}{Remark}
\newcounter{rem}%makes pointer correct
\providecommand{\remname}{Remark}

\makeatletter
\NewDocumentEnvironment{rem}{o}
 {%
  \IfValueTF{#1}
    {\innerrem[#1]\refstepcounter{rem}\def\@currentlabel{(#1)}}%
    {\innerrem}%
 }
 {%
  \endinnerrem
 }
\makeatother

\newtheorem*{innerdefn}{Definition}%%placeholder
\newcounter{defn}%makes pointer correct
\providecommand{\defnname}{Definition}

\makeatletter
\NewDocumentEnvironment{defn}{o}
 {%
  \IfValueTF{#1}
    {\innerdefn[#1]\refstepcounter{defn}\def\@currentlabel{(#1)}}%
    {\innerdefn}%
 }
 {%
  \endinnerdefn
 }
\makeatother

\newtheorem*{scratch}{Scratch Work}


\newtheorem*{namedconj}{Conjecture}
\newcounter{conj}%makes pointer correct
\providecommand{\conjname}{Conjecture}
\makeatletter
\NewDocumentEnvironment{conj}{o}
 {%
  \IfValueTF{#1}
    {\innerconj[#1]\refstepcounter{conj}\def\@currentlabel{(#1)}}%
    {\innerconj}%
 }
 {%
  \endinnerconj
 }
\makeatother

\newtheorem*{poll}{Poll question}
\newtheorem{tps}{Think-Pair-Share}[section]
%\newtheorem{br}{In-class Problem}[section]
\newtheorem*{cs}{Crowd Sourced Proof}

\newlist{checklist}{itemize}{2}
\setlist[checklist]{label=$\square$}

\newenvironment{obj}{
	\textbf{Learning Objectives.} By the end of class, students will be able to:
		\begin{itemize}}
		{\!.\end{itemize}
		}

\newenvironment{pre}{
	\begin{description}
	}{
	\end{description}
}


\newcounter{br}%makes pointer correct
\providecommand{\brname}{In-class Problem}

\newenvironment{br}[1][2in]%
{%Env start code
\problemEnvironmentStart{#1}{In-class Problem}
\refstepcounter{br}
}
{%Env end code
\problemEnvironmentEnd
}

\newcounter{ex}%makes pointer correct
\providecommand{\exname}{Homework Problem}
\newenvironment{ex}[1][2in]%
{%Env start code
\problemEnvironmentStart{#1}{Homework Problem}
\refstepcounter{ex}
}
{%Env end code
\problemEnvironmentEnd
}



\newenvironment{sketch}
 {\begin{proof}[Sketch of Proof]}
 {\end{proof}}
%\newenvironment{hint}
%  {\begin{proof}[Hint]}
%  {\end{proof}}

\newcommand{\gt}{>}
\newcommand{\lt}{<}
\newcommand{\N}{\mathbb N}
\newcommand{\Q}{\mathbb Q}
\newcommand{\Z}{\mathbb Z}
\newcommand{\C}{\mathbb C}
\newcommand{\R}{\mathbb R}
\renewcommand{\H}{\mathbb{H}}
\newcommand{\lcm}{\operatorname{lcm}}
\newcommand{\nequiv}{\not\equiv}
\newcommand{\ord}{\operatorname{ord}}
\newcommand{\ds}{\displaystyle}
\newcommand{\floor}[1]{\left\lfloor #1\right\rfloor}
\newcommand{\legendre}[2]{\left(\frac{#1}{#2}\right)}



%%%%%%%%%%%%




%Imports for cross references
\externaldocument{otherResults}
\externaldocument{Week1Spring24}
\externaldocument{Week2Spring24}
\externaldocument{Week3Spring24}
\externaldocument{Week4Spring24}
\externaldocument{Week5Spring24}


\StrBetween*[1,1]{\currfilename}{Week}{Sp}[\week]

\title{Week \week--MAT-255 Number Theory}

\begin{document}
%%%%%%%%%%%%%%%%%%%%%%%%%
\section{Monday, February 19: Chinese Remainder Theorem}
%%%%%%%%%%%%%%%%%%%%%%%%%%

\begin{obj}
\item Solve system of linear equations in one variable.
 \item Prove the Chinese Remainder Theorem.
\end{obj}


\begin{pre}
    \item[Reading] None
\end{pre}

%%%%%%%%%%%%%%%%%%%%%%%%%%
\subsection{Multiplicative inverses (20 min)}
%%%%%%%%%%%%%%%%%%%%%%%%%%
From Friday

\begin{cor*}[Corollary 2.8]\label{cor:condition-invertible}
    Let $a,m\in\Z$ with $m>0.$ The linear congruence in one variable $ax\equiv 1\pmod{m}$ has a solution if and only if $(a,m)=1$. If $(a,m)=1,$ then the solution is unique modulo $m$. 
\end{cor*}

\begin{defn}[multiplicative inverse of $a$ modulo $m$]\label{defn:mult-inv} Let $a,m\in\Z$ with $m>0$ and $(a,m)=1.$
    We call the unique incongruent solution to $ax\equiv 1\pmod m$ the \emph{multiplicative inverse of $a$ modulo $m$}.
\end{defn}

\begin{example}\label{example:mult-inv}
    Examples of multiplicative inverses:
    \begin{itemize}
        \item $5(3)\equiv 1\pmod{7}$ so $3$ is the multiplicative inverse of $5$ modulo $7$ and $5$ is the multiplicative inverse of $3$ modulo $7$.
        \item $9(5)\equiv 1\pmod{11}$ so $5$ is the multiplicative inverse of $9$ modulo $11$ and $9$ is the multiplicative inverse of $5$ modulo $11$.
        \item $8(-4)\equiv 8(7)\equiv 1\pmod{11}$ so $7\equiv -4\pmod{11}$ is the multiplicative inverse of $8$ modulo $11$ and $8$ is the multiplicative inverse of $7\equiv -4\pmod{11}$ modulo $11$.
        \item $8(5)\equiv 1\pmod{13}$ so $5$ is the multiplicative inverse of $8$ modulo $13$ and $8$ is the multiplicative inverse of $5$ modulo $13$.
    \end{itemize}

    Example using multiplicative inverses:
    \begin{align*}
        6!& \equiv 6\ast 5\ast 4\ast 3\ast 2\ast 1\pmod{7}\\
          & \equiv 6\ast 5(3)\ast 4(2)\ast 1\pmod{7}\\
          & \equiv 6 \pmod{7}
    \end{align*}
\end{example}

\begin{tps}
    Find $10!\pmod{11}$ and $12!\pmod{13}.$ Is there a pattern?

    
    \begin{solution}
        \begin{align*}
            10! & \equiv 10\ast 9\ast 8\ast 7\ast 6\ast 5\ast 4\ast 3\ast 2\ast 1\pmod{11}\\
                & \equiv 10\ast 9(5)\ast 8(7)\ast 6(2)\ast 4(3)\ast 1\pmod{11}\\
                & \equiv 1\pmod{11}
        \end{align*}

        \begin{align*}
            12! & \equiv 12\ast 11\ast 10\ast 9\ast 8\ast 7\ast 6\ast 5\ast 4\ast 3\ast 2\ast 1\pmod{13}\\
                & \equiv 12\ast 11(6) \ast 10(4)\ast 9(3)\ast 8(5)\ast 7(2) \ast 1\pmod{13}\\
                & \equiv 1\pmod{13}
        \end{align*}

        For a prime $p,$ $(p-1)!\equiv 1\pmod{p}$.
    \end{solution}
\end{tps}
\begin{remark}
    We do need the condition that $p$ is prime. For example, $3!\equiv 2\pmod{4},$ and $8!\equiv 0\pmod{9}.$
\end{remark}
%%%%%%%%%%%%%%%%%%%%%%%%%%
\subsection{Simultaneous Linear congruences in one variable (30 min)}


\begin{example}
    Consider the system of linear equations 
    \begin{align*}
        x &\equiv 2 \pmod{5}\\
        x &\equiv 3 \pmod{7}\\
        x &\equiv 1 \pmod{8}.
    \end{align*}

    A slow way to find an integer $x$ that satisfies all three congruences is to write out the congruence classes:
    \begin{align*}
        2, 2+5, 2+5(2), \boxed{2+5(3)}, \dots\\
        3, 3+7, \boxed{3+7(2)}, 3+7(3), \dots\\
        1, 1+8, 1+8(2), \boxed{1+8(3)}, \dots
    \end{align*}
    and see what integers are on all three lists. In addition to being tedius, we this doesn't help find \emph{all} such integers.

    To find all such integers, define $M=5(7)(8)=280,$ and $M_1=\frac{M}{5}=7(8),M_2=\frac{M}{7}=5(8),M_3=\frac{M}{8}=5(7).$ Then each $M_i$ is relatively prime to $M$ by construction. Thus, by \nameref{cor:condition-invertible} the congruences
    \begin{align*}
        M_1x_1 & \equiv 1\pmod 5, & 7(8)x_1&\equiv x_1 \equiv 1\pmod 5\\
        M_2x_2 & \equiv 1\pmod 7, & 5(8)x_2 &\equiv 5x_2 \equiv 1\pmod 7\\
        M_3x_3 & \equiv 1\pmod 8, & 5(7)x_3&\equiv 3x_3\equiv 1\pmod 8
    \end{align*}
    have solutions. Thus, $x_1\equiv 1\pmod{5}, x_2\equiv 3\pmod{7},$ and $x_3\equiv 3\pmod{8}.$

    Note that 
    \begin{align*}
        M_1x_1(2)&=56(1)(2)\equiv 2\pmod 5, & M_2&\equiv M_3\equiv 0\pmod{5}\\
        M_2x_2(3)&=40(3)(3)\equiv 3\pmod 7, & M_1&\equiv M_3\equiv 0\pmod{7}\\
        M_3x_3(1)&=35(3)(1)\equiv 1\pmod 8, & M_1&\equiv M_2\equiv 0\pmod{8}
    \end{align*}

    Thus, \[x=M_1x_1(2)+M_2x_2(3)+M_3x_3(1)=56(1)(2)+40(3)(3)+35(3)(1)\]
    is a solution to all three congruences.
\end{example}

\begin{thm*}[Chinese Remainder Theorem]\label{CRT}
    Let $m_1,m_2,\dots m_k$ be pairwise relatively prime positive integers (that is, any pair $\gcd(m_i,m_j)=1$ when $i\neq j$). Let $b_1, b_2,\dots, b_k$ be integers. Then the system of congruences 
   \begin{align*}
    x&\equiv b_1 \pmod{m_1}\\
    x&\equiv b_2 \pmod{m_2}\\
       &\vdots\\
     x&\equiv b_n \pmod{m_k}
   \end{align*}
   has a unique solution modulo $M=m_1m_2\dots m_k$. This solution has the form 
   \[x=M_1x_1b_1+M_2x_2b_2+\cdot+M_kx_kb_k,\] where $M_i=\frac{M}{m_i}$ and $M_i x_i\equiv 1 \pmod{m_i}$.
   \end{thm*}
   \begin{proof} Let $m_1,m_2,\dots m_k$ be pairwise relatively prime positive integers.
    We start by constructing a solution modulo $M=m_1m_2\dots m_k$. By construction, $M_i=\frac{M}{m_i}$ is an integer. Since each the $m_i$ are pairwise relatively prime, $\left(M_i, m_i\right)=1$. Thus, by \nameref{cor:condition-invertible}, for each $i$ there is an integer $x_i$ where $M_i x_i\equiv 1 \pmod{m_i}$. Thus $M_i x_i b_i\equiv b_i\pmod{m_i}$. We also have that $(M_i, m_j)=m_j$ when $i\neq j$, so $M_i b_i\equiv 0 \pmod{m_j}$ when $i\neq j$.  Let 
    \[x=M_1x_1b_1+M_2x_2b_2+\cdot+M_kx_kb_k.\] 
    Then $x\equiv M_i x_i b_i\equiv b_i\pmod{m_i}$ for each $i=1,2,\dots,k$ and $x\equiv M_i x_i b_i\equiv 0\pmod{m_j}$ when $i\neq j.$ Thus, we have found a solution to the system of equivalences.
    
    To show the solution is unique modulo $M,$ consider two solutions $x_1,x_2.$ Then $x_1\equiv x_2\pmod{m_i}$ for each $i=1,2,\dots,k.$ Thus $m_i\mid x_2-x_1$. Since $(m_i,m_j)=1$ when $i\neq j,$ $M=[m_1,m_2,\dots,m_k]$ and $M\mid x_2-x_1.$ Thus, $x_1\equiv x_2\pmod M.$ 
   \end{proof}
   
%%%%%%%%%%%%%%%%%%%%%%%%%%

%%%%%%%%%%%%%%%%%%%%%%%%%
\section{Wednesday, February 21: Wilson's Theorem}
%%%%%%%%%%%%%%%%%%%%%%%%%%

\begin{obj}
    \item Characterize when $a$ is its own inverse modulo a prime.

    \item Prove Wilson's Theorem and its converse
\end{obj}


\begin{pre}
    \item[Reading] Strayer, Section 2.4

    \item[Turn in]
    Does this match with your conjecture from Exercise 5? If not, what is the difference?
\end{pre}

%%%%%%%%%%%%%%%%%%%%%%%%%%
\subsection{Wilson's Theorem (50 min)}
%%%%%%%%%%%%%%%%%%%%%%%%%%


\begin{lem*}[Lemma 2.10]\label{lem:sqrt1}
    Let $p$ be a prime number and $a\in\Z.$ Then $a$ is its own inverse modulo $m$ if and only if $a\equiv \pm 1\pmod{p}.$
\end{lem*}

\begin{proof}
    Let $p$ be a prime number and $a\in\Z.$ Then $a$ is its own inverse modulo $m$ if and only if $a^2\equiv 1 \pmod{p}$ if and only if $p\mid a^2-1=(a-1)(a+1).$ Since $p$ is prime, $p\mid a-1$ or $a+1$ by \nameref{lem:irreducible-prime}. Thus, $a\equiv\pm 1\pmod p.$
\end{proof}

\begin{corollary}\label{cor:sqrt1}
    Let $p$ be a prime. Then $x^2\equiv 1\pmod{p}$ if and only if $x\equiv \pm 1\pmod{p}.$
\end{corollary}

\begin{remark}
    It is important to note why we require $p$ is prime. \nameref{lem:irreducible-prime} is only true for primes: 
    \begin{itemize}
        \item $8\mid ab$ is true when $8\mid a,$ $8\mid b,$ $4\mid a$ and $2\mid b,$ or $2\mid a$ and $4\mid b.$
    \end{itemize}
    Let $a=2k+1$ for some integer $k.$ Then 
    \[a^2=4k^2+4k+1=4k(k+1)+1.\]
    Since either $k$ or $k+1$ is even, $a^2=8m+1$ for some $m\in\Z.$ Thus, $a^2\equiv 1\pmod 8$ for all odd integers $a\in\Z.$

    \begin{itemize}
        \item When $a\equiv 1\pmod 8,$ then $8\mid (a-1).$
        \item When $a\equiv 3\pmod 8,$ then $8k=a-3$ for some $k\in\Z.$ Thus $2\mid (a-1)$ and $4\mid (a+1)$.
        \item When $a\equiv 5\pmod 8,$ then $8k=a-5$ for some $k\in\Z.$ Thus $4\mid (a-1)$ and $2\mid (a+1)$.
        \item When $a\equiv 7\pmod 8,$ then $8\mid (a+1).$
    \end{itemize}
\end{remark}


\begin{thm*}[Wilson's Theorem]\label{Wilson}
    Let $p$ be a prime number. Then \[(p-1)!\equiv -1\pmod{p}.\]
\end{thm*}

\begin{proof}
    When $p=2,$ $(2-1)!=1\equiv -1 \pmod{2}.$ Now consider $p$ an odd prime. By \nameref{cor:condition-invertible}, each $a=1,2,\dots,p-1$ has a unique multiplicative inverse modulo $p.$ \nameref{lem:sqrt1} says the only elements that are their own multiplicative inverse are $1$ and $p-1$. Thus $(p-2)!$ is the product of $1$ and $\frac{p-3}{2}$ pairs of $a,a^\prime$ where $a a^\prime\equiv 1\pmod{p}.$ Therefore, 
    \begin{align*}
        (p-2)! & \equiv 1\pmod{p}\\
        (p-1)! & \equiv p-1\equiv -1 \pmod{p}.\qedhere
    \end{align*}
\end{proof}

Wilson's Theorem is normally stated as above, but the converse is also true. It can also be a (very ineffective) prime test.
\begin{prop*}[Proposition 2.12]\label{Wilson-converse}
    Let $n$ be a positive integer. If $(n-1)!\equiv 1\pmod{n},$ then $n$ is prime.
\end{prop*}

\begin{proof}
    Let $a$ and $b$ be positive integers where $ab=n.$ It suffices to show that if $1\leq a < n,$ then $a=1.$ If $a=n,$ then $b=1.$ If  $1\leq a < n,$ then $a\mid (n-1)!$ by the definition of factorial. Then $(n-1)!\equiv -1\pmod{n}$ implies $a\mid (n-1)!+1$ by transitivity of division. Thus, $a\mid (n-1)!+1-(n-1)!=1$ by linear combination and $a=1.$ Therefore, the only positive factors of $n$ are $1$ and $n,$ so $n$ is prime.
\end{proof}


\begin{br}[Part of Strayer, Chapter 2 Exercise 47]
    Let $p$ be an odd prime. Use (a) $\left(\left(\frac{p-1}{2}\right)!\right)\equiv (-1)^{(p+1)/2} \pmod{p}$ to show
    \begin{enumerate}[label=(\alph*)]
        \setcounter{enumi}{1}
        \item If $p\equiv 1\pmod{4},$ then $\left(\left(\frac{p-1}{2}\right)!\right)^2\equiv -1 \pmod{p}$
        
        \item If $p\equiv 3\pmod{4},$ then $\left(\left(\frac{p-1}{2}\right)!\right)^2\equiv 1 \pmod{p}$
    \end{enumerate}

    \begin{solution}

        \begin{enumerate}[label=(\alph*)]
            \setcounter{enumi}{1}
            \item Let $p$ be a prime with $p\equiv 1 \pmod 4.$ Then $p=4k+1$ for some $k\in\Z.$ From part (a), 
            \[\left(\left(\frac{p-1}{2}\right)!\right)\equiv (-1)^{(p+1)/2} \equiv (-1)^{(4k+1+1)/2}\equiv (-1)^{2k+1}\equiv -1 \pmod{p}.\]

            \item Let $p$ be a prime with $p\equiv 3 \pmod 4.$ Then $p=4k+3$ for some $k\in\Z.$ From part (a), 
            \[\left(\left(\frac{p-1}{2}\right)!\right)\equiv (-1)^{(p+1)/2} \equiv (-1)^{(4k+3+1)/2}\equiv (-1)^{2k+2}\equiv 1 \pmod{p}.\]
            
        \end{enumerate}
        
    \end{solution}
\end{br}


\begin{thm*}[On Paper 2, Polynomial Factorization option]
    Let $p$ be a prime number. The congruence $x^2\equiv -1 \pmod p$ has solutions if and only if $p=2$ or $p\equiv 1 \pmod 4.$
\end{thm*}

%%%%%%%%%%%%%%%%%%%%%%%%%%
\section{Friday, February 23: Euler's Theorem and Fermat's Little Theorem}
%%%%%%%%%%%%%%%%%%%%%%%%%%
%%%%%%%%%%%%%%%%%%%%%%%%%%
\begin{obj}
	\item Define and find a reduced residue system modulo $m$
	\item Define the Euler $\phi$-function $\phi(n)$
	\item Prove Euler's Generalization of Fermat's Little Theorem
\end{obj}

\begin{pre} \item[Read] Strayer, Section 2.5

    \item[Turn in]  
    
    Exercise 50. Prove that $9^{10} = 1\pmod{11}$ by following the steps of the proof of Fermat's Little Theorem.
    
    \begin{solution}
        Consider the $10$ integers given by $9,2(9), 3(9),\dots, 9(10).$ Note that $11\mid 9i$ for $i=1,2,\dots,10$ since $11$ is prime and $11\nmid 10$ and $11\nmid i.$ By \nameref{cor:condition-invertible}, since $(9,11)=1$ if $9i\equiv 9j\pmod{11}$ implies $i\equiv j\pmod{11}.$ Therefore, no two of $9,2(9), 3(9),\dots, 9(10)$ are congruent modulo $11.$ So the least nonnegative residues modulo $11$ of the integers $9,2(9), 3(9),\dots, 9(10),$ taken in some order, must be $1,2,\dots, p-1.$ Then \[(9)(2(9)) (3(9))\cdots (9(10))\equiv (1)(2)\cdots (10)\pmod{11}\] or, equivalently, \[9^{10}10!\equiv 10!\pmod{11}.\] By \nameref{Wilson}, the congruence above becomes $-9^{10}\equiv -1\pmod{11},$ which is equivalent to $9^{10}\equiv 1\pmod{11}.$
    \end{solution}
\end{pre}
%%%%%%%%%%%%%%%%%%%%%%%%%%
\subsection{Quiz (10 min)}
%%%%%%%%%%%%%%%%%%%%%%%%%%

%%%%%%%%%%%%%%%%%%%%%%%%%%
\subsection{Euler's Generalization of Fermat's Little Theorem (40 min)}
%%%%%%%%%%%%%%%%%%%%%%%%%%

There are several different ways to present the material in Sections 2.4 through 2.6. In class, we will do the other order: Fermat's Little Theorem to prove Wilson's Theorem. I will keep the result numbering from the book, so they will be out of order.


\begin{defn}[reduced residue system modulo $m$]\label{defn:reduced-res-sys}
    Let $m$ be a positive integer. We say that $\{r_1,r_2,\dots,r_k\}$ is a \emph{reduced residue system modulo $m$} if 
    \begin{itemize}
        \item $(r_i,m)=1$ for all $i=1,2,\dots,k,$
        \item $r_i\not \equiv r_j \pmod {m}$ when $i\neq j,$
        \item for all $a\in\Z$ with $(a,m)=1,$ $a\equiv r_1\pmod{p}$ for some $i=1,2,\dots,k.$ 
    \end{itemize}
\end{defn}

\begin{example}\label{example:reduced-sys}
    
    \begin{itemize}
        \item The sets $\{1,2,3,4,5,6\}$ and $\{5,10,15,20,25,30,35\}$ are both reduced residue systems modulo $7.$
        
        \item If $p$ is prime, then $\{1,2,\dots,p-1\}$ is a complete residue system modulo $p.$ If $p\neq 5,$ $\{5,10,\dots, 5(p-1)\}$ is a complete residue system modulo $p.$
        
        \item The sets $\{1,5,7,11\}$ and $\{5,25,35,55\}$ are both reduced residue systems modulo $12.$
    \end{itemize}
\end{example}


\begin{lem*}[Porism 2.18]\label{lem:reduced-sys}
    Let $m$ be a positive integer and let $\{r_1,r_2,\dots,r_k\}$ be a reduced residue system modulo $m.$ If $a\in\Z$ with $(a,m)=1,$ then $\{ar_1,ar_2,\dots,ar_k\}$ is a reduced residue system modulo $m.$
\end{lem*}
This result is also implicitly used in the proof of \nameref{FlT} since $\{1,2,\dots,p-1\}$ is a reduced residue system.

\begin{proof}
    Let $\{r_1,r_2,\dots,r_k\}$ be a reduced residue system modulo $m$ and $a\in\Z$ with $(a,m)=1.$ Since $\{r_1,r_2,\dots,r_k\}$ and $\{ar_1,ar_2,\dots,ar_k\}$ have the same number of elements, it suffices to show that $(ar_i,m)=1$ and  $ar_i\not\equiv ar_j\pmod{m}$ for $i\neq j.$ If there exist some prime $p$ such that $p\mid(ar_i,m)$ then $p\mid ar_i$ and $p\mid m$ by \autoref{defn:gcd}. By \nameref{lem:irreducible-prime}, $p\mid a$ or $p\mid r_i$, so either $p\mid (a,m)$ or $p\mid (r_i,m).$ which is a contradiction. Thus, $(ar_i,m)=1.$

    By \nameref{prop-equiv-gcd}, $ar_i\equiv ar_j\pmod{m}$ if and only $r_i\equiv r_j \pmod{\tfrac{m}{(a,m)}}.$ Since $(a,m)=1,$ $ar_i\not\equiv ar_j\pmod{m}$ when $i\neq j$.
\end{proof}


\begin{defn}[Euler $\phi$-function]\label{defn:phi-fn}
    Let $n$ be a positive integer. The \emph{Euler $\phi$-function} $\phi(n)$ is \[\phi(n)=\#\{a\in\Z : a>0 \textnormal{ and } (a,m)=1\}.\]
\end{defn}


\begin{remark}
    For a positive integer $m,$ $\phi(m)$ is the number of reduced residues modulo $m$
\end{remark}
\begin{example}\label{example:phi}
    
    \begin{itemize}
        \item $\phi(7)=6$
        
        \item If $p$ is prime, $\phi(p)=p-1$
        
        \item $\phi(12)=4$
    \end{itemize}
\end{example}

\begin{thm*}[Euler's Generalization of \nameref{FlT}]\label{thm:euler-FlT}
    Let $a,m\in\Z$ with $m>0.$ If $(a,m)=1,$ then \[a^{\phi(m)}\equiv 1\pmod{m}.\]
\end{thm*}



\begin{cor*}[Fermat's Little Theorem]\label{FlT}
    Let $p$ be prime and $a\in\Z.$ If $p\nmid a,$ then \[a^{p-1}\equiv 1\pmod{p}.\]
\end{cor*}

\begin{proof}
    Let $p$ be prime and $a\in\Z,$ then $(a,p)=1$ if and only if $p\nmid a.$ Since $\phi(p)=p-1,$ $a^{p-1}\equiv 1\pmod{p}.$
\end{proof}

\begin{warning}
    The converse of both of these theorems is false. The easiest example is $1^k\equiv 1\pmod{m}$ for all positive integers $k, m$. Also note that $2^{341}\equiv 2\pmod{341}$. Since $(2,341)=1,$ there exists an integer $a$ such that $2a\equiv 1\pmod{341}.$ Thus \[a2^{341}\equiv (2a)2^{340}\equiv 2^{340}\equiv 2a\equiv 1\pmod{341}.\]
    However, $341=(11)(31).$
\end{warning}

\begin{proof}[Proof of \nameref{thm:euler-FlT}]
    Let $m$ be a positive integer and let $\{r_1,r_2,\dots,r_{\phi(m)}\}$ be a reduced residue system modulo $m.$ If $a\in\Z$ with $(a,m)=1,$ then $\{ar_1,ar_2,\dots,ar_{\phi(m)}\}$ is a reduced residue system modulo $m$ by \nameref{lem:reduced-sys}. Thus, for all $i=1,2,\dots, \phi(m),$ then $r_i\equiv a r_j\pmod{m}$ for some $j=1,2,\dots,\phi(m).$ Thus \[r_1 r_2\cdots r_{\phi(min)}\equiv ar_1 ar_2\cdots ar_{\phi(min)}\equiv a^{\phi(m)}r_1 r_2\cdots r_{\phi(m)} \pmod{m}.\]

    Since $(r_i,m)=1,$ there exists $x_i\in\Z$ such that $r_i x_i\equiv 1\pmod{m}.$ Thus, 
    \begin{align*}
        r_1 x_1 r_2 x_2\cdots r_{\phi(min)} x_{\phi(m)}&\equiv a^{\phi(m)} r_1 x_1 r_2 x_2\cdots r_{\phi(min)} x_{\phi(m)} \pmod{m}\\
        1\equiv a^{\phi(m)} \pmod{m}.\qedhere
    \end{align*}
\end{proof}

\end{document}