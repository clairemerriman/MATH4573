\documentclass[letterpaper, 11 pt]{article}
<<<<<<< Updated upstream
\usepackage{amssymb, latexsym, amsmath, amsthm, graphicx, amsthm,alltt,color, listings,multicol,xr-hyper,hyperref,aliascnt,enumitem}
=======
\usepackage{amssymb, latexsym, amsmath, amsthm, graphicx, amsthm,alltt,color, listings,multicol,hyperref}
\usepackage[capitalise,nameinlink]{cleveref}
>>>>>>> Stashed changes
\usepackage{xfrac}

\usepackage{parskip}
\usepackage[,margin=0.7in]{geometry}
\setlength{\textheight}{8.5in}

\usepackage{epstopdf}

\DeclareGraphicsExtensions{.eps}
\usepackage{tikz}


\usepackage{tkz-euclide}
%\usetkzobj{all}
\tikzstyle geometryDiagrams=[rounded corners=.5pt,ultra thick,color=black]
\colorlet{penColor}{black} % Color of a curve in a plot


\usepackage{subcaption}
\usepackage{float}
\usepackage{fancyhdr}
\usepackage{pdfpages}
\newcounter{includepdfpage}
\usepackage{makecell}


\usepackage{currfile}
\usepackage{xstring}




\graphicspath{  
{./otherDocuments/}
}

\author{Claire Merriman}
\newcommand{\classday}[1]{\def\classday{#1}}

%%%%%%%%%%%%%%%%%%%%%
% Counters and autoref for unnumbered environments
% Not needed??
%%%%%%%%%%%%%%%%%%%%%
<<<<<<< Updated upstream
\theoremstyle{plain}


\newtheorem*{namedthm}{Theorem}
\newcounter{thm}%makes pointer correct
\providecommand{\thmname}{Theorem}
=======

\crefname{problem}{problem}{problems}


% \theoremstyle{plain}


% \newtheorem*{namedthm}{Theorem}
% \newcounter{thm}%makes pointer correct
% \providecommand{\thmname}{Theorem}
>>>>>>> Stashed changes

\makeatletter
\NewDocumentEnvironment{thm*}{o}
 {%
  \IfValueTF{#1}
    {\namedthm[#1]\refstepcounter{thm}\def\@currentlabel{(#1)}}%
    {\namedthm}%
 }
 {%
  \endnamedthm
 }
\makeatother


\newtheorem*{namedprop}{Proposition}
\newcounter{prop}%makes pointer correct
\providecommand{\propname}{Proposition}

\makeatletter
\NewDocumentEnvironment{prop*}{o}
 {%
  \IfValueTF{#1}
    {\namedprop[#1]\refstepcounter{prop}\def\@currentlabel{(#1)}}%
    {\namedprop}%
 }
 {%
  \endnamedprop
 }
\makeatother

\newtheorem*{namedlem}{Lemma}
\newcounter{lem}%makes pointer correct
\providecommand{\lemname}{Lemma}

\makeatletter
\NewDocumentEnvironment{lem*}{o}
 {%
  \IfValueTF{#1}
    {\namedlem[#1]\refstepcounter{lem}\def\@currentlabel{(#1)}}%
    {\namedlem}%
 }
 {%
  \endnamedlem
 }
\makeatother

\newtheorem*{namedcor}{Corollary}
\newcounter{cor}%makes pointer correct
\providecommand{\corname}{Corollary}

\makeatletter
\NewDocumentEnvironment{cor*}{o}
 {%
  \IfValueTF{#1}
    {\namedcor[#1]\refstepcounter{cor}\def\@currentlabel{(#1)}}%
    {\namedcor}%
 }
 {%
  \endnamedcor
 }
\makeatother

\theoremstyle{definition}
\newtheorem*{annotation}{Annotation}
\newtheorem*{rubric}{Rubric}

\newtheorem*{innerrem}{Remark}
\newcounter{rem}%makes pointer correct
\providecommand{\remname}{Remark}

\makeatletter
\NewDocumentEnvironment{rem}{o}
 {%
  \IfValueTF{#1}
    {\innerrem[#1]\refstepcounter{rem}\def\@currentlabel{(#1)}}%
    {\innerrem}%
 }
 {%
  \endinnerrem
 }
\makeatother

\newtheorem*{innerdefn}{Definition}%%placeholder
\newcounter{defn}%makes pointer correct
\providecommand{\defnname}{Definition}

\makeatletter
\NewDocumentEnvironment{defn}{o}
 {%
  \IfValueTF{#1}
    {\innerdefn[#1]\refstepcounter{defn}\def\@currentlabel{(#1)}}%
    {\innerdefn}%
 }
 {%
  \endinnerdefn
 }
\makeatother

\newtheorem*{scratch}{Scratch Work}


\newtheorem*{namedconj}{Conjecture}
\newcounter{conj}%makes pointer correct
\providecommand{\conjname}{Conjecture}
\makeatletter
\NewDocumentEnvironment{conj}{o}
 {%
  \IfValueTF{#1}
    {\innerconj[#1]\refstepcounter{conj}\def\@currentlabel{(#1)}}%
    {\innerconj}%
 }
 {%
  \endinnerconj
 }
\makeatother

\newtheorem*{poll}{Poll question}
\newtheorem{tps}{Think-Pair-Share}[section]


\newenvironment{obj}{
	\textbf{Learning Objectives.} By the end of class, students will be able to:
		\begin{itemize}}
		{\!.\end{itemize}
		}

<<<<<<< Updated upstream
\newenvironment{pre}{
	\begin{description}
	}{
	\end{description}
}
=======

\ifinstructornotes
\newenvironment{pre}
  {{\textbf Reading assignment:}
  \begin{description}
    }{
	\end{description}
  }
\else
\newenvironment{pre}{ 
  \begin{trivlist}
  \item[]}
  {\end{trivlist}}
\fi
>>>>>>> Stashed changes


\newcounter{ex}%makes pointer correct
\providecommand{\exname}{Homework Problem}
\newenvironment{ex}[1][2in]%
{%Env start code
\problemEnvironmentStart{#1}{Homework Problem}
\refstepcounter{ex}
}
{%Env end code
\problemEnvironmentEnd
}

\newcommand{\inlineAnswer}[2][2 cm]{
    \ifhandout{\pdfOnly{\rule{#1}{0.4pt}}}
    \else{\answer{#2}}
    \fi
}


\ifhandout
\newenvironment{shortAnswer}[1][
    \vfill]
        {% Begin then result
        #1
            \begin{freeResponse}
            }
    {% Environment Ending Code
    \end{freeResponse}
    }
\else
\newenvironment{shortAnswer}[1][]
        {\begin{freeResponse}
            }
    {% Environment Ending Code
    \end{freeResponse}
    }
\fi

\let\question\relax
\let\endquestion\relax

\newtheoremstyle{ExerciseStyle}{\topsep}{\topsep}%%% space between body and thm
		{}                      %%% Thm body font
		{}                              %%% Indent amount (empty = no indent)
		{\bfseries}            %%% Thm head font
		{}                              %%% Punctuation after thm head
		{3em}                           %%% Space after thm head
		{{#1}~\thmnumber{#2}\thmnote{ \bfseries(#3)}}%%% Thm head spec
\theoremstyle{ExerciseStyle}
\newtheorem{br}{In-class Problem}

\newenvironment{sketch}
 {\begin{proof}[Sketch of Proof]}
 {\end{proof}}


\newcommand{\gt}{>}
\newcommand{\lt}{<}
\newcommand{\N}{\mathbb N}
\newcommand{\Q}{\mathbb Q}
\newcommand{\Z}{\mathbb Z}
\newcommand{\C}{\mathbb C}
\newcommand{\R}{\mathbb R}
\renewcommand{\H}{\mathbb{H}}
\newcommand{\lcm}{\operatorname{lcm}}
\newcommand{\nequiv}{\not\equiv}
\newcommand{\ord}{\operatorname{ord}}
\newcommand{\ds}{\displaystyle}
\newcommand{\floor}[1]{\left\lfloor #1\right\rfloor}
\newcommand{\legendre}[2]{\left(\frac{#1}{#2}\right)}



%%%%%%%%%%%%





\title{Week 3--MATH 4573 Elementary Number Theory}

\begin{document}

\maketitle
\tableofcontents

%%%%%%%%%%%%%%%%%%%%%%%%%
\section{Monday, January 25: More primes}
%%%%%%%%%%%%%%%%%%%%%%%%%%
Reading: Section 2.1 Jones \& Jones

{\bf Turn in:} How does this section relate to your previous understanding of primes? 


%%%%%%%%%%%%%%%%%%%%%%%%%
\subsection{Primes (45 minutes)}
%%%%%%%%%%%%%%%%%%%%%%%%%%
\begin{lem}[Lemma 2.1 and Corollary 2.2]
 If $p$ is prime and $a_1,\dots, a_k$ are integers such that $p$ divides $a_1a_2\cdots a_k$, then 
 
\begin{enumerate}
 \item either $p\mid a_i$ or $\gcd(a_i,p)=1$ for each $a_i$,
 \item $p$ divides $a_i$ for some $i$.
\end{enumerate}
\end{lem}
\begin{proof}
 
\begin{enumerate}
 \item By definition $\gcd(a_i,p)$ is a positive divisor of $p$. Thus, it is either $p$ or $1$ since $p$ is prime.
 \item We use induction on $k$. If $k=1,$ then $p\mid a_i$ and we are done.
 
 If $k=2$, then either $p\mid a_1$ and we are done, or $\gcd(a_1,p)=1$ by part (a). Then by Bezout's identity, there exists integers $u,v$ such that $1=a_1u+pv$. Thus, $a_2=a_1a_2u+a_2pv$. By assumption $p\mid a_1a_2$, so $p$ divides the linear combination $a_1a_2u+a_2pv=a_2$.
 
 Now assume that $k>2$ and that the result holds for all products of $k-1$ factors. Then we can write $a=a_1\dots a_{k-1}$ and $b=a_k$ and $p\mid ab$. By the $k=2$ case, it follows that either $p\mid a$ or $p\mid b$. In the first case, $p\mid a=a_1\cdots a_{k-1}$ means that $p\mid a_i$ for some $i$ by the induction hypothesis. In the other case, $p\mid b$ means that $p\mid a_k$ by construction. \qedhere
 \end{enumerate}
\end{proof}

For a polynomial $f(x)=a_n x^n+a_{n-1}x^{n-1}+\cdots+a_1 x+a_0$, where $a_i\in\Z$, whether or not the $a_i$ are divisible by some prime $p$ helps determine if the polynomial can be factored.
\begin{br}[Exercise 2.1, 5 minutes]
 Prove that if $p$ is prime and $p\mid a^k$, then $p\mid a$, and thus $p^k\mid a^k$. 
 
 Is this still valid if $p$ is composite? Prove or provide a counterexample.
\end{br}

\begin{thm} [The Fundamental Theorem of Arithmetic, restatement of Textbook Theorem 2.3] Every integer $n\ge 2$ can be factored into a product of primes $n=p_1 p_2 p_3\dots p_n$ in a unique way (up to rearrangement).\end{thm}
\begin{proof}
We work by induction. We can check that $2=2$, $3=3$ and $4=2^2$ all have prime factorizations. So assume now that all numbers $n$ in the range $2\le n<N$ can be factored in the specified way, and we will show that $N$ can be factored this way as well. If $N$ is prime, then $N$ is its own prime factorization. Otherwise $N$ is composite and has some factor $2\le n_1< N$. Thus we can write $N=n_1n_2$. But $n_2$ must also be in the range $2\le n_2 < N$. Therefore $n_1$ and $n_2$ must be factorable into primes, say $n_1=p_1p_2\dots p_r$ and $n_2= q_1q_2\dots q_s$. Then $N=p_1p_2\dots p _r q_1q_2\dots q_s$. By induction, all integers $n\ge 2$ can be factored into a product of primes.

Now we must show this product is unique up to rearrangement. Suppose that $n$ has two factorizations $p_1p_2\dots p_r$ and $q_1q_2\dots q_s$, and let us assume that $r\le s$. By our previous theorem, since $p_1$ divides $q_1q_2\dots q_s$, it must divide one of the $q_i$'s, and, by rearranging, we can assume that $p_1\mid q_1$. But $q_1$ is prime and only has $1$ and $q_1$ as prime factors, so therefore, $p_1=q_1$. Therefore we have $p_2p_3\dots p_r = q_2q_3\dots q_s$. 

We can repeat this process (and repeatedly rearrange the $q_i$'s as necessary) to show that $p_2=q_2$, and then $p_3=q_3$, and so on until we show that $1=q_{r+1}q_{r+2}\dots q_s$. If $s>r$, then the number on the right-hand side here is greater than $1$, which is impossible, so we have that $r=s$ and $p_i=q_i$ for $1\le i \le r$. This proves that the factorization is unique up to rearrangement.
\end{proof}

So let's suppose we want to factor an integer $n$. Let's start with $d=2$. 

We check to see if $d\mid n$. If it does, then we factor $d$ out of $n$ to get $n'=n/d$. We record the divisor $d$ and henceforth work with $n'$ in place of $n$, restarting at this step. Otherwise, we increment $d$ by one and repeat this step.

Once $n=1$, we have all the factors.


Now there are various ways we could try to speed this up. You might think we could only work with those trial divisors $d$ that are prime, but it would take more time to check if $d$ is a prime than it would to just see if it divides $n$. If it's composite, it won't, because we'll have already taken all its factors out of $n$ beforehand.

One thing we can do is only check $d$ up to $\sqrt{n}$. This is because $n$ cannot have two prime factors that are each $>\sqrt{n}$. So if we reach this point, whatever $n$ remains must be prime. So that takes us down from a worst case scenario of $d=n$ to a worst case scenario of $d=\sqrt{n}$. That's a big improvement. It's still very slow.


\vspace{1cm}

We can  write out a prime factorization (of a positive integer) by
\[
a=\prod_p p^{e_p}
\]
where $e_p$ is the number of times $p$ divides $a$ evenly. Note that $\prod$ denote a product in the same way that $\sum$ denotes a sum, and that the subscript $p$ means that the product ranges over all primes.

This is what we are doing when we write $12=2^2\cdot 3$.

This gives us the textbook's formulation of the Fundamental Theorem of Arithmetic




%%%%%%%%%%%%%%%%%%%%%%%%%%
\subsection{An asside on unique factorization (10 minutes)}
%%%%%%%%%%%%%%%%%%%%%%%%%%
Let $\Z[\sqrt{-6}]=\{a+b\sqrt{-6}: a,b\in\Z\}$.

\begin{defn}
 An element $a+b\sqrt{-6}$ is \emph{irreducible in $\Z[\sqrt{-6}]$} if it cannot be expressed as a product of two elements for $\Z[\sqrt{6}]$ except with the trivial factorizations
 \[a+b\sqrt{-6}=(1)(a+b\sqrt{6})=(-1)(-a-b\sqrt{6}).\]
\end{defn}

We also define the \emph{norm} of a number in $\Z[\sqrt{-6}]$ to be $||a+b\sqrt{-6}||=a^2+6b^2$. The product of the norm is the norm of the product. That is $||xy||=||x|||y||$ for $x,y\in\Z[\sqrt{-6}].$ Why?

\begin{example} Examples from Zoom poll questions: 

Question 1: $2=2+0\sqrt{-6}, ||2||=2^2+6(0^2)=4$, Question 2: $5=5+0\sqrt{-6}, ||5||=5^2+6(0^2)=25$, and Question 3: $2+\sqrt{-6}=2^2+6(1^2)=10$.
\end{example}

Think about: How would the fundamental theorem of arithmetic (Theorem 1.16) change if we said 1 is a prime number?

%%%%%%%%%%%%%%%%%%%%%%%%%%
\section{Wednesday, January 27: More distribution of primes}
%%%%%%%%%%%%%%%%%%%%%%%%%%
Reading: Section 2.2, distribution of primes

{\bf Turn in}: After Theorem 2.9, the book says ``There are also infinitely many primes of the form $4q+1$; however the proof is a little more subtle\dots Where does the method of the proof of Theorem 2.9 break down in this case?"

Answer this question.

%%%%%%%%%%%%%%%%%%%%%%%%%%
\subsection{Review of reading and announcements (5 minutes)}
%%%%%%%%%%%%%%%%%%%%%%%%%%
%%%%%%%%%%%%%%%%%%%%%%%%%%
\subsection{Finishing Unique factorizations (10 minutes)}
%%%%%%%%%%%%%%%%%%%%%%%%%%

\begin{br}[10 minutes]
Consider $\Z[\sqrt{-6}]$. 
\begin{enumerate}
 	\item Prove that $2$ and $5$ are irreducible elements of $\Z[\sqrt{-6}].$
	\item Prove that $2+\sqrt{-6}$ and $2-\sqrt{-6}$ are irreducible elements of $\Z[\sqrt{-6}].$
	\item Prove that there are elements of $\Z[\sqrt{-6}]$ with two different factorizations into irreducible elements, and thus $\Z[\sqrt{-6}]$ does not have unique factorizations.
\end{enumerate}
\end{br}

%%%%%%%%%%%%%%%%%%%%%%%%%
\subsection{Prime-power factorization (20 minutes)}
%%%%%%%%%%%%%%%%%%%%%%%%%%

Supposed that integers $a$ and $b$ have factorizations
\[a=p_1^{e_1}\dots p_k^{e_k}\quad\textrm{and}\quad b=p_1^{f_1}\dots p_k^{f_k}\]
for primes $p_i$ and integers $e_i,f_i\geq0$. Then we have the following formulas:

\begin{br}[5 minutes]
\begin{align*}
ab&=p_1^{e_1+f_1}\cdots p_k^{e_k+f_k}\\
\frac{a}{b}&=p_1^{e_1-f_1}\cdots p_k^{e_k-f_k}\\ 
a^m&=p_1^{me_1}\cdots p_k^{me_k}\\
\gcd(a,b)&=p_1^{\min(e_1,f_1)}\cdots p_k^{\min(e_k,f_k)}\\
\lcm(a,b)&=p_1^{\max(e_1,f_1)}\cdots p_k^{\max(e_k,f_k)}
\end{align*}
 Prove each of these formulas.
\end{br}

For a prime $p$, we use the notation $p^e||n$ to indicate that $e$ is the highest power of $p$ that divides $n$.

\begin{lem}[Lemma 2.4]
If $a_1,\dots a_r$ are mutually coprime integers and $a_1\cdots a_r$ is an $m^{th}$ power (ie, $a_1\cdots a-r=b^m$ for some integer $b$) for some integer $m\geq 2,$ then each $a_i$ is an $m^{th}$ power.
\end{lem}
\begin{proof}
 It follows from the formula for $a^m$ that a positive integer is an $m^{th}$ power if and only if the exponent of each prime in its prime-power factorization is divisible by $m$ (recall, equals signs are if and only if statements). If $a=a_1\cdots a_r$, where the factors $a_i$ are mutually coprime (ie, pairwise relatively prime), then each prime that divides $a$ divides exactly one of the $a_i$. Thus, $p^{me}$ appears in the prime factorization of $a$, if and only it also appears in the prime factorization of exactly one of the $a_i$. Since is is true for all prime power divisors of $a$, each $a_i$ is also an $m^{th}$ power.
\end{proof}

The assumption that the $a_i$ are pairwise relatively prime is important! The integers $4, 6, 9$ are relatively prime as a triple, but not in pairs since $\gcd(4,6)=2, \gcd(6,9)=3$. Both $4$ and $9$ are perfect squares, but $6$ is not. However $4*6*9=216=6^3$.

%%%%%%%%%%%%%%%%%%%%%%%%%
\subsection{Counting primes (25 minutes, some moved to Friday)}
%%%%%%%%%%%%%%%%%%%%%%%%%%
The reading serves as an introduction to counting the \emph{distribution} or \emph{density} of primes. Much of \emph{analytic number} theory is dedicated to proving these types of results. We will only go over a few in detail.

\begin{thm}[Euclid's infinitude of primes, Theorem 2.6]
There are infinitely many primes 
\end{thm}
\begin{proof}
 We proceed by contradiction. Assume that there are finitely many prime numbers, $p_1,p_2,\dots, p_k$. Consider \[m=p_1p_2\cdots p_k+1.\]
Since $m$ is an integer greater than $1$, the Fundamental Theorem of Arithmetic (Theorem 2.3) implies that there exists some $p$ prime such that $p\mid m$. By our assumption, $p$ must be in the list $p_1,p_2,\dots, p_k$. Then $p\mid p_1p_2\cdots p_k$ and $p\mid m$. Thus, $p\mid m-p_1p_2\cdots p_k=1$. However, the only positive divisor of $1$ is $1$. Thus, we have reached a contradiction and there must be infinitely many primes.
\end{proof}

\begin{thm}
 There are infinitely many primes of the form $4q+3$.
\end{thm}

Reviewing this proof was part of the reading assignment. Questions on that proof?



%%%%%%%%%%%%%%%%%%%%%%%%%
\section{Friday, January 29: Finishing up primes}
%%%%%%%%%%%%%%%%%%%%%%%%%%
Reading: Read the three attached introductions to modular arithmetic: Jones \& Jones pages 37-40, Granville page 61-62, and Strayer pages 39-41.

In class, we will discuss editing and giving feedback on writing. In addition to analyzing the different writing styles, we will create rubrics for giving feedback. We will create ``sentence stems" that can be used for giving feedback such as ``This sentence helps me understand, because\dots"

{\bf Turn in}: Give one potential sentence stem for giving feedback on mathematical writing.

Pick which reading you like the most and which you like the least and answer the following questions about each of them:
\begin{enumerate}
 \item Why did you pick those two readings? What makes you like or dislike them?
 \item For each reading, identify something that the author does that helps you understand the material.
 \item For each reading, identify something that the author does that is confusing.
 \end{enumerate}


%%%%%%%%%%%%%%%%%%%%%%%%%
\subsection{Announcements (5 minutes)}
%%%%%%%%%%%%%%%%%%%%%%%%%%
Hints to Homework 2, Problem 1b are linked from the \emph{Divisibility} discussion board.

How to read Gradescope score: \url{https://help.gradescope.com/article/axm932lptr-student-view-submission}

%%%%%%%%%%%%%%%%%%%%%%%%%
\subsection{Counting primes (20 minutes)}
%%%%%%%%%%%%%%%%%%%%%%%%%%
\begin{br}[Exercise 2.6, 5 minutes]
Prove that every prime $p\neq3$ has the form $3n+1$ or $3n+2$ for some integer $n$. Prove that there are infinitely many primes of the form $3n+2$. 
\end{br}
\begin{hint}
 The 
\end{hint}


\begin{thm}[Reformulation of Exercise 2.7]
 There are arbitrarily large gaps in the primes. In other words, for any positive integer $k$, there exist $k$ consecutive composite integers.
\end{thm}
\begin{br}[Exercise 2.7, 5 minutes]
Find five consecutive composite integers. Show that for each integer $k\geq 1$, there is a sequence of $k$ consecutive composite integers.
\end{br}

Now, what about how many primes there are? We let $\pi(x)$ denote the number of primes up to $x$. So $\pi(2)=1$, $\pi(3)=2$, $\pi(4)=2$ and so on.

\begin{thm}[Prime Number Theorem]
\[\lim_{n\to\infty}\frac{\pi(x)}{x/\log x}=1\] 
\end{thm}

\begin{conj}[Twin Prime Conjecture]
 There are infinitely many prime numbers $p$ for which $p+2$ is also a prime number. 
\end{conj}

%%%%%%%%%%%%%%%%%%%%%%%%%
\subsection{Creating feedback rubrics (20 minutes)}
%%%%%%%%%%%%%%%%%%%%%%%%%%

Now we will look at the reading assignments and creating feedback rubrics. 

In your breakout rooms, answer:
\begin{br}
 What was one thing that you found helpful in one of the readings?
What is one piece of advice you would like to give one of the authors?
\end{br}


An example of a graded student homework is on the January 29 Participation assignment.




\end{document}




