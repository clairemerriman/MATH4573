\documentclass[letterpaper, 11 pt]{article}
\usepackage{amssymb, latexsym, amsmath, amsthm, graphicx, amsthm,alltt,color, listings,multicol,xr-hyper,hyperref,aliascnt,enumitem}
\usepackage{xfrac}

\usepackage{parskip}
\usepackage[,margin=0.7in]{geometry}
\setlength{\textheight}{8.5in}

\usepackage{epstopdf}

\DeclareGraphicsExtensions{.eps}
\usepackage{tikz}


\usepackage{tkz-euclide}
%\usetkzobj{all}
\tikzstyle geometryDiagrams=[rounded corners=.5pt,ultra thick,color=black]
\colorlet{penColor}{black} % Color of a curve in a plot


\usepackage{subcaption}
\usepackage{float}
\usepackage{fancyhdr}
\usepackage{pdfpages}
\newcounter{includepdfpage}
\usepackage{makecell}


\usepackage{currfile}
\usepackage{xstring}




\graphicspath{  
{./otherDocuments/}
}

\author{Claire Merriman}
\newcommand{\classday}[1]{\def\classday{#1}}

%%%%%%%%%%%%%%%%%%%%%
% Counters and autoref for unnumbered environments
% Not needed??
%%%%%%%%%%%%%%%%%%%%%
\theoremstyle{plain}


\newtheorem*{namedthm}{Theorem}
\newcounter{thm}%makes pointer correct
\providecommand{\thmname}{Theorem}

\makeatletter
\NewDocumentEnvironment{thm*}{o}
 {%
  \IfValueTF{#1}
    {\namedthm[#1]\refstepcounter{thm}\def\@currentlabel{(#1)}}%
    {\namedthm}%
 }
 {%
  \endnamedthm
 }
\makeatother


\newtheorem*{namedprop}{Proposition}
\newcounter{prop}%makes pointer correct
\providecommand{\propname}{Proposition}

\makeatletter
\NewDocumentEnvironment{prop*}{o}
 {%
  \IfValueTF{#1}
    {\namedprop[#1]\refstepcounter{prop}\def\@currentlabel{(#1)}}%
    {\namedprop}%
 }
 {%
  \endnamedprop
 }
\makeatother

\newtheorem*{namedlem}{Lemma}
\newcounter{lem}%makes pointer correct
\providecommand{\lemname}{Lemma}

\makeatletter
\NewDocumentEnvironment{lem*}{o}
 {%
  \IfValueTF{#1}
    {\namedlem[#1]\refstepcounter{lem}\def\@currentlabel{(#1)}}%
    {\namedlem}%
 }
 {%
  \endnamedlem
 }
\makeatother

\newtheorem*{namedcor}{Corollary}
\newcounter{cor}%makes pointer correct
\providecommand{\corname}{Corollary}

\makeatletter
\NewDocumentEnvironment{cor*}{o}
 {%
  \IfValueTF{#1}
    {\namedcor[#1]\refstepcounter{cor}\def\@currentlabel{(#1)}}%
    {\namedcor}%
 }
 {%
  \endnamedcor
 }
\makeatother

\theoremstyle{definition}
\newtheorem*{annotation}{Annotation}
\newtheorem*{rubric}{Rubric}

\newtheorem*{innerrem}{Remark}
\newcounter{rem}%makes pointer correct
\providecommand{\remname}{Remark}

\makeatletter
\NewDocumentEnvironment{rem}{o}
 {%
  \IfValueTF{#1}
    {\innerrem[#1]\refstepcounter{rem}\def\@currentlabel{(#1)}}%
    {\innerrem}%
 }
 {%
  \endinnerrem
 }
\makeatother

\newtheorem*{innerdefn}{Definition}%%placeholder
\newcounter{defn}%makes pointer correct
\providecommand{\defnname}{Definition}

\makeatletter
\NewDocumentEnvironment{defn}{o}
 {%
  \IfValueTF{#1}
    {\innerdefn[#1]\refstepcounter{defn}\def\@currentlabel{(#1)}}%
    {\innerdefn}%
 }
 {%
  \endinnerdefn
 }
\makeatother

\newtheorem*{scratch}{Scratch Work}


\newtheorem*{namedconj}{Conjecture}
\newcounter{conj}%makes pointer correct
\providecommand{\conjname}{Conjecture}
\makeatletter
\NewDocumentEnvironment{conj}{o}
 {%
  \IfValueTF{#1}
    {\innerconj[#1]\refstepcounter{conj}\def\@currentlabel{(#1)}}%
    {\innerconj}%
 }
 {%
  \endinnerconj
 }
\makeatother

\newtheorem*{poll}{Poll question}
\newtheorem{tps}{Think-Pair-Share}[section]


\newenvironment{obj}{
	\textbf{Learning Objectives.} By the end of class, students will be able to:
		\begin{itemize}}
		{\!.\end{itemize}
		}

\newenvironment{pre}{
	\begin{description}
	}{
	\end{description}
}


\newcounter{ex}%makes pointer correct
\providecommand{\exname}{Homework Problem}
\newenvironment{ex}[1][2in]%
{%Env start code
\problemEnvironmentStart{#1}{Homework Problem}
\refstepcounter{ex}
}
{%Env end code
\problemEnvironmentEnd
}

\newcommand{\inlineAnswer}[2][2 cm]{
    \ifhandout{\pdfOnly{\rule{#1}{0.4pt}}}
    \else{\answer{#2}}
    \fi
}


\ifhandout
\newenvironment{shortAnswer}[1][
    \vfill]
        {% Begin then result
        #1
            \begin{freeResponse}
            }
    {% Environment Ending Code
    \end{freeResponse}
    }
\else
\newenvironment{shortAnswer}[1][]
        {\begin{freeResponse}
            }
    {% Environment Ending Code
    \end{freeResponse}
    }
\fi

\let\question\relax
\let\endquestion\relax

\newtheoremstyle{ExerciseStyle}{\topsep}{\topsep}%%% space between body and thm
		{}                      %%% Thm body font
		{}                              %%% Indent amount (empty = no indent)
		{\bfseries}            %%% Thm head font
		{}                              %%% Punctuation after thm head
		{3em}                           %%% Space after thm head
		{{#1}~\thmnumber{#2}\thmnote{ \bfseries(#3)}}%%% Thm head spec
\theoremstyle{ExerciseStyle}
\newtheorem{br}{In-class Problem}

\newenvironment{sketch}
 {\begin{proof}[Sketch of Proof]}
 {\end{proof}}


\newcommand{\gt}{>}
\newcommand{\lt}{<}
\newcommand{\N}{\mathbb N}
\newcommand{\Q}{\mathbb Q}
\newcommand{\Z}{\mathbb Z}
\newcommand{\C}{\mathbb C}
\newcommand{\R}{\mathbb R}
\renewcommand{\H}{\mathbb{H}}
\newcommand{\lcm}{\operatorname{lcm}}
\newcommand{\nequiv}{\not\equiv}
\newcommand{\ord}{\operatorname{ord}}
\newcommand{\ds}{\displaystyle}
\newcommand{\floor}[1]{\left\lfloor #1\right\rfloor}
\newcommand{\legendre}[2]{\left(\frac{#1}{#2}\right)}



%%%%%%%%%%%%




\title{Week 4--MATH 4573 Elementary Number Theory}

\begin{document}

\maketitle
\tableofcontents

%%%%%%%%%%%%%%%%%%%%%%%%%
\section{Monday, February 1: Modular arithmetic}
%%%%%%%%%%%%%%%%%%%%%%%%%%
Reading: Section 3.1 through the paragraph after Exercise 3.3 (page 43) of Jones and Jones, Modular arithmetic

{\bf Turn in}: Exercise 3.2, Without using a calculator, find: 
\begin{enumerate}[(a)]
 	\item the least non-negative residue of $34*17 \pmod{29}$; 
	\item the least absolute residue of $19*14 \pmod{23}$;
	\item the remainder when 510 is divided by 19;
	\item the final decimal digit of $1!+2!+3!+...+10!$
\end{enumerate}
%%%%%%%%%%%%%%%%%%%%%%%%%
\subsection{Primality testing and factorization (10 minutes)}
%%%%%%%%%%%%%%%%%%%%%%%%%%
We will spend more time on divisibility testing when we talk about cryptography. 

\begin{lem}[Lemma 2.14]
 An integer $n>1$ is composite if and only if it is divisible by some prime $p\leq\sqrt{n}$.
\end{lem}
\begin{proof} In one direction, assume that $p\mid n$ for some prime $p\leq \sqrt{n}$.
Then since $1<p\leq \sqrt{n}<n$, $n$ must be composite by definition.

In the other direction, assume that $n$ is composite. Then $n=ab$ for some positive integers $1<a,b<n$. If both $a$ and $b$ are greater than $\sqrt{n}$, then $ab>n$. Thus, at least one of $a$ or $b$ is less than the $\sqrt{n}$. Without loss of generality, say that $a\leq \sqrt{n}$. By the Fundamental Theorem of Arithmetic, there exists a prime $p$ that divides $a$. Thus, $p\leq \sqrt{n}$.
\end{proof}

%%%%%%%%%%%%%%%%%%%%%%%%%
\subsection{Review of Reading (5 minutes)}
%%%%%%%%%%%%%%%%%%%%%%%%%%

Congruences generalize the concept of evenness and oddness. We typically think of even and odd as ``divisible by 2'' and ``not divisible by 2'', but a more useful interpretation is even means ``there is no remainder when divided by 2'' and ``there is a remainder of 1 when divided by 2''. This interpretation gives us more flexibility when replacing $2$ by larger numbers. Instead of ``divisible'' or ``not divisible'' we have several gradations.


For example, we have the sets ``remainder of 0 when divided by 3," ``remainder of 1 when divided by 3," and ``remainder of 2 when divided by 3."

%%%%%%%%%%%%%%%%%%%%%%%%%
\subsection{Modular arithmetic (40 minutes)}
%%%%%%%%%%%%%%%%%%%%%%%%%%


\begin{defn} We say that $a$ is congruent to $b$ modulo $m$ and write $a \equiv b \pmod{m}$ if $a$ and $b$ have the same remainder when divided by $m$. We generally apply this for $a,b\in \mathbb{Z}$ and an  integer $m \geq 2$. 

If $a$ and $b$ are not equivalent $\!\pmod{m}$, we write $a\nequiv b\pmod{m}$.
\end{defn}

\begin{poll}
 Select all numbers that are congruent to:
 
Question 1: $5 \pmod 2$
 
\begin{itemize}
\item Answer 1: -1
\item Answer 2: 3
\item Answer 3: 6
\item Answer 4: -10
\end{itemize}
Question 2: $10\pmod{4}$
 \begin{itemize}
\item Answer 1: -6
\item Answer 2: -2
\item Answer 3: 0
\item Answer 4: 5
\end{itemize}

Question 3: $100 \pmod 9$
\begin{itemize}
\item Answer 1: -8
\item Answer 2: -17
\item Answer 3: 1
\item Answer 4: 10
\end{itemize}
\end{poll}

\begin{solution}
 Options and answers posted after class.
 
 Question 1: $-1\equiv 3\equiv 5 \pmod 2$
 Question 2: $-6\equiv-2\equiv 10 \pmod 4$
 Question 3: $-8\equiv -17\equiv 1\equiv 10 \pmod 9$
\end{solution}

Be careful with this idea and negative values. Make sure you understand why $-2\equiv 1\pmod{3}$ or $-10\equiv 4\pmod{7}$.

Sometimes it will be helpful to think of congruences one way, sometimes the other. The latter is helpful for understanding why $-2$, $1$, $4$, and $7$ are all equivalent modulo $3$.

While this is a useful starting point for understanding congruences, it's only a starting point. Here's are two alternate definitions: 
\begin{lem}\label{moddefs}[Lemma 3.1 expanded]
\begin{enumerate}
\item $a$ and $b$ are equivalent modulo $m$ if and only if $m$ divides $a-b$.
\item $a$ and $b$ are equivalent modulo $m$ if and only if they differ by a multiple of $m$.
\end{enumerate}
\end{lem}
\begin{br}[Lemma 3.1 expanded, 5 minutes]
 Prove that 
\begin{enumerate}
	\item $a$ and $b$ are equivalent modulo $m$ if and only if $m$ divides $a-b$.
	\item $a$ and $b$ are equivalent modulo $m$ if and only if they differ by a multiple of $m$.
\end{enumerate}
\end{br}
This is why it is helpful to know that every integer $m$ divides $0$.

We can use the idea of congruences to simplify some of the arguments from section 2.5. First, let's prove Lemma 3.3

\begin{br}[Lemma 3.3 expanded, 5 minutes]
 Let $a,b,c,d$ denote integers, then:
\begin{enumerate}[(a)]
\item $a\equiv b \pmod{m}$ and $b\equiv c \pmod{m}$ implies $a\equiv c \pmod{m}$
\item\label{add} $a\equiv b \pmod{m}$ and $c\equiv d \pmod{m}$ implies $a+c \equiv b+d \pmod{m}$ 
\item\label{multiply} $a\equiv b\pmod{m}$ and $c\equiv d \pmod{m}$ implies $ac\equiv bd \pmod{m}$.
\item $a\equiv b \pmod{m}$ and $d\mid m$, $d>0$ implies $a\equiv b \pmod{d}$
\item $a\equiv b \pmod{m}$ implies $ac\equiv bc \pmod{mc}$ for $c>0$.
\end{enumerate}
Parts \eqref{add}, \eqref{multiply} are proved in the book.
\end{br}

Let's look at the various parts of this expanded Lemma 3.3 and see what they tell us about how congruences and moduli work.

The first two parts tell us that congruences behave a lot like equality. 

The last two parts tell us how we can manipulate the modulus $m$. We can replace the modulus with a divisor of it at any time. We can replace the modulus with a multiple of it, provided we multiply $a$ and $b$ by the same multiple.

The remaining two parts are perhaps the most useful: they let us do arithmetic with congruences. In particular, they imply that whenever I'm adding or multiplying numbers, I can replace the numbers I have with any equivalent number that is more convenient to use. So for example, $37*210\equiv 1*0 \pmod{2}$ or $651262*697016 \equiv 2*6 \pmod{10}$.

Note that this doesn't work for powers. It is not the case that $2^{20} \equiv 2^0 \pmod{2}$. Also, unlike for regular equality, we cannot cancel a common factor of both sides unless it is relatively prime to $m$. That is to say $ac \equiv bc \pmod{m}$ implies $a\equiv b \pmod{m}$ if $gcd(c,m)=1$. (If $gcd(c,m)>1$ we'll talk more on this later.)



%%%%%%%%%%%%%%%%%%%%%%%%%
\section{Wednesday, February 3: Modular arithmetic}
%%%%%%%%%%%%%%%%%%%%%%%%%%
Reading: The rest of Section 3.1 of Jones and Jones, Modular arithmetic

{\bf Turn in}: Part of Exercise 3.4,
Prove that the following polynomials have no integer roots:
\begin{enumerate}[(a)]
\item $x^3-x+1$
\end{enumerate}
%%%%%%%%%%%%%%%%%%%%%%%%%
\subsection{Student questions (5 minutes)}
%%%%%%%%%%%%%%%%%%%%%%%%%%
Question of how to prove that $5$ is irreducible in $\Z[\sqrt{-6}]$.

Notes from class linked from the Participation assignment.

%%%%%%%%%%%%%%%%%%%%%%%%%
\subsection{Finish material from Monday (30 minutes)}
%%%%%%%%%%%%%%%%%%%%%%%%%%
\begin{prop} A natural number is divisible by $3$ if and only if the sum of its (base $10$) digits is divisible by $3$.
\end{prop}
\begin{proof} Let a  number $n\in \mathbb{N}$ have base $10$ expansion $a_ka_{k-1}a_{k-2}\dots a_1a_0$ so that $n=\sum_{i\le k} a_i 10^i$.  By the definition of congruences, we have that $3\mid n$ if and only if $n\equiv 0 \pmod{3}$. Using the fact that $10\equiv 1 \pmod{3}$, we know by Lemma 3.3 that any time we multiply by $10$ (modulo $3$) we could instead just multiply by $1$ (modulo $3$). So,
\[
n \equiv \sum_{i\le k} a_i 10^i \equiv \sum_{i\le k} a_i 1^i \equiv \sum_{i\le k} a_i \pmod{3}.
\]
Therefore $n\equiv 0\pmod{3}$ if and only if the sum of the digits of $n$ is $0\pmod{3}$. 
\end{proof}

\begin{br}[5 minutes]
 List all integers in the range 1 to 100 that are congruent to 7 mod 17. 
\end{br}

Go over this as a class. 


\begin{br}[5 minutes, with previous problem]
 Find all positive integers $m$ for which the following statements are true:
\begin{enumerate}
 \item $13\equiv 5 \pmod m$
 \item $10\equiv 9 \pmod m$
 \item $-7\equiv 6 \pmod m$.
\end{enumerate}
\end{br}

Why study congruence relations? Why not stick to good old integers all the time? 

There's two major reasons. As we've already seen, calculations get simplified for modular arithmetic. We'll see later that taking MASSIVE powers of MASSIVE numbers is a fairly doable feat in modular arithmetic. The second reason is that by reducing a question from all integers to modular arithmetic, we often have an easier time seeing what is not allowed. 

Here's an example of this second phenomenon. Consider $a^2+b^2=c^2$, the Pythagorean relation. Suppose we have a solution to this. Since equality holds, then the weaker fact of congruence modulo 3 must also hold. So $a^2+b^2\equiv c^2\pmod{3}$. Because of what we said before about multiplication, we can reduce all the values here to either $0$, $1$, or $2$. But $0^2\equiv 0 \pmod{3}$, $1^2\equiv 1 \pmod{3}$, and $2^2\equiv 4 \equiv 1 \pmod{3}$.  Since it is impossible to have $c^2\equiv 2 \pmod{3}$, we cannot have that $a^2+b^2\equiv 2 \pmod{3}$. Thus at least one of $a$ or $b$ must be $0\pmod{3}$. We have learned something about what the possible solutions to $a^2+b^2=c^2$ look like by studying a congruence.

%%%%%%%%%%%%%%%%%%%%%%%%%%
\subsection{More on basics of modular arithmetic (20 minutes)}
%%%%%%%%%%%%%%%%%%%%%%%%%%
The point of the next theorem is to say that DIVISION is different and we must be careful with it. Here's the special rule.

\begin{thm}
\begin{enumerate}
\item $ax\equiv ay \pmod{m}$ if and only if $x\equiv y \pmod{\frac{m}{\gcd(a,m)}}$.
\item If $ax \equiv ay \pmod{m}$ and $(a,m)=1$ then $x\equiv y \pmod{m}$.
\item $x\equiv y \pmod{m_i}$ for $i=1,2,\dots, r$ if and only if $x\equiv y \pmod{[m_1,m_2,\dots,m_r]}$.
\end{enumerate}
\end{thm}

This second part gives an idea of why prime values of $m$ are so helpful!


The proof of the lemma is left for homework.

I will prove the first part for $a=6, m=8$, and the third part for $i=2$. The general proof is left for homework.
\begin{proof} 
\begin{enumerate}
 \item (Proof for $a=6, m=8$, and $\gcd(a,m)=2$) If $6x\equiv 6y \pmod 8$, then $6y-6x=8z$ for some integer $z$. Thus, 
 
 \[
\frac{6}{2}(y-x) = \frac{8}{2}z,
\]
and so $\frac{8}{2}=4$ divides $3 (y-x)$.  Rewriting Corollary 10, we get $\gcd(6,8)=2*\gcd(3,4)$.
 Therefore, the only way $4$ divides $3 (y-x)$ is if it divides $y-x$. Then by definition $x\equiv y \pmod{4}$.

In the direction, suppose $x\equiv y \pmod{\frac{8}{2}}$. Then $4\mid (x-y)$. Multiplying through by $\gcd(6,8)=2$ gives $8\mid 2x-2y$. Since $\frac{6}{2}\in\Z$, we have $8\mid 3(2x-2y)$. Thus, $6x\equiv 6y \pmod 8$.

\item Left for homework

\item (Proof for $i=2$). If $x\equiv y \pmod{m_1}$ and $x\equiv y \pmod{m_2}$, then $m_i\mid (y-x)$ for $i=1,2$. That is, $y-x$ is a common multiple of $m_1$ and $m_2$. Therefore, by Exercise 1.14, $[m_1,m_2]\mid (y-x)$. This implies $x\equiv y \pmod{[m_1,m_2]}$

In the other direction, if  $x\equiv y \pmod{[m_1,m_2]}$, then $x\equiv y \pmod{m_i}$ by expanded Lemma 3.3 part 4, since $m_i\mid [m_1,m_2]$. \qedhere
\end{enumerate}
\end{proof}

%%%%%%%%%%%%%%%%%%%%%%%%%
\section{Friday, February 5: Modular arithmetic with polynomials}
%%%%%%%%%%%%%%%%%%%%%%%%%%
Reading: No reading. Projects due.

%%%%%%%%%%%%%%%%%%%%%%%%%
\subsection{Questions from students (5 minutes)}
%%%%%%%%%%%%%%%%%%%%%%%%%%
%%%%%%%%%%%%%%%%%%%%%%%%%%
\subsection{Modular arithmetic with polynomials (10 minutes)}
%%%%%%%%%%%%%%%%%%%%%%%%%%
We are going to look at polynomials, we will look at some very basic results now and then go into specific cases.

\begin{lem}[Lemma 3.5]
 Let $f$ denote a polynomial with integer coefficients. If $a\equiv b \pmod m$ then $f(a)=f(b) \pmod m$.
\end{lem}

\begin{example}
 Let $f(x)=x^2+1$. Then if $a\equiv b\pmod 5$, then $f(a)\equiv f(b) \pmod 5$. That is, $a^2+1\equiv b^2 +1 \pmod 5$. If $a=3, b=8$, then this means $3^2+1\equiv 10\equiv 65 \equiv 8^2+1\pmod 5$.
\end{example}

\begin{proof}
 We can write $f(x)=c_nx^n+c_{n-1}x^{n-1}+\cdots+c_1x+c_0$ for some integers $c_i$. Then by Theorem 2.1 part 4, we have that $a\equiv b\pmod m$ implies $a^2\equiv b^2\pmod m, a^3\equiv b^3 \pmod m, \dots, a^n\equiv b^n\pmod m$. Also by Theorem 2.1 part 4, we have $c_i a^i\equiv c_i b^i \pmod m$. Finally, putting everything together with Theorem 2.1 part 3, we have that $c_na^n+c_{n-1}a^{n-1}+\cdots+c_1a+c_0\equiv c_nb^n+c_{n-1}b^{n-1}+\cdots+c_1b+c_0 \pmod m$.
\end{proof}

\subsection{Residue classes, lots of definitions (40 minutes)}

\begin{defn}
  If $x\equiv y \pmod{m}$ then $y$ is called a residue of $x\pmod{m}$. 
  
 A set $x_1,x_2,\dots, x_m$ is called a \emph{complete residue system modulo $m$} if every integer $y$ there exists exactly one $x_j$ such that $y\equiv x_j \pmod{m}$. 
 \end{defn}
 
 Implication: If you have a complete residue system $\pmod m$, then $x_i\not\equiv x_j \pmod m$ if $i\neq j$.



\begin{defn}
 Let $r_1,r_2,\dots, r_m$ denote a complete residue system modulo $m$. The \emph{number of solutions of $f(x)=0 \pmod m$} is the number of $r_i$ such that $f(r_i)=0 \pmod m$. That is, each solution $r_1$ is the representative of a congruence class of solutions, and we only count the number of congruence classes. This agrees with asking for the number of incongruent solutions (why?).
\end{defn}

Answer ``why" as a class.

\begin{defn}
 Let $f(x)=a_nx^n+a_{n-1}x^{n-1}+\cdots+a_1x+a_0$. The \emph{degree of the polynomial $\!\pmod m$} is the largest integer $j$ such that $a_j\not\equiv 0 \pmod m$. If all of the $a_i\equiv 0 \pmod m$, then the degree is 0.
\end{defn}


\begin{defn}
 
 The set of all numbers $x$ satisfying $x\equiv a \pmod{m}$ is the \emph{arithmetic progression} $\dots, a-m , a, a+m, a+2m, \dots$ and this is called a \emph{residue class or congruence class modulo $m$}. 
\end{defn}

\begin{br}[2 minutes]
 Find a complete residue system mod 17 where every number is a multiple of 3.
\end{br}

Discussion as a class: The congruence classes of 17 are: 
\begin{align*}
 \dots, -34, -17, 0, 17, 34, \dots\\
 \dots, -33, -16, 1, 18, 35, \dots\\
 \dots, -32, -15, 2, 19, 36, \dots\\
 \dots, -31, -14, 3, 20, 37, \dots\\
 \dots, -30, -13, 4, 21, 38, \dots\\
 \dots, -29, -12, 5, 22, 39, \dots\\
 \dots, -28, -11, 6, 23, 40, \dots\\
 \dots, -27, -10, 7, 24, 41, \dots\\
 \dots, -26, -9, 8, 25, 42, \dots\\
 \dots, -25, -8, 9, 26, 43, \dots\\
 \dots, -24, -7, 10, 27, 44, \dots\\
 \dots, -23, -6, 11, 28, 45, \dots\\
 \dots, -22, -5, 12, 29, 46, \dots\\
 \dots, -21, -4, 13, 30, 47, \dots\\
 \dots, -20, -3, 14, 31, 48, \dots\\
 \dots, -19, -2, 15, 32, 49, \dots\\
 \dots, -18, -1, 16, 33, 50, \dots\\
\end{align*}

Each infinite list is one congruence class. To get a complete residue system, pick exactly one element from each list.

\begin{defn}
  A \emph{reduced residue system modulo $m$} is a set of integers $r_i$ such that $\gcd(r_i,m)=1$, and for every $x$ coprime to $m$ is congruent modulo $m$ to exactly one member $r_i$ of the set. 
\end{defn}

Now we need to know if this is well defined. How do we know if $\gcd(r,m)=1$ is something that is preserved over a given residue class? By the following result:

\begin{prop} If $b\equiv c \pmod{m}$ then $\gcd(b,m)=\gcd(c,m)$. \end{prop}
\begin{br}[5 minutes] Prove that
if $b\equiv c \pmod{m}$, then $\gcd(b,m)=\gcd(c,m)$. 
\end{br}

Finally, we want to look out for some weird behavior in with modular arithmetic.
\begin{br}[5 minutes]
 Find $a,b,m\in\Z$ such that $ab\equiv 0 \pmod m$ but $a\nequiv0\pmod m$ and $b\nequiv 0 \pmod n$.
\end{br}

Go over as a class.

\end{document}

