\documentclass{ximera}  
\title{Participation assignment for March 23}  
\begin{document}  
\begin{abstract}  
This assignment looks at arithmetic functions in number theory. We have already seen the examples $\phi(n)=\#\{x: 0<x< n, (x,n)=1\}$ and the floor/ greatest integer function $\lfloor x \rfloor=$greatest integer less than $x$. So far, we have a formula for the value of $\phi(n)$ when $n$ is prime or the product of two distinct primes. We will prove a general formula for the $\phi$ function and look at some other functions.
\end{abstract}  
\maketitle  

\section{Euler $\phi$ function: number of relatively prime positive integers less than $n$}
Recall from before break that $\phi(p)=p-1$ if and only if $p$ is prime. We also proved that $\phi(pq)=(p-1)(q-1)$ for primes $p$ and $q$. 

On the homework, you will prove that for relatively prime positive integers $m$ and $n$,  $\phi(mn)=\phi(m)\phi(n)$. Since Ximera can really only handle numerical answers, let's prove this is true for a particular example:

\begin{question}
Let us prove that $\phi(20)=\phi(4)\phi(5)$. First, note that $\phi(4)=\answer{2}$ and $\phi(5)=\answer{4}$, so $\phi(20)=\answer{8}$.
\begin{enumerate}
 \item A number $a$ is relatively prime to $20$ if and only if $a$ is relatively prime to $\answer{4}$ and $\answer{5}$ (first blank should be smaller than second blank for the automatic grading to work, both should be relevant to what we are trying to show). 
 \item  
 We can partition the positive integers less that $20$ into 
 \begin{align*}
& 0\equiv\answer{4}\equiv\answer{8}\equiv\answer{12}\equiv\answer{16}\pmod 4\\
& 1\equiv\answer{5}\equiv\answer{9}\equiv\answer{13}\equiv\answer{17}\pmod 4\\
& 2\equiv\answer{6}\equiv\answer{10}\equiv\answer{14}\equiv\answer{18}\pmod 4\\
& 3\equiv\answer{7}\equiv\answer{11}\equiv\answer{15}\equiv\answer{19}\pmod 4
\end{align*}

For any $b$ in the range $0,1,2,3$, define $s_b$ to be the number of integers $a$ in the range $0,1,2,\dots, 19$ such that $a\equiv b \pmod 4$ and $\gcd(a,20)=1$. Thus, $s_0=\answer{0}, s_1=\answer{4}, s_2=\answer{0}$, and $s_3=\answer{4}$.

We can see that when $(b,4)=1$, $s_b=\phi(\answer{5})$ and when $(b,4)>1$, $s_b=\answer{0}$.

\item $\phi(20)=s_0+s_1+s_2+s_3$. Why? 

All of the positive integers less than or equal to $20$ is in exactly one of the congruence classes above. The $s_i$ count how many integers in each congruence class are relatively prime to $20$. If we add them up, we have counted all positive integers less than or equal to $20$.

\item We have seen that $\phi(20)=s_0+s_1+s_2+s_3$, that when $(b,4)=1$, $s_b=\phi(5)$, and that when $(b,4)>1$, $s_b=0$. Thus, we can say that $\phi(20)=0+\phi(\answer{5})+0+\phi(\answer{5})$. To finish the 	``proof" we show that there are $\phi(\answer{4})$ integers $b$ where $(b,4)=1$. 

There are $\answer{4}$ congruence classes modulo 4. Of these, $\answer{2}=\phi(\answer{4})$ have elements that are relatively prime to $20$. Thus, $\phi(20)=\phi(4)\phi(5)$.
\end{enumerate}

\end{question}

\section{Other common number theory functions}
We are going to look at some other functions that show up in analytic number theory. 
\begin{itemize}
 \item $d(n)$ is the number of positive divisors of $n$. For example, $d(12)=\answer{6}$. We introduce the notation $\displaystyle\sum_{d\mid n}$ as ``the sum over the divisors of $n$,'' called the \emph{divisor sum}. For the normal sum: $\displaystyle\sum_{i=1}^n 1=\answer{n}$. Then, $\displaystyle\sum_{d\mid n}1=\answer{d(n)}$.
 
 \item $\sigma(n)$ is the sum of positive divisors of $n$. For example, $\sigma(12)=\answer{28}$. Then $\displaystyle\sum_{d\mid n}\answer{d}=\sigma(n)$.
 
 \item $\sigma_k(n)$ is sum of the $k^{th}$ powers of positive divisors of $n$. For example, $\sigma_2(12)=1^2+2^2+3^2+4^2+6^2+12^2=210.$ Generally, $\displaystyle\sum_{d\mid n}\answer{d^k}=\sigma_k(n)$.

 \item $\omega(n)$ is the number of distinct prime divisors of $n$. For example, $\omega(12)=\answer{2}$. We can modify the divisor sum to sum over prime divisors of $n$,  $\displaystyle\sum_{p\mid n}$. Then, $\displaystyle\sum_{p\mid n} \answer{1}=\omega(n)$.
 
 \item $\Omega(n)$ is the number of primes dividing $n$ counting multiplicity. For example, $\Omega(12)=\answer{3}$. Then $\displaystyle\sum_{\answer{p^\beta}\mid n} 1=\Omega(n)$.

\end{itemize}

\end{document}