\documentclass{ximera}  
\title{Preclass assignment for March 23}  
\begin{document}  
\begin{abstract}  
This assignment is designed to give you practice using this format.
\end{abstract}  
\maketitle  

This format allows you to answer questions directly in your browser. It will also grade some of the questions immediately. Let's test it out!

\begin{question}
 Did you find this page?
 
\begin{multipleChoice}
 \choice[correct]{Yes}
 \choice{No}
\end{multipleChoice}
\end{question}

We can also do math!
\begin{question}
 Find the smallest, nonnegative solution to $x\equiv 10 \pmod 6$. $x=\answer{4}$.
\end{question}

%Free response questions won't grade automatically, but they will allow you to type math. Here are a few things to know: 
%\begin{itemize}
% \item You begin and end math with \$. For example \verb| $x^5=1$| gives  $x^5=1$.
%\item Here are some symbols and commands that will be helpful:
%\verb|$\equiv$|  for $\equiv$, \verb|$\pm$| for $\pm$,   \verb|$\pmod{10}$| for $\pmod{10}$, \verb|$\left(\frac{1}{5}\right)$| for  $\left(\frac{1}{5}\right)$ and \verb$\lfloor x \rfloor$| for $\lfloor x \rfloor$.
%
%\item More commands are available in the TeXsymbols file \url{https://osu.instructure.com/courses/71111/files?preview=19736910}
%\end{itemize}
%
%\begin{question}  
%Write the following using TeX code in the free response box: The congruence $x^2+1\equiv 0 \pmod{5}$ has solutions $x\equiv 2 \pmod{5}$ and $x\equiv 3 \pmod{5}$. Thus,  $\left(\frac{-1}{5}\right)=1.$
%\begin{freeResponse}
%\begin{verbatim}
%The congruence $x^2+1\equiv 0 \pmod{5}$ has solutions 
%$x\equiv 2 \pmod{5}$ and $x\equiv 3 \pmod{5}$. 
%Thus,  $\left(\frac{-1}{5}\right)=1$.
%\end{verbatim}
%\end{freeResponse}  
%\end{question}
\end{document}