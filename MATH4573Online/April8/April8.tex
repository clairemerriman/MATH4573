\documentclass{ximera}

%\newtheorem{theorem}{Theorem}%[section] % reset theorem numbering for each section
%\newtheorem*{theorem*}{Theorem}%[section] % reset theorem numbering for each section
% \newtheorem{prop}[theorem]{Proposition}
% \newtheorem{lem}[theorem]{Lemma}
% \newtheorem{ex}{Example}


\title{April 8--Sums of squares}  
\begin{document}  
\begin{abstract}  
We finally determine which integers can be written as the sum of two squares of integers!
\end{abstract}
\maketitle  

\subsection{Sum of Two Squares}
Which integers can be represented as the sum of two perfect squares? 

\begin{selectAll}
 \choice[correct] {1}
 \choice[correct] {2}
 \choice {3}
 \choice[correct] {4}
 \choice[correct] {5}
 \choice {6}
 \choice {7}
 \choice[correct] {8}
 \choice[correct] {9}
 \choice[correct] {10}
  \choice {11}
 \choice {12}
 \choice[correct] {13}
 \choice {14}
 \choice {15}
 \end{selectAll}
 \begin{selectAll}
 \choice[correct] {16}
 \choice[correct] {17}
 \choice[correct] {18}
 \choice {19}
 \choice [correct]{20} 
 \choice {21}
 \choice {22}
 \choice {23}
 \choice {24}
 \choice[correct] {25}
 \choice[correct] {26}
 \choice {27}
 \choice {28}
 \choice[correct] {29}
 \choice {30}
\end{selectAll}

Note that this sum is not necessarily unique: $25=5^2+0^2=4^3+3^2$. Try to conjecture whether or not $374^{695}$ can be written as the sum of two squares. We found back in the congruences chapter that $n$ cannot be written written as the sum of two squares if $n\equiv 3 \pmod 4$. In order to establish which integers are expressible as the sum of two squares, we will find necessary and sufficient conditions for the Diophantine equation $x^2+y^2=n$ to have solutions.

\begin{theorem}
 Let $n_1,n_2\in\mathbb{Z}$ with $n_1,n_2>0$. If $n_1$ and $n_2$ are expressible as the sum of two squares of integers, then $n_1n_2$ is expressible as the sum of two squares of integers.
\end{theorem}
\begin{proof}
Participation assignment
 %Let $a,b,c,d\in\mathbb{Z}$ such that $n_1=a^2+b^2$ and $n_2=c^2+d^2$. Then $n_1n_2=(a^2+b^2)(c^2+d^2)=(ac+bd)^2+(ad-bc)^2$.
\end{proof}

\begin{example}
 Since $13=\answer{3}^2+\answer{2}^2$ and $17=\answer{4}^2+\answer{1}^2$ are each expressible as the sum of two squares, $13*17=221=\answer{14}^2+\answer{-5}^2$.
\end{example}

We will finally prove that every prime that congruent to $1 \pmod 4$ is expressible as the sum of two squares.

\begin{theorem}[Primes as sums of squares]
 If $p$ is a prime such that $p\equiv 1 \pmod 4$, then there exists $x,y\in\mathbb{Z}$ such that $x^2+y^2=kp$ for some $k\in\mathbb{Z}$ and $0<k<p$.
\end{theorem}
\begin{proof}
 Since $p\equiv 1 \pmod 4$, we have that $\left(\frac{-1}{p}\right)=1$. Thus, there exists $x\in\mathbb{Z}$ with $0<x\leq\frac{p-1}{2}$ such that $x^2\equiv -1 \pmod p$. Then, $p\mid x^2+1$, and we have that $x^2+1=kp$ for some $k\in\mathbb{Z}$. Thus, we found $x$ and $y=1$. Since $x^2+1$ and $p$ are positive, so is $k$. Also, \[kp=x^2+y^2<\left(\frac{p}{2}\right)^2+1<p^2\] implies $k<p$.
\end{proof}

The next theorem will finally prove that primes $p\equiv 1 \pmod 4$ and $p=2$ can be written as the sum of two square integers.

\begin{theorem}
 If $p$ is a prime number such that $p\not\equiv 3\pmod 4$, then $p$ is expressible as the sum of two squares of integers.
\end{theorem}
\begin{proof}
 When $p=2=1^2+1^2$, we are done.
 
 Assume that $p\equiv 1\pmod 4$. Let $m$ be the least integer such that there exists $x,y\in\mathbb{Z}$ with $x^2+y^2=mp$ and $0<m<p$ as in the previous theorem. We show that $m=1$. Assume, by way of contradiction, that $m>1$. Let $a,b\in\mathbb{Z}$ such that \[a\equiv x\pmod m,\quad \frac{-m}{2}<a\leq\frac{m}{2}\] and \[b\equiv y\pmod m,\quad \frac{-m}{2}<b\leq\frac{m}{2}.\] Then \[a^2+b^2\equiv x^2+y^2=mp\equiv 0\pmod m,\] and so there exists $k\in\mathbb{Z}$ with $k>0$ such that $a^2+b^2=km$. (Why?)
 
 Now, \[(a^2+b^2)(x^2+y^2)=(km)(mp)=km^2p.\] By the participation assignment, $(a^2+b^2)(x^2+y^2)=(ax+by)^2+(ay-bx)^2$, so $(ax+by)^2+(ay-bx)^2=km^2p$. Since $a\equiv x\pmod m$ and $b\equiv y\pmod m$, \[ax+by\equiv x^2+y^2\equiv0\pmod m\] and \[ay-bx\equiv xy-yx\equiv0\pmod m\] so $\frac{ax+by}{m},\frac{ay-bx}{m}\in\mathbb{Z}$ and \[\left(\frac{ax+by}{m}\right)^2+\left(\frac{ay-bx}{m}\right)^2=\frac{km^2p}{m^2}=kp.\]  Now, $\frac{-m}{2}<a\leq\frac{m}{2}$ and $\frac{-m}{2}<b\leq\frac{m}{2}$  imply that $a^2\leq\frac{m^2}{4}$ and $b^2\leq\frac{m^2}{4}$. Thus, $km=a^2+b^2\leq\frac{m^2}{2}$. Thus, $k\leq \frac{m}{2}<m$, but this contradicts that $m$ is the smallest such integer.
 
 Thus, $m=1$.
\end{proof}

We finish with a characterization of which integers are expressible as the sum of two square integers and some examples.

\begin{theorem}
Let $n\in\mathbb{Z}$ with $n>0$. Then $n$ is expressible as the sum of two squares if and only if every prime factor congruent to $3$ modulo  $4$ occurs to an even power in the prime factorization of $n$.
\end{theorem}
\begin{proof}
 ($\Rightarrow$) Assume that $p$ is an odd prime number and that $p^{2i+1},i\in\mathbb{Z}$ occurs in the prime factorization of $n$. We will show that $p\equiv 1 \pmod 4$. Since $n$ is expressible as the sum of two squares of integers, there exist $x,y\in\mathbb{Z}$ such that $n=x^2+y^2.$ Let $(x,y)=d, a=\frac{x}{d},b=\frac{y}{d}$ and  $m=\frac{n}{d^2}$. Then $(a,b)=1$ and $a^2+b^2=m$. Let $p^j,j\in\mathbb{Z}$ be the largest power of $p$ dividing $d$. Then $p^{(2i-1)-2j}\mid m$; since $(2i+1)-2j)\geq 1$, we have $p\mid m$. Now, $p\nmid a$ since $(a,b)=1$. Thus, there exists $z\in\mathbb{Z}$ such that $az\equiv b \pmod p$. Then $m=a^2+b^2\equiv a^2+(az)^2\equiv a^2(1+z^2)\pmod p$.
 
 Since $p\mid m$, we have \[a^2(1+z^2)\equiv 0\pmod p\] or $p\mid a^2(1+z^2)$ or $z\equiv -1 \pmod p$. Thus, $-1$ is a quadratic residue modulo $p$, so $p\equiv 1 \pmod 4$. By contrapositive, any prime factor congruent to $3$ modulo $4$ occurs to an even power in the prime factorization of $n$ as desired.
 
 ($\Leftarrow$) Assume that every prime factor of $n$ congruent to $3$ modulo $4$ occurs to an even power in the prime factorization of $n$. Then $n$ can be written as $n=m^2p_1p_2\dots  p_r$ where $m\in\mathbb{Z}$ and $p_1,p_2,\dots,p_r$ are distinct prime numbers equal to $2$ or equivalent to $1$ modulo $4$. Now, $m^2=m^2+0^2$, so is expressible as the sum of two squares, and each $p_1$ is also expressible as the sum of two squares by the theorem labeled Primes as Sums of Squares. Thus, by the first theorem of the day, $n$ is expressible as the sum of two squares.
\end{proof}

\begin{example}
 Determine whether $374^{695}$ is expressible as the sum of two squares. The prime factorization of $374$ is $2*11*17$. So $374^{695}=2^{695}11^{695}17^{695}$ Thus, $374^{695}$
 
\begin{multipleChoice}
 \choice{is}
 \choice[correct]{is not}
\end{multipleChoice}
expressible as the sum of two squares.
\end{example}

\begin{example}
 Express $4410$ as the sum of two squares by splitting into factors that can be written as the sum of two squares.
 
 The prime factorization of $4410$ is $2*3^2*5*7^2$. We group this into $4410=(\answer{2}*\answer{7}^2)(\answer{3}^2*\answer{5})=\answer{98}*\answer{45}.$ By inspection, the larger of these factors is $\answer{98}=\answer{7}^2+\answer{7}^2$ and the smaller is $\answer{45}=\answer{6}^2+\answer{3}^2$.
 
 The method from the participation assignment gives $4410=\answer{63}^2+\answer{21}^2$.
\end{example}
\end{document}