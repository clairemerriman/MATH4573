\documentclass{ximera}

\theoremstyle{plain}
\newtheorem{thm}{Theorem}%[section] % reset theorem numbering for each section
\newtheorem*{thm*}{Theorem}%[section] % reset theorem numbering for each section
\newtheorem{prop}[thm]{Proposition}
\newtheorem{lem}[thm]{Lemma}
\theoremstyle{definition}

\title{Quadratic reciprocity}  
\begin{document}  
\begin{abstract}  
Introducing quadratic reciprocity.
\end{abstract}  
\maketitle  

We are going to explore the relationship between $\left(\frac{p}{q}\right)$ and $\left(\frac{q}{p}\right)$. Let's look at an example:
\begin{question}
 We want to know if $3$ is a quadratic residue modulo $107$. It would be a lot easier to check if $107$ is a quadratic residue modulo $3$. We know that $107\equiv \answer{2}
 \pmod 3$, so $\left(\frac{107}{3}\right)=\answer{-1}
 $. It would be nice if this also gave us $\left(\frac{3}{107}\right)$.
\end{question}

\begin{question}
 Another example: Find $\left(\frac{p}{5}\right)$ and $\left(\frac{5}{p}\right)$.
 
\begin{tabular}{|l||l|l|l|l|l|}\hline
$p$&3&5&7&11&13\\\hline\hline
$\left(\frac{p}{5}\right)$&$\answer{-1}$&$\answer{0}$&$\answer{-1}$&$\answer{1}$&$\answer{-1}$\\\hline
$\left(\frac{5}{p}\right)$&-1&0&-1&1&-1\\\hline
\end{tabular}
\end{question}

\begin{question}
 Another example: Find $\left(\frac{p}{7}\right)$ and $\left(\frac{7}{p}\right)$.
 
\begin{tabular}{|l||l|l|l|l|l|}\hline
$p$&3&5&7&11&13\\\hline\hline
$\left(\frac{p}{7}\right)$&$\answer{-1}$&$\answer{-1}$&0&$\answer{1}$&$\answer{-1}$\\\hline
$\left(\frac{7}{p}\right)$&$\answer{1}$&$\answer{-1}$&0&-1&-1\\\hline
\end{tabular}
\end{question}

This gives some evidence for our theorem:
\begin{thm}
 Let $p$ and $q$ be odd primes with $p\neq q$. 
\begin{itemize}
 \item if $p\equiv 1 \pmod 4$ or $q\equiv 1 \pmod 4$, then $\left(\frac{p}{q}\right)=\left(\frac{q}{p}\right)$
 \item if $p\equiv q\equiv 3 \pmod 4$, then $\left(\frac{p}{q}\right)=-\left(\frac{q}{p}\right)$
\end{itemize}
\end{thm}

Our goal for Friday is to prove this.
\end{document}
