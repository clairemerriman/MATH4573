\documentclass{ximera}  
\theoremstyle{plain}
\newtheorem{thm}{Theorem}

\title{Preclass assignment for March 25}  
\begin{document}  
\begin{abstract}  
We will review a some points about primitive roots, quadratic residues, and the Legendre symbol from before break, then finish those sections.
\end{abstract}  
\maketitle  

\begin{question}
For a prime $p$, a primitive root there exists modulo $p$.
\begin{multipleChoice}
 \choice[correct] {Always}
 \choice {Sometimes}
 \choice {Never}
\end{multipleChoice}
\end{question}

\begin{question}
 If $n=pq$ where $p$ and $q$ are distinct primes, then there exists a primitive root modulo $n$.
 \begin{multipleChoice}
 \choice{Always}
 \choice[correct] {Sometimes}
 \choice {Never}
\end{multipleChoice}
\end{question}

\begin{question}
 If $n=2^k$ and $k\geq 3$, then there exists a primitive root modulo $n$.
 \begin{multipleChoice}
 \choice{Always}
 \choice {Sometimes}
 \choice[correct] {Never}
\end{multipleChoice}
\end{question}

\begin{question}
 If $n=km$ where $k$ and $m$ are relatively prime and greater than 2, then there exists a primitive root modulo $n$.
 \begin{multipleChoice}
 \choice{Always}
 \choice {Sometimes}
 \choice[correct] {Never}
\end{multipleChoice}
\end{question}

\begin{question}
There exists primitive roots modulo $n$ when for $n=$
 \begin{selectAll}
 \choice[correct] {1}
 \choice[correct] {$p$ a prime}
 \choice[correct] {4}
\choice {$2^m$ for $m\geq 3$}
\choice[correct] {$p^m$ for $p$ an odd prime}
\choice[correct] {$2 p^m$ for $p$ an odd prime}
\choice{$n$ a composite number with at least two distinct odd prime factors}
\end{selectAll}
\end{question}


%%%%%%%%%%%%%%%%%%%%%%%%%%%%%%%%%
\begin{question}
 Let $p>2$ be a prime, and let $a$ be an integer between $0$ and $p-1$.
 
\begin{itemize}
 \item If $a$ is a quadratic residue modulo $p$, then $a^{\frac{p-1}{2}}=\answer{1}
 $.
 \item If $a$ is a quadratic nonresidue modulo $p$, then $a^{\frac{p-1}{2}}=\answer{-1}
 $.
 \item Otherwise, $a^{\frac{p-1}{2}}=\answer{0}
 $.
\end{itemize}
\end{question}

\begin{question}
 Euler's identity:  Let $p>2$ be a prime, and let $a$ be an integer. Then $\left(\answer{\frac{a}{p}}
 \right)\equiv a^{\frac{p-1}{2}} \pmod p$.
\end{question}

\begin{thm}
 Let $p>2$ be prime.
\begin{itemize}
 \item If $p\equiv 1 \pmod 4$, then $-1$ is a quadratic residue modulo $p$.
 \item If $p\equiv 3 \pmod 4$, then $-1$ is a quadratic nonresidue modulo $p$.
\end{itemize}
\end{thm}
\begin{proof}
 For an arbitrary prime $p>2$, Euler's identity tells us that $\left(\frac{-1}{p}\right)\equiv (-1)^{\frac{p-1}{2}} \pmod p$. Note that, we have that $\left(\frac{-1}{p}\right)$ is either $+1$ or $-1$ by definition, and $(-1)^{\frac{p-1}{2}}$ is also either $+1$ or $-1$. Since $1\not\equiv -1 \pmod p$, the two sides of the congruence are actually equal. That is, $\left(\frac{-1}{p}\right)= (-1)^{\frac{p-1}{2}} $.
 
 The completion of the proof involves applying the answer to the preclass assignment, and the proof is on homework 9.
 \end{proof}
 
\begin{question}
Let $p>2$ be prime, and let $a$ and $b$ be integers between $1$ and $p-1$.
\begin{itemize}
\item If $ab$ is a quadratic residue, then
\begin{selectAll}
\choice[correct]
{$a$ and $b$ are both quadratic residues}
\choice[correct]
{$a$ and $b$ are both quadratic nonresidues}
\choice
{One of $a$ and $b$ is a quadratic residue and the other is a quadratic nonresidue}
\end{selectAll}
 \item If $ab$ is a quadratic nonresidue, then
\begin{selectAll}
\choice[correct]
{$a$ and $b$ are both quadratic residues}
\choice[correct]
{$a$ and $b$ are both quadratic nonresidues}
\choice
{One of $a$ and $b$ is a quadratic residue and the other is a quadratic nonresidue}
\end{selectAll}
\end{itemize}
 
\end{question}


%%%%%%%%%%%%%%%%%%%%%%%%%%%%%%%%%
\section{Quadratic reciprocity}
We are going to explore the relationship between $\left(\frac{p}{q}\right)$ and $\left(\frac{q}{p}\right)$. Let's look at an example:
\begin{question}
 We want to know if $3$ is a quadratic residue modulo $107$. It would be a lot easier to check if $107$ is a quadratic residue modulo $3$. We know that $107\equiv \answer{2}
 \pmod 3$, so $\left(\frac{107}{3}\right)=\answer{-1}
 $. It would be nice if this also gave us $\left(\frac{3}{107}\right)$.
\end{question}

\begin{question}
 Another example: Find $\left(\frac{p}{5}\right)$ and $\left(\frac{5}{p}\right)$.
 
\begin{tabular}{|l||l|l|l|l|l|}\hline
$p$&3&5&7&11&13\\\hline\hline
$\left(\frac{p}{5}\right)$&$\answer{-1}$&$\answer{0}$&$\answer{-1}$&$\answer{1}$&$\answer{-1}$\\\hline
$\left(\frac{5}{p}\right)$&-1&0&-1&1&-1\\\hline
\end{tabular}
\end{question}

\begin{question}
 Another example: Find $\left(\frac{p}{7}\right)$ and $\left(\frac{7}{p}\right)$.
 
\begin{tabular}{|l||l|l|l|l|l|}\hline
$p$&3&5&7&11&13\\\hline\hline
$\left(\frac{p}{7}\right)$&$\answer{-1}$&$\answer{-1}$&0&$\answer{1}$&$\answer{-1}$\\\hline
$\left(\frac{7}{p}\right)$&$\answer{1}$&$\answer{-1}$&0&-1&-1\\\hline
\end{tabular}
\end{question}

This gives some evidence for our theorem:
\begin{thm}
 Let $p$ and $q$ be odd primes with $p\neq q$. 
\begin{itemize}
 \item if $p\equiv 1 \pmod 4$ or $q\equiv 1 \pmod 4$, then $\left(\frac{p}{q}\right)=\left(\frac{q}{p}\right)$
 \item if $p\equiv q\equiv 3 \pmod 4$, then $\left(\frac{p}{q}\right)=-\left(\frac{q}{p}\right)$
\end{itemize}
\end{thm}

Our goal for Friday is to prove this.



\end{document}