\documentclass{ximera}

\theoremstyle{plain}
\newtheorem{thm}{Theorem}%[section] % reset theorem numbering for each section
\newtheorem*{thm*}{Theorem}%[section] % reset theorem numbering for each section
\newtheorem{prop}[thm]{Proposition}
\newtheorem{lem}[thm]{Lemma}
\theoremstyle{definition}

\title{Quadratic residues}  
\begin{document}  
\begin{abstract}  
We will review a some points about quadratic residues and the Legendre symbol from before break, and finish those sections.
\end{abstract}  
\maketitle  

\begin{question}
 Let $p>2$ be a prime, and let $a$ be an integer between $0$ and $p-1$.
 
\begin{itemize}
 \item If $a$ is a quadratic residue modulo $p$, then $a^{\frac{p-1}{2}}=\answer{1}
 $.
 \item If $a$ is a quadratic nonresidue modulo $p$, then $a^{\frac{p-1}{2}}=\answer{-1}
 $.
 \item Otherwise, $a^{\frac{p-1}{2}}=\answer{0}
 $.
\end{itemize}
\end{question}

\begin{question}
 Euler's identity:  Let $p>2$ be a prime, and let $a$ be an integer. Then $\left(\answer{\frac{a}{p}}
 \right)\equiv a^{\frac{p-1}{2}} \pmod p$.
\end{question}

\begin{thm}
 Let $p>2$ be prime.
\begin{itemize}
 \item If $p\equiv 1 \pmod 4$, then $-1$ is a quadratic residue modulo $p$.
 \item If $p\equiv 3 \pmod 4$, then $-1$ is a quadratic nonresidue modulo $p$.
\end{itemize}
\end{thm}
\begin{proof}
 For an arbitrary prime $p>2$, Euler's identity tells us that $\left(\frac{-1}{p}\right)\equiv (-1)^{\frac{p-1}{2}} \pmod p$. Note that, we have that $\left(\frac{-1}{p}\right)$ is either $+1$ or $-1$ by definition, and $(-1)^{\frac{p-1}{2}}$ is also either $+1$ or $-1$. Since $1\not\equiv -1 \pmod p$, the two sides of the congruence are actually equal. That is, $\left(\frac{-1}{p}\right)= (-1)^{\frac{p-1}{2}} $.
 
 The completion of the proof involves applying the answer to the preclass assignment, and the proof is on homework 9.
 \end{proof}
 
\begin{question}
Let $p>2$ be prime, and let $a$ and $b$ be integers between $1$ and $p-1$.
\begin{itemize}
\item If $ab$ is a quadratic residue, then
\begin{selectAll}
\choice[correct]
{$a$ and $b$ are both quadratic residues}
\choice[correct]
{$a$ and $b$ are both quadratic nonresidues}
\choice
{One of $a$ and $b$ is a quadratic residue and the other is a quadratic nonresidue}
\end{selectAll}
 \item If $ab$ is a quadratic nonresidue, then
\begin{selectAll}
\choice[correct]
{$a$ and $b$ are both quadratic residues}
\choice[correct]
{$a$ and $b$ are both quadratic nonresidues}
\choice
{One of $a$ and $b$ is a quadratic residue and the other is a quadratic nonresidue}
\end{selectAll}
\end{itemize}
 
\end{question}
\end{document}