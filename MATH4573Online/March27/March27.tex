\documentclass{ximera}

%\newtheorem{theorem}{Theorem}%[section] % reset theorem numbering for each section
%\newtheorem*{theorem*}{Theorem}%[section] % reset theorem numbering for each section
\newtheorem{prop}[theorem]{Proposition}
\newtheorem{lem}[theorem]{Lemma}
\theoremstyle{definition}

\title{March 27 -- Lemma's for quadratic reciprocity}  
\begin{document}  
\begin{abstract}  
We will prove the two lemmas needed in order to prove quadratic reciprocity.\end{abstract}  
\maketitle  

We want to prove 
\begin{theorem}[Quadratic reciprocity]
 Let $p$ and $q$ be primes with $p\neq q$, then 
\begin{itemize}
 \item if $p\equiv 1 \pmod 4$ or $q\equiv 1 \pmod 4$, then $\left(\frac{p}{q}\right)=\left(\frac{q}{p}\right)$.
 \item if $p\equiv q \equiv 3 \pmod 4$, then $\left(\frac{p}{q}\right)=-\left(\frac{q}{p}\right)$.
\end{itemize}
\end{theorem}

Let's try some examples: 
\begin{question}
 $\left(\frac{11}{47}\right)=\answer{-1}
* \left(\frac{47}{11}\right)$. We can reduce $47\equiv\answer{3}
 \pmod{11}$, which 
\begin{multipleChoice}
 \choice[correct] {is}
 \choice {is not}
\end{multipleChoice}
a quadratic residue modulo 11. Thus,  $\left(\frac{11}{47}\right)=\answer{-1}
$ and $ \left(\frac{47}{11}\right)=\answer{1}
.$ 
\end{question}

\begin{question}
 $\left(\frac{3}{107}\right)=\answer{-1}
* \left(\frac{107}{3}\right)$. We can reduce $107\equiv\answer{2}
 \pmod{3}$, which 
\begin{multipleChoice}
 \choice{is}
 \choice[correct]  {is not}
\end{multipleChoice}
a quadratic residue modulo 3. Thus,  $\left(\frac{107}{3}\right)=\answer{-1}
$ and $ \left(\frac{3}{107}\right)=\answer{1}
.$ 
\end{question}

We are going to restate quadratic reciprocity as
\begin{theorem}[Restatement of quadratic reciprocity]
 Let $p$ and $a$ be odd primes with $p\neq q$. Then \[\left(\frac{p}{q}\right)\left(\frac{q}{p}\right)=(-1)^{\frac{p-1}{2}\frac{q-1}{2}}.\]
\end{theorem}

\begin{theorem}The restatement of quadratic reciprocity implies quadratic reciprocity.
\end{theorem}
\begin{proof}
 Let $p$ and $q$ be odd primes with $p\neq q$. We assume that $\left(\frac{p}{q}\right)\left(\frac{q}{p}\right)=(-1)^{\frac{p-1}{2}\frac{q-1}{2}}$ is true. Then we have two cases:
 
\begin{itemize}
 \item $p\equiv 1 \pmod 4$ or $q\equiv 1 \pmod 4$ [To show $\left(\frac{p}{q}\right)=\answer{1}
* \left(\frac{q}{p}\right)$.]

Without loss of generality, we assume $p\equiv 1 \pmod 4$. Then there exists a $k\in\mathbb{Z}$ such that $p=4k+1$. This implies that $\frac{p-1}{2}=2k$. Thus, \[\left(\frac{p}{q}\right)\left(\frac{q}{p}\right)=(-1)^{\frac{p-1}{2}\frac{q-1}{2}}=1^\frac{q-1}{2}=1.\]
Thus, we have that $\left(\frac{p}{q}\right)$ and $\left(\frac{q}{p}\right)$ must either both be $+1$ or both be $-1$.

\item $p\equiv q\equiv 3 \pmod 4$[To show $\left(\frac{p}{q}\right)=\answer{-1}
* \left(\frac{q}{p}\right)$.]
 There exists $k,m\in\mathbb{Z}$ such that $p=4k+3$ and $q=4m+3$. This implies that $\frac{p-1}{2}=2k+1$ and $\frac{q-1}{2}=2k+1$. Thus, \[\left(\frac{p}{q}\right)\left(\frac{q}{p}\right)=(-1)^{\frac{p-1}{2}\frac{q-1}{2}}=(-1)^\frac{q-1}{2}=-1.\]
Thus, we have that exactly one of $\left(\frac{p}{q}\right)$ and $\left(\frac{q}{p}\right)$ is $+1$ and the other is $-1$. \qedhere
\end{itemize}
\end{proof}

In order to prove this, we first need to prove two rather technical lemmas. Then we will use a geometric proof to finish.

\begin{theorem}[Gauss's lemma]
Let $p$ be an odd prime number and let $a\in\mathbb{Z}$ with $p\nmid a$. Let $n$ be the number of least positive residues of the integers $a,2a,\dots, \frac{p-1}{2} a$ that are greater than $\frac{p}{2}$. Then 
\[\left(\frac{a}{p}\right)=(-1)^n.\]
\end{theorem}

\begin{question}
 Use Gauss's lemma to find $\left(\frac{6}{11}\right)$. We need to find $n$, the number of least nonnegative positive residues of the integers $6,2*6,3*6,4*6,5*6$ greater that $\answer{5.5}
 $
 We have 
 
\begin{align*}
 6\equiv \answer{6}
 \pmod{11}\\
 2*6\equiv\answer{1}
 \pmod{11}\\
 3*6\equiv\answer{7}
 \pmod{11}\\
 4*6\equiv\answer{2}
 \pmod{11}\\
 5*6\equiv\answer{8}
 \pmod{11}
\end{align*}
Thus, $n=\answer{3}
$ and $(-1)^n=\answer{-1}
$.
\end{question}

We now prove Gauss's lemma.
\begin{proof}
 Let $r_1,r_2,\dots r_n$ be the least nonnegative residues of the integers $a,2a,\dots,\frac{p-1}{2}a$ that are greater than $\frac{p}{2}$ and $s_1,s_2,\dots,s_m$ be the least nonnegative residues that are less that $\frac{p}{2}$. Note that no $r_i$ or $s_j$ is 0, since $p$ does not divide any of $a,2a,\dots \frac{p-1}{2}$. Consider the $\frac{p-1}{2}$ integers given by \[p-r_1,p-r_2,\dots,p-r_n,s_1,s_2,\dots,s_m.\]
 We want to show that these integers are the integers from $1$ to $\frac{p-1}{2}$ inclusive in some order. Since each integer is less than or equal to $\frac{p-1}{2}$, it suffices to show that no two of these integers are congruent modulo $p$. 
 
If $p-r_i\equiv p-r_j \pmod p$ for some $i\neq j$, then $r_i\equiv r_j \pmod p$, but this implies that there exists some $k_i,k_j\in\mathbb{Z}$ such that $r_i=k_ia\equiv k_ja=r_j\pmod p$ with $k_i\neq k_j$ and $1\leq k_i,k_j\leq\answer{\frac{p-1}{2}}
$. Since 
\begin{multipleChoice}
 \choice[correct] {$p\nmid a$}
 \choice {$p\mid a$}
\end{multipleChoice}
 we know that the multiplicative inverse of $a$ modulo $p$ 
\begin{multipleChoice}
 \choice[correct] {exists}
 \choice {does not exist}
\end{multipleChoice}
 and thus $k_i\equiv k_j \pmod p$, a contradiction. Thus, no two of the first $n$ integers are congruent modulo $p$. 
 
 Similarly, no two of the second $m$ integers are congruent. Now, if $p-r_i\equiv s_j \pmod p$, for some $i$ and $j$, then $-r_i\equiv s_j \pmod p$. Thus, there exists $k_i,k_j\in\mathbb{Z}$ such that $-r_i=-k_ia\equiv k_ja=s_j\pmod p$ with $k_i\neq k_j$ and $1\leq k_i,k_j\leq\frac{p-1}{2}$. Since $p\nmid a$, we know that the multiplicative inverse of $a$ modulo $p$ exists, and thus $-k_i\equiv k_j \pmod p$, a contradiction.
Thus, the $\frac{p-1}{2}$ integers $p-r_1,p-r_2,\dots,p-r_n,s_1,s_2,\dots,s_m$ are the integers $1,2,\dots,\frac{p-1}{2}$ in some order. 

Then, \[(p-r_1)(p-r_2)\cdots(p-r_n)s_1s_2\cdots s_m\equiv\frac{p-1}{2}! \pmod p\]
implies that \[(-1)^nr_1r_2\cdots r_ns-1s_2\cdots s_m\equiv\frac{p-1}{2}! \pmod p.\]
By the definition of $r_i$ and $s_j$, we have 
\[(-1)^na(2a)(3a)\cdots(\frac{p-1}{2}a)\equiv\frac{p-1}{2}! \pmod p.\]
By reordering, we have 
\[(-1)^na^{\frac{p-1}{2}}\frac{p-1}{2}!\equiv\frac{p-1}{2}! \pmod p.\]
Thus, $(-1)^na^{\frac{p-1}{2}}\equiv 1 \pmod p$, and $a^{\frac{p-1}{2}}\equiv (-1)^n \pmod p$. By Euler's criterion, we get that $\left(\frac{a}{p}\right)\equiv(-1)^n \pmod p$. Since both sides of the congruence must be $\pm1,$ we have $\left(\frac{a}{p}\right)=(-1)^n $.
\end{proof}

We are going to prove a result about $\left(\frac{2}{p}\right)$ before our next technical lemma.

\begin{theorem}
 Let $p$ be an odd prime. Then 
\begin{equation*}
 \left(\frac{2}{p}\right)=(-1)^{\frac{p^2-1}{8}}=
\begin{cases}
 1& if\ p\equiv 1,7 \pmod 8\\
 -1 & if\ p\equiv 3,5 \pmod 8.
\end{cases}
\end{equation*}
\end{theorem}
\begin{proof}
 By Gauss's Lemma, we have that $\left(\frac{2}{p}\right)=(-1)^n,$ where $n$ is the number of least positive residues of the integers $2,2*2,2*3,\dots,\frac{p-1}{2}$ that are greater than $\frac{p}{2}$. Let $k\in\mathbb{Z}$ with $1\leq k\leq \frac{p-1}{2}$. Then $2k<\answer{\frac{p}{2}}
 $ if and only  if $k<\frac{p}{4};$ so $\left\lfloor\answer{\frac{p}{4}}
 \right\rfloor$ of the integers $2,2*2,2*3,\dots,\frac{p-1}{2}$ that are less than $\frac{p}{2}$, where $\lfloor\cdot\rfloor$ is the greatest integer (or floor) function. So, $\frac{p-1}{2}-\left\lfloor\frac{p}{4}\right\rfloor$  of these integers are greater than $\frac{p}{2}$, from which 
 \[\left(\frac{2}{p}\right)=(-1)^{\frac{p-1}{2}-\left\lfloor\frac{p}{4}\right\rfloor}\] by Gauss's Lemma. For the first equality, it suffices to show that 
 \[\frac{p-1}{2}-\left\lfloor\frac{p}{4}\right\rfloor\equiv \frac{p^2-1}{8} \pmod 2.\]
 
 If $p\equiv 1 \pmod 8$, the $p=8k+1$ for some $k\in\mathbb{Z}$. That gives us
 \[\frac{p-1}{2}-\left\lfloor\frac{p}{4}\right\rfloor=\frac{(8k+1)-1}{2}--\left\lfloor\frac{8k+1}{4}\right\rfloor=4k-2k=2k\equiv 0 \pmod 2\] and
 \[\frac{p^2-1}{8}=\frac{8k+1)^2-1}{8}=8k^2+2k\equiv 0\pmod 2.\]
 Thus,  holds when $p\equiv 1 \pmod 8$. The rest of the cases are part of homework 9.
\end{proof}

\begin{theorem}[Rephrased textbook Theorem 3.3]
 Let $p$ be an odd prime number and let $a\in\mathbb{Z}$ with $p\nmid a$ and $a$ odd. If \[N=\sum_{j=1}^{\frac{p-1}{2}}\left\lfloor\frac{ja}{p}\right\rfloor,\] then \[\left(\frac{a}{p}\right)=(-1)^N.\]
\end{theorem}
Where $\lfloor\cdot\rfloor$ is the greatest integer (or floor) function. This gives us another way of computing Legendre symbols. Let's look at an example before diving into the technical proof.

\begin{question}
 Use this lemma to find $\left(\frac{7}{11}\right)$. We have
 \begin{align*}
 N&=\sum_{j=1}^{\answer{5}
 }
 \left\lfloor\frac{j7}{11}\right\rfloor= \left\lfloor\frac{7}{11}\right\rfloor+ \left\lfloor\frac{14}{11}\right\rfloor+\left\lfloor\frac{21}{11}\right\rfloor+ \left\lfloor\frac{28}{11}\right\rfloor+\left\lfloor\frac{35}{11}\right\rfloor\\
 &=\answer{0}
 +\answer{1}
 +\answer{1}
 +\answer{2}
 +\answer{3}
 \\&=
 \answer{7}
 \end{align*}
 So $\left(\frac{7}{11}\right)=(-1)^{\answer{7}
 }=\answer{-1}
 .$
\end{question}

\begin{proof}
 Let $r_1,r_2,\dots,r_n$ are the least nonnegative representatives of $a,2a,3a,\dots,\frac{p-1}{2}a$ modulo $p$ which are greater than $\frac{p}{2}$ and $s_1,s_2,\dots,s_m$ be the least nonnegative representatives of $a,2a,3a,\dots,\frac{p-1}{2}a$ modulo $p$ which are less than $\frac{p}{2}$.  Then for each $j=1,2, \dots, \frac{p-1}{2}$ we have that \[ja=p\left\lfloor\frac{ja}{p}\right\rfloor+\textrm{(remainder depending on $j$)}\]
 where each of $r_1,r_2, \dots, r_n,s_1,s_2,\dots,s_m$ appears exactly once as a remainder. 
 
By adding the $\frac{p-1}{2}$ equations above, we get
\begin{equation}\label{sumja}\sum_{j=1}^{\frac{p-1}{2} }ja=\sum_{j=1}^{\frac{p-1}{2}}p\left\lfloor\frac{ja}{p}\right\rfloor+\sum_{j=1}^n r_j+\sum_{j=1}^m s_j
\end{equation}

The integers $p-r_1,p-r_2,\dots,p-r_n,s_1,s_2,\dots,s_m$ are precisely the integers from $1$ to $\frac{p-1}{2}$ in some order, so we have 
\begin{equation}\label{sumj}
 \sum_{j=1}^{\frac{p-1}{2}} j=\sum_{j=1}^n (p-r_j)+\sum_{j=1}^m s_j=pn-\sum_{j=1}^n r_j+\sum_{j=1}^m s_j
\end{equation}

We subtract \eqref{sumj} from \eqref{sumja} to get 
\begin{align*}
 \sum_{j=1}^{\frac{p-1}{2} }ja- \sum_{j=1}^{\frac{p-1}{2}} j&=\sum_{j=1}^{\frac{p-1}{2}}p\left\lfloor\frac{ja}{p}\right\rfloor+\sum_{j=1}^n r_j+\sum_{j=1}^m s_j -\left( pn-\sum_{j=1}^n r_j+\sum_{j=1}^m s_j \right)\\
 &=\sum_{j=1}^{\frac{p-1}{2}}p\left\lfloor\frac{ja}{p}\right\rfloor -pn +2\sum_{j=1}^n r_j.
\end{align*}
Now, we can factor the left hand side to get 
\[(\answer{a-1}
)\sum_{j=1}^{\frac{p-1}{2}} j=\sum_{j=1}^{\frac{p-1}{2}}p\left\lfloor\frac{ja}{p}\right\rfloor -pn +2\sum_{j=1}^n r_j.\]
Reducing both sides of the equation modulo 2 gives
\[0\equiv \sum_{j=1}^{\frac{p-1}{2}}p\left\lfloor\frac{ja}{p}\right\rfloor -n \pmod 2\] since $p\equiv \answer{1}
\pmod 2$. Equivalently $n\equiv \sum_{j=1}^{\frac{p-1}{2}}p\left\lfloor\frac{ja}{p}\right\rfloor\pmod 2$.

Thus, $n\equiv N \pmod 2$, thus $\left(\frac{a}{p}\right)=(-1)^n=(-1)^N$.
\end{proof}
\end{document}